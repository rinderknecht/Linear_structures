%%-*-latex-*-

\documentclass[10pt]{article}

\usepackage[T1]{fontenc}
\usepackage{pst-tree,pst-eps}
\usepackage{amsmath,amssymb}
\usepackage[scaled=0.8]{beramono}

\newcommand\erlcode[1]{\texttt{#1}}
\newcommand\erlnode[1]{\TR{\erlcode{#1}}}
\newcommand\erlbox[1]{\TR{\fbox{\erlcode{#1}}}}

\begin{document}

\TeXtoEPS
\pstree[nodesep=2pt,levelsep=20pt,treesep=14pt]{\erlnode{join}}{
  \erlnode{[]}
  \TR{\rnode[r]{Qa}{\erlcode{Q}}}
}
\quad\raisebox{-3pt}{\(\xrightarrow{\alpha}\)}\quad
\pstree[levelsep=20pt,treesep=14pt]{\TR{\rnode[cb]{Qup}{\(\circ\)}}}{}
\qquad\qquad
\pstree[nodesep=2pt,levelsep=20pt,treesep=14pt]{\erlnode{join}}{
  \pstree{\erlnode{|}}{
    \TR{\rnode[br]{I}{\erlcode{I}}}
    \TR{\rnode[br]{P}{\erlcode{P}}}
  }
  \TR{\rnode[br]{Qb}{\erlcode{Q}}}
}
\quad\raisebox{-3pt}{\(\xrightarrow{\beta}\)}\quad
\pstree[nodesep=2pt,levelsep=15pt,treesep=14pt]%
  {\TR{\rnode[bl]{cons}{\erlcode{|}}}}{
  \Tn
  \Tn
  \pstree{\TR{\rnode[bl]{joinP}{\rnode[br]{joinQ}{\erlnode{join}}}}}{
    \Tn
  }
}
\nccurve[angleA=-90,angleB=0,nodesepB=1pt,nodesepA=-1pt]{->}{Qup}{Qa}
\nccurve[angleA=-130,angleB=-55,nodesepA=2pt,nodesepB=1pt,ncurvB=1]{->}{cons}{I}
\nccurve[angleA=-120,angleB=-45,nodesepA=5pt]{->}{joinP}{P}
\nccurve[angleA=-60,angleB=-50,nodesepA=5pt,nodesepB=1pt,ncurvA=1]{->}{joinQ}{Qb}
\endTeXtoEPS

\end{document}
