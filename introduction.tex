%%-*-latex-*-

\chapter{Introduction}
%\addcontentsline{toc}{chapter}{Introduction}

Consider the factorial function, defined by the following equations:
\begin{align*}
  1! &= 1,\\
  n! &= n \cdot (n-1)!,\quad \text{if \(n > 1\)}.
\end{align*}
The expression `\(n!\)' reads as `the factorial of the positive
integer~\(n\),' where the factorial is a mathematical function
symbolised by~`!'. The equations mean that the \emph{function
  call}~\(n!\) is the product of the consecutive positive integers up
to~\(n\), which can be written informally as
\begin{equation*}
n! = n \cdot (n-1) \cdot \ldots \cdot 1.
\end{equation*}
Let us use the equational definition to compute \(3!\) as an example:
\begin{equation*}
3! = 3 \cdot (3-1)! = 3 \cdot 2! = 3 \cdot (2 \cdot (2-1)!) = 3
\cdot (2 \cdot 1!) = 3 \cdot (2 \cdot (1)) = 6.
\end{equation*}
Let us be more precise about how this simple computation is carried
out by tagging the equations with Greek letters:
\begin{align*}
  1! &\eqn{\alpha} 1,\\
  n! &\eqn{\beta} n \cdot (n-1)!, \quad \text{if \(n > 1\)}.
\end{align*}
The previous computation is rewritten from left to right as follows:
\begin{equation*}
3! \eqn{\beta} 3 \cdot (3-1)! = 3 \cdot 2! \eqn{\beta} 3 \cdot (2 \cdot
(2-1)!) = 3 \cdot (2 \cdot 1!) \eqn{\alpha} 3 \cdot (2 \cdot (1)) = 6.
\end{equation*}
Two things are worth noticing. Firstly, not all equality
symbols~(\(=\)) are annotated. The unadorned ones correspond to
arithmetic operations, like computing \(3-1 = 2\), and thus are not
part of the definition of the factorial function per se. Secondly,
when a function call is to be replaced by the right\hyp{}hand side of
an equation, it is sometimes first surrounded by parentheses, like
\(2!\)~being replaced by \(\textbf{(}2 \cdot (2-1)!\textbf{)}\)
instead of simply \(2 \cdot (2-1)!\). The purpose is to avoid any
interference with the \emph{control context} of the call, that is, the
part of the formula in which the call is embedded. In the current
example, there is no ambiguity because the control context is always
`\(n \cdot \erlcode{\textvisiblespace}\)', where the underscore
(\erlcode{\textvisiblespace}) is a place\hyp{}holder for the call in
question, but let us recall why parentheses are necessary in \(x \cdot
(y + z)\). The first step, that is, \(3!  \eqn{\beta} 3 \cdot
(3-1)!\), does not require extra parentheses because there was no
control context around~\(3!\) since it is the original call.

\bigskip

\paragraph{Follow the arrows.}

The way the previous equations were used suggests another notation,
which provides further hints about their usage:
\begin{alltt}
fact(\(1\)) \(\xrightarrow{\smash[t]{\alpha}}\) 1;
fact(\(n\)) \(\xrightarrow{\smash[t]{\beta}}\) \(n\) *
fact(\(n-1\)), \textrm{if \(n>1\).}
\end{alltt}
We replaced the factorial function \emph{symbol} (\(!\)) by the
function \emph{name} \erlcode{fact}. Function names are placed before
the value they apply to, for example, \erlcode{fact(\(n\))} instead
of~\(n!\) which is usual in mathematics. Equality symbols have been
replaced by arrows (\(\rightarrow\)) oriented from left to right. It
is a fundamental property that equations are not oriented: every time
\(x = y\) is true, \(y = x\) is as well and vice versa. In practice,
when using equations to define a function, they are often implicitly
oriented from left to right in order to hint at how they should be
used to compute. But, in theory, nothing impedes writing
\begin{align*}
  1! &= 1,\\
  n \cdot (n-1)! &= n!, \quad \text{if} \; n > 1.
\end{align*}
By replacing~(\(=\)) with~(\(\rightarrow\)), an orientation is imposed
on the equations, which become \emph{rewrite rules}. By restricting
the way an equation is used, that is, either only from left to right
or only from right to left, some generality may be lost, that is, some
formulas might not be computed anymore, but it is a necessary
compromise to automate computations. It is sometimes possible to
complete rewrite rules with more rules in order to regain the lost
expressivity when orienting the initial equations, but this is far
off\hyp{}topic here and we shall focus on designing rewrite rules from
the start, instead of deriving them from a set of equations.

When considering rewrite rules, we must distinguish the left\hyp{}hand
side from the right\hyp{}hand side of~(\(\rightarrow\)). For example,
the expressions used as input to a function in the right\hyp{}hand
side are called \emph{arguments}, for instance, \(n-1\)~is an argument
of~\erlcode{fact}. The left\hyp{}hand sides are restricted to be
function calls and their arguments are either constants, like~\(1\),
or \emph{variables}, which are names denoting \emph{values}. Values
are expressions which cannot be computed further, like an
integer. These variables on the left\hyp{}hand side are called
\emph{parameters}. So the \(n\)~in \erlcode{fact(\(n\))} is a
parameter of \erlcode{fact} and \(n\) in
`\erlcode{\(n\)~*~\textvisiblespace}' is simply called a
variable. Parameters refer potentially to an infinite number of
values, for example, the parameter~\(n\) in \erlcode{fact(\(n\))} can
be bound to any integer. The previous computation is now written as
follows:
\begin{alltt}
fact(3) \(\xrightarrow{\beta}\) 3 * fact(3-1)       \(=\) 3 * fact(2)
        \(\xrightarrow{\smash[t]{\beta}}\) 3 * (2 * fact(2-1)) \(=\) 3 * (2 * fact(1))
        \(\xrightarrow{\smash[t]{\alpha}}\) 3 * (2 * (1))       \(=\) 6\textrm{.}
\end{alltt}
The final integer is the value of the initial function call. A value
contains neither function calls nor arithmetic operators, because such
calls would need themselves to be thoroughly rewritten. For the
moment, values can only be integers. It is often clearer to implicitly
compose intermediary arithmetic operations~(\(=\)) with the current
rewrite and write in short
\begin{alltt}
fact(3) \(\xrightarrow{\beta}\) 3*fact(2) \(\xrightarrow{\smash[t]{\beta}}\) 3*(2*fact(1)) \(\xrightarrow{\smash[t]{\alpha}}\) 3*(2*(1)) \(=\) 6\textrm{.}
\end{alltt}
Note how the last rewrite (\(\xrightarrow{\smash[t]{\alpha}}\)) must be
followed by a series of multiplications~\erlcode{3*(2*1)} because each
individual multiplication had to be delayed until \erlcode{fact(1)} be
computed. This could have been anticipated because the call to
\erlcode{fact} in the right\hyp{}hand side of the rewrite rule
(\(\xrightarrow{\smash[t]{\beta}}\)), that is, the underlined text in
\begin{alltt}
fact(\(n\)) \(\xrightarrow{\smash[t]{\beta}}\) \(n\) * \underline{\erlcode{fact(\(n\)-1)}}
\end{alltt}
has the non\hyp{}empty control context
`\erlcode{\(n\)~*~\textvisiblespace}'. To understand why this is
important, let us consider a slightly longer series of
rewrites:\label{trace:fact_5} \begin{alltt}
fact(5) \(\xrightarrow{\beta}\) 5 * fact(4)
        \(\smashedrightarrow{\beta}\) 5 * (4 * fact(3))
        \(\smashedrightarrow{\beta}\) 5 * (4 * (3 * fact(2)))
        \(\smashedrightarrow{\beta}\) 5 * (4 * (3 * (2 * fact(1))))
        \(\smashedrightarrow{\alpha}\) 5 * (4 * (3 * (2 * (1))))    \(=\) 120\textrm{.}
\end{alltt}
 It is clear that each
rewrite by (\(\xrightarrow{\smash[t]{\beta}}\)) yields a longer
expression. An expression can contain function calls, arithmetic
operations, numerical constants and variables. Therefore, \emph{values
  are expressions but the converse does not always hold.}  Let us
focus now only on the shapes of the previous computation:
\begin{alltt}
fact(5) \(\xrightarrow{\smash[t]{\beta}}\) \fbcode{5 * fact(4)}
        \(\xrightarrow{\smash[t]{\beta}}\) \fbcode{5 * (4 * fact(3))}
        \(\xrightarrow{\smash[t]{\beta}}\) \fbcode{5 * (4 * (3 * fact(2)))}
        \(\xrightarrow{\smash[t]{\beta}}\) \fbcode{5 * (4 * (3 * (2 * fact(1))))}
\end{alltt}
This phenomenon suggests that a great deal of space, that is, computer
\emph{memory}, is needed to keep the expressions before the final,
long arithmetic computations. The example leads to induce that the
larger term occurring in the computing of \erlcode{fact(\(n\))} is the
one just before (\(\xrightarrow{\smash[t]{\alpha}}\)) is applied and its
size is likely to be proportional to~\(n\), since all the integers
from \(n\)~to~\(1\) had to be kept until the end.

The dual concept of space is time. By orienting the original
equations, the definition becomes more operational, it conveys the
notions of `before' a rewrite, `after' a rewrite and, transitively, of
a `series' of rewrites as a historical record. Therefore, the number
of rewrites could be used as a measure of the time needed to compute
the value of a function call. For instance, \erlcode{fact(\(n\))}
requires \(n\)~rewrites by (\(\xrightarrow{\smash[t]{\beta}}\)) plus
one by (\(\xrightarrow{\smash[t]{\alpha}}\)), hence a total
of~\(n+1\).

\bigskip

\paragraph{Introducing \Erlang.}

\Erlang is a \emph{functional programming language}, which means that
it supports a programming style mainly based on the design and
processing of mathematical functions. In \Erlang, the previous rewrite
system is written
\begin{verbatim}
fact(1)            -> 1;
fact(N) when N > 1 -> N * fact(N-1).
\end{verbatim}
Notice how the syntax is similar to mathematics. A difference is that,
in \Erlang, the first letter of a variable must be set in upper case,
like~\erlcode{N}, \erlcode{Number} or \erlcode{Var}, to distinguish it
from a function name, whose first letter must be set in lower case,
like~\erlcode{plusOne}. Let us remark that, mimicking a usage in
English prose, each rewrite rule is either followed by a semicolon or
a period, the entire definition being conceived as a complete
sentence. Also, \erlcode{when} is a~\emph{keyword}, that is to say, it
looks like a function name but it is in practice reserved only for
expressing conditions on a rewrite rule, like \erlcode{N~>~1}. In
\Erlang, a rewrite rule is called a~\emph{clause}. Its left\hyp{}hand
side is called the~\emph{head} and the right\hyp{}hand side
the~\emph{body}. A condition on a clause is called a~\emph{guard}. (As
a side note, the lexical conventions, syntax and vocabulary of \Erlang
have been drawn indirectly from the \Prolog programming language.)
The structure of a clause in \Erlang is summed up in
\fig~\vref{fig:clause}.
\begin{figure}[t]
\centering
\includegraphics[bb=214 602 397 646]{clause}
\caption{Structure of a clause in \Erlang\label{fig:clause}}
\end{figure}
A head has always the shape of a function call. A body is an
expression. The guard in the definition of~\erlcode{fact} implies that
there is no overlap between the two clauses. By overlap, we mean that,
for a given function call, either no clause apply, thus no rewrite is
entailed, or only one does. It is thus impossible for two clauses to
carry on the next calculation. This becomes clearer if the clauses are
rewritten in the following way:
\begin{alltt}
fact(N) \textbf{when N = 1} -> N;
fact(N) when N > 1 -> N * fact(N-1).
\end{alltt}
Obviously, the two guards are mutually exclusive, that is, when one is
true the other one is false, thus enabling exactly one clause or none
if~\erlcode{N~<~1}. As a consequence, the previous definition could
have been written instead as follows:
\begin{verbatim}
fact(N) when N > 1 -> N * fact(N-1);
fact(1)            -> 1.
\end{verbatim}
The pieces of source code up to now are not complete \Erlang programs
for an \Erlang program to be self\hyp{}contained needs to be a
\emph{module}. A module is a unit of compilation containing a
collection of function definitions. The module name must be the
basename of the file containing the program. For example, if the
definition above is found in a file named~\erlcode{math1.erl}, the
corresponding module name must be~\erlcode{math1}:
\begin{alltt}
\textbf{-module(math1).}\hfill% \emph{Drop the file extension} .erl
\textbf{-export([fact/1]).}
fact(1)            -> 1;
fact(N) when N > 1 -> N * fact(N-1).
\end{alltt}
The \erlcode{-export} line lists the function names which can be
called from outside the module, that is, either from another module or
from the \Erlang \emph{shell}. A shell is an application which reads
commands entered by some user, interprets them, prints a result or an
error message and waits for further commands. The `\erlcode{/1}'
part of `\erlcode{fact/1}' means that the function
named~\erlcode{fact} which has exactly one parameter is exported. The
reason for specifying the number of parameters is that \Erlang
functions can be overloaded, that is, it is possible to define two
functions with the same name as long as they differ in the number of
their parameters.

In order to test some examples with \erlcode{fact/1}, we first have to
launch the \Erlang shell. Depending on your operating system, your
programming environment may vary greatly. Here, we shall assume a
command\hyp{}line interface, like the ones available in a terminal for
the \Unix operating systems. The \Erlang shell is an application which
allows us to interactively compile and call functions from modules and
its name is likely to be~\erlcode{erl}. Here is the start of a session
with the shell:
\begin{alltt}
\$ erl
Erlang (BEAM) emulator version 5.6.3 [source] [smp:2]
[async-threads:0] [kernel-poll:false]

Eshell V5.6.3  (abort with ^G)
1> \(\talloblong\)
\end{alltt}
The first line is the command to run the shell. The last line is the
prompt of the \Erlang shell, the number meaning that the shell is
waiting for the first command. Note that the operating system prompt
is denoted by a dollar sign (\erlcode{\$}). The character
\(\talloblong\) denotes the blinking prompt of the \Erlang shell where
typing will occur. If we want to close the shell and return to the
operating system shell, just type~`\erlcode{q().}' (standing
for~`quit'). Each command must be terminated by a
period~(\erlcode{.})  and followed by a pressure on the return key.
\begin{alltt}
1> q().
ok
2> \$ \textvisiblespace
\end{alltt}
The character \erlcode{\textvisiblespace} represents where text is to
be typed in the operating system shell. But before quitting the
\Erlang shell, the first action usually consists in calling the
\Erlang compiler to process some module we want to use. This is done
by the command~`\erlcode{c}', whose argument is the module name. In
our example, the filename is \texttt{math1.erl}:
\begin{alltt}
1> c(math1).
\{ok,math1\}
2> \(\talloblong\)
\end{alltt}
The compilation was successful, as the \erlcode{ok} suggests. Let us
compute some factorials now:
\begin{alltt}
2> math1:fact(4).
24
3> math1:fact(-3).
** exception error: no function clause matching
math1:fact(-3)
4> \(\talloblong\)
\end{alltt}
The error message is very legible. In this book, we will rarely copy
and paste the input to and the output from the \Erlang shell. We will
not write complete modules as well because we want to focus on the
programming itself and delegate the practical aspects to a user manual
or a textbook oriented towards practice.

\bigskip

\paragraph{A classic variation.}

Consider another way of writing the factorial:
\begin{verbatim}
-module(math2).
-export([fact_tf/1]).
fact_tf(N) when N >= 1 -> fact_tf(N,1).
fact_tf(1,A)           -> A;
fact_tf(N,A)           -> fact_tf(N-1,A*N).
\end{verbatim}
\noindent The \Erlang comparison (\erlcode{>=}) stands for `greater
or equal,' which is usually written (\erlcode{=>}) in other
programming languages. This is because the sequence of
symbols~(\erlcode{=>}) reminded too much of the logical
implication~(\(\Rightarrow\)). Notice how there are actually two
definitions, each one concluded by a period. Both functions are named
\erlcode{fact\_tf} but the first one takes only one parameter (the
first clause), whilst the second one takes two parameters (the two
last clauses). In other words, the former is completely qualified by
\erlcode{fact\_tf/1} and the latter by \erlcode{fact\_tf/2}. We can
remark also that \Erlang allows us to write the definition of a
function, such as \erlcode{fact\_tf/2}, \emph{after} another
definition making use of it, such as \erlcode{fact\_tf/1}. The
function~\erlcode{fact\_tf/2} is not meant to be called from outside
the module, the rationale being that, instead of delaying the
multiplications, exactly one multiplication is going to be computed at
each rewrite, thus, in the end, nothing remains to be done---there is
no instance of the control context to resume. The reason for this new
version is hence to minimise the total memory needed to calculate the
factorial. In order to follow an example with this new
module~\erlcode{math2}, let us revert for a moment to the mathematical
notation, so the clauses can be labelled with Greek letters and easily
distinguished, but let us keep the \Erlang variables:
\begin{alltt}
fact_tf(N)   \(\xrightarrow{\smash[t]{\alpha}}\) fact_tf(N,1),      \textrm{if} N > 0.
fact_tf(1,A) \(\xrightarrow{\smash[t]{\beta}}\) A;
fact_tf(N,A) \(\xrightarrow{\smash[t]{\gamma}}\) fact_tf(N-1,A*N).
\end{alltt}
The previously considered function call \erlcode{fact\_tf(5)} is
rewritten as
\begin{alltt}
fact_tf(5) \(\xrightarrow{\smash[t]{\alpha}}\) fact_tf(5,1)\textrm{,}        \textrm{since} 5 > 1\textrm{,}
           \(\xrightarrow{\smash[t]{\gamma}}\) fact_tf(5-1,1*5)  \(=\) fact_tf(4,5)
           \(\xrightarrow{\smash[t]{\gamma}}\) fact_tf(4-1,5*4)  \(=\) fact_tf(3,20)
           \(\xrightarrow{\smash[t]{\gamma}}\) fact_tf(3-1,20*3) \(=\) fact_tf(2,60)
           \(\xrightarrow{\smash[t]{\gamma}}\) fact_tf(2-1,60*2) \(=\) fact_tf(1,120)
           \(\xrightarrow{\smash[t]{\beta}}\) 120\textrm{.}
\end{alltt}
The reason why \erlcode{fact\_tf(5)} = \erlcode{fact(5)} is that
\begin{equation}
  (((1 \cdot 5) \cdot 4) \cdot 3) \cdot 2
= 5 \cdot (4 \cdot (3 \cdot (2 \cdot 1))).\label{eq:fact5}
\end{equation}
This equality holds because, in general, for all numbers~\(x\),
\(y\)~and~\(z\),
\begin{enumerate}

  \item \label{mult_assoc} the multiplication is associative: \(x \cdot (y
    \cdot z) = (x \cdot y) \cdot z\);

  \item \label{mult_one} the number \(1\) is neutral with respect to
    (\(\cdot\)): \(x \cdot 1 = 1 \cdot x = x\).

\end{enumerate}
To show exactly why, let us write~(\(\eqn{\ref{mult_assoc}}\))
and~(\(\eqn{\ref{mult_one}}\)) to denote, respectively, the use of
associativity and neutrality, then lay out the following equalities
leading from the left\hyp{}hand side to the right\hyp{}hand side of
the purported equality~\eqref{eq:fact5}:
\begin{align*}
  (((1 \cdot 5) \cdot 4) \cdot 3) \cdot 2
  &\eqn{\ref{mult_one}}   ((((1 \cdot 5) \cdot 4) \cdot 3) \cdot 2)
  \cdot 1\\
  &\eqn{\ref{mult_assoc}}   (((1 \cdot 5) \cdot 4) \cdot 3) \cdot (2
    \cdot 1)\\ 
  &\eqn{\ref{mult_assoc}} ((1 \cdot 5) \cdot 4) \cdot (3 \cdot (2
    \cdot 1))\\
  &\eqn{\ref{mult_assoc}} (1 \cdot 5) \cdot (4 \cdot (3 \cdot (2 \cdot
    1))\\
  &\eqn{\ref{mult_assoc}} 1 \cdot (5 \cdot (4 \cdot (3 \cdot (2 \cdot
    1))))\\
  &\eqn{\ref{mult_one}} 5 \cdot (4 \cdot (3 \cdot (2 \cdot 1))).\quad
    \text{\textsc{qed}.}
\end{align*}
Furthermore, if we do not want to rely upon the neutrality of~\(1\),
we could set the initial call to \erlcode{fact\_tf(N-1,N)} and stop
when the number is~\(0\), instead of~\(1\) (change in
bold):\label{code:fact_tf_alpha}
\begin{alltt}
fact_tf(N) when N > 0 \(\xrightarrow{\smash[t]{\alpha}}\) fact_tf(N\textbf{-1},\textbf{N}).
fact_tf(\textbf{0},A)          \(\xrightarrow{\smash[t]{\beta}}\) A;
fact_tf(N,A)          \(\xrightarrow{\smash[t]{\gamma}}\) fact_tf(N-1,A*N).
\end{alltt}
The same example now runs as
\begin{alltt}
fact\_tf(5) \(\xrightarrow{\smash[t]{\alpha}}\) fact\_tf(5-1,5)     \(=\) fact\_tf(4,\ \ 5)
           \(\xrightarrow{\smash[t]{\gamma}}\) fact\_tf(4-1,5*4)   \(=\) fact\_tf(3, 20)
           \(\xrightarrow{\smash[t]{\gamma}}\) fact\_tf(3-1,20*3)  \(=\) fact\_tf(2, 60)
           \(\xrightarrow{\smash[t]{\gamma}}\) fact\_tf(2-1,60*2)  \(=\) fact\_tf(1,120)
           \(\xrightarrow{\smash[t]{\gamma}}\) fact\_tf(1-1,120*1) \(=\) fact\_tf(0,120)
           \(\xrightarrow{\smash[t]{\beta}}\) 120\textrm{.}
\end{alltt}
This new version relies on the following equality which can be proved
only by means of associativity:
\begin{equation*}
  (((5 \cdot 4) \cdot 3) \cdot 2) \cdot 1
= 5 \cdot (4 \cdot (3 \cdot (2 \cdot 1))).
\end{equation*}
The number of rewrites seems greater than with the factorial defined
in module~\erlcode{math1}, precisely one more step due to the
clause~\clause{\alpha}. But this version presents an advantage in
terms of memory usage, as long as it is understood that all integers
within a certain range imposed by the hardware architecture occupy the
same space at run\hyp{}time. This means that we assume, for instance,
that the memory needed to store the number~\erlcode{120} is the same
as for the number~\erlcode{5}. Then the shape of the previous rewrites
is
\begin{alltt}
fact_tf(5) \(\xrightarrow{\smash[t]{\alpha}}\) \fbcode{fact_tf(0,10)}
           \(\xrightarrow{\smash[t]{\gamma}}\) \fbcode{fact_tf(0,10)}
           \(\xrightarrow{\smash[t]{\gamma}}\) \fbcode{fact_tf(0,10)}
           \(\xrightarrow{\smash[t]{\gamma}}\) \fbcode{fact_tf(0,10)}
           \(\xrightarrow{\smash[t]{\gamma}}\) \fbcode{fact_tf(0,10)}
           \(\xrightarrow{\smash[t]{\beta}}\) \fbcode{120}\textrm{.}
\end{alltt}
It is safe to induce that this version uses a constant chunk of
memory, while the version in module~\erlcode{math1} uses an increasing
amount of memory, more precisely a space proportional to~\erlcode{N}
when computing \erlcode{fact\_tf(N)}. This phenomenon has been
anticipated by the keen reader who noticed that there is no control
context for the calls in the clauses of module \erlcode{math2}, so
there are no delayed computations that accumulate until the last
step. As a conclusion, module~\texttt{math2} is always preferable to
\texttt{math1}.

Let us dwell a bit longer on the former. The relative order of the
clauses \clause{\beta}~and~\clause{\gamma} is significant because the
two corresponding heads can overlap, that is, there exists a
\emph{substitution} for the parameters which leads to identical
heads. In this case, by substituting~\erlcode{0} for~\erlcode{N}, the
heads of clauses \clause{\beta}~and~\clause{\gamma} become
identical. Which one should be used when computing
\erlcode{fact\_tf(0,A)}? The intention is to use clause
\clause{\beta}, but how can this preference be expressed?  The problem
could be remedied by adding a condition to clause \clause{\gamma}:
\begin{alltt}
fact_tf(0,A) \(\xrightarrow{\smash[t]{\beta}}\) A;
fact_tf(N,A) \(\xrightarrow{\smash[t]{\gamma}}\) fact_tf(N-1,A*N), \textrm{if} N \(\neq\) 0.
\end{alltt}
In \Erlang, this is expressed as a guard:
\begin{verbatim}
fact_tf(0,A)              -> A;
fact_tf(N,A) when N =/= 0 -> fact_tf(N-1,A*N).
\end{verbatim}
Clearly, the two clauses now can be considered irrespectively of their
relative order, so the following is just fine:
\begin{verbatim}
fact_tf(N,A) when N =/= 0 -> fact_tf(N-1,A*N);
fact_tf(0,A)              -> A.
\end{verbatim}
With the aim to avoid writing too many guards, \Erlang assumes an
implicit order on the clauses as they are written. As a result, when
looking for a match between a function call and a head in a clause,
the firstly written clause is considered first, then, in case of match
failure, the secondly written clause is examined etc. That is why
\begin{verbatim}
fact_tf(0,A) -> A;
fact_tf(N,A) -> fact_tf(N-1,A*N).
\end{verbatim}
is valid because it is implicit that, in the second clause,
\erlcode{N} cannot be~\erlcode{0}, since the first clause would have
been selected otherwise. Also, this means that the following
definition is wrong.
\begin{alltt}
fact_tf(N,A) \(\xrightarrow{\smash[t]{\gamma}}\) fact_tf(N-1,A*N);
fact_tf(0,A) \(\xrightarrow{\smash[t]{\beta}}\) A.
\end{alltt}
Indeed, the second clause would never be chosen: \erlcode{N}
in the first head can always be replaced by any number,
including~\erlcode{0}. As a consequence, the computation never
ends. For example:
\begin{alltt}
fact_tf(3) \(\xrightarrow{\smash[t]{\alpha}}\) fact_tf( 3,1) \(\xrightarrow{\smash[t]{\gamma}}\) fact_tf( 2,3)
           \(\xrightarrow{\smash[t]{\gamma}}\) fact_tf( 1,6) \(\xrightarrow{\smash[t]{\gamma}}\) fact_tf( 0,0)
           \(\xrightarrow{\smash[t]{\gamma}}\) fact_tf(-1,0) \(\xrightarrow{\smash[t]{\gamma}}\) fact_tf(-2,0) \(\rightarrow \ldots\)
\end{alltt}
In other words, the \Erlang definition `loops' forever and the
process running it has to be interrupted. The memory of the computer
will not be exhausted because we already know that this version of
factorial, albeit incorrect, uses a constant amount of space.

Still, would not it be better to guard~\erlcode{fact\_tf(N,A)} from
a negative~\erlcode{N}? For instance, by writing
\begin{verbatim}
fact_tf(N)   when N >= 1 -> fact_tf(N-1,N).
fact_tf(0,A)             -> A;
fact_tf(N,A) when N > 0  -> fact_tf(N-1,A*N).
\end{verbatim}
Indeed, this is a more robust function \erlcode{fact\_tf/2}, since it
now fails on incorrect input, that is, on negative or
zero~\erlcode{N}, instead of entering an infinite loop. But note how
the guard~\erlcode{N~>~0} is evaluated each time the head
of~\erlcode{fact\_tf(N,A)} is successfully matched, which incurs a
time penalty---even though we only take into account the number of
rewrites to approximate the real time. Therefore, the only scenario
where the guard is useful is when calling directly
\erlcode{fact\_tf/2} without passing through the body
of~\erlcode{fact\_tf/1}. Since the only aim of~\erlcode{fact\_tf/2} is
to serve as an auxiliary to~\erlcode{fact\_tf/1}, it could be
envisaged that the guard be left off, while making sure that no direct
calls to~\erlcode{fact\_tf/2} are made in the rest of the
module. Typically, a comment would remind the programmer of this
requisite:
\begin{alltt}
fact_tf(N) when N > 1 -> fact_tf(N-1,N).

% fact_tf(N,A) \emph{assumes} N > 0.
fact_tf(0,A) -> A;
fact_tf(N,A) -> fact_tf(N-1,A*N).
\end{alltt}

\medskip

\paragraph{Fibonacci.}
\label{fibonacci}

Consider the Fibonacci function defined as follows:
\begin{alltt}
fib(0)            \(\xrightarrow{\smash[t]{\alpha}}\) 1;
fib(1)            \(\xrightarrow{\smash[t]{\beta}}\) 1;
fib(N) when N > 1 \(\xrightarrow{\smash[t]{\gamma}}\) fib(N-1) + fib(N-2).
\end{alltt}
Beware that some authors define \erlcode{fib(0) -> 0}, leading to the
sequence to be shifted by one: \(0\),\(1\),\(1\),\(2\),\(3\),\(5\)
etc. instead of our \(1\),\(1\),\(2\),\(3\),\(5\) etc. The peculiarity
here is that the right\hyp{}hand side of clause~\clause{\gamma}
contains two function calls instead of one, as in the factorial
function definition. This yields a cascade of choices as to which call
is computed next. For instance, we underline the call to be rewritten
in the rewrites of~\erlcode{fib(3)}:
\begin{alltt}
\underline{fib(3)} \(\xrightarrow{\smash[t]{\gamma}}\)            \underline{fib(2)} + fib(1)
       \(\xrightarrow{\smash[t]{\gamma}}\) \obp{}fib(1) + \underline{fib(0)}\cbp{} + fib(1)
       \(\xrightarrow{\smash[t]{\alpha}}\) (fib(1) +    \obp{}1\cbp) + \underline{fib(1)}
       \(\xrightarrow{\smash[t]{\beta}}\) (\underline{fib(1)} +    (1)) +    \obp{1}\cbp
       \(\xrightarrow{\smash[t]{\beta}}\) (   \obp{1}\cbp +    (1)) +    (1) = (2) + (1) = 3\textrm{,}
\end{alltt}
but the following rewrites are also admissible:
\begin{alltt}
\underline{fib(3)} \(\xrightarrow{\smash[t]{\gamma}}\)            fib(2) + \underline{fib(1)}
       \(\xrightarrow{\smash[t]{\beta}}\)            \underline{fib(2)} +    (1)
       \(\xrightarrow{\smash[t]{\gamma}}\) (\underline{fib(1)} + fib(0)) +    (1)
       \(\xrightarrow{\smash[t]{\beta}}\) (   (1) + \underline{fib(0)}) +    (1)
       \(\xrightarrow{\smash[t]{\alpha}}\) (   (1) +    (1)) +    (1) = (2) + (1) = 3\textrm{.}
\end{alltt}
The strategy of \Erlang does not specify which call is rewritten
first. The only constraint is that, for a call to be rewritten, its
arguments must be values, that is, they do not contain calls
(including arithmetic operations) themselves. We shall encounter later
the case where function calls are arguments to other function calls
and see what \Erlang does in this case. In the previous example, the
two original calls~\erlcode{fib(2)} and~\erlcode{fib(1)} fulfill this
requirement. Even if the order of rewriting is not revealed to the
programmer, it is guaranteed to be irrelevant. Moreover, the order is
always the same for each execution of the program and only one call is
computed at each step: these are the \emph{deterministic} and
\emph{sequential} paradigms, respectively. Indeed, it may be
imaginable that two or more calls are computed at the same time,
giving birth to the (implicit) \emph{concurrent} paradigm of
computation. Something similar to the fictitious \newlength\Parallel
\settowidth\Parallel{\(_{\gamma\|\gamma}\)}
\begin{alltt}
\underline{fib(3)} \(\MyArrow{\Parallel}{\gamma}\)            \underline{fib(2)} + \underline{fib(1)}
       \(\xrightarrow{\smash[t]{\gamma\|\beta}}\) (\underline{fib(1)} + \underline{fib(0)}) +    (1)
       \(\xrightarrow{\smash[t]{\beta\|\alpha}}\) (   (1) +    (1)) +    (1) = (2) + (1) = 3\textrm{.}
\end{alltt}
But this is not the way \Erlang works by default.

A look again at the two rewrites above reveals that \erlcode{fib(1)}
is computed twice. In order to know whether this is a coincidence, it
is worth trying a longer example and use the concurrent strategy to
save space:
\newlength\ArrowLen
\settowidth\ArrowLen{\(\xrightarrow{\smash[t]{\gamma\|\gamma}}\)}
\def\MyVdots{\(\makebox[\ArrowLen][c]{\(\vdots\)}\)}
\begin{alltt}
\underline{fib(5)} \(\MyArrow{\Parallel}{\gamma}\)            \underline{fib(4)} +            \underline{fib(3)}
       \(\xrightarrow{\smash[t]{\gamma\|\gamma}}\) (\underline{fib(3)} + \underline{fib(2)}) + (\underline{fib(2)} + \underline{fib(1)})
       \(\xrightarrow{\smash[t]{\gamma\|\gamma}}\) ((\underline{fib(2)} + \underline{fib(1)}) + (\underline{fib(1)} + \underline{fib(0)}))
            + ((\underline{fib(1)} + \underline{fib(0)}) + 1)
       \(\xrightarrow{\smash[t]{\hphantom{\gamma\|\gamma}}} \ldots\)
\end{alltt}
It is clear now that some computations are duplicated, leading to the
following slow and memory\hyp{}consuming \Erlang program:
\begin{alltt}
fib(0) -> 1;
fib(1) -> 1;
fib(N) -> fib(N-1) + fib(N-2).\hfill% \emph{Assuming} N > 1
\end{alltt}
In order to visualise the inefficiency, it is handy to use a
two\hyp{}dimensional representation of the calls such as the ones
unfolding from~\erlcode{fib(5)} and shown in
\fig~\vref{fig:fib5_tree}.
\begin{figure}[t]
\centering
\includegraphics[bb=71 631 320 721]{fib5_tree}
\caption{Call tree of \erlcode{fact(5)}
\label{fig:fib5_tree}}
\end{figure}
This kind of graphic is called a \emph{tree}\label{def:tree} and
\fig~\vref{fig:fib5_shape} shows its general shape. The disks are
called \emph{nodes} and the segments which connect two nodes are
called \emph{edges}. There are several kinds of nodes: the one at the
top is called the \emph{root} and the grey ones are called the
\emph{leaves}. Reading the tree top\hyp{}down, a node is directly
connected to some other nodes or none.
\begin{figure}[b]
\centering
\includegraphics{fib5_shape}
\caption{Shape of the call tree of
  \erlcode{fib(5)} in \fig~\ref{fig:fib5_tree}
\label{fig:fib5_shape}}
\end{figure}
If any, these nodes are called the \emph{children} of the upper node
and the latter is the \emph{parent} of the formers. In our particular
case, the tree is a \emph{call tree} because each node represents a
function call, the root being the initial call. Moreover, each node
has two or zero children because each is an instance of a rewrite rule
of the~\erlcode{fib} definition. For instance, clause \clause{\gamma}
corresponds to the tree pattern in \fig~\vref{fig:fib_shape}.  It is
now clear that some subtrees are repeated, the biggest one being
rooted at the function calls~\erlcode{fib(3)}. Note also that the
value of the root is the number or leaves, for example, the value
of~\erlcode{fib(5)} is~\erlcode{8}. An interesting conclusion that can
be drawn from this analysis is that implicit concurrency does not
always result in an optimal computation, since some calls may be
duplicated in parallel rewrites and, therefore, a better course of
action is to find a faster version of the Fibonacci function within
the sequential default strategy of \Erlang.
\begin{figure}[H]
\centering
\includegraphics{generative_fib}
\caption{Generative tree pattern for the Fibonacci
  function\label{fig:fib_shape}}
\end{figure}

\paragraph{Linear structures.}
\label{chap:lists}

Until now, the only data type has been integers. In many situations,
this is not versatile enough. Consider for example the need for
summing a finite series of integers, like \(11 + 5 + 7 + 3\). We must
capture the concept of `a finite series of values' in \Erlang and
make it amenable to computation. This is the very purpose of the
\emph{list} data structure. A list is either empty, in which case it
is denoted in \Erlang by the symbols~\erlcode{[]}, or
non\hyp{}empty. In the latter case, it is equivalent to say that it
contains at least one \emph{element}, also called~\emph{item}. Since a
series is totally ordered, there must be a first item in a
non\hyp{}empty list. In \Erlang, the first item is called
the~\emph{head}---not to be confused with the head of a clause. The
items following the head, if any, wholly constitute a list by
themselves, called the~\emph{tail}. For instance, the list containing
the item~\erlcode{1789} is written
\begin{center}
\erlcode{[1789 | []]}.
\end{center}
Note the vertical bar which joins the head \erlcode{1789} and the tail
\erlcode{[]}: it is an operator, that is, a predefined function
denoted by a special symbol, called \emph{push}, also called
\emph{cons}. The list containing items~\erlcode{1848}
and~\erlcode{1789}, in this order, is specified as
\begin{center}
\erlcode{[1848 | [1789 | []]]}.
\end{center}
The head is now~\erlcode{1848} and the tail is
\erlcode{[1789|[]]}. Variables can stand for any \Erlang value, in
particular, any item or tail. For instance,
\begin{center}
\erlcode{[Head | Tail]}
\end{center}
denotes a non\hyp{}empty list. A list with at least two items would be
written
\begin{center}
\erlcode{[First | [Second | Tail]]},
\end{center}
where~\erlcode{Tail} represents the tail of the tail of the whole
list. Again here, \Erlang follows the syntactic convention of the
\Prolog programming language. If a list, with or without variables, is
used in the head of a clause, it is called a \emph{pattern}. Actually,
parameters and integers are patterns.

\medskip

\paragraph{Summing.}

Consider now how to define a function~\erlcode{sum/1} summing all the
integers in a given list. First, remember that a list is either empty
or not, which leads to lay out the following canvas:
\begin{alltt}
sum(   []) -> \fbcode{N + sum(L)};
sum([N|L]) -> \fbcode{N + sum(L)}.
\end{alltt}
\noindent We have two patterns: \erlcode{[]} (which is also a value)
and~\erlcode{[N|L]}. The empty boxes are place holders for \Erlang
expressions to be determined. Firstly, it is worth noticing that the
order of the clauses is not relevant here, since there cannot be any
overlap between them (in other words, the two cases exclude each
other), so
\begin{alltt}
sum([N|L]) -> \fbcode{N + sum(L)};
sum(   []) -> \fbcode{N + sum(L)}.
\end{alltt}
\noindent makes perfect sense as well. Thereafter, the programmer
should focus separately on each body to be completed.

Let us take under examination the case of the empty list, for
instance, and ponder what should be the outcome of summing no
integers. It is very important to convince oneself that this matter
does not make any sense and, in particular, the temptation of
answering~\erlcode{0} must be vigorously dispelled. Natural numbers
are often taught in primary school by repeatedly displaying
heterogeneous collections of objects and stating that they contain the
same `number' of objects. The pupil is then expected to grasp the
numerical concept by inductive reasoning. Of course, zero is naturally
identified with the empty set because this pedagogy introduces the
number as a cardinal. This conceptualisation of zero hinders the
comprehension of algebra, where zero is abstractly and
\emph{positively} postulated as satisfying the equation \(x + 0 = x\),
for all numbers~\(x\) (and assuming \(x + y = y + x\)). Formal
thinking is essential to learn programming, therefore, we must
convince ourselves that there is no natural relationship between zero
and the empty list. Sometimes they can be associated, sometimes not:
it all depends on the problem at hand.

Returning to the function~\erlcode{sum/1}, once it is understood that
the empty list should not be accepted as an argument, the
corresponding clause is simply removed and the definition boils down
to
\begin{alltt}
sum([N|L]) -> \fbcode{N + sum(L)}.
\end{alltt}
In search for inspiration, turning to small and meaningful examples is
often the way to go. For instance, what if we compute the function
call \erlcode{sum([5|[3|[7|[]]]])}? By comparing the shape of this
call to the head of the only remaining clause, it should be apparent
that if~\erlcode{5} is substituted for the variable \erlcode{N} and
\erlcode{[3|[7|[]]]} is substituted for \erlcode{L}, the head becomes
identical to the function call. This is called a \emph{match} and the
whole process \emph{matching}. It enables the clause to be used for
computing the call. Since what is expected to be calculated by the
body is~\(5 + 3 + 7\) or~\(3 + 5 + 7\) or~\(7 + 5 + 3\) etc., it seems
natural to favour the combinations where the head~\erlcode{5} stands
out, that is, either at the end of the sum or at the beginning: \(5 +
\dots\) or~\(\ldots + 5\). Choosing arbitrarily the first form, the
clause becomes
\begin{alltt}
sum([N|L]) -> \textbf{N +} \fbcode{sum(L)}.
\end{alltt}
\noindent It remains to understand the ellipsis in the formula. It
stands either for~\(3 + 7\) or~\(7 + 3\). Is there a direct way to
relate these additions to the tail \erlcode{L}? In the example, the
tail corresponds to \erlcode{[3|[7|[]]]}. It should strike us as
obvious that what remains to be done is exactly the same computation
as previously but performed on the remaining numbers, that is, on the
tail~\erlcode{L}. This is thus written:
\begin{alltt}
sum([N|L]) -> N + \textbf{sum(L)}.\hfill% \emph{A recursive call}
\end{alltt}
\noindent This kind of clause, where the body contains a function call
to the function being defined, while not being exceptional, is
qualified by a special adjective in informatics: \emph{recursive}. The
corresponding definition, even if only one of its defining clause is
recursive, is said to be recursive as well. Note that it is the
definition that is recursive, not the function. Generally, a statement
is recursive if it is self\hyp{}referential. For instance, the
definition of \erlcode{fact\_tf/2} in module \erlcode{math2} is
recursive. Once a definition is reached, it is wise to check it
against another function call and unroll all the calls until a value
is found:
\begin{alltt}
sum([1|[2|[3|[]]]]) \(\rightarrow\) 1 + sum([2|[3|[]]])
                    \(\rightarrow\) 1 + (2 + sum([3|[]]))
                    \(\rightarrow\) 1 + (2 + (3 + sum([])))
                    \(\nrightarrow\)
\end{alltt}
The broken arrow (\(\nrightarrow\))\label{broken_arrow} means that no
clause is available to continue computing the function call
\erlcode{sum([])}. Indeed, the only clause at hand only accepts, that
is `is matched by,' non\hyp{}empty lists. But the case of the empty
list has been previously discarded as meaningless, so what should be
done? Did we make a mistake?

While the sum of all the integers of the empty list indeed cannot be
computed, the sum on the list containing \emph{exactly one} integer
can be, which means that the example above could simply terminate by
means of an additional clause applied to \erlcode{sum([3|[]])}:
\begin{alltt}
sum([N| L]) -> N + sum(L)\textbf{;}
\textbf{sum([N|[]]) -> N}.\hfill% \emph{New clause for singletons}
\end{alltt}
First, we must understand that the variable \erlcode{N} in the two
clauses denote, in general, different values. In other words, the
previous definition is exactly equivalent to
\begin{alltt}
sum([N| L]) -> N + sum(L);
sum([\textbf{M}|[]]) -> \textbf{M}.\hfill% \emph{Renaming}
\end{alltt}
Second, the attentive reader has already noticed that the given
complete definition still fails to compute the input above. The reason
is that the head of the first clause is matched by all non\hyp{}empty
lists, that is, all lists with at least one item, in particular, the
lists with exactly one item, called \emph{singletons}. But the second
clause, which has just been added, is also matched by singletons, so
the heads of the two clauses overlap partially. Nevertheless, \Erlang
specifies that the first matching head will select the corresponding
body, that is to say, heads are tried in turn until one is matched by
the arguments of the function call to be computed. If none is matched,
an error occurs at run\hyp{}time, as symbolised by the broken arrow
(\(\nrightarrow\)) above. In our example, the first clause is always
matched until the empty list is considered again (as bound
by~\erlcode{L}), which leads to the exact same failure as presented
previously. The fix consists simply in swapping the clauses and write
instead
\begin{alltt}
sum([N|[]]) -> N;\hfill% \emph{Must come first}
sum([N| L]) -> N + sum(L).
\end{alltt}
For the sake of legibility, let us label the arrows:
\begin{alltt}
sum([N|[]]) \(\xrightarrow{\smash[t]{\alpha}}\) N;
sum([N| L]) \(\xrightarrow{\smash[t]{\beta}}\) N + sum(L).
\end{alltt}
and revisit, now successfully, our running example:
\begin{alltt}
sum([1|[2|[3|[]]]]) \(\xrightarrow{\smash[t]{\beta}}\) 1 + sum([2|[3|[[]]]])
                    \(\xrightarrow{\smash[t]{\beta}}\) 1 + (2 + sum([3|[]]))
                    \(\xrightarrow{\smash[t]{\alpha}}\) 1 + (2 + (3))         \(=\) 6.
\end{alltt}
\noindent As a side\hyp{}note, it is nevertheless possible to write
the second clause first if a guard is used:
\begin{alltt}
sum([N| L]) when L =/= [] -> N + sum(L);\hfill% L \(\neq\) []
sum([N|[]])               -> N.
\end{alltt}
but this version, although more explicit, is generally avoided by good
programmers, who prefer concise definitions and always keep in mind
the order \Erlang applies to the clauses.

\medskip

\paragraph{Efficiency.}

What can be said about the speed and the memory usage of the function
\erlcode{sum/1}? The number of rewrites clearly equals the number of
integers in the list because every integer is either matched with one
pattern head or the other. Hence, if the initial function is called on
a list of \(n\) integers, the number of steps to reach the result is
\(n\): \(n-1\) times using clause \clause{\beta} and one time using
clause \clause{\alpha}. Taking a slightly longer list can provide a
hint about memory usage:
\begin{alltt}
sum([1|[2|[3|[4|[]]]]]) \(\xrightarrow{\smash[t]{\beta}}\) 1 + sum([2|[3|[4|[]]]])
                        \(\xrightarrow{\smash[t]{\beta}}\) 1 + (2 + sum([3|[4|[]]]))
                        \(\xrightarrow{\smash[t]{\beta}}\) 1 + (2 + (3 + sum([4|[]])))
                        \(\xrightarrow{\smash[t]{\alpha}}\) 1 + (2 + (3 + (4)))       \(=\) 10\textrm{.}
\end{alltt}
which prompts us to consider only the sizes of the right\hyp{}hand
sides:
\begin{alltt}
sum([1|[2|[3|[4|[]]]]]) \(\xrightarrow{\smash[t]{\beta}}\) \fbcode{1 + sum([2|[3|[4|[]]]])}
                        \(\xrightarrow{\smash[t]{\beta}}\) \fbcode{1 + (2 + sum([3|[4|[]]]))}
                        \(\xrightarrow{\smash[t]{\beta}}\) \fbcode{1 + (2 + (3 + sum([4|[]])))}
                        \(\xrightarrow{\smash[t]{\alpha}}\) \fbcode{1 + (2 + (3 + (4)))}\textrm{.}
\end{alltt}
It seems that the total memory usage increases slowly and then reduces
sharply after the last rewrite step. But this calls for a closer
examination. In particular, the character font used here is
proportional, which implies that, for instance, (\erlcode{+}) takes
more horizontal space on the line than (\erlcode{|}). Omitting blanks
and keeping a monospace font gives
\begin{verbatim}
sum([1|[2|[3|[4|[]]]]]) -> 1+sum([2|[3|[4|[]]]])
                        -> 1+(2+sum([3|[4|[]]]))
                        -> 1+(2+(3+sum([4|[]])))
                        -> 1+(2+(3+(4))).
\end{verbatim}
it seems now that the expressions are of constant size until clause
\(\alpha\) applies. Moreover, even if (\erlcode{+}) were instead
written \erlcode{plus}, its occurrence should not be considered as
taking more memory than (\erlcode{+}) because names are only
tags. Also, what about the parentheses and the blanks? Should they be
considered meaningful, as far as memory allocation is concerned? All
these considerations bring to the fore the need for a finer
understanding of how \Erlang functions and data are usually
represented at run\hyp{}time but, because these encodings depend
strongly on the compiler and the hardware architecture, it would be
inappropriate to rely on too detailed a description. Therefore, it is
sufficient and appropriate here to provide a model of expressions
which allows to draw conclusions about memory usage which hold up to a
proportional constant, while reminding the reader that this model is
unlikely implemented as it is presented next.

\bigskip

\paragraph{Memory.}

Up to now, the only elements composing expressions have been
variables, integers, function calls, arithmetic operators and lists of
expressions. (Incidentally, this is a recursive definition.) The key
is to realise that the essence of an expression is best captured by a
two\hyp{}dimensional representation, as opposed to a line of
punctuated text. This kind of model is widespread in mathematics as
matrices, graphics, graphs etc. The best suited here is a kind of tree
called \emph{abstract syntax tree}. We introduced trees on
page~\pageref{def:tree} in the context of the \emph{call trees} of a
function computing the Fibonacci numbers.

Consider in \fig~\vref{fig:sum} the abstract syntax tree of the
expression \erlcode{1+(2+sum([3|[4|[]]]))} and the function call it
contains in \fig~\vref{fig:sum_call_tree}.
\begin{figure}[!b]
\centering
\subfloat[Expression\label{fig:sum}]{%
  \includegraphics[bb=71 613 134 721]{sum_tree}%
}
\qquad
\subfloat[\ Instances of the control context \erlcode{N
    + \textvisiblespace}\label{fig:sum:context}]{%
  \includegraphics[bb=59 613 113 721]{sum_context}%
}
\quad
\subfloat[Call\label{fig:sum_call_tree}]{%
  \includegraphics[bb=71 654 111 721]{sum_call_tree}%
}
\qquad
\subfloat[Argument\label{fig:sum:arg}]{%
  \includegraphics[bb=59 671 125 721]{sum_arg}%
}
\caption{\erlcode{1+(2+sum([3|[4|[]]]))}}
\end{figure}
The leaves here are the integers~\erlcode{1}, \erlcode{2},
\erlcode{3}~and~\erlcode{4} and the empty list~\erlcode{[]}. The inner
nodes, that is, the ones which are not leaves, are either arithmetic
operators, like~(\erlcode{+}), or function names, like~\erlcode{sum},
or the \emph{list constructor}~(\erlcode{|}). By taking as origin the
node \erlcode{sum}, the abstract syntax tree can be split into the
part below it, that is, the argument (see \fig~\vref{fig:sum:arg}),
and the part above, called \emph{instances of the control context}
(see \fig~\vref{fig:sum:context}).

The main interest of abstract syntax trees is that no parentheses are
required, because a sub\hyp{}expression is denoted by a subtree, that
is, a tree embedded into another. Moreover, blank characters are
absent as well. This allows us to focus on the essential. To
illustrate the gained practicality, consider again the previous
computation in full, from left to right (the node to be rewritten is
boxed), in \fig~\vref{fig:sum_1234}.
\begin{figure}[t]
\centering
\includegraphics[bb=71 606 360 721]{sum_1234}
\caption{Computation of \erlcode{sum([1,2,3,4])}\label{fig:sum_1234}}
\end{figure}
It is now clear that the instances of the control context accumulate
so as to grow in inverse proportion to the argument's length: integers
move one by one from the argument to the control context and the
associated operation changes from a push (\erlcode{|}) to an
addition. Therefore, if we use as memory unit a node, the total memory
is indeed constant---except for the last step.

\label{def:call_stack_heap}
The part of the memory where the instances of the control context
accumulate is called the \emph{call stack}. Since the instances have
always the same shape (an addition), the size of the call stack can be
thought of as proportional to the control context size, the unit of
measure being a node. The original function call and its arguments are
also stored in the call stack, except if the arguments are lists, in
which case the call stack contains a \emph{reference} to a different
part of the memory, called the \emph{heap}, where the lists are
stored. The exact nature of this reference is not relevant
here. Suffice to say that it is an oriented edge ending downwardly at
a node (\erlcode{|}) or \erlcode{[]}. Consider again the running
example and what remains in the call stack, step after step, in
\fig~\vref{fig:sem_1234_stack}.
\begin{figure}[b]
\centering
\includegraphics[bb=71 634 360 721]{sum_1234_stack}
\caption{Call stack while computing \erlcode{sum([1,2,3,4])}
\label{fig:sem_1234_stack}}
\end{figure}
The arrow represents a reference to a list in the heap, as it is the
unique argument of \erlcode{sum/1} (the list is not shown). This
example shows that the call stack grows until it contains the result
or a reference to it if it is located in the heap, while the heap size
will decrease so that the total memory, that is, the size of the call
stack plus the heap size, remains constant. In case the value of a
call is a list, the call stack would contain a reference to the heap,
where the value in question resides.

\smallskip

\paragraph{Tail form.}

On most operating systems, the maximum size of the call stack is often
much smaller than the maximum heap size, therefore, if this is a
concern, it is wise to write programs that do not overload the call
stack. In the case of the function \erlcode{sum/1}, if the input list
contains \(n\) integers, the size of the call stack increases
proportionally to \(n\). Is there a way to write another definition of
the function so that the heap is mainly used instead? The best is to
try to express \erlcode{sum/1} without using any control context at
all: the size of the call stack will then be constant because, in this
configuration, only the arguments will reside in the stack---the
arguments which are lists are stored in the heap and a reference to
them is kept in the call stack instead. This kind of function
definition is said to be in \emph{tail form}. Such a definition was
presented before, \vpageref{code:fact_tf_alpha}, as a variation on the
first factorial function: \verbatiminput{fact_tf.def} The abstract
syntax trees denoting the three bodies are either a value (see
\fig~\vref{fig:fact:A}) or a function call whose arguments contain no
user\hyp{}defined function call (see \fig
\ref{fig:fact:fact_N_1}~and~\vref{fig:fact:fact_N_A}). Arithmetic
expressions \erlcode{N-1} and \erlcode{A*N} call \emph{operators},
that is, built\hyp{}in functions represented by symbols.
\begin{figure}[t]
\centering
\subfloat[\erlcode{fact\_tf(N-1,N)}\label{fig:fact:fact_N_1}]{%
  \includegraphics[bb=55 672 125 721]{fact_N_1} } \qquad
\subfloat[\erlcode{A}\label{fig:fact:A}]{%
  \includegraphics[bb=56 672 93 721]{A}}%
\quad
  \subfloat[\erlcode{fact\_tf(N-1,A*N)}\label{fig:fact:fact_N_A}]{%
    \includegraphics[bb=61 672 147 721]{fact_N_A} }
\caption{Abstract syntax trees from \erlcode{fact\_tf/1} and
  \erlcode{fact\_tf/2}}
\end{figure}

Is there a tail form definition of \erlcode{sum/1}? First, let us call
this new function \erlcode{sum\_tf/1} (\emph{\textbf{t}ail
  \textbf{f}orm}) in order to avoid any confusion. The idea, very much
alike the factorial, is to define an auxiliary function
\erlcode{sum\_tf/2} with a supplementary argument which accumulates
partial results. This kind of argument is called an
\emph{accumulator}. The new version should hence look like as follows:
\begin{alltt}
sum_tf(L)        -> sum_tf(\fbcode{0},L).
sum_tf(A,[N|[]]) -> \fbcode{sum_tf(N+A,L)};
sum_tf(A,[N| L]) -> \fbcode{sum_tf(N+A,L)}.
\end{alltt}
or, equivalently,
\begin{alltt}
sum_tf(L)        -> sum_tf(L,\fbcode{0}).
sum_tf([N|[]],A) -> \fbcode{sum_tf(N+A,L)};
sum_tf([N| L],A) -> \fbcode{sum_tf(L,N+A)}.
\end{alltt}
Notice that, just as with \erlcode{sum/1}, the call
\erlcode{sum\_tf([])} fails without a warning and we may find this
behaviour questionable. Indeed, it can be considered inappropriate in
the framework of software engineering, where programming large and
robust applications is a requisite, but this book focuses primarily on
programming in the small, therefore the programs introduced here are
purposefully fragile; in other words, they may fail on some
undesirable inputs instead of providing the user with some nice
warnings, error messages or, even better, managing to get the compiler
itself to reject such programs.

Since it was decided that an accumulator is needed, we must be clear
on what kind of data it holds. As said previously, an accumulator
usually contains a part of the final result. From a different
perspective, an accumulator can be regarded as a partial trace of all
the previous rewrite steps. Here, since the final result is an
integer, we bear in mind that the accumulator ought to be a number as
well.

There is no need to fill the above canvas (the boxes) from the first
line to the last: this is a program, not an essay. Perhaps the best
method is to first lie down the heads of the clauses and make sure
none is missing and that none is useless (taking into account the
implicit ordering from first to last). Second, we pick up the clause
whose body seems the easiest to guess. For instance, the first clause
of \erlcode{sum/2} seems simple enough because it applies when only
one number, \erlcode{N}, remains in the input list. Since the
accumulator \erlcode{A} holds the partial sum up to now, only
\erlcode{N} remains to be processed. Therefore, the answer is
\erlcode{N+A} or \erlcode{A+N}:
\begin{alltt}
sum_tf(L)        -> sum_tf(L,\fbcode{0}).
sum_tf([N|[]],A) -> \textbf{N+A};
sum_tf([N| L],A) -> \fbcode{sum_Tf(L,N+A)}.
\end{alltt}
The second clause of \erlcode{sum\_tf/2} is chosen next. It applies
when the input list is not empty and its first item is \erlcode{N} and
the remaining are in \erlcode{L}. Until now, the partial sum is the
accumulator \erlcode{A}. It is clear that a recursive call is needed
here, because if the body were \erlcode{N+A} again, then the rest of
the integers, \erlcode{L}, would be useless. So the process must be
resumed with another input:
\begin{alltt}
sum_tf(L)        -> sum_tf(L,\fbcode{0}).
sum_tf([N|[]],A) -> N+A;
sum_tf([N |L],A) -> \textbf{sum_tf(}\fbcode{L}\textbf{,}\fbcode{N+A}\textbf{)}.
\end{alltt}
The question now is to find what the new list and the new accumulator
should be in this last clause. What is known about the list?
\erlcode{N} and \erlcode{L}. What can be done with \erlcode{N}? Well,
the same that was done before, in the first clause of
\erlcode{sum\_tf/2}, that is, let us add it to the accumulator:
\begin{alltt}
sum_tf(L)        -> sum_tf(L,\fbcode{0}).
sum_tf([N|[]],A) -> N+A;
sum_tf([N| L],A) -> sum_tf(\fbcode{L},\textbf{N+A}).
\end{alltt}
This way, the new accumulator is \erlcode{N+A}, which is fine since the
purpose of the accumulator is to hold the partial sum until the
present number, which is \erlcode{N} now. What new list of numbers
should be used? It is clear that \erlcode{N} cannot be reused here,
because it has already been added to the accumulator, and it must not
be added twice. This means that it is not needed anymore. Remains
\erlcode{L}, which is what is sought, since it represents all the
remaining numbers to be added to the accumulator:
\begin{alltt}
sum_tf(L)        -> sum_tf(L,\fbcode{0}).
sum_tf([N|[]],A) -> N+A;
sum_tf([N| L],A) -> sum_tf(\textbf{L},N+A).
\end{alltt}
The last unfinished business is the initial value of the
accumulator. It is important not to rush and to deal with this value
at the last moment. What kind of operation is being carried out on the
accumulator? Additions. Without knowing anything about the integers in
\erlcode{L}, as it is the case in the clauses of \erlcode{sum\_tf/1},
what integer could be taken as an initial value? It is well known
that, for all \(n\), \(0 + n = n + 0 = n\), thus \erlcode{0}~appears
to be the only possible value here, because it does not change the
total sum:
\begin{alltt}
sum_tf(L)        -> sum_tf(L,\textbf{0}).
sum_tf([N|[]],A) -> N+A;
sum_tf([N| L],A) -> sum_tf(L,N+A).
\end{alltt}
The last step consists in trying some examples after the labelling
\begin{alltt}
sum_tf(L)        \(\xrightarrow{\smash[t]{\alpha}}\) sum_tf(L,0).
sum_tf([N|[]],A) \(\xrightarrow{\smash[t]{\beta}}\) N+A;
sum_tf([N| L],A) \(\xrightarrow{\smash[t]{\gamma}}\) sum_tf(L,N+A).
\end{alltt}
Consider our running example again:
\begin{alltt}
sum\_tf([1|[2|[3|[4|[]]]]])
  \(\xrightarrow{\smash[t]{\alpha}}\) sum\_tf([1|[2|[3|[4|[]]]]],0)
  \(\xrightarrow{\smash[t]{\gamma}}\) sum\_tf([2|[3|[4|[]]]],1+0) \(=\) sum\_tf([2|[3|[4|[]]]],1)
  \(\xrightarrow{\smash[t]{\gamma}}\) sum\_tf([3|[4|[]]],2+1)     \(=\) sum\_tf([3|[4|[]]],3)
  \(\xrightarrow{\smash[t]{\gamma}}\) sum\_tf([4|[]],3+3)         \(=\) sum\_tf([4|[]],6)
  \(\xrightarrow{\smash[t]{\beta}}\)  4 + 6                     \(=\) 10\textrm{.}
\end{alltt}
Let us recall here the run
\begin{verbatim}
sum([1|[2|[3|[4|[]]]]]) -> 1+sum([2|[3|[4|[]]]])
                        -> 1+(2+sum([3|[4|[]]]))
                        -> 1+(2+(3+sum([4|[]])))
                        -> 1+(2+(3+(4))).
\end{verbatim}
The difference between \erlcode{sum\_tf/1} and \erlcode{sum/1} lies
not in the result (both functions are indeed equivalent) but in the
way the additions are performed. They are equivalent because
\begin{equation*}
4 + (3 + (2 + (1 + 0))) = 1 + (2 + (3 + 4)).
\end{equation*}
This equality holds because, for all numbers \(x\), \(y\) and
\(z\),\label{proof_sum}
\begin{enumerate}

  \item \label{add_assoc} the addition is associative: \(x + (y + z) =
    (x + y) + z\),

  \item \label{add_comm} the addition is symmetric: \(x + y = y +
    x\),

  \item \label{add_zero} zero is a right\hyp{}neutral number: \(x+0 =
    x\).

\end{enumerate}
To show exactly why, let us write (\(\eqn{\ref{add_assoc}}\)),
(\(\eqn{\ref{add_comm}}\)) and (\(\eqn{\ref{add_zero}}\)) to denote,
respectively, the use of associativity, symmetry and neutrality, and
lay out the following equalities leading from one expression to the
other:
\begin{align*}
4 + (3 + (2 + (1 + 0)))
  &\eqn{\ref{add_zero}}  4 + (3 + (2 + 1))\\
  &\eqn{\ref{add_comm}}  (3 + (2 + 1)) + 4\\
  &\eqn{\ref{add_comm}}  ((2 + 1) + 3) + 4\\
  &\eqn{\ref{add_comm}}  ((1 + 2) + 3) + 4\\
  &\eqn{\ref{add_assoc}} (1 + 2) + (3 + 4)\\
  &\eqn{\ref{add_assoc}} 1 + (2 + (3 + 4)).\quad \text{\textsc{qed}.}
\end{align*}
This seems a bit heavy for such a small program. Is there a way to
rewrite further \erlcode{sum\_tf/1} so that less hypotheses are needed
to prove the equivalence with \erlcode{sum/1}? Let us start with the
most obvious difference: the use of zero. This zero is the initial
value of the accumulator and its sole purpose is to be added to the
first number in the list. We could then simply first load the
accumulator with this number, so the neutrality of zero is no more
required:
\begin{alltt}
sum_tf(\textbf{[N|}L\textbf{]})    -> sum_tf(L,\textbf{N}).
sum_tf([N|[]],A) -> N+A;
sum_tf([N| L],A) -> sum_tf(L,N+A).
\end{alltt}
But this definition of \erlcode{sum\_tf/1} fails on lists containing
exactly one number, because \erlcode{L} can be empty. Therefore, we
must allow the list to be empty in the definition of
\erlcode{sum\_tf/2}:
\begin{alltt}
sum_tf([N|L])   -> sum_tf(L,N).
\textbf{sum_tf(   [],A) -> A;}
sum_tf([N|L],A) -> sum_tf(L,N+A).
\end{alltt}
Now, we can easily get rid of the hypothesis that the addition is
symmetric: by replacing \erlcode{N+A} by
\erlcode{A+N}:\label{code:sum_tf}
\begin{alltt}
sum_tf([N|L])   -> sum_tf(L,N).
sum_tf(   [],A) -> A;
sum_tf([N|L],A) -> sum_tf(L,\textbf{A+N}).
\end{alltt}
Let us relabel the arrows
\begin{alltt}
sum_tf([N|L])   \(\xrightarrow{\smash[t]{\alpha}}\) sum_tf(L,N).
sum_tf(   [],A) \(\xrightarrow{\smash[t]{\beta}}\) A;
sum_tf([N|L],A) \(\xrightarrow{\smash[t]{\gamma}}\) sum_tf(L,A+N).
\end{alltt}
and consider again our running example:
\begin{alltt}
sum\_tf([1|[2|[3|[4|[]]]]])
          \(\xrightarrow{\smash[t]{\alpha}}\) sum\_tf([2|[3|[4|[]]]],1)
          \(\xrightarrow{\smash[t]{\gamma}}\) sum\_tf([3|[4|[]]],1+2) \(=\) sum\_tf([3|[4|[]]],3)
          \(\xrightarrow{\smash[t]{\gamma}}\) sum\_tf([4|[]],3+3)     \(=\) sum\_tf([4|[]],6)
          \(\xrightarrow{\smash[t]{\gamma}}\) sum\_tf([],4+6)         \(=\) sum\_tf([],10)
          \(\xrightarrow{\smash[t]{\beta}}\)  10\textrm{.}
\end{alltt}
This time, the series of additions corresponds to the expression
\(((1+2)+3)+4\), which we can prove equal to \(1+(2+(3+4))\) by means
of the associativity of (\erlcode{+}) only:
\begin{equation*}
((1 + 2) + 3) + 4
  \eqn{\ref{add_assoc}} (1 + 2) + (3 + 4)
  \eqn{\ref{add_assoc}} 1 + (2 + (3 + 4)).\quad \text{\textsc{qed}.}
\end{equation*}
What about the speed and the memory usage of \erlcode{sum\_tf/1}?
While this is not an exact measure, let us assume that the speed is
proportional to the number of rewrites. Then, it is easy to convince
ourselves that each step by means of clauses \clause{\beta} and
\clause{\gamma} process exactly one integer from the input list, so
the total number of rewrite steps is the number of integers plus one
due to the initial rewrite through clause \clause{\alpha}. In other
words, if the initial input list contains \(n\)~integers, the number
of rewrites is exactly~\(n+1\). When \(n\)~is large and taking into
account the assumption we made about this measure of speed, this means
that the execution time is proportional to the size of the input, for
all inputs. In particular, if the input size (taken here as the length
of the list) triples, then the running time also triples.

Consider the abstract syntax trees of the rewritten expressions in
\fig~\vref{fig:sum_tf_1234},
\begin{figure}[t]
\centering
\includegraphics[bb=71 606 358 721]{sum_tf_1234}
\caption{Abstract syntax trees of \erlcode{sum\_tf([1,2,3,4])}
    \(\twoheadrightarrow\) \erlcode{10} \label{fig:sum_tf_1234}}
\end{figure}
where the arrow (\(\twoheadrightarrow\)) means `at least one rewrite.'
Two expressions \(e_1\)~and~\(e_2\) are said to be equal, noted \(e_1
\mathrel{\equiv} e_2\), if \(e_1 \twoheadrightarrow v\) and \(e_2
\twoheadrightarrow v\). For instance, \(\erlcode{sum([1,2,3,4])}
\mathrel{\equiv} \erlcode{sum([5,5])}\). The intermediary trees where
the abstract syntax tree of \erlcode{A+N} appears have been skipped to
show the point, which is that the size of the trees strictly decreases
and the size of the call stack is constant (the boxed nodes all remain
at the first level). There is no control context to the recursive call
of \erlcode{sum\_tf/2}, so no call stack space is needed for it, but
the call stack is not empty because the initial function call to
rewrite is always put in the call stack, including the arguments which
are integers and the references to lists in the heap (but not the
arguments which are lists). The parts of the trees which reside in the
call stack at each step are shown in
\fig~\vref{fig:sum_tf_1234_stack}.
\begin{figure}[h]
\centering
\includegraphics[bb=71 686 360 721]{sum_tf_1234_stack}
\caption{Call stack while computing \erlcode{sum\_tf([1,2,3,4])}
\label{fig:sum_tf_1234_stack}}
\end{figure}
As mentionned earlier, the downward arrows stand for references to
values in the heap, where the first argument, being a list, is
stored. It becomes clear now that the size of the call stack remains
constant and the heap size decreases during the computation, hence so
does the total memory size used by the program. As a conclusion, the
\erlcode{sum\_tf/1} is to be preferred over \erlcode{sum/1} because it
consumes less call stack and heap usage also decreases during the
calculation after the first rewrite.

Does this mean that \erlcode{sum\_tf/1} should always be used when
summing integers?

Let us consider again the previous example. In mathematical notations,
what is aimed at are sums of the type \(S_n = 1 + 2 + \dots + n\). By
writing it from right to left, this is still \(S_n = n + (n-1) + \dots
+ 1\). Adding side by side these two equalities yields
\begin{align*}
S_n + S_n   &= (1 + n) + (2 + (n-1)) + \dots + (n + 1),\\ 
2 \cdot S_n &= \underbrace{(n+1) + \ldots + (n+1)}_{n \,\; \text{times}}
           = (n + 1) \cdot n,
\end{align*}
so \(S_n = n(n+1)/2\). In particular, it comes without a surprise that
\(S_4 = 4 \cdot (4+1)/2=10\). In \Erlang, this is the program
\begin{verbatim}
s(N) when N >= 0 -> N*(N+1)/2.
\end{verbatim}
This function is very efficient, both in terms of speed and memory
usage, since only one step, involving only arithmetic operators, is
needed. The moral of this story is that sometimes a specialised
version is more suitable.

\medskip

\paragraph{Multiplying.}

Consider this time multiplying all the integers in a list. The first
thing that should come to the mind is that this problem is very
similar to the previous one, only the arithmetic operator is
different, so the following definition can be written immediately, by
modification of \erlcode{sum/1}:
\begin{verbatim}
mult([N|[]]) -> N;
mult([N| L]) -> N * mult(L).
\end{verbatim}
Similarly, a definition in tail form can be derived from
\erlcode{sum\_tf/1}:\label{code:mult_tf}
\begin{verbatim}
mult_tf([N|L])   -> mult_tf(L,N).
mult_tf(   [],A) -> A;
mult_tf([N|L],A) -> mult_tf(L,A*N).
\end{verbatim}
This program inherits the constant call stack of \erlcode{sum\_tf/1}.
Moreover, the reason why \erlcode{mult\_tf/1} is equivalent to
\erlcode{mult/1} is the same reason why \erlcode{sum\_tf/1} is
equivalent to \erlcode{sum/1}: the arithmetic operator (\erlcode{*})
is associative, just as (\erlcode{+}) is.

What could be improved that could not be in \erlcode{sum\_tf/1}? In
other words, what can speed up a long series of multiplications? The
occurrence of at least one zero, for example. In that case, it is not
necessary to keep multiplying the remaining numbers, because the
result is going to be zero anyway. This optimisation can be done by
setting apart the case when \erlcode{N} is \erlcode{0}:
\begin{alltt}
mult_tf([N|L])   -> mult_tf(L,N).
mult_tf(   [],A) -> A;
\textbf{mult_tf([0|L],A) -> 0;}\hfill% \emph{Improvement.}
mult_tf([N|L],A) -> mult_tf(L,A*N).
\end{alltt}
How often a zero occurs in the input? In the \emph{worst case}, there
is no zero and thus the added clause is useless. But if it is known
that zero is likely to be in the input with a probability higher than
for other numbers, this added clause could be useful in the long term,
that is, on the average time of different runs of the
program. Actually, even if the numbers are uniformly random over an
interval including zero, it makes sense to keep the clause.

Here have been reviewed different ways of assessing the computation
time: one is to count the exact number of rewrite steps for any input;
another is to find, when the input is large for some measure, an
equivalent function, for example, saying that the running time \(t_n\)
is proportional to the input of size \(n\) can be written \(t_n
\mathrel{\sim} an\), as \(n \rightarrow \infty\), for some positive
constant \(a\) and measure \(n\); another way consists in considering
the running time when the input is large and has the topology that
maximally slows down the execution (the worst case); another one is to
consider the average running time for random inputs.
