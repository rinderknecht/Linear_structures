%%-*-latex-*-

\chapter{Joining and reversing}

Consider writing a function \erlcode{join/2} with two lists as
parameters and which computes a list containing all the items of the
first list followed by all the items of the second one. In other
words, all function calls \erlcode{join(\(P\),\(Q\))} are rewritten
into a list containing all the items of \(P\) followed by all the
items of \(Q\). For instance, the function call
\erlcode{join([3|[5|[]]],[2|[]])} has value
\erlcode{[3|[5|[2|[]]]]}. Since the above requirement is a bit vague,
it is important to try out several extreme cases in order to
understand more precisely what is expected from this function. These
extreme cases are special configurations of the arguments. Both of
them being lists, a hint comes naturally to mind because it is well
known that lists must either be empty or non\hyp{}empty. This simple
observation leads to consider four distinct cases: both lists are
empty; the first is empty and the second is not; the first is not
empty and the second is; both are not empty:
\begin{verbatim}
join(   [],   [])
join(   [],[J|Q])
join([I|P],   [])
join([I|P],[J|Q])
\end{verbatim}
It is important not to rush to write the bodies. Are some cases left
out? No, because there are exactly two arguments which can each be
either empty or not, leading to exactly \(2 \cdot 2 = 4\) cases. Then,
we write the corresponding \Erlang canvas:
\begin{alltt}
join(   [],   []) -> \fbcode{[I|join(P,[J|Q])]};
join(   [],[J|Q]) -> \fbcode{[I|join(P,[J|Q])]};
join([I|P],   []) -> \fbcode{[I|join(P,[J|Q])]};
join([I|P],[J|Q]) -> \fbcode{[I|join(P,[J|Q])]}.
\end{alltt}
Let us ponder which clause looks easier, because there is no reason to
complete them in order if do not want to. When reading the patterns, a
clear mental representation of the situation should arise. Here, it
seems that the first clause is the easiest: what is the list made of
all the items of the (first) empty list followed by all the items of
the (second) empty list? Since the empty list, by definition, contains
no item, the answer is the empty list:
\begin{alltt}
join(   [],   []) -> \textbf{[]};
join(   [],[J|Q]) -> \fbcode{[I|join(P,[J|Q])]};
join([I|P],   []) -> \fbcode{[I|join(P,[J|Q])]};
join([I|P],[J|Q]) -> \fbcode{[I|join(P,[J|Q])]}.
\end{alltt}
Perhaps we might wonder whether this case should be dropped. In other
words, is it a meaningful case? Yes, because the vague English
description of the behaviour of \erlcode{join/2} mandates the result
be always a list made of items of other lists (the arguments), so, in
the absence of any items to ``put'' in the result, the resulting list
``remains'' empty.

It should be evident enough that the second and third clauses should
lead to the same result, because when appending a non\hyp{}empty list
to the right of an empty list is the same as appending it to the left:
the result is always the non\hyp{}empty list in question.
\begin{alltt}
join(   [],   []) -> [];
join(   [],[J|Q]) -> \textbf{[J|Q]};\hfill% \emph{Clause symmetrical...}
join([I|P],   []) -> \textbf{[I|P]};\hfill% \emph{...to this one.}
join([I|P],[J|Q]) -> \fbcode{[I|join(P,[J|Q])]}.
\end{alltt}
The last clause is the trickiest. What does the pattern in the clause
head reveal about the situation? That both lists are not empty, more
precisely, the head of the first list is denoted by the
variable~\erlcode{I}, the corresponding tail (which can either be
empty or not) is~\erlcode{P}, and similarly for the second list and
\erlcode{J}~and~\erlcode{Q}. Are these bricks enough for building the
body, that is, the next step towards the result? We understand that
appending two lists \(P\)~and~\(Q\) preserves in the result the total
order of the items in \(P\)~and~\(Q\) but also the relative order of
the items of~\(P\) with respect to the items of~\(Q\). In other words,
the items of~\(P\) must be present in the result before the items of
\(Q\) and the items from~\(P\) must be in the same order as in~\(P\)
(same for~\(Q\)). With this ordering in mind, it is worth sketching
the following
\begin{alltt}
join([I|P],[J|Q]) -> \fbcode{II} I \fbcode{II} P \fbcode{II} J \fbcode{II} Q \fbcode{II}.
\end{alltt}
In one rewrite step, can the item~\erlcode{I} be placed in its final
position, that is, in a position from where it does not need to be
moved again?  The answer is yes: it must be the head of the
result. 
\begin{alltt}
join([I|P],[J|Q]) -> [I | \fbcode{II} P \fbcode{II} J \fbcode{II} Q \fbcode{II}].
\end{alltt}
What about the other head, variable~\erlcode{J}? It should remain the
head of~\erlcode{Q}:
\begin{alltt}
join([I|P],[J|Q]) -> [I | \fbcode{II} P \fbcode{II} [J | Q]].
\end{alltt}
What is the relationship between \erlcode{P}~and~\erlcode{[J|Q]}? We
can be tempted by
\begin{alltt}
join([I|P],[J|Q]) -> [I | [P | [J | Q]]].\hfill% ?
\end{alltt}
which, despite being correct \Erlang, is flawed. The reason is that,
while \Erlang allows lists to be used as items for other lists (in
other words, lists can be arbitrarily embedded in other lists),
\erlcode{P}~should not be used as an item here, as it is when it is
used as a head for \erlcode{[J|Q]}. Consider the running example
\erlcode{join([3|[5|[]]],[2|[]])}: here, the head of the clause in
question bounds \erlcode{I}~to~\erlcode{3},
\erlcode{P}~to~\erlcode{[5|[]]}, \erlcode{J}~to~\erlcode{2} and
\erlcode{Q}~to~\erlcode{[]}, therefore the corresponding putative body
\erlcode{[I|[P|[J|Q]]]} is actually
\erlcode{[3|[\textbf{[5|[]]}|[2|[]]]]}, which is different from the
expected result \erlcode{[3|[\textbf{5}|[2|[]]]]} in that
\erlcode{[5|[]]} is not~\erlcode{5}.

How can \erlcode{P}~and~\erlcode{[J|Q]} be joined? Well, is this
question not familiar yet? There is a list, here named~\erlcode{P},
which has to be appended to the left of another list, whose shape is
described by the pattern~\erlcode{[J|Q]}. Of course, this is exactly
the purpose of the function \erlcode{join/2} being defined. Therefore,
what we need here is a recursive call:
\begin{alltt}
join(   [],   []) -> [];
join(   [],[J|Q]) -> [J|Q];
join([I|P],   []) -> [I|P];
join([I|P],[J|Q]) -> \textbf{[I|join(P,[J|Q])]}.
\end{alltt}
\noindent Even before proofing this idea against some examples, we
should consider on the spot whether this kind of recursive definition
is likely to terminate for all valid inputs. Indeed, while \Erlang
allows any kind of recursive definition, this does not mean that all
lead to terminating programs. Let us consider
\begin{verbatim}
loop(N) -> loop(N).
\end{verbatim}
which is a valid program but which does not end for any input and uses
a constant amount of memory. (Some other ill\hyp{}conceived programs
run until they exhaust the memory of its corresponding process.) Hence
we must discriminate between terminating and non\hyp{}terminating
recursive calls. There is no general automatic method which can always
answer that question and that is why the \Erlang compiler does not
check the termination property. In practice, however, some heuristics
work quite well, as we shall see. First, the pattern and the call must
be compared for a match, and a measure of the arguments, for instance,
the size in the case of a list, must be decided upon and kept in
mind. In the case of \erlcode{loop/1}, there is only one argument,
which is an integer and the head of the clause is exactly the same as
the body: \erlcode{loop(N)}. It is not surprising that such definition
never ends. But then what about the following?
\begin{verbatim}
loop(N) -> loop(N+1).
\end{verbatim}
In this situation, the clause head is different from the recursive
call. A rather ``natural'' measure on an integer is induced by the
total ordering on the mathematical integers, that is, we rely upon
\erlcode{N~<~N~+~1}. This implies that the recursive call is going to
be performed on increasingly larger numbers without an end. Let us
consider a slight variation:
\begin{verbatim}
loop(N) -> loop(N-1).
\end{verbatim}
Here \erlcode{N - 1 < N}, but that does not seem to make any
difference as far as termination is concerned. But a clause can be
added to \emph{filter}, that is, catch, the terminating value of the
argument. For instance
\begin{verbatim}
loop(0) -> 7;
loop(N) -> loop(N-1).
\end{verbatim}
This is a program which always terminate with the value \erlcode{7},
for example,
\[
\erlcode{loop(3)} \rightarrow \erlcode{loop(2)} \rightarrow
\erlcode{loop(1)} \rightarrow \erlcode{loop(0)} \rightarrow
\erlcode{7}.
\]
Two conditions are sufficient to have a terminating recursive
definition: the recursive calls must have smaller arguments than their
heads and some clauses must provide pattern heads for the smallest
arguments. But we may object that the variant definition
\begin{verbatim}
loop(0) -> 7;
loop(N) -> loop(N+1).
\end{verbatim}
also terminates on \erlcode{7} for all \erlcode{N < 0}. This remark
reveals by the way that
\begin{verbatim}
loop(0) -> 7;
loop(N) -> loop(N-1).
\end{verbatim}
fails to terminate if \erlcode{N~<~0}, so it should have been written
instead
\begin{verbatim}
loop(0)            -> 7;
loop(N) when N > 0 -> loop(N-1).
\end{verbatim}
Returning to the other loop, the following terminates for all
\erlcode{N~<~1}:
\begin{verbatim}
loop(0)            -> 7;
loop(N) when N < 0 -> loop(N+1).
\end{verbatim}
These observations allow us to formulate more precisely the previous
criteria: a recursive definition always terminates if the size of the
arguments of the recursive calls change so that the series of calls
gets strictly closer, that is, converges, to some cases which
terminate in one step, called \emph{base cases}. In the above
definitions, the base case is \erlcode{loop(0)} and the recursive
calls bring one step closer to it.

If we apply this criterion to \erlcode{join/2}, we can see that the
first argument ``becomes'' a shorter list in one rewrite:
\erlcode{P}~is shorter by one item in comparison with \erlcode{[I|P]},
whilst the second argument does not change. Since the direction of
change is towards a smaller first argument, base cases are needed to
filter the minimum value of the first argument. The initial reasoning
about the requirements revealed that the first argument can be the
empty list and two complementary base cases are already written:
\erlcode{join([],[])} and \erlcode{join([],[J|Q])}. Therefore, the
function \erlcode{join/2} terminates on all input lists.

By the way, an assumption was made earlier about the loop
\begin{verbatim}
loop(N) -> loop(N).
\end{verbatim}
which was too strong. It was indeed implicitly taken for granted that
\erlcode{N}~was an integer, but this statement does not arise from the
examination of the code. Actually, \erlcode{N}~could perfectly well be
a list, since nothing is done with it. The moral of this story is that
we should not trust naming conventions when inferring the type of a
variable, especially when inspecting the code of another
programmer. It is not because a variable is spelled~\erlcode{N} that
it is an integer, it is its uses in the program that make it an
integer or not. Of course, this remark does not entail that no naming
convention should be followed. In
\begin{verbatim}
loop(0)            -> 7;
loop(N) when N < 0 -> loop(N+1).
\end{verbatim}
there are two constraints which entail that \erlcode{N}~must be an
integer: the guard \erlcode{N~<~0} and the arithmetic operation
\erlcode{N+1}. Continuing on this track, the definition of
\erlcode{join/2} could be examined in terms of the nature, or types,
of the items in the lists. Let us consider again the definition
\verbatiminput{join_4.def} Do \erlcode{I}~and~\erlcode{J} have to be
integers? By looking at the bodies, it appears that none of these
variables are involved in any kind of computation or guard that would
force them to denote numbers; they are passed around until they reach
their respective position in the final result. Spurred on by
curiosity, we should try a test with embedded lists, so let us label
the arrows of the previous definition:
\begin{alltt}
join(   [],   []) \(\smashedrightarrow{\alpha}\) [];
join(   [],[J|Q]) \(\smashedrightarrow{\beta}\) [J|Q];
join([I|P],   []) \(\smashedrightarrow{\gamma}\) [I|P];
join([I|P],[J|Q]) \(\smashedrightarrow{\delta}\) [I|join(P,[J|Q])].
\end{alltt}
and try for instance
\begin{alltt}
\erlcode{join([3|[[4|[]]|[]]],[[]|[]])}
                           \(\smashedrightarrow{\delta}\) \erlcode{[3|join([[4|[]]|[]],[[]|[]])]}
                           \(\smashedrightarrow{\delta}\) \erlcode{[3|[[4|[]]|join([],[[]|[]])]]}
                           \(\smashedrightarrow{\beta}\) \erlcode{[3|[[4|[]]|[[]|[]]]]}\textrm{.}
\end{alltt}
It may be unclear what the arguments in this example are. The first
list contains two items: the first one is the number~\erlcode{3} and
the last one is the \emph{list} containing the single
number~\erlcode{4}. The second argument is the list containing a
single item, the empty list. Fortunately, \Erlang proposes an
alternative syntax which is more legible and the previous example is
preferably written
\begin{alltt}
\erlcode{join([3,[4]],[[]])} \(\smashedrightarrow{\delta}\) \erlcode{[3|join([[4]],[[]])]}
                   \(\smashedrightarrow{\delta}\) \erlcode{[3,[4]|join([],[[]])]}
                   \(\smashedrightarrow{\beta}\) \erlcode{[3,[4],[]]}\textrm{.}
\end{alltt}
There are two simplifying conventions:
\begin{itemize}

  \item for all expression \(e\), we have \(\erlcode{[\(e\)|[]]}
    \mathrel{\equiv} \erlcode{[\(e\)]}\);

  \item for all expressions \(e_0\) and \(e_1\), and all lists \(L\),
    we have \[\erlcode{[\(e_0\)|[\(e_1\)|\(L\)]]} \mathrel{\equiv}
    \erlcode{[\(e_0\),\(e_1\)|\(L\)]}.\]

\end{itemize}
In particular, a list whose items are all known can be specified by
enumerating all these items separated by commas, for example, the list
\erlcode{[3|[1|[0|[]]]]}~is the same as~\erlcode{[3,1,0]} and
\erlcode{[5|[]]}~is the same as~\erlcode{[5]}. About the second
simplification, note that \erlcode{[3|[1|L]]}~is the same as
\erlcode{[3,1|L]}, but \erlcode{[3|[1|L]]}~\emph{is not}
\erlcode{[3,1,L]}, because \erlcode{L}~is not an item of the whole
list: it is the rest of the whole list following the two first items.

As a final note, it is easy to convince oneself that the function
\erlcode{join/2} is \emph{associative}, that is, for all lists
\(L_1\), \(L_2\) and \(L_3\), the following holds:
\[
  \erlcode{join(\(L_1\),join(\(L_2\),\(L_3\)))} \mathrel{\equiv}
  \erlcode{join(join(\(L_1\),\(L_2\)),\(L_3\))}.
\]

\medskip

\paragraph{Correctness and completeness.}
\label{correctness}

It is extremely common that, in the course of designing a function, we
focus on some aspect, some clause, and then on some other part of the
code, and we may miss the forest for the trees. When we settle for a
function definition, the next step is to check whether it is correct
and complete with respect to the conception we had of its expected
behaviour.

We say that a definition is \emph{correct} if all function calls we
can make with it are rewritten by the clauses into the expected result
\emph{and} every call failing was expected to fail. By failure, we
mean that a \emph{stuck expression} is reached, that is, an expression
which can not be further rewritten into a value. We came across such an
expression \vpageref{broken_arrow} and represented this situation with
a broken arrow \((\nrightarrow)\).

We say that a definition is \emph{complete} if all function calls that
we expect to be computable are indeed computable. In other words, we
must also check that the definition enables rewriting into values all
the inputs we deem acceptable.

How do we check that the last definition of \erlcode{join/2} is
correct and complete? If the concept of ``expected result'' is not
formally defined, typically by means of mathematics, we resort to
\emph{code review} and \emph{testing}. One important aspect of the
reviewing process consists in verifying again the heads of the
definition and see if all possible inputs are accepted or not. In case
some inputs are not matched by the heads, we must justify that fact
and record the reason in a comment. The heads of \erlcode{join/2}
match all the combinations of two lists, whether they are empty or
not, and this is exactly what was expected: no more, no less. The next
step is to inspect the bodies and wonder twofold: (1)~Are the bodies
rewritten into the expected type of value, for all function calls?
(2)~Are the function calls being provided the expected type of
arguments?  These checks stem from the fact that \Erlang does not
include \emph{type inference} at compile\hyp{}time. Other functional
languages, like \OCaml and \Haskell, would have their compilers
automatically establish these properties. The examination of the
bodies in the definition of \erlcode{join/2} confirms that
\begin{itemize}

  \item the bodies of the clauses \clause{\alpha}, \clause{\beta} and
    \clause{\gamma} are lists containing the same kind of items as the
    arguments;

  \item the arguments of the unique recursive call in clause
    \clause{\delta} are lists made of items from the parameters;

  \item assuming that the recursive call has the expected type, we
    deduce that the body of the last clause is a list made of items
    from the arguments.

\end{itemize}
As a conclusion, the two questions above have been positively
answered. Notice how we had to assume that the recursive call already
had the type we were trying to establish for the current
definition. There is nothing wrong with this reasoning, called
\emph{inductive}, and it is rife in mathematics. We shall revisit it in
different contexts.

The following stage consists in testing the definition. This means to
define a set of inputs which lead to a set of outputs and failures
that are all expected. For example, it is expected that
\erlcode{join([],[])} be rewritten into~\erlcode{[]}, so we could
assess the validity of this statement by running the code. And the
function call indeed passes the test. How should we choose the inputs
meaningfully? There are no general rules, but some guidelines are
useful. One is to consider the empty case or the smaller case,
whatever that means in the context of the function. For example, if
some argument is a list, then let us try the empty list. If some
argument is a nonnegative integer, then let us try zero. Another
advice is to have at least \emph{test cases}, that is, some function
calls whose values are known, which exert each clause. In the case of
\erlcode{join/2}, there are four clauses to be covered by the test
cases.

\medskip

\paragraph{Improving joins.}

When we are convinced that the function we just defined is correct and
complete, it is often worth considering again the code for
improvement. There are several directions in which improvements, often
called \emph{optimisations} although the result is not optimal, can be
achieved:
\begin{itemize}

  \item Can we rewrite the definition so that in all or some cases it
    is faster? 

  \item Is there an equivalent definition which uses less memory in
    all or some cases?

  \item Can we shorten the definition by using less clauses (perhaps
    some are useless or redundant) or shorter bodies?

  \item Can we use less parameters in the definition? (This is related
    to memory usage.)

  \item Do we need a tail form version of the same function?
    (Again, this is related to memory, more precisely, the call
    stack.)

\end{itemize}
Let us reconsider \erlcode{join/2}:
\verbatiminput{join_4.def}
and focus our attention on the two first clauses, whose common point
is to have the first list being empty. It is clear now that the
bodies, which are values, are, in both clauses, the second list,
whether it is empty (first clause) or not (second clause). Therefore,
there is no need to discriminate on the structure of the second list
when the first one is empty and we can equivalently write
\begin{alltt}
\textbf{join(   [],    Q) -> Q;}\hfill% Q \emph{is a list}
join([I|P],   []) -> [I|P];
join([I|P],[J|Q]) -> [I|join(P,[J|Q])].
\end{alltt}
Note how the new definition does not ascertain that \erlcode{Q} is a
list---hence the comment---so it is not strictly equivalent to the
original definition: now \erlcode{join([],5)}
\(\rightarrow\)~\erlcode{5}. Let us compromise by favouring the
conciseness of the latter definition.

Let us consider next the two last clauses and look for common
patterns. It turns out that, in the penultimate clause, the first list
is matched as \erlcode{[I|P]} but nothing is done with \erlcode{I} and
\erlcode{P} except \emph{rebuilding} \erlcode{[I|P]} in the body. This
suggests that we could simplify the clause as follows:
\begin{alltt}
join(   [],    Q) -> Q;
join(    \textbf{P},   []) -> \textbf{P};\hfill% P \emph{is a non-empty list}
join([I|P],[J|Q]) -> [I|join(P,[J|Q])].
\end{alltt}
It is important to check that changing \erlcode{[I|P]}
into~\erlcode{P} does not affect the matching, that is to say, exactly
the same inputs which used to match the head are still matching
it. Indeed, it is possible in theory that the new~\erlcode{P} matches
an empty list. Can we prove that \erlcode{P}~is always non\hyp{}empty?
The head of the penultimate clause matches only if the previous clause
did not match, since \Erlang tries the heads in the order of the
writing, that is, top\hyp{}down. Therefore, we know that
\erlcode{P}~can not be the empty list, because \erlcode{[]}~is used in
the previous clause \emph{and} the second parameter can be any
list. Nevertheless, as happened before, \erlcode{P}~is not necessarily
a list anymore, for instance, \erlcode{join(5,[])}
\(\rightarrow\)~\erlcode{5}. Again, we will ignore this
side\hyp{}effect and choose the conciseness of the latter definition.

In the last clause, we observe that the second argument, matched as
\erlcode{[J|Q]}, is simply passed over to the recursive call, thus it
is useless to distinguish \erlcode{J}~and~\erlcode{Q} and we can try
\begin{alltt}
join(   [], Q) -> Q;
join(    P,[]) -> P;
join([I|P], \textbf{Q}) -> [I|join(P,\textbf{Q})].\hfill% \emph{Can} Q \emph{be} []\emph{?}
\end{alltt}
Again, we must make sure that \erlcode{Q}~can not match an empty list:
it can not be empty because, otherwise, the previous head would have
matched the call. As it is, the penultimate head is included in the
last, that is, all input matching the last could match the
penultimate, if we let aside the order used by \Erlang, which leads us
to consider whether the function would still be correct if
\erlcode{Q}~could be empty after all. If so, then the penultimate
clause would be useless. Let us label the clauses as follows:
\begin{alltt}
join(   [], Q) \(\smashedrightarrow{\alpha}\) Q;
join(    P,[]) \(\smashedrightarrow{\beta}\) P;
join([I|P], Q) \(\smashedrightarrow{\gamma}\) [I|join(P,Q)].
\end{alltt}
Let \(L\)~be some list containing \(n\)~items, which we write
informally as \(L = [I_0, I_1, \dots, I_{n-1}]\). The subscript~\(i\)
in~\(I_i\) is the \emph{position} of the item in the list, the head of
the list being at position~\(0\). Then \Erlang would rewrite in one
step
\begin{alltt}
join(\(L\),[]) \(\smashedrightarrow{\beta}\) \(L\).
\end{alltt}
Had clause~\clause{\beta} been erased, we would have had instead the
series
\begin{tabbing}
\erlcode{join(\(L\),[])} \= \(\smashedrightarrow{\gamma}\) \= \erlcode{[\(I\sb{0}\)|join([\(I\sb{1}\),\(\ldots\),\(I\sb{n-1}\)],[])]}\\
           \> \(\smashedrightarrow{\gamma}\) \> \erlcode{[\(I\sb{0}\)|[\(I\sb{1}\)|join([\(I\sb{2}\),\(\ldots\),\(I\sb{n-1}\)],[])]]}\\
           \> \(=\) \> \erlcode{[\(I\sb{0}\),\(I\sb{1}\)|join([\(I\sb{2}\),\(\ldots\),\(I\sb{n-1}\)],[])]}\\
           \> \(\smashedrightarrow{\gamma}\) \> \erlcode{[\(I\sb{0}\),\(I\sb{1}\)|[\(I\sb{2}\)|join([\(I\sb{3}\),\(\ldots\),\(I\sb{n-1}\)],[])]]}\\
           \> \(=\) \> \erlcode{[\(I\sb{0}\),\(I\sb{1}\),\(I\sb{2}\)|join([\(I\sb{3}\),\(\ldots\),\(I\sb{n-1}\)],[])]}\\
           \> \ \(\vdots\)\\
           \> \(\smashedrightarrow{\gamma}\) \> \erlcode{[\(I\sb{0}\),\(I\sb{1}\),\(\ldots\),\(I\sb{n-1}\)|join([],[])]}\\
           \> \(\smashedrightarrow{\alpha}\) \> \erlcode{[\(I\sb{0}\),\(I\sb{1}\),\(\ldots\),\(I\sb{n-1}\)|[]]}\\
           \> \(=\) \>
           \erlcode{[\(I\sb{0}\),\(I\sb{1}\),\(\ldots\),\(I\sb{n-1}\)]}\\
           \> \(:=\) \> \(L\)\textrm{.}
\end{tabbing}
(The notation \((:=)\) means ``equal by definition,''
sometimes called an \emph{assignment}. It must be distinghished from
\((=)\), which is an equality \emph{deduced} from some other property.)
In short, we found
\[
\erlcode{join(\(L\),[])} \twoheadrightarrow L.
\]
This means that clause~\clause{\beta} is useless, since its removal
allows us to reach the same result~\(L\), although more slowly:
\(n\)~steps by clause~\clause{\gamma} plus~\(1\) by
clause~\clause{\alpha}, instead of one step by
clause~\clause{\beta}. We are hence in a situation where we discover
that the original definition was already specialised for speed when
the second list is empty. If we remove clause~\clause{\beta}, the
program is shorter but becomes slower in that special case. This kind
of dilemma is quite common in programming and there is sometimes no
clear\hyp{}cut answer as to what is the best design. Perhaps another
argument can here tip the scale slightly in favour of the
removal. Indeed, whilst the removal slows down some calls, it makes
the number of rewrites easy to remember: it is the number of items of
the first list plus~\(1\); in particular, the length of the second
argument is irrelevant. So let us settle for\label{code:join}

\verbatiminput{join_2.def}

\noindent Let us note that \erlcode{join(5,[])} fails again, as it
does in the original version.

When programming medium or large applications, it is recommended to
use evocative variables, like~\erlcode{ListOfProc}, instead of
enigmatic ones, like~\erlcode{L}. But in this presentation we are
mostly concerned with short programs, not software engineering, so
short variables will do. Nevertheless, we need to opt for a naming
convention so we can easily recognise the type of the variables across
function definitions. Let us decide that \erlcode{L}, \erlcode{P},
\erlcode{Q}~and~\erlcode{R} are lists, while
\erlcode{I}~and~\erlcode{J} denote elements, also called items, of
those lists (which can be lists themselves).

\medskip

\paragraph{Tail form.}

As the rewrite of \erlcode{join(\(L\),[])} shows, the definition of
\erlcode{join/2} is not in tail form. Indeed, a function call is
embedded in the second clause and it has the control context
\erlcode{[I|\textvisiblespace]}, the mark \erlcode{\textvisiblespace}
standing for the location of the call. We may wonder whether a tail
form variant is necessary or not. This decision lies entirely on
practical grounds, typically, the maximum size of the call stack in
comparison to the call stack usage as a function of the expected size
of the input. In the case under consideration, the call stack usage
can be considered as proportional to the size of the first list, as
the list grows. More precisely, we ended up with a definition of
\erlcode{join/2} for which the number of computational steps
is~\(n+1\) if the length of the first argument is~\(n\). Therefore, if
we expect some calls to \erlcode{join/2} to have a first argument
longer than the maximum size of the call stack allocated by the
process running the program, it is wise to write a tail form version
of \erlcode{join/2}, otherwise a crash is likely. But it is never
necessary in theory, hence experience is paramount here. For the sake
of the argument, let us imagine that we need such a tail form
variant. Perhaps it is also useful here to state unambiguously that it
is always possible to rewrite a function definition into an equivalent
tail form definition.

For the moment, instead of presenting an automatic transformation, let
us envisage an empirical approach. We previously considered the
transformations of \erlcode{sum/1} into \erlcode{sum\_tf/1},
\vpageref{code:sum_tf}, and of \erlcode{mult/1} into
\erlcode{mult\_tf/1}, \vpageref{code:mult_tf}. The idea consisted in
adding an \emph{accumulator} (or \emph{accumulative parameter}) which
would hold a partial result. This partial result was computed by
applying the control context to the previous value of the accumulator,
thus, for instance, the clause
\begin{alltt}
sum([N|L]) -> \textbf{N +} sum(L).
\end{alltt}
became
\begin{alltt}
sum_tf([N|L],A) -> sum_tf(L,A\textbf{+N}).
\end{alltt}
But we noted that this works only because the addition operation is
associative, that is, \(x + (y + z) = (x + y) + z\). In the case of
\erlcode{join/2}, as stated above, the control context is
\erlcode{[I|\textvisiblespace]} and the operator~(\erlcode{|}) is not
associative: \(\erlcode{[X|[Y|Z]]} \not\equiv
\erlcode{[[X|Y]|Z]}\). Let us try nevertheless and see exactly what
happens. We want a new function \erlcode{join/3} whose shape
is\label{trans_join}
\begin{alltt}
join(   [],Q,\textbf{A}) -> \fbcode{join([],[I|Q],A)};
join([I|P],Q,\textbf{A}) -> \fbcode{join([],[I|Q],A)}.
\end{alltt}
The new parameter~\erlcode{A} is the accumulator in question. Since we
want to push~\erlcode{I} on it, it must be a list. Furthermore, its
initial value should be the empty list, otherwise extraneous items
would be found in the result. Therefore, the definition in tail form,
equivalent to \erlcode{join/2} and named \erlcode{join\_tf/2}, calls
\erlcode{join/3} with the extra argument~\erlcode{[]}:
\begin{verbatim}
join_tf(P,Q) -> join(P,Q,[]).
\end{verbatim}
Let us go back to \erlcode{join/3} and push \erlcode{I} on
\erlcode{A}:
\begin{alltt}
join(   [],Q,A) \(\smashedrightarrow{\alpha}\) \fbcode{join([],[I|Q],A)};
join([I|P],Q,A) \(\smashedrightarrow{\beta}\) \textbf{join(P,Q,[I|A])}.
\end{alltt}
What is the accumulator with respect to the expected result? We
already know that it can not be a partial result, because
(\erlcode{|})~is not associative. So some more work has to be done
with \erlcode{A}~\emph{and}~\erlcode{Q}, but, first, we should
understand what \erlcode{A}~contains at this point and unfolding a
call, with a piece of paper and a pencil, is quite enlightening. Let
\(L\) be a list of \(n\)~items \([I_0,I_1,\dots,I_{n-1}]\). We have
the following:
\begin{tabbing}
\erlcode{join(\(L\),\(Q\),[])} \= \(\smashedrightarrow{\beta}\) \= \erlcode{join([\(I\sb{1}\),\(\ldots\),\(I\sb{n-1}\)],\(Q\),[\(I\sb{0}\)])}\\
              \> \(\smashedrightarrow{\beta}\) \> \erlcode{join([\(I\sb{2}\),\(\ldots\),\(I\sb{n-1}\)],\(Q\),[\(I\sb{1}\),\(I\sb{0}\)])}\\
              \> \ \(\vdots\)\\
              \> \(\smashedrightarrow{\beta}\) \> \erlcode{join([],\(Q\),[\(I\sb{n-1}\),\(I\sb{n-2}\),\(\ldots\),\(I\sb{0}\)])}\\
              \> \(\smashedrightarrow{\alpha}\) \> \fbcode{join([],[I|Q],A)}\textrm{.}
\end{tabbing}
Therefore \erlcode{A} in the head of clause~\clause{\alpha} is bound
to a list which contains the same items as the original first
argument~\(L\), but in \emph{reverse order}. In other words, given the
call \erlcode{join(P,Q,[])}, the parameter~\erlcode{A} in the first
head of \erlcode{join/3} holds~\erlcode{P} reversed. What can we do
with \erlcode{A}~and~\erlcode{Q} in order to reach the result?  The
key is to realise that the answer depends on the contents
of~\erlcode{A}, which, therefore, needs to be matched more accurately:
is~\erlcode{A} empty or not? This leads to split
clause~\clause{\alpha} into \(\alpha_0\) and \(\alpha_1\):
\newlength\Split\settowidth\Split{\(_{\alpha_0}\)}
\begin{alltt}
join(   [],Q,   \textbf{[]}) \(\MyArrow{\Split}{\alpha\sb{0}}\) \fbcode{join([],[I|Q],A)};
join(   [],Q,\textbf{[I|A]}) \(\MyArrow{\Split}{\alpha\sb{1}}\) \fbcode{join([],[I|Q],A)};
join([I|P],Q,    A) \(\MyArrow{\Split}{\beta}\) join(P,Q,[I|A]).
\end{alltt}
Notice that clauses \clause{\alpha_0}~and~\clause{\alpha_1} could be
swapped, as they filter completely distinct cases. The body of
clause~\clause{\alpha_0} is easy to guess: it must be~\erlcode{Q},
since it corresponds to the case when we want to append~\erlcode{Q} to
the empty list:
\begin{alltt}
join(   [],Q,   []) \(\MyArrow{\Split}{\alpha\sb{0}}\) \textbf{Q};
join(   [],Q,[I|A]) \(\MyArrow{\Split}{\alpha\sb{1}}\) \fbcode{join([],[I|Q],A)};
join([I|P],Q,    A) \(\MyArrow{\Split}{\beta}\) join(P,Q,[I|A]).
\end{alltt}
How do we relate \erlcode{Q}, \erlcode{I}~and~\erlcode{A} in clause
\clause{\alpha_1} with the result we are looking for? Given the call
\erlcode{join(P,Q,[])}, we know that \erlcode{[I|A]}~is~\erlcode{P}
reversed, so item \erlcode{I}~is last in~\erlcode{P} and it should be
on top of~\erlcode{Q} in the result. Now, we found above that
\erlcode{[I|Q]} is the end of the result. But what should we do with
\erlcode{A}? The key is to realise that we need to start the same
process again, that is, we need another recursive call:
\begin{alltt}
join(   [],Q,   []) \(\MyArrow{\Split}{\alpha\sb{0}}\) Q;
join(   [],Q,[I|A]) \(\MyArrow{\Split}{\alpha\sb{1}}\) \textbf{join([],[I|Q],A)};
join([I|P],Q,    A) \(\MyArrow{\Split}{\beta}\) join(P,Q,[I|A])\textrm{.}
\end{alltt}
To test the correctness of this definition, we can try a small
example:
\begin{alltt}
join([1,2,3],[4,5],[]) \(\MyArrow{\Split}{\beta}\) join([2,3],      [4,5],    [1])
                       \(\MyArrow{\Split}{\beta}\) join(  [3],      [4,5],  [2,1])
                       \(\MyArrow{\Split}{\beta}\) join(   [],      [4,5],[3,2,1])
                       \(\MyArrow{\Split}{\alpha\sb{1}}\) join(   [],    [3,4,5],  [2,1])
                       \(\MyArrow{\Split}{\alpha\sb{1}}\) join(   [],  [2,3,4,5],    [1])
                       \(\MyArrow{\Split}{\alpha\sb{1}}\) join(   [],[1,2,3,4,5],     [])
                       \(\MyArrow{\Split}{\alpha\sb{0}}\) [1,2,3,4,5]\textrm{.}
\end{alltt}
As a conclusion, the tail form version of \erlcode{join/2}, called
\erlcode{join\_tf/2}, requires an auxiliary function \erlcode{join/3}
with an accumulator whose purpose is to reverse the first
argument:\label{code:join_tf} \begin{alltt}
join_tf(P,Q)        \(\smashedrightarrow{\alpha}\) join(P,Q,[]).
join(   [],Q,   []) \(\smashedrightarrow{\beta}\) Q;
join(   [],Q,[I|A]) \(\smashedrightarrow{\gamma}\) join([],[I|Q],A);
join([I|P],Q,    A) \(\smashedrightarrow{\delta}\) join(P,Q,[I|A]).
\end{alltt}
 We also know
what to do when the control context is not made of a call to some
associative operator: push the variables it contains and when the
input list is empty, pop them and apply the context.

\medskip

\paragraph{Efficiency.}

The number of steps to rewrite \erlcode{join\_tf(\(P\),\(Q\))} into a
value is greater than with \erlcode{join(\(P\),\(Q\))}, as we guessed
while writing the previous example. Indeed, assuming that
\erlcode{P}~contains \(n\)~items, we have
\begin{itemize} 

  \item one step to obtain \erlcode{join(\(P\),\(Q\),[])}, by clause
    \clause{\alpha};

  \item \(n\) steps to reverse \(P\) in the accumulator, by clause
    \clause{\delta};

  \item \(n\) steps to reverse the accumulator on top of \(Q\), by
    clause \clause{\gamma};

  \item one step when the accumulator is finally empty, by clause
    \clause{\beta}.

\end{itemize}
Thus, the total number of steps is \(2n+2\), which is twice the delay
of the previous version. When we wrote two different versions of the
factorial function, \erlcode{fact\_tf/1} in tail form and
\erlcode{fact/1}, the number of rewrites was, respectively,
\(n+1\)~versus~\(n\). So why is the difference much greater between
\erlcode{join/2} and \erlcode{join\_tf/2}?  The explanation is found
in the heterogeneous natures of the accumulators. In the case of the
factorial, the operation applied to the accumulator is a
multiplication and the accumulator is, at all times, a partial
result. In the case of appending a list to another, the operation
applied to the accumulator consists in pushing an item onto a list and
has to be undone later: the accumulator is not a partial result but a
\emph{temporary list} used to hold the items of the first list in
reverse order. We shall find many occurrences of this
situation. Meanwhile, it is important to remember that a tail form
variant of a function operating on lists may lead to a slower program,
even though the call stack remains of constant size. Also, the derived
definition in tail form may be longer, as illustrated by
\erlcode{join\_tf/2}: four clauses instead of two.

It may be argued that any instance of a control context has to be
stored in the call stack and later removed to be combined with the
result of said call, therefore, since we only measure the number of
user\hyp{}defined rewrites, these removals from the call stack are not
accounted for, despite incurring a delay. This is a valid point, but
this delay is arguably much shorter than the allocation time because,
for reasons we will expand in a later chapter, deallocation of data in
the heap takes significantly more time than in the call stack.

As a final note on efficiency, we can summarise the \emph{execution
  trace} above as a product (composition) of clauses: \(\alpha \cdot
\delta^n \cdot \gamma^n \cdot \beta\), or, simply,
\(\alpha\delta^n\gamma^n\beta\). The number of clauses
\(\comp{sum\_tf}{n}\) of \erlcode{sum\_tf(\(L\))}, when the list~\(L\)
contains \(n\)~integers, is then the length of the trace, given that
the length of a clause~\(c\), noted~\(\abs{c}\), is~\(1\):
\begin{align*}
\comp{sum\_tf}{n} 
  &:= \abs{\alpha\delta^n\gamma^n\beta}
   = \abs{\alpha} + \abs{\delta^n} + \abs{\gamma^n} + \abs{\beta}
   = 1 + \abs{\delta} \cdot n + \abs{\gamma} \cdot n + 1\\
   &= 2n + 2.
\end{align*}

\medskip

\paragraph{Polymorphism.}
\label{polymorphism}

The function \erlcode{join/2} presents a characteristic which makes it
different from \erlcode{fact/1}, \erlcode{sum/1} and \erlcode{mult/1}
because the nature, usually called \emph{type}, of the items in the
lists to be joined is irrelevant. For example, we showed previously
that \erlcode{join([3,[]],[[]])} can be computed, in spite the lists
to be joined containing integers and/or lists. In the case of the
other mentioned functions, the argument must either be an integer or a
list containing integers, otherwise the computation fails because the
involved arithmetic operators can not be applied to something else
other than an integer. When the evaluation of a function call never
depends on the type of (some part of) its arguments, the function is
said to be \emph{polymorphic}. This does not mean that any type would
fit a polymorphic function: \erlcode{join/2} does require its first
argument to be a list, what does not matter is the type of the items
it contains.

\medskip

\paragraph{Function definitions as metadata.}

The previous discussions on obtaining equivalent definitions which are
in tail form suppose to consider programs as data. At this point, it
is a methodological standpoint only and we do not mean that functions
can be processed as lists (we shall come back on this later), but,
more informally, we mean that definitions can be transformed into
other definitions and that this is often an excellent method, as
opposed to trying to figure out from scratch the final definition. It
would have been probably more difficult to write the tail form variant
of \erlcode{join/2} without having first designed the version not in
tail form. It is in general not a good idea to start head\hyp{}on by
defining a function in tail form because it may either be unnecessary,
the maximum allowed size for the call stack being large enough, or
lead to a mistake since these kinds of definitions are usually more
involved.

Sometimes the result is correct but overly involved. Let us consider a
simple case by defining a function \erlcode{last/1} such that
\erlcode{last(\(L\))}~computes the last item of the non\hyp{}empty
list~\(L\). The correct approach is to forget about tail forms and aim
straight at the heart of the problem. We know that \(L\)~can not be
empty, so let us start with the following head:
\begin{alltt}
last([I|L]) -> \fbcode{last([J|L])}.
\end{alltt}
Can we reach the result in one step? No, because we do not know
whether \erlcode{I}~is the sought item: we need to know more
about~\erlcode{L}. This additional information about the structure of
\erlcode{L} is given by more precise heads: \erlcode{L}~can be empty
or not. This is
\begin{alltt}
last([I|   \textbf{[]}]) -> \fbcode{last([J|L])};
last([I|\textbf{[J|L]}]) -> \fbcode{last([J|L])}.
\end{alltt}
The first head can be simplified as follows:
\begin{alltt}
last(      [I]) -> \fbcode{last([J|L])};
last([I|[J|L]]) -> \fbcode{last([J|L])}.
\end{alltt}
The first body is easy to guess:
\begin{alltt}
last(      [I]) -> \textbf{I};
last([I|[J|L]]) -> \fbcode{last([J|L])}.
\end{alltt}
In the last clause, how do \erlcode{I}, \erlcode{J}~and~\erlcode{L}
relate to the result? Can we reach it in one step? No, despite we know
that \erlcode{I}~is \emph{not} the result, we still don't know whether
\erlcode{J}~is, so we have to start over again, which means a
recursive call is required:
\begin{alltt}
last(      [I]) -> I;
last([I|[J|L]]) -> \textbf{last(}\fbcode{[J|L]}\textbf{)}.\hfill% I \emph{is useless}
\end{alltt}
Note how knowing that some specific part of the input is not useful to
build the output is useful knowledge. By the way, \Erlang allows us to
silence these useless parts by writing instead an underscore as in
\begin{alltt}
last(      [I]) -> I;
last([\textbf{\_}|[J|L]]) -> last(\fbcode{[J|L]}).\hfill% \emph{Anonymous list head}
\end{alltt}
We can not call recursively \erlcode{last(L)} because \erlcode{L}~may
be empty and the call fail, which would mean that the answer
was~\erlcode{J}. Therefore, we must call with \erlcode{[J|L]} to give
\erlcode{J} the opportunity to be the last:
\begin{alltt}
last(      [I]) -> I;
last([_|[J|L]]) -> last(\textbf{[J|L]}).
\end{alltt}
As we advocated previously, the next phase consists in testing this
definition for correctness and completeness, using meaningful
examples. Let us assume that we are convinced that this definition is
correct and complete. What's next? Let us try to improve upon it. Let
us look for common patterns that can be simplified. For instance, we
observe that \erlcode{[J|L]}~is used as a whole, in other words,
\erlcode{J}~and~\erlcode{L} are not used separately in the second
body. Therefore, it is worth trying to replace the pattern by a more
general one, in this case, a variable:
\begin{alltt}
last(  [I]) -> I;
last([_|\textbf{L}]) -> last(\textbf{L}).\hfill% L \emph{is non-empty}
\end{alltt}
This transformation is correct because the case where \erlcode{L}~is
empty has already been matched by the first head. Notice also that we
considered a definition as some data (We should write more accurately
\emph{metadata} since definitions are not data processed by the
program, but by the programmer.) which we transformed, step by step,
making sure that each step starts at a correct and complete definition
and leads to an equivalent one, so a series of transformations
compulsorily yields an equivalent function. In passing, it is now
obvious that the definition we found for \erlcode{last/1} is in tail
form.

What if we had tried to find directly a definition in tail form? We
might have recalled that such definitions need an accumulator and we
would have tried perhaps something along these lines:
\begin{verbatim}
last(L)         -> last__(L,0).
last__(   [],A) -> A;
last__([I|L],A) -> last__(L,I).
\end{verbatim}
The first observation may be about the function name
\erlcode{last\_\_}. Why not write the following, in accordance with
the style up to now?
\begin{alltt}
last(L)       -> \textbf{last}(L,0).
\textbf{last}(   [],A) -> A;
\textbf{last}([I|L],A) -> \textbf{last}(L,I).
\end{alltt}
There would be no confusion between \erlcode{last/1} and
\erlcode{last/2} because, each taking a different number of arguments,
they are logically considered different in \Erlang. The reason why we
recommend to distinguish the names and, in general, to use one name
for only one function (inside a module), is that this discipline
enables the compiler to catch the error consisting in forgetting one
argument. For instance, the program
\begin{alltt}
last(L)       -> last(L,0).
last(   [],A) -> A;
last([I|L],A) -> \textbf{last(L)}.\hfill% \emph{Argument silently missing}
\end{alltt}
contains an error that goes unreported, while
\begin{alltt}
last(L)         -> last__(L,0).
last__(   [],A) -> A;
last__([I|L],A) -> \textbf{last\_\_(L)}.\hfill% \emph{Error reported}
\end{alltt}
raises an error in the compiler, which allows us to realise an error
was made and helps in fixing it. However, for didactic purposes in
this book, we will not always follow this recommendation of having
unique function names. The possibility to use the same name for
different functions which can be otherwise distinguished by the number
of their arguments is called \emph{overloading}. Overloading of
methods in the programming languages \Java and \Cpp is allowed, but
the rules used to distinguish amongst the different methods sharing
the same name is different than in \Erlang, as it makes use of the
number of parameters but also their static types.

Computing the call \erlcode{last([1,2,3])} with the original
definition, we find that the three clauses are covered until the
correct result is found, that is, \erlcode{3}. Because we recommended
previously to make some litmus test and the argument is a list, we try
the empty list and find that \(\erlcode{last([])} \rightarrow
\erlcode{last\_\_([],0)} \rightarrow \erlcode{0}\), which is
unexpected, since this test ought to fail (\erlcode{last/1} is not
defined for the empty list). Can we fix this case? Yes, this can be
done in a simple way: let us change the head of \erlcode{last/1} so
that only non\hyp{}empty lists are matched. We find here a case where
more information on the structure of the input is needed and a
variable is too general a pattern. We need instead
\begin{alltt}
last(\textbf{[I|L]})     -> last__(\textbf{[I|L]},0).\hfill% \emph{Fixed}
last__(   [],A) -> A;
last__([I|L],A) -> last__(L,I).
\end{alltt}
This emendation seems to go against an improvement we made earlier,
when we replaced \erlcode{[J|L]} by \erlcode{L}, but it does not: here
we want to exclude some input, that is, we do not seek an equivalent
function, whilst before the purpose was to simplify and obtain an
equivalent function.

This last definition of \erlcode{last/1} is correct and complete but a
careful review should raise some doubts about its actual
simplicity. For example, the initial value of the accumulator, given
in the unique body of \erlcode{last/1} is \erlcode{0}, but this value
is never used, because it is discarded immediately after in the second
body of \erlcode{last\_\_/2}, as it is conspicuous if we write instead
the equivalent
\begin{alltt}
last([I|L])     -> last__([I|L],\textbf{0}).\hfill% \emph{Why zero?}
last__(   [],A) -> A;
last__([I|L],\textbf{\_}) -> last__(L,I).\hfill% A \emph{was useless}
\end{alltt}
Indeed, we could write the equivalent definition:
\begin{alltt}
last([I|L])     -> last__([I|L],\textbf{7}).\hfill% \emph{Why not 7?}
last__(   [],A) -> A;
last__([I|L],_) -> last__(L,I).
\end{alltt}
The initial value of the accumulator here does not even need to be an
integer, it could be of whatever type, like \erlcode{[4,[]]}. We
should better give up this overly complicated definition, which is the
result of a methodology that does not consider programs as data and is
founded on the wrong assumption that definitions in tail form always
require an accumulator: they do not, in general. Take for example the
polymorphic identity: \erlcode{id(X) -> X.} It is trivially in tail
form. In passing, being in tail form has nothing to do, in general,
with recursion, despite the widespread and unfortunate locution
``tail\hyp{}recursive function.'' A recursive definition may be in tail
form, but a definition in tail form may not be recursive, as
\erlcode{id/1}.

\medskip

\paragraph{List reversal.}

How do we reverse a given list? Let us explore two approaches: the
first one tries to analyse the problem in terms of a previous problem,
the second reuses, more abstractly, some insight gained in solving a
previous larger or similar problem. Let \erlcode{srev/1} be the
function such that \erlcode{srev(\(L\))} is rewritten in a list which
contains the items of list \(L\) in reverse order. It is easy to write
the heads:
\begin{alltt}
srev(   []) -> \fbcode{HH srev(L) HH I HH};
srev([I|L]) -> \fbcode{HH srev(L) HH I HH}.
\end{alltt}
We must ponder whether \erlcode{srev/1} is defined for the empty list,
since the original question left that point out. It was said that the
result is a list containing all the items of the input and, since the
input contains no item, it implies that the result is a list that
contains no items. This makes sense as this is the very definition of
the empty list:
\begin{alltt}
srev(   []) -> \textbf{[]};
srev([I|L]) -> \fbcode{HH srev(L) HH I HH}.
\end{alltt}
We must wonder now if we can reach the result in one step in the
remaining body. Clearly no, because we have to reverse \erlcode{L},
that is, a recursive call \erlcode{srev(L)} must be placed
somewhere. Where is the item \erlcode{I} located in the final result?
Since the purpose is to reverse \erlcode{[I|L]}, we deduce that
\erlcode{I} must be located at the end of the resulting list. So the
situation so far is as follows:
\begin{alltt}
srev(   []) -> [];
srev([I|L]) -> \fbcode{HH} \textbf{srev(L)} \fbcode{HH} \textbf{I} \fbcode{HH}.
\end{alltt}
As usual, the empty boxes are a reminder of some relationship to be
found between the pieces already in place. For instance, we know that
the body must be a list, so should we try the following?
\begin{alltt}
srev(   []) -> [];
srev([I|L]) -> [srev(L) | I].\hfill% \emph{Hmm...}
\end{alltt}
For this to work we need to check the types of the expressions
involved at this point. We know that \erlcode{srev(L)} is a list made
of items from \erlcode{L}, because it is a recursive call; item
\erlcode{I} is an item from \erlcode{[I|L]}. But the position of
\erlcode{srev(L)} in our proposal is that of the head of a list and
\erlcode{I} is the tail of the same, which is wrong: they should be
reversed to be homogeneous. But the relative order of
\erlcode{srev(L)} and \erlcode{I} above is correct, so we must think
of some other relationship. What we want is to join \erlcode{srev(L)}
and \erlcode{I}, which reminds us of the function \erlcode{join/2}:
\begin{alltt}
srev(   []) -> [];
srev([I|L]) -> \textbf{join(}srev(L)\textbf{,}I\textbf{)}.\hfill% \emph{Still...}
\end{alltt}
Again, let us check the types. We know that in
\erlcode{join(\(P\),\(Q\))}, both \(P\) and \(Q\) must be lists. But
\erlcode{I} is an item, so it does not fit in as it is. The solution
consists in making a singleton from \erlcode{I}:\label{code:srev}

\verbatiminput{srev.def}

\noindent Before we test this definition, a keen observer may have
raised an objection to the last argument: it is possible that
\erlcode{I} be a list. Indeed, but what was exactly meant was that
\erlcode{I} is considered as an item \emph{with respect to
  \erlcode{srev(L)}}, not in an absolute sense. This is best
understood by looking at the abstract syntax tree of the input
list. Let us take for instance \erlcode{[3,[1,[]],[]]}. This value is
a short\hyp{}hand for \erlcode{[3|[[1|[[]|[]]]|[[]|[]]]]}, whose tree
is given in \fig~\vref{fig:list_example}.
\begin{figure}
\centering
\includegraphics[bb=71 631 171 721]{list_example}%
\caption{Abstract Syntax Tree of \erlcode{[3,[1,[]],[]]}
\label{fig:list_example}}
\end{figure}
Following the rightmost edges from the root to the rightmost
\erlcode{[]} (the dashed line), the left subtrees (the solid
lines) are the items of the embedding list \erlcode{[3,[1,[]],[]]}:
\erlcode{3}, \erlcode{[1,[]]} and \erlcode{[]}. It does not matter
that one item is an integer while the other two are lists, what does
matter is their relative positions in the abstract syntax tree because
it defines their interpretation as items of the whole list. Perhaps we
can also remark that the definition of \erlcode{srev/1} we came upon
contains a call inside a call. We found \vpageref{fibonacci} a similar
situation with a \emph{na\"{\i}ve} definition of the Fibonacci
function:
\begin{verbatim}
fib(0)            -> 1;
fib(1)            -> 1;
fib(N) when N > 1 -> fib(N-1) + fib(N-2).
\end{verbatim}
Remember that (\erlcode{+}) is an operator, that is, a predefined
function, so its arguments are function calls themselves:
\erlcode{fib(N-1)} and \erlcode{fib(N-2)}. When multiple calls are
present in a body, there could be several ways to rewrite it, but we
saw that the strategy used by \Erlang consists in rewriting first the
most embedded calls and then their immediately embedding calls
etc. until the outermost call is evaluated. But the order in which two
calls which are arguments to the same function or arithmetic operator
are evaluated is not specified in \Erlang. In the case of
\erlcode{srev/1}, there is no choice and the first call to be computed
is \erlcode{srev(\(L\))}, because it is the most deeply embedded, and
then the immediately embedding
\erlcode{join(\(\overline{L}\),[\(I\)])}, where \(\overline{L}\)
stands for the value of \erlcode{srev(\(L\))} previously computed. In
the case of \erlcode{fib/1}, it is purposefully left unspecified which
call of \erlcode{fib(N-1)} or \erlcode{fib(N-2)} is rewritten first.

Another observation is that our definition of \erlcode{srev/1} is not
in tail form: there is a control context
\erlcode{join(\textvisiblespace,[\(I\)])} around the recursive call
\erlcode{srev(\(L\))}. (Let us keep in mind that the fact that the
call is recursive does not matter here.) Therefore, we should expect
the maximum size of the call stack to grow as the input size
grows. Let us take as an example \erlcode{srev([3,2,1])}, whose
expected outcome must be \erlcode{[1,2,3]}, and let us follow the
rewrites on the abstract syntax trees by drawing the node to be
rewritten at each step in a box. Let us label the clauses defining
\erlcode{srev/1} with the Greek letters \(\alpha\) and \(\beta\):
\begin{alltt}
srev(   []) \(\smashedrightarrow{\alpha}\) [];
srev([I|L]) \(\smashedrightarrow{\beta}\) join(srev(L),[I]).
\end{alltt}
\Fig~\vref{fig:srev_321} shows the first rewrites until there are no
more calls to \erlcode{srev/1}.
\begin{figure}[t]
\centering
\includegraphics[bb=71 620 324 721]{srev_321_0}
\includegraphics[bb=66 626 311 721]{srev_321_1}
\caption{\erlcode{srev([3,2,1])} \(\twoheadrightarrow\)
\erlcode{join(join(join([],[1]),[2]),[3])}
\label{fig:srev_321}}
\end{figure}
Let us label the clauses defining \erlcode{join/2} as follows
\begin{alltt}
join(   [],Q) \(\smashedrightarrow{\gamma}\) Q;
join([I|P],Q) \(\smashedrightarrow{\delta}\) [I|join(P,Q)].
\end{alltt}
and resume the rewrites in \fig~\vref{fig:rev_321ast}.  This
computation can also be followed in a flat representation as
well:\label{rev_321bis}
\begin{tabbing}
\erlcode{srev([3,2,1])} \= \(\smashedrightarrow{\beta}\) \= \erlcode{join(srev([2,1]),[3])}\\
  \> \(\smashedrightarrow{\beta}\) \> \erlcode{join(join(srev([1]),[2]),[3])}\\
  \> \(\smashedrightarrow{\beta}\) \>
                \erlcode{join(join(join(srev([]),[1]),[2]),[3])}\\
  \> \(\smashedrightarrow{\alpha}\) \>
                \erlcode{join(join(join([],[1]),[2]),[3])}\\
  \> \(\smashedrightarrow{\gamma}\) \> \erlcode{join(join([1],[2]),[3])}\\
  \> \(\smashedrightarrow{\delta}\) \> \erlcode{join([1|join([],[2])],[3])}\\
  \> \(\smashedrightarrow{\gamma}\) \> \erlcode{join([1|[2]],[3])}\\
  \> \(=\)               \> \erlcode{join([1,2],[3])}\\
  \> \(\smashedrightarrow{\delta}\) \> \erlcode{[1|join([2],[3])]}\\
  \> \(\smashedrightarrow{\delta}\) \> \erlcode{[1|[2|join([],[3])]]}\\
  \> \(=\)               \> \erlcode{[1,2|join([],[3])]}\\
  \> \(\smashedrightarrow{\gamma}\) \> \erlcode{[1,2|[3]]}\\
  \> \(=\)               \> \erlcode{[1,2,3]}\textrm{.}
\end{tabbing}
There are~\(10\) steps: \(4\)~with clauses
\clause{\alpha}~and~\clause{\beta} (\erlcode{srev/1}), plus \(6\)~with
the clauses \clause{\gamma}~and~\clause{\delta}
(\erlcode{join/2}). The contents of the call stack, at any time, is
the branch from the root to the boxed node, which is the call to be
rewritten next, as shown in \fig~\vref{fig:rev_steps} (the final
downward arrow is a reference to the result in the heap). We see that
the longest extant of the call stack was reached after the third step,
when \fbox{\erlcode{srev}} is the deeper in the abstract syntax
tree. As we shall understand, the occurrence of the number~\(3\) here
is no mere coincidence.
\begin{figure}[H]
\centering
\includegraphics[bb=71 627 366 721]{join_join_123_0}
\includegraphics[bb=71 627 364 721]{join_join_123_1}
\caption{\erlcode{join(join([1],[2]),[3])} \(\twoheadrightarrow\)
  \erlcode{[1,2,3]}
\label{fig:rev_321ast}}
\end{figure}

What would have happened, had we chosen \erlcode{join\_tf/2} (tail
form) instead of \erlcode{join/2}? Let us recall the definition of
\erlcode{join\_tf/2}:
\begin{alltt}
join_tf(P,Q)        \(\smashedrightarrow{\alpha}\) join(P,Q,[]).
join(   [],Q,   []) \(\smashedrightarrow{\beta}\) Q;
join(   [],Q,[I|A]) \(\smashedrightarrow{\gamma}\) join([],[I|Q],A);
join([I|P],Q,    A) \(\smashedrightarrow{\delta}\) join(P,Q,[I|A]).
\end{alltt}

and define an alternative definition \erlcode{srev\_alt/1} as follows:
\begin{alltt}
srev_alt(   []) \(\smashedrightarrow{\epsilon}\) [];
srev_alt([I|L]) \(\smashedrightarrow{\zeta}\) join_tf(srev_alt(L),[I]).
\end{alltt}
Then, our example unfolds as follows.\label{rev_321}
\begin{alltt}
\erlcode{srev\_alt([3,2,1])}
    \(\smashedrightarrow{\zeta}\) \erlcode{\textbf{join\_tf}(srev\_alt([2,1]),[3])}
    \(\smashedrightarrow{\zeta}\) \erlcode{join\_tf(join\_tf(srev\_alt([1]),[2]),[3])}
    \(\smashedrightarrow{\zeta}\) \erlcode{join\_tf(join\_tf(join\_tf(srev\_alt([]),[1]),[2]),[3])}
    \(\smashedrightarrow{\epsilon}\) \erlcode{join\_tf(join\_tf(join\_tf([],[1]),[2]),[3])}
    \(\smashedrightarrow{\alpha}\) \erlcode{join\_tf(join\_tf(join([],[1],[]),[2]),[3])}
    \(\smashedrightarrow{\beta}\) \erlcode{join\_tf(join\_tf([1],[2]),[3])}
    \(\smashedrightarrow{\alpha}\) \erlcode{join\_tf(join([1],[2],[]),[3])}
    \(\smashedrightarrow{\delta}\) \erlcode{join\_tf(join([],[2],[1]),[3])}
    \(\smashedrightarrow{\gamma}\) \erlcode{join\_tf(join([],[1,2],[]),[3])}
    \(\smashedrightarrow{\beta}\) \erlcode{join\_tf([1,2],[3])}
    \(\smashedrightarrow{\alpha}\) \erlcode{join([1,2],[3],[])}
    \(\smashedrightarrow{\theta}\) \erlcode{join([2],[3],[1])}
    \(\smashedrightarrow{\delta}\) \erlcode{join([],[3],[2,1])}
    \(\smashedrightarrow{\delta}\) \erlcode{join([],[2,3],[1])}
    \(\smashedrightarrow{\gamma}\) \erlcode{join([],[1,2,3],[])}
    \(\smashedrightarrow{\beta}\) \erlcode{[1,2,3]}\textrm{.}
\end{alltt}
\begin{figure}[h]
\centering
\includegraphics[bb=71 633 371 721]{srev_call_stack_0}
\includegraphics[bb=71 653 376 720]{srev_call_stack_1}
\caption{Call stack of \protect\figs~\protect\vrefrange{fig:srev_321}{fig:rev_321ast}\label{fig:rev_steps}}
\end{figure}
These rewrites can also be viewed two\hyp{}dimensionally as abstract
syntax trees in \fig~\vref{fig:join_tf_123}, 
\begin{figure}
\centering
\vspace{-4pt}
\includegraphics[bb=71 628 339 721]{join_tf_0}
\includegraphics[bb=71 646 335 721]{join_tf_1}
\includegraphics[bb=71 626 317 720]{join_tf_2}
\hspace{-6pt}
\includegraphics[bb=71 646 255 720]{join_tf_3}
\hspace{12pt}
\includegraphics[bb=71 646 301 720]{join_tf_4}
\includegraphics[bb=71 626 325 720]{join_tf_5}
\caption{\erlcode{join\_tf(join\_tf(join\_tf([],[1]),[2]),[3])}
\(\!\!\!\twoheadrightarrow\!\!\!\) \erlcode{[1,2,3]}
\label{fig:join_tf_123}}
\end{figure}
where the last step \((\smashedrightarrow{\epsilon})\) of
\erlcode{srev\_alt/1} is shown first. There are \(16\) steps: \(4\)
with the definition of \erlcode{srev\_alt/1}, plus \(12\) with the
definition of \erlcode{join/2}. The contents of the call stack, at any
time, is the branch from the root to the boxed node in
\fig~\vref{fig:join_tf_123_stack} (steps 1--4 are omitted as they are
already displayed in \fig~\vref{fig:rev_steps}).
\begin{figure}
\centering
\includegraphics[bb=71 643 365 720]{join_tf_call_stack_0}
\bigskip
\includegraphics[bb=71 674 310 720]{join_tf_call_stack_1}
\bigskip
\includegraphics[bb=71 693 313 720]{join_tf_call_stack_2}
\bigskip
\includegraphics[bb=71 693 365 720]{join_tf_call_stack_3}
\caption{Call stack of \protect\fig~\protect\vref{fig:join_tf_123}
\label{fig:join_tf_123_stack}}
\end{figure}

What can we conclude from using \erlcode{join/2} to define
\erlcode{srev/1} and \erlcode{join\_tf/1} to define
\erlcode{srev\_alt/1}?

On the example \erlcode{srev\_alt([3,2,1])}, using
\erlcode{join\_tf/2} (tail form) leads to longer rewrites (\(16\)
against \(10\)) and the maximum call stack capacity is the same (since
it is reached after the same third step in both cases). We may not be
surprised to find a greater delay by using a tail form definition, but
it may come as a surprise that there is no gain in the call stack
allocation.

Is perhaps the reason that the test was carried out on a list too
short to show otherwise? When considering such a question, instead of
trying a longer input by hand, it is worth revisiting the previous
test and try to generalise the process for a list of \(n\) items:
\erlcode{[\(I_0\),\(\dots\),\(I_{n-1}\)]}. The third rewrite,
involving only the definition of \erlcode{srev/1}, leads to the
abstract syntax tree in \fig~\vref{fig:srev_iota}, where the dotted
edge stands for the repetition of the same schema: a node
\erlcode{join} with a right subtree \erlcode{[\(I_i\)]}. The number of
nodes in the call stack is thus \(n+1\) and this holds whichever
version for joining two lists, \erlcode{join/2} or
\erlcode{join\_tf/2}, is chosen. Therefore, a lesson to remember from
this short analysis is that the usage of a definition in tail form
does not automatically lead to gains in terms of call stack allocation
or speed---indeed, a slowdown is likely. One way to recover the
benefit of definitions in tail form is to make sure that \emph{all}
the functions that are composed are either in tail form or do not use
call stack space at least in proportion of the input size.

Therefore, let us turn our attention back to the design of
\erlcode{srev/1}. We would like to find a definition equivalent to
\erlcode{srev/1} but in tail form. There are at least two options left
open: either start from scratch or try to modify the program which is
correct and complete but not in tail form. Another way to analyse the
problem consists in realising that \emph{the definition of
  \erlcode{join\_tf/2} already includes the concept of list reversal.}
In other words, instead of calling \erlcode{join/2} in the definition
of \erlcode{srev/1}, let us realise that the design of
\erlcode{srev/1} is already embedded in the definition
\erlcode{join\_tf/2} and write accordingly another version of
\erlcode{srev/1}, called \erlcode{rev/1} in tail form. It is important
to realise that \erlcode{rev/1} is not derived mechanically from
\erlcode{srev/1}: it implements another design. The definition
\vpageref{code:join_tf} shows that clause \clause{\delta} has the
effect of reversing the first argument into the accumulator (the third
argument). Moreover, clause \clause{\gamma} reverses the accumulator
on the second argument. Thus, we already know how to reverse a list:
simply use an extra argument to accumulate the items of the list to be
reversed, since the accumulation consists in pushing the items one by
one on top of the initially empty list, the resulting value of the
accumulator is the input list reversed. In \Erlang, this is
implemented as follows:\label{code:rev_join}\label{code:rev}
\verbatiminput{rev.def} All the clauses are in tail form, hence is the
whole definition. The accumulator is named \erlcode{Q}. We must bear
in mind that the reason why this definition is efficient is not due to
it being in tail form, but because the algorithm, that is, the underlying
idea, is better.
\begin{figure}[t]
\centering
\includegraphics[bb=71 650 179 721]{srev_iota}%
\caption{\erlcode{srev([\(I_0\),\(\ldots\),\(I_{n-1}\)])} after \(n\)
  rewrites\label{fig:srev_iota}}
\end{figure}

\medskip

\medskip

\paragraph{Exercises.}
\label{ex:joining_and_reversing}

\noindent [Answers \vpageref{ans:joining_and_reversing}.]
\begin{enumerate}

  \item Define a function \erlcode{len\_tf/1} in tail form such that
    \erlcode{len\_tf(\(L\))} is rewritten to the number of items in
    the list \(L\). Do you need an auxiliary function? How many
    rewrites are needed to reach the result when \(L\) contains \(n\)
    items?

  \item Define a function \erlcode{penul/1} such that
    \erlcode{penul(\(L\))} is rewritten to the penultimate item in
    list \(L\). Make sure to avoid any confusion in the syntax of
    lists between the comma (\erlcode{,}) and the vertical bar
    (\erlcode{|}). Test for correctness and completeness. If \(L\)
    contains \(n\) items, how many rewrites are needed to reach the
    result? Is your definition in tail form?

  \item Define a function \erlcode{rep\_fst/1} such that the call
    \erlcode{rep\_fst(\(L\))}, where \(L\) is a list, is rewritten
    into a list of same length as \(L\) but containing only the first
    item of \(L\) repeated.

  \item Same with \erlcode{rep\_lst/1}, but repeating the last item
    instead of the first.

  \item Consider the following alternative definition for reversing a list.
\begin{verbatim}
rev_bis(   []) -> [];
rev_bis([I|L]) -> join(rev_bis(L),[I],[]).

join(   [],Q,   []) -> Q;
join(   [],Q,[I|A]) -> join([],[I|Q],A);
join([I|P],Q,    A) -> join(P,Q,[I|A]).
\end{verbatim}
    Lay out the rewrites of \erlcode{rev\_bis([3,2,1])} with this
    definition, both with the abstract syntax trees and the flat
    representation. Compare with the definition of \erlcode{srev/1}:
    \verbatiminput{srev.def} \verbatiminput{join_2.def}

\end{enumerate}
