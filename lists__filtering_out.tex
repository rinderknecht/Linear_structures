%%-*-latex-*-

\chapter{Filtering out}
\label{chap:filtering_out}

\paragraph{First occurrence.}

Let us consider the definition of a function \erlcode{rm\_fst/2} such
that the call \erlcode{rm\_fst(\(I\),\(L\))} is rewritten into~\(L\)
if \(I\)~does not belong to the list~\(L\), otherwise it is rewritten
to a list identical to~\(L\) but without the first occurrence
of~\(I\). This is our \emph{specification}. For instance, we expect
the following rewrites:\label{rm_fst_ex}
\begin{align*}
\erlcode{rm\_fst(3,[])}         & \twoheadrightarrow \erlcode{[]};\\
\erlcode{rm\_fst([],[])}        & \twoheadrightarrow \erlcode{[]};\\
\erlcode{rm\_fst([5,2],[3,[]])} & \twoheadrightarrow \erlcode{[3,[]]};\\
\erlcode{rm\_fst(4,[[],[1,2],4,[],4])}
 & \twoheadrightarrow \erlcode{[[],[1,2],[],4]};\\
\erlcode{rm\_fst([],[4,[1,2],[],[],4])}
 & \twoheadrightarrow \erlcode{[4,[1,2],[],4]}.
\end{align*}

\paragraph{First attempt.}

Let us try a direct approach. In particular, at this point, it is
important \emph{not} to seek a definition in tail form. Tail form must
be considered as an optimisation and early optimisation is the root of
all evil. The first idea that may come to mind is to define an
auxiliary function \erlcode{mem/2} such that the call
\erlcode{mem(\(I\),\(L\))} checks whether a given item \(I\)~is in a
given list~\(L\), because that notion of membership appears in the
wording of the specification. But two problems would arise. Firstly,
what would be the result of such a function?  Secondly, what would be
the added delay to use it? For the sake of the argument, let us try
this line of thought. A list can either be empty or not, so let us
make two clauses:
\begin{alltt}
mem(I,   []) -> \fbcode{mem(I,L)};
mem(I,[J|L]) -> \fbcode{mem(I,L)}.
\end{alltt}
Note that we introduced a variable \erlcode{J}, distinct from variable
\erlcode{I}. \emph{Two different variables may or may not denote the
  same value, but two occurrences of the same variable must denote the
  same value.} Had we written instead
\begin{alltt}
mem(I,   []) -> \fbcode{mem(I,L)};
mem(I,[\textbf{I}|L]) -> \fbcode{mem(I,L)}.\hfill% \emph{Error: two occurrences of} I
\end{alltt}
one case would be missing, namely when the head of the list is not the
item sought for, for instance, \erlcode{mem(3,[4])} would fail due to
a match failure. Now, what is the first body? The first head matches
if the list is empty. In particular, this means that the item is not
in the list, since, by definition, an empty list is a list containing
no item. How do we signify that? Since the original problem is silent
on the matter, it is said to be \emph{underspecified}. We may think
that zero would be a token of choice to denote the absence of the item
in the list:
\begin{alltt}
mem(I,   []) -> \textbf{0};
mem(I,[\textbf{J}|L]) -> \fbcode{mem(I,L)}.
\end{alltt}
But this would be a mistake because there is no natural relationship
between the concept of emptiness and the number zero. (Why not~\(7\)?)
Zero is best understood algebraically as the number noted~\(0\) such
that \(0 + n = n + 0 = n\), for any number~\(n\). Then, let us try the
empty list:
\begin{alltt}
mem(I,   []) -> \textbf{[]};\hfill% I \emph{is not reused}
mem(I,[J|L]) -> \fbcode{mem(I,L)}.
\end{alltt}
The next step is to find a way to actually compare the value
of~\erlcode{I} to the value of~\erlcode{J}. We can use the rule above
about variables: two occurrences of the same variable mean that they
hold the same value. Therefore
\begin{alltt}
mem(I,   []) -> [];
mem(I,[\textbf{I}|L]) -> \fbcode{mem(I,L)}.
\end{alltt}
was not so bad, after all? Indeed, but we know now that a case is
missing, so let us add it at the end:
\begin{alltt}
mem(I,   []) -> [];
mem(I,[I|L]) -> \fbcode{mem(I,L)}\textbf{;}
\textbf{mem(I,[J|L]) ->} \fbcode{mem(I,L)}.\hfill% I \(\neq\) J
\end{alltt}
Since the run\hyp{}time, for instance, a virtual machine, of \Erlang
scans the heads top\hyp{}down, if the last head matches the current
call, it means that \erlcode{I}~and~\erlcode{J} denote different
values. Now, what is the second body? It stands after we found that
the item we were looking for is present at the head of the list, thus
it belongs to the list. How do we signify that? We may think of ending
with the item itself, the rationale being that if the result is the
empty list, then the item is not in the input list, otherwise the
result is the item itself (and it is present in the input list):
\begin{alltt}
mem(I,   []) -> [];
mem(I,[I|L]) -> \textbf{I};\hfill% \emph{Ignoring} L
mem(I,[J|L]) -> \fbcode{mem(I,L)}.
\end{alltt}
The last body is easier to guess since it deals with the case where
the head of the list (\erlcode{J}) is not the item we seek
(\erlcode{I}), so a recursive call which ignores~\erlcode{J} should
come to mind:
\begin{alltt}
mem(I,   []) -> [];
mem(I,[I|L]) -> I;
mem(I,[J|L]) -> \textbf{mem(I,L)}.\hfill% \emph{Ignoring} J
\end{alltt}
We could even simplify this definition by replacing variables in the
heads where they occur only once and which do not occur in their
corresponding bodies (that is, their value is ignored) with an
underscore:
\begin{alltt}
mem(\textbf{\_},   []) -> [];\hfill% \emph{Here,}
mem(I,[I|\textbf{\_}]) -> I;\hfill% \emph{here}
mem(I,[\textbf{\_}|L]) -> mem(I,L).\hfill% \emph{and here}
\end{alltt}
That is the reason why we choose the names of the auxiliary functions,
that is, functions that are not expected to be used by the user
directly (and are not exported outside their module), to end with two
underscores: it is a syntactical error in \Erlang to have a pattern
containing two successive underscores standing for parameters and we
sometimes use one underscore at a time in the names, so two
underscores avoid any confusion.

Some tests would increase the confidence that this definition is
correct and complete with respect to the specification. Let us label
the clauses first:
\begin{alltt}
mem(_,   []) \(\smashedrightarrow{\zeta}\) [];
mem(I,[I|_]) \(\smashedrightarrow{\eta}\) I;
mem(I,[_|L]) \(\smashedrightarrow{\theta}\) mem(I,L).
\end{alltt}
Then we could try the following cases:
\begin{alltt}
     mem(3,[]) \(\smashedrightarrow{\zeta}\) []\textrm{,}
    mem(3,[1]) \(\smashedrightarrow{\theta}\) mem(3,[])    \(\smashedrightarrow{\zeta}\) []\textrm{,}
mem(3,[1,3,2]) \(\smashedrightarrow{\theta}\) mem(3,[3,2]) \(\smashedrightarrow{\eta}\) 3\textrm{.}
\end{alltt}
The code seems to work: item~\erlcode{I}~is in list~\erlcode{L} if the
result is~\erlcode{I}, otherwise it is~\erlcode{[]}. However, the
program is not correct. The hidden and flawed assumption that lead to
the error is ``items can not be lists,'' in spite of the examples
given for illustrating the expected behaviour of \erlcode{rm\_fst/2},
which contained cases where the item was a list. In particular, an
item can be the empty list and this situation leads to an ambiguity
with our definition of \erlcode{mem/2}:
\[
\erlcode{mem([],[])}   \smashedrightarrow{\zeta} \erlcode{[]},\quad
\erlcode{mem([],[[]])} \smashedrightarrow{\eta} \erlcode{[]}.
\]
It is impossible to distinguish the two cases, the first one meaning
absence and the second presence of the item, because they both end
with the empty list. This problem could have been discovered earlier
or during testing. Empirical evidence advises to test a given
definition with the empty list in as many locations as possible in the
input. If we wish to maintain the polymorphism of the solution to our
problem, that is, if we really want the definition to accept lists of
any kind of items, there is no way out. Actually, the definition we
came up with is correct if the items are all integers, but we have no
way yet to implement this restriction. We shall indeed meet other
kinds of values later, making this simple question of membership
straightforward to answer. But, for now, let us backtrack and ask
ourselves whether using~\erlcode{mem/2}, assuming we have it, is
really a good idea.

\medskip

\paragraph{Better approach.}

Let us suppose that the input list contains the item at the
end. Using~\erlcode{mem/2} to find it leads to a complete traversal of
the input list. Then another traversal from the beginning (the head of
the list) is needed to make a new list without the last item, so, in
total, two traversals are performed. But a better idea consists in
\emph{interleaving} these two passes into one because the problem
stems from the fact that \erlcode{mem/2}~forgets about the items which
are not the item of interest, thus, when it is found or known to be
absent, there is no way to remember the past items in order to make
the result. By interleaving, we mean that during the traversal, the
concepts of membership and of ``rebuilding a list without an item''
are combined, instead of being used separately as two function
calls. A similar situation was encountered in the design of a function
reversing a list: the version which made use of a \emph{function} to
join two lists was less efficient than the version which included the
\emph{concept} of joining. The algorithm consists in memorising all
visited items, so they are not forgotten and, if the item is not
found, the resulting list is rebuilt from them; if found, the result
is built from them \emph{and} the remaining, unvisited, items. There
are usually two ways to implement this auxiliary memory of visited
items: either by using an accumulative parameter, called
\emph{accumulator}, or, quite simply, by means of the control context
of a recursive call. At this point, it is important to recall a
cardinal guideline: Do not try first to design a definition in tail
form, but opt instead for a direct approach. Once a correct and
complete definition has been reached, if the data set is too big to
fit in the call stack as allocated by the operating system, then a
tail form definition should be obtained by transforming the current
definition.  In some simple cases, a direct approach can be in tail
form, but the point still holds in general: at first, let us ignore
all concerns about the definition being in tail form. Accordingly, let
us use the control context of a recursive call to record the visited
items. A list being either empty or not, the following two clauses
come naturally to mind:
\begin{alltt}
rm_fst(I,   []) -> \fbcode{[J|rm_fst(I,L)]};
rm_fst(I,[J|L]) -> \fbcode{[J|rm_fst(I,L)]}.
\end{alltt}
Then, just as we tried with~\erlcode{mem/2}, we must distinguish the
case when \erlcode{I}~is the same as~\erlcode{J}:
\begin{alltt}
rm_fst(I,   []) -> \fbcode{[J|rm_fst(I,L)]};
\textbf{rm_fst(I,[I|L]) ->} \fbcode{[J|rm_fst(I,L)]}\textbf{;}\hfill% \emph{Equality constraint}
rm_fst(I,[J|L]) -> \fbcode{[J|rm_fst(I,L)]}.
\end{alltt}
Now, we know that the last clause deals with the case when the item
\erlcode{I}~is not~\erlcode{J}, thus we must memorise~\erlcode{J} and
go on comparing~\erlcode{I} with the other items in~\erlcode{L} (if
any). This is where the recursive call with a control context,
discussed above, is set as \erlcode{[J|\textvisiblespace]}:
\begin{alltt}
rm_fst(I,   []) -> \fbcode{[J|rm_fst(I,L)]};
rm_fst(I,[I|L]) -> \fbcode{[J|rm_fst(I,L)]};
rm_fst(I,[J|L]) -> \textbf{[J|rm_fst(I,L)]}.
\end{alltt}
Very importantly, let us remark that the position of~\erlcode{J} in
the result is the same as in the input (the head of a list). The
second clause corresponds to the case where the item we are looking
for, namely~\erlcode{I}, is found to be the head of the current list,
which is a sub\hyp{}list of the original input. A list made of
successive items from the beginning of a given list is called a
\emph{prefix} of the latter. When a list is a sub\hyp{}list of
another, that is, it is made of successive items including the last,
it is called a \emph{suffix}. We know that the~\erlcode{I} in
\erlcode{[I|L]} is the first occurrence of~\erlcode{I} in the original
list (the one in the first call), because we wouldn't be dealing with
this case \emph{again}: the specification states that this first
occurrence must be absent from the resulting list; since it is now at
the head of a suffix, we just need to end with~\erlcode{L},
\emph{which we do not visit}:
\begin{alltt}
rm_fst(I,   []) -> \fbcode{[J|rm_fst(I,L)]};
rm_fst(I,[I|L]) -> \textbf{L};
rm_fst(I,[J|L]) -> [J|rm_fst(I,L)].
\end{alltt}
The first clause handles the case where we traversed the whole
original list (up to \erlcode{[]}) without finding \erlcode{I}. Thus
the result is simply the empty list because the empty list without
\erlcode{I} is the empty list:
\begin{alltt}
rm_fst(I,   []) -> \textbf{[]};
rm_fst(I,[I|L]) -> L;
rm_fst(I,[J|L]) -> [J|rm_fst(I,L)].
\end{alltt}
We can simplify a little bit by muting variables whose values are
useless and finally settle for\label{code:rm_fst}
\begin{alltt}
rm_fst(\textbf{\_},   []) -> [];\hfill% \emph{Here}
rm_fst(I,[I|L]) -> L;
rm_fst(I,[J|L]) -> [J|rm_fst(I,L)].
\end{alltt}
Note that it is incorrect to write
\begin{alltt}
rm_fst(_,   []) -> [];
rm_fst(\textbf{\_},[\textbf{\_}|L]) -> L;\hfill% \emph{Error}
rm_fst(I,[J|L]) -> [J|rm_fst(I,L)].\hfill% \emph{Becomes useless}
\end{alltt}
We must understand the underscore (\erlcode{\_}) as an unknown
variable, yet unique in the pattern where it occurs. So two
underscores in the same pattern, as we mistakenly just put, may or
\emph{may not} denote the same value. Only variables which are not
used in the body \emph{and} which are not repeated in the head can be
correctly replaced by underscores. Perhaps is it timely now to comment
on the name we chose: \erlcode{rm\_fst}. It obviously stands for
``remove (the) first (occurrence),'' but, as we came to realise, nothing
is actually removed from the original list: every time a new list is
reconstructed from it, but without the first occurrence of a given
item. The usual conception of removing supposes a permanent state, an
object or entity of some sort, such that, after the removal, some part
of it is lacking. This idea has no avatar in functional languages,
like \Erlang, whose run\hyp{}time environments build new values from
old ones only by copying and sharing. This, of course, bears the
question of how and when is old data discarded by the
run\hyp{}time. This task is appointed to the \emph{garbage collector},
which can be thought of as a concurrent process with access to the
memory of the program being executed. There are many kinds of
strategies to determine when some data is definitely useless and we
shall not expand on this complicated subject here. Just let us imagine
that there is an invisible oracle which is taking care of the
deallocation of useless data, that is, releasing memory space for other
usages. The second question that is also logically entailed is how
costly is it to copy parts from the old data, for instance, the arguments of a
function call, to the new data, for instance, the result of a function
call. For example, in the definition of \erlcode{rm\_fst/2}, the
second body is \erlcode{L}, which is a suffix of the original (input)
list. Is it necessarily duplicated in memory or can it be shared
between the input and the output? We shall answer this question later.

Let us run some tests now and, in order to follow them easily, it is
handy to label the clauses with some Greek
letters:\label{code:rm_fst_alpha}

\begin{alltt}
rm_fst(_,   []) \(\smashedrightarrow{\alpha}\) [];
rm_fst(I,[I|L]) \(\smashedrightarrow{\beta}\) L;
rm_fst(I,[J|L]) \(\smashedrightarrow{\gamma}\) [J|rm_fst(I,L)].
\end{alltt}


\noindent We can then try some of the examples given
\vpageref{rm_fst_ex}.
\begin{tabbing}
\erlcode{rm\_fst([],[4,[1,2],[],[],4])}\\
\erlcode{rm\_fst([5,2],[3,[]])} \= \(\smashedrightarrow{\gamma}\) \=\kill 
\> \(\smashedrightarrow{\gamma}\) \> \erlcode{[4|rm\_fst([],[[1,2],[],[],4])]}\\
\> \(\smashedrightarrow{\gamma}\) \> \erlcode{[4|[[1,2]|rm\_fst([],[[],[],4])]]}\\
\> \(=\) \> \erlcode{[4,[1,2]|rm\_fst([],[[],[],4])]}\\
\> \(\smashedrightarrow{\beta}\) \> \erlcode{[4,[1,2]|[[],4]]}\\
\> \(=\) \> \erlcode{[4,[1,2],[],4]}.\\
\erlcode{rm\_fst([5,2],[3,[]])}
\> \(\smashedrightarrow{\gamma}\) \> \erlcode{[3|rm\_fst([5,2],[[]])]}\\
\> \(\smashedrightarrow{\gamma}\) \> \erlcode{[3|[[]|rm\_fst([5,2],[])]]}\\
\> \(=\) \> \erlcode{[3,[]|rm\_fst([5,2],[])]}\\
\> \(\smashedrightarrow{\alpha}\) \> \erlcode{[3,[]|[]]}\\
\> \(=\) \> \erlcode{[3,[]]}.
\end{tabbing}
Once we are convinced that our definition is correct and
complete with respect to the specification, there is a little extra
worth testing: we can check \emph{what happens for inputs which are
  not expected by the specification.} Our specification says at one
point that the second argument of \erlcode{rm\_fst/2} is a list. What
happens if we supply an integer instead? For example, we have
\erlcode{rm\_fst([],3)} \(\nrightarrow\). We have a match failure,
that is, the rewrites are stuck, which means that our definition is
not \emph{robust}, that is, it fails abruptly on unspecified
inputs. When programming in the small, as we do here, robustness is
usually not a concern because we want to focus on learning the
language and how to express simple algorithms with it, but when
developing programs whose customers are other people or simply when
making large applications, even for ourselves, it is important to make
the code robust by means of error signalling, catching and
display. Notice that a program can be complete but not robust, because
completeness is relative to what is specified (all valid inputs must
be accepted and not lead to an error), whilst robustness is relative
to what is left unspecified. In the present case, what can be done to
add robustness to our definition of \erlcode{rm\_fst/2}? We can add a
\emph{last} clause to handling the case where the second argument is
not a list and make the body be the empty list:
\begin{alltt}
rm_fst(_,   []) -> [];
rm_fst(I,[I|L]) -> L;
rm_fst(I,[J|L]) -> [J|rm_fst(I,L)]\textbf{;}
\textbf{rm_fst(_,    _) -> []}.\hfill% \emph{A catch-all clause}
\end{alltt}
But why the empty list in the first place? Why not? Any kind of list
would be fine, actually. The first clause has become useless, because
the empty list is matched by the underscore, as the underscore is a
special variable and variables match anything, so we can simplify the
definition:
\begin{alltt}
rm_fst(I,[I|L]) -> L;
rm_fst(I,[J|L]) -> [J|rm_fst(I,L)];
rm_fst(_,    _) -> [].
\end{alltt}
Now \erlcode{rm\_fst/2} never fails but \Erlang does not enforce that
this function will never be called with a second argument which is not
a list. Other functional languages, like \Haskell or \OCaml, feature
\emph{type systems} that enable their compilers to prove that all
calls are \emph{type\hyp{}safe}, that is, their computations will
never lead to an error due to the nature of some argument or,
generally, to values being used with wrong assumptions, for instance, they
forbid expressions like \erlcode{[] + 1}, which is allowed by \Erlang
compilers. In \OCaml, the question of the robustness of the definition
of \erlcode{rm\_fst\_\_/3} would not be raised, since the compiler
would refuse to compile a program with a call to it with a second
argument that would not be a list. Thus there would be no need and
even no way to add a catch\hyp{}all clause as we did here. It may be
argued, of course, that the caller will not know that it called
\erlcode{rm\_fst/2} with a wrong second argument. These considerations
are akin to discussing the merits and weaknesses of scripting
languages, which try very hard to ignore errors by defaulting on
special values (like the empty string) to keep running, as opposed to
programming languages that impose strong requirements at
compile\hyp{}time about the way values are processed. If defaulting on
the empty list is considered an issue, \Erlang allows another kind of
value, in addition to integer and lists: \emph{atoms}. Atoms are
enumerated values whose name only is known. In other words, an atom is
known by its name and the fact that it is unique in the whole program,
that is, another atom with a different name is a different value. By
contrast, two different variables can denote the same value. Atoms
follow the same lexical rules as function names, that is: they must
start with a lowercase letter and may be followed by any number of
letters, numbers or underscores. For instance \erlcode{error} and
\erlcode{empty} and \erlcode{hello} are valid atoms and we can trust
that they are pairwise different, that is, their unknown values are
distinct when considered two by two: \erlcode{error} \(\neq\)
\erlcode{empty} etc.

Here is the definition of a function which distinguishes between lists
and non\hyp{}list arguments:\label{code:is_a_list}
\begin{verbatim}
is_a_list(   []) -> yes;
is_a_list([_|_]) -> yes;
is_a_list(    _) -> no.
\end{verbatim}
As any other kind of value, atoms can be mixed within more complex
values, for instance, \erlcode{[8,blue,yellow,17,green]}. Atoms are handy to
signal errors because they are unique identifiers, therefore they
cannot be confused with any other kind of data the function computes
and so can be detected easily by the caller. Consider this robust
version of \erlcode{rm\_fst/2} which distinguishes errors from
specified computations:
\begin{alltt}
rm_fst(_,   []) -> [];
rm_fst(I,[I|L]) -> L;
rm_fst(I,[J|L]) -> [J|rm_fst(I,L)];
rm_fst(_,    _) -> \textbf{error}.\hfill% \emph{An atom}
\end{alltt}
Then a function calling \erlcode{rm\_fst/2} can make the difference
between a normal rewrite and an error by using an atom as a pattern:
\begin{alltt}
caller(I,L)   -> check(rm_fst(I,L)).
check(error)  -> \fbcode{[........]};\hfill% \emph{Atom as a pattern}
check(Result) -> \fbcode{[........]}.
\end{alltt}
In order to develop further our programming skills, let us imagine
that a tail form definition of \erlcode{rm\_fst/2} is needed. The best
is to start from the version not in tail form and see where it fails
to be in tail form. Only one body is not in tail form: the last,
precisely because we use the control context to store the items which
are not the one we are looking for. We hence need to add an
accumulative parameter \erlcode{A} to the definition, which becomes
\erlcode{rm\_fst/3}, and we define a new function
\erlcode{rm\_fst\_tf/2} which must be, in the end, equivalent to
\erlcode{rm\_fst/2} but in tail form (we frame the code to be
reconsidered and in bold is the change):
\begin{alltt}
rm_fst_tf(I,L)    -> rm_fst(I,L,\fbcode{[]}).
rm_fst(_,   [],\textbf{A}) -> \fbox{[]};
rm_fst(I,[I|L],\textbf{A}) -> \fbox{L};
rm_fst(I,[J|L],\textbf{A}) -> \fbox{[J|rm_fst_tf(I,L,\textbf{A})]}.
\end{alltt}
We learnt from the transformation from \erlcode{join/2} to
\erlcode{join\_tf/2}, \vpagerefrange{trans_join}{code:join_tf}, that
we can push the variable in the control context, that is, \erlcode{J},
onto the accumulator \erlcode{A}:
\begin{alltt}
rm_fst_tf(I,L)    -> rm_fst(I,L,\fbcode{[]}).
rm_fst(_,   [],A) -> \fbox{[]};
rm_fst(I,[I|L],A) -> \fbox{L};
rm_fst(I,[J|L],A) -> rm_fst_tf(I,L,\textbf{[J|A]}).
\end{alltt}
Now \erlcode{J} is pushed on \erlcode{A}, not on the call anymore,
therefore the order of the items is reversed: the second argument is
reversed on the accumulator until either the end is reached or
\erlcode{I} is found. If the latter (second clause of
\erlcode{rm\_fst/3}), we cannot just return \erlcode{L} as we did
because there is no control context to provide the past items anymore:
we need to use the accumulator. We already know that \erlcode{A}
contains the previous items in reverse order and \erlcode{L} contains
the remaining, unvisited, items. So we can try the following:
\begin{alltt}
rm_fst_tf(I,L)    -> rm_fst(I,L,\fbcode{[]}).
rm_fst(_,   [],A) -> \fbox{[]};
rm_fst(I,[I|L],A) -> \textbf{join(rev(A),L)};
rm_fst(I,[J|L],A) -> rm_fst_tf(I,L,[J|A]).
\end{alltt}
At this point, we should hear a bell ringing because this body reminds
us of \erlcode{srev/1} defined as \verbatiminput{srev.def} and about
which we determined that its delay was asymptotically quadratic; in
other words, it is slow. The reason of its slowness was due to the
repeated calls to \erlcode{join/2} with the length of the first
argument ranging from \(n-1\) to \(0\), where \(n\) is the length of
the input list. This deficiency was remedied by using an auxiliary
function \erlcode{rev\_join/2} with an accumulator:
\verbatiminput{rev.def} A look back at \fig~\vref{fig:rev_321ast}
suggests that if the input list has length \(n\), then the first item
is pushed \(n-1\) times in total, the second item \(n-2\) etc. until
the last item is unmoved. So the problem boiled down to the
reprocessing the same data over and over again, not to the function
\erlcode{join/2} being slow or evil in itself.

\medskip

\paragraph{Improvements.}

Then what should we do to improve on this slow scheme? The study of
the previous definition gives the key. The purpose is to reverse a
list (\erlcode{A}) on top of another (\erlcode{L}). This is what
\erlcode{rev\_join/2} \emph{efficiently} does:
\begin{alltt}
rm_fst_tf(I,L)    -> rm_fst(I,L,\fbcode{[]}).
rm_fst(_,   [],A) -> \fbox{[]};
rm_fst(I,[I|L],A) -> \textbf{rev_join(A,L)};
rm_fst(I,[J|L],A) -> rm_fst_tf(I,L,[J|A]).
\end{alltt}
It is incorrect to write \erlcode{rev\_join(L,A)} in stead of
\erlcode{rev\_join(A,L)}. We can now finish the definition because we
know exactly the contents of the accumulator: it contains a reversed
prefix of the list up to the current situation. In the case of the
second clause, the situation in question is that the end of the
original list has been reached (\erlcode{[]}). In other words,
\erlcode{A} contains the complete original list in reverse
order. Hence, the complete definition is
\begin{alltt}
rm_fst_tf(I,L)    -> rm_fst(I,L,\fbcode{[]}).
rm_fst(_,   [],A) -> \textbf{rev(A)};
rm_fst(I,[I|L],A) -> rev_join(A,L);
rm_fst(I,[J|L],A) -> rm_fst_tf(I,L,[J|A]).
\end{alltt}
The work is not done yet, as the initial value of the accumulator has
to be determined. (As a side note, this example shows that a program
is not compulsorily best written from top to bottom. After all,
programs are not English essays and programming is perhaps more akin
to painting.) Since it is a list in which another is being reversed to
be later reversed again, it is important to start with an empty list,
otherwise extraneous initial items would be found in the final value:
\begin{alltt}
rm_fst_tf(I,L)    -> rm_fst(I,L,\textbf{[]}).
rm_fst(_,   [],A) -> rev(A);
rm_fst(I,[I|L],A) -> rev_join(A,L);
rm_fst(I,[J|L],A) -> rm_fst_tf(I,L,[J|A]).
\end{alltt}
Compiling this definition, which needs first to be embedded in a
module, displays an error:
\begin{center}
\texttt{function rm\_fst\_tf/3 undefined}
\end{center}
in the last clause. Indeed, it should be \erlcode{rm\_fst}
instead of \erlcode{rm\_fst\_tf}. So the correct tail form definition
is\label{code:rm_fst_tf}
\begin{alltt}
rm_fst_tf(I,L)    -> rm_fst(I,L,[]).
rm_fst(_,   [],A) -> rev(A);
rm_fst(I,[I|L],A) -> rev_join(A,L);
rm_fst(I,[J|L],A) -> \textbf{rm_fst}(I,L,[J|A]).\hfill% \emph{Fixed}
\end{alltt}
More precisely, \erlcode{rm\_fst\_tf/2} is in tail form because
the definition of a function is in tail form if, and only if,
\begin{enumerate}

  \item the bodies of its definition are in tail form. A body is in
    tail form if it is a constant (either a variable, an integer, an
    atom or a list made of these or empty) or a function call whose
    arguments are either constants or arithmetic expressions.

  \item the functions called in these bodies are in tail form too.

\end{enumerate}
The definition of ``being in tail form'' is syntactical in nature, that is,
it depends only on the shape of the code, not on what it computes, and
it is recursive. The definition of the function itself does not need
to be recursive, since, for example, \erlcode{id(X) -> X.}  is a
definition in tail form which is not recursive. For example, if
\erlcode{rev/1} were not in tail form, then \erlcode{rm\_fst/3} would
not be in tail form as well, despite having the right shape (no
control contexts for the calls) and so would not
\erlcode{rm\_fst\_tf/2} be in tail form.

\medskip

\paragraph{Membership test.}

Before analysing the performance of \erlcode{rm\_fst/2} and
\erlcode{rm\_fst\_tf/2}, let us backtrack and find a definition to the
restive \erlcode{mem/2}. The key is to use some atom to distinguish
the case when the sought item is present and when it is absent. Let us
use two atoms, one for each case:
\begin{alltt}
mem(_,   []) -> \textbf{false};
mem(I,[I|_]) -> \textbf{true};
mem(I,[_|L]) -> mem(I,L).
\end{alltt}
Note that the choice of the atom, just like a variable, is important
for the legibility of the program. It is important not to mislead the
caller by writing, for instance
\begin{alltt}
mem(_,   []) -> \textbf{true};
mem(I,[I|_]) -> \textbf{false};
mem(I,[_|L]) -> mem(I,L).
\end{alltt}
despite this being a valid definition. Anyhow, whatever the atoms, the
specification and the comments should state their interpretation
explicitly. Here, we would expect something like: ``The function
\erlcode{mem/2} determines the membership of an item to a list. The
function call \erlcode{mem(\(I\),\(L\))} results in the atom
\erlcode{true} if, and only if, item \(I\) is an element of list
\(L\), otherwise the result is the atom \erlcode{false}. If \(L\) is
not a list, the call fails immediately due to a match failure.'' We
insist in using slanted letters, for instance, \(I\) or \(L\), instead of
\erlcode{I} or \erlcode{L}, because \(I\), for example, denotes
\emph{any} \Erlang value whatsoever, while \erlcode{I} denotes any
\Erlang value \emph{in a specific clause}. Technically, \(I\) and
\(L\) are called \emph{metavariables} because they denote \Erlang
expressions independently of any particular definition, as opposed to
variables which denote a value in a particular clause of a particular
definition.

On a last note, we should underline that the multiple occurrences of
\erlcode{I} in the definition of \erlcode{mem/2} implies an equality
test at run\hyp{}time that can be slow if \erlcode{I} denotes a large
value. However, for the sake of simplicity, we will not take this kind
of implicit computational overhead into account when assessing the
speed of a function. As a rough approximation of the wall\hyp{}clock
time, we shall only count the number of functions calls needed to
evaluate a given function call, what we call its \emph{delay}. As an
extreme example demonstrating that this is not the whole story, mind
\begin{alltt}
tautology(X) -> eq(X,X).
eq(X,X) -> true;
eq(_,_) -> false.
\end{alltt}

\smallskip

\paragraph{Delays.}

Let us assess the delay of \erlcode{rm\_fst/2} and
\erlcode{rm\_fst\_tf/2}. Contrary to previous functions up to now, the
delay of \erlcode{rm\_fst(\(I\),\(L\))} depends on the position of the
first occurrence of item \(I\) in the list \(L\), of length \(n\),
because of clause \clause{\beta} in
\begin{alltt}
rm_fst(_,   []) \(\smashedrightarrow{\alpha}\) [];\hfill% \emph{Absent}
rm_fst(I,[I|L]) \(\smashedrightarrow{\beta}\) L;\hfill% \emph{Present}
rm_fst(I,[J|L]) \(\smashedrightarrow{\gamma}\) [J|rm_fst(I,L)].\hfill% \emph{Search further}
\end{alltt}
More precisely, there are three cases:
\begin{itemize}

  \item \(L\) is empty;

  \item \(L\) is not empty and \(I\) is absent;

  \item \(L\) is not empty and \(I\) occurs at position
    \(k\), counting the head of \(L\) as position \(0\).

\end{itemize}
In the first case, the delay is \(1\), by clause
\clause{\alpha}. Otherwise, the recursive call in clause
\clause{\gamma} is repeated \(n\) times until either clause
\clause{\alpha} or \clause{\beta} applies. If the item is absent, then
\clause{\alpha} applies and the delay is \(n+1\). Otherwise, it is
\(k+1\). All these execution traces can be formally summarised as
\(\gamma^n\alpha+\gamma^k\beta\), where the addition signifies a
disjunction, that is, in operational terms, a choice. (Composition of
clauses do not commute, for instance, \(\gamma\alpha \neq \alpha\gamma\), but
disjunction does, for instance, \(\alpha + \beta = \beta + \alpha\).) For
example, the delay of \erlcode{rm\_fst(1,[4,2,1,7,1])} is \(3\)
because clause \clause{\gamma} is used \(2\) times and, finally,
clause \clause{\beta} once, which corresponds to the execution trace
\(\gamma^2\beta\). If we interpret \(k=n\) as meaning that \(I\) is
absent from a non\hyp{}empty \(L\), then the delay is always \(k+1\),
which is concisely expressed by \(\abs{\gamma^k(\alpha+\beta)} = k
\cdot \abs{\gamma} + \abs{\alpha + \beta} = k +1\). The length of a
disjunction of clauses, for instance, \(\abs{\alpha + \beta}\), is defined
only if the disjoint sub\hyp{}traces have the same length, here,
\(\abs{\alpha} = \abs{\beta} = 1\). We cannot extend the formula
\(k+1\) to be valid when \(L\) is empty, as it would require \(k=0\)
and this case means that \(I\) is the head of \(L\). If we note
\(\comp{rm\_fst}{n,k}\) the delay for computing the call
\erlcode{rm\_fst(\(I\),\(L\))}, where \(L\) contains \(n\) items and
\(k\) is the position at which the first occurrence of \(I\) occurs in
\(L\), then\label{comp_rm_fst}
\[
\comp{rm\_fst}{n,k} = k + 1,\,\; \text{with} \,\; 0 \leqslant k
\leqslant n,
\]
with the convention that \(k=n\) if it is missing. If the list is
empty, that is, \(n=0\), then the inequalities imply that \(k=0\), but
this is not interpreted as \(I\) being the head of \(L\).

It is now easy to deduce what are the best and worst cases for the
delay, that is, for a given size \(n\) of the input, what are the
configurations that lead to a minimum or maximum number of rewrites to
reach the result, that is, a value.

Let us note \(\best{rm\_fst}{n}\) the delay in the best case and
\(\worst{rm\_fst}{n}\) the delay in the worst case. The above formulas
define them immediately. The minimum value of \(\comp{rm\_fst}{n,k}\)
is when \(k=0\) (keep in mind that \(n\) is fixed in this analysis),
that is, the item we sought is the head of the list. If the list is
empty, that is, \(n=0\), then there is only one case and the delay is
\(1\). The worst case happens when \(k=n\), that is, when the item is
absent from the list. For \(n \geqslant 0\),
\begin{align*}
\best{rm\_fst}{n}  &:= 
  \min_{0 \leqslant k \leqslant n}{\comp{rm\_fst}{n,k}} = 1,\\
\worst{rm\_fst}{n} &:=
  \max_{0 \leqslant k \leqslant n}{\comp{rm\_fst}{n,k}} = n + 1
\mathrel{\sim} n,\,\; \text{as} \,\; n \rightarrow \infty.
\end{align*}
The best and worst delays provide bounds for the delay, but more can
be said. The \emph{average delay} is the arithmetic mean of the delays
on all possible inputs. Usually, the number of inputs is infinite so
some additional assumptions are usually necessary, depending on the
problem at hand. In the case of \erlcode{rm\_fst/2}, the only thing
that matters is the position of the item to be removed and it does not
suppose anything about the other items, for instance, their relative order or
value. The problem is thus equivalent to being given a list \(L\) and
considering all its items for removal. This leads to define the
average delay as
\[
\ave{rm\_fst}{n}
  :=
    \frac{1}{n}\sum_{I \in L}\delay{rm\_fst(\(I\),\(L\))}.
\]
Consider that the average delay is not the measure of the delay of a
single operation because it may be that no configuration of the input
leads to the average delay. It may also not be the most probable delay
of a single operation, because this depends on the distribution of all
the delays. Let us resume our calculation. Since an item in a list is
uniquely referred to by its position, it is equivalent to write
\begin{equation}
\ave{rm\_fst}{n}
  = \frac{1}{n}\sum_{k=0}^{n-1}{\comp{rm\_fst}{n,k}}
  = \frac{1}{n}\sum_{k=0}^{n-1}{(k+1)}
  = \frac{1}{n}\sum_{k=1}^{n}{k}
  = \frac{n+1}{2}.\label{ave_rm_fst}
\end{equation}
We cannot compare \(\ave{rm\_fst}{n}\) and \(\worst{rm\_fst}{n}\)
because the latter assumed that the item is absent, whilst the former
assumed the contrary. The worst case when the item is present happens
when the item is last, that is, \(k=n-1\), the delay is then
\(
\worst{rm\_fst}{n} = \comp{rm\_fst}{n,n-1} = (n-1) + 1 = n.
\)
Therefore,
\begin{equation}
\ave{rm\_fst}{n} \mathrel{\sim} \tfrac{1}{2}\worst{rm\_fst}{n},\,\;
\text{as} \,\; n \rightarrow \infty.\label{eq:ave_rm_fst}
\end{equation}
In other words, the asymptotic average delay is midway between the
best and worst delays (the best delay is \(1\)).

For the sake of versatility, let us prove that \(\sum_{k=1}^{n}{k} =
n(n+1)/2\) in yet another way, by telescoping successive terms of the
series \((n^2)_{n>0}\), that is, summing the differences \((n+1)^2 -
n^2\). The key is that these differences have a simple, closed
expression: \((n+1)^2 - n^2 = 2n + 1\).\label{tele_squares}
\begin{align*}
  && (1+1)^2 - \boxed{1^2} &= 2^2 - 1^2 = 2 \cdot 1 + 1\\
+ && (2+1)^2 - 2^2         &= 3^2 - 2^2 = 2 \cdot 2 + 1\\
+ &&&\;\;\vdots\\
+ && \boxed{(n+1)^2} - n^2 &= 2 \cdot n + 1
\intertext{\rule{\linewidth}{0.4pt}}
\Rightarrow 
  &&\boxed{(n+1)^2} - \boxed{1^2}
  &= 2 \sum_{k=1}^{n}{k} + n.
\end{align*}
Finally: \(\sum_{k=1}^{n}{k} = n(n+1)/2\). Let us consider now the
delay for calls \erlcode{rm\_fst\_tf(\(I\),\(L\))}, with
\begin{alltt}
rm_fst_tf(I,L)    \(\smashedrightarrow{\alpha}\) rm_fst(I,L,[]).
rm_fst(_,   [],A) \(\smashedrightarrow{\beta}\) rev(A);
rm_fst(I,[I|L],A) \(\smashedrightarrow{\gamma}\) rev_join(A,L);
rm_fst(I,[J|L],A) \(\smashedrightarrow{\delta}\) rm_fst(I,L,[J|A]).
\end{alltt}
Just as before, there are three cases:
\begin{itemize}

  \item \(L\) is empty;

  \item \(L\) is not empty and \(I\) is absent;

  \item \(L\) is not empty and \(I\) occurs at position
    \(k\), counting the head of \(L\) as position \(0\).

\end{itemize}
In the first case, clause \clause{\alpha} is used once, then clause
\clause{\beta} once and finally we have to add the delay of the call
\erlcode{rev([])}, which is \(2\). The total delay is thus \(4\).

Let us now assume that \(L\) contains \(n > 0\) items and that the
item \(I\) we are looking for is missing from \(L\). Then the call
\erlcode{rm\_fst\_tf(\(I\),\(L\))} leads to
\begin{itemize}

  \item clause \clause{\alpha} being used once,

  \item clause \clause{\delta} being used \(n\) times,

  \item clause \clause{\beta} being used once,

  \item clause \clause{\gamma} being unused.

\end{itemize}
This execution trace is formally \(\alpha\delta^n\beta\), but it is
partial as we must take into account the delays of the calls other
than to \erlcode{rm\_fst/3}: only \erlcode{rev(A)} remains and we know
that its delay is always the length of \erlcode{A} plus \(2\) (on page
\pageref{delay:rev_join}). What is the length of the accumulator in
the body of clause \clause{\beta}? The body of clause \clause{\delta}
gives the answer: the accumulator contains all the items from the
input list but in reverse order, that is, its length is also
\(n\). Therefore the total delay is \((1)+(n)+(1+(n+2)) = 2n + 4\).

Now let us suppose that \(I\) does occur in \(L\) at position \(k\),
counting the head of the list as being at position \(0\). Then
\begin{itemize}

  \item clause \clause{\alpha} is used once;

  \item clause \clause{\delta} is used \(k\) times, from positions
    \(0\) to \(k-1\);

  \item clause \clause{\beta} is unused;

  \item clause \clause{\gamma} is used once.

\end{itemize}
Formally, this is the partial execution trace
\(\alpha\delta^k\gamma\), which must be completed by taking into
account the call \erlcode{rev\_join(A,L)}. We know from equation
\eqref{delay:rev_join} \vpageref{delay:rev_join} that
\(\comp{rev\_join}{n} = n + 1\), where \(n\) is the length of the
first argument. In clause \clause{\gamma} this argument is \erlcode{A}
and we know that it contains the first \(k\) items of \(L\), because
clause \clause{\delta} was used \(k\) times. Therefore, we have to add
\(k+1\) steps to the previous count, so the total delay is \(1 + k + 1
+ (k+1) = 2k + 3\), with \(k < n\).

We may wonder why there is a difference in the additive constants of
the two last cases: \(2n+4\) versus \(2k+3\). Looking again at our
calculations reveals that the difference comes from calling
\erlcode{rev(A)} in clause \clause{\beta}, which in turn calls
\erlcode{rev\_join(A,[])}: this adds one more rewrite, whereas we
could call \erlcode{rev\_join/2} directly, just like in clause
\clause{\gamma}. It is in fact equivalent to define
\erlcode{rm\_fst\_tf/2} as follows (change in bold):
\begin{alltt}
rm_fst_tf(I,L)    \(\smashedrightarrow{\alpha}\) rm_fst(I,L,[]).
rm_fst(_,   [],A) \(\smashedrightarrow{\beta}\) \textbf{rev\_join(A,[])};
rm_fst(I,[I|L],A) \(\smashedrightarrow{\gamma}\) rev_join(A,L);
rm_fst(I,[J|L],A) \(\smashedrightarrow{\delta}\) rm_fst(I,L,[J|A]).
\end{alltt}
Now the delay when the item is absent becomes \(2n+3\), mirroring
nicely the delay when it first occurs at position \(k\), that is,
\(2k+3\). Note that making the previous change in order to save one
function call is not something useful in general. We did it because it
helped evince that there is a common concept shining through the
different cases: if we take the convention that \(k=n\) means that
\(I\) is not in \(L\), then the delay is \(2k+3\) if \(L\) is not
empty. If it is empty, the delay is \(1+1+(0+1)=3\), which would be
obtained by taking \(k=0\) in \(2k+3\), but this would normally be
interpreted as meaning that \(I\) is the first item of \(L\), that is,
the head of \(L\). Hence the empty list seems to be a special case. If
we really want to have one single formula for the delay, we can take
the additional convention that \(k=-1\) means that the list is
empty. This leads to a delay of \(2 \cdot (-1)+3=1\). Can we modify
the definition so that becomes true without changing the other cases?
Yes, by adding a clause:
\begin{alltt}
\textbf{rm_fst_tf(I,[])   -> [];}
rm_fst_tf(I, L)   -> rm_fst(I,L,[]).
rm_fst(_,   [],A) -> rev_join(A,[]);
rm_fst(I,[I|L],A) -> rev_join(A,L);
rm_fst(I,[J|L],A) -> rm_fst(I,L,[J|A]).
\end{alltt}
Again, this last modification is only made with a didactic purpose in
mind and it may not be worthwhile in general, but it does allow one to
roundly say that
\[
\comp{rm\_fst\_tf}{n,k} = 2k+3,\,\; \text{where} \,\; -1 \leqslant k
\leqslant n,
\]
\(k\) being the position of \(I\) in \(L\), whose length is \(n\);
\(k=-1\) means that \(L\) is empty, so \(n=0\); \(k=n\) means that
\(I\) is absent from a non\hyp{}empty \(L\).

We can now deduce what are the best and worst cases, by respectively
minimising and maximising the two previous delays when \(k\)
varies---recall that the best and worst cases are defined for a given
\(n\). The best case is when \(k=0\) and the worst is when
\(k=n\). Thus, the best case happens when \(I\) is the head of \(L\)
and the worst case is when \(I\) is absent from \(L\). If the list is
empty, that is, \(k=-1\), it means that \(L\) is empty, that is, \(n=0\),
and, thus, there is only one case which is both the best and the
worst. If we note \(\worst{rm\_fst\_tf}{n}\) the delay in the worst
case and \(\best{rm\_fst\_tf}{n}\) the delay in the best case, we
have, for \(n > 0\),
\begin{gather*}
\comp{rm\_fst\_tf}{0,-1} = \best{rm\_fst\_tf}{0} =
\worst{rm\_fst\_tf}{0} = 1,\\
\best{rm\_fst\_tf}{n}  = 3,\quad \worst{rm\_fst\_tf}{n} = 2n+3
\mathrel{\sim} 2n,\,\; \text{as} \,\; n \rightarrow \infty.
\end{gather*}
How does \erlcode{rm\_fst/2} and \erlcode{rm\_fst\_tf/2} compare in
terms of delays? When comparing delays, we must be cautious because,
while it is always mathematically possible to relate real numbers,
doing so is not necessarily meaningful. For instance, let us suppose
that we have two functions \erlcode{f/1} and \erlcode{g/1} and we
determined their exact delays \(\comp{f}{n}\) and \(\comp{g}{n}\). If
both functions are equivalent, that is, if for all the same inputs
they compute the same outputs, it is meaningful to compare their exact
delays and also their asymptotic delays. If their delays are only
known in some worst case, it is important to make sure that the worst
case of \erlcode{f/1} is the worst case of \erlcode{g/1} as well,
otherwise the comparison is meaningless. For instance, it may be
possible that the best case of \erlcode{f/1} is the worst case of
\erlcode{g/1}. The same applies to best cases. Here, we have the exact
delays of \erlcode{rm\_fst/2} and \erlcode{rm\_fst\_tf/2}. We simply
deduce that
\begin{align*}
  \comp{rm\_fst}{0,-1} &= \comp{rm\_fst\_tf}{0,-1},
& \comp{rm\_fst}{n,k} &< \comp{rm\_fst\_tf}{n,k},\,\; \text{for} \,\;
  n > 0.
\end{align*}
Therefore, \(\comp{rm\_fst}{n,k} \leqslant \comp{rm\_fst\_tf}{n,k}\),
for all \(n \geqslant 0\). More precisely, we can study the asymptotic
behaviour of
\[
\comp{rm\_fst\_tf}{n,k}/\comp{rm\_fst}{n,k} = (2k+3)/(k+1)
\mathrel{\sim} 2,\,\; \text{as} \,\; k \rightarrow \infty.
\]
In other words:
\(
\comp{rm\_fst\_tf}{n,k} \mathrel{\sim} 2 \cdot \comp{rm\_fst}{n,k},
\,\; \text{as} \,\; k \rightarrow \infty \,\; \text{and} \,\; k <
n.
\)
Asymptotically, \erlcode{rm\_fst/2} is thus twice as fast as
\erlcode{rm\_fst\_tf/2}, although it needs to store in the call stack
\(n\) items, whilst \erlcode{rm\_fst\_tf/2}, being in tail form, uses
a small constant amount of call stack---just one call, actually.

Assuming now that some item \(I\) is in the list \(L\), what is the
average delay of \erlcode{rm\_fst\_tf(\(I\),\(L\))}? We already expect
this to be the arithmetic mean of the delays for all positions. Let us
denote \(\ave{rm\_fst\_tf}{n}\) this quantity, which is not an integer
in general. Formally, it is defined as
\[
\ave{rm\_fst\_tf}{n}
  = \frac{1}{n}\sum_{k=0}^{n-1}{\comp{rm\_fst\_tf}{n,k}}
  = \frac{1}{n}\sum_{k=0}^{n-1}{(2k+3)} = n + 2.
\]
When the item is present in the list, the worst case happens when it
is the last, that is, \(k=n-1\), so the delay is then
\[
\worst{rm\_fst\_tf}{n} 
  = \comp{rm\_fst\_tf}{n,n-1} = 2(n-1)+3 = 2n + 1.
\]
Therefore, \(\ave{rm\_fst\_tf}{n} \mathrel{\sim} \tfrac{1}{2}
\worst{rm\_fst\_tf}{n}\), as \(n \rightarrow \infty\). This is the
same situation as with \(\ave{rm\_fst}{n}\).

\medskip

\paragraph{Alternate definition.}

Let us envisage an alternative definition of
\erlcode{rm\_fst/2}:\label{rm_fst_alt}
\begin{alltt}
rm_fst_alt(I,L)     \(\smashedrightarrow{\alpha}\) rm_fst__(I,L,[]).
rm_fst__(_,   [],A) \(\smashedrightarrow{\beta}\) \textbf{A};
rm_fst__(I,[I|L],A) \(\smashedrightarrow{\gamma}\) \textbf{join(A,L)};
rm_fst__(I,[J|L],A) \(\smashedrightarrow{\delta}\) rm_fst__(I,L,\textbf{join(A,[J])}).
\end{alltt}
The expressions in bold in clauses \clause{\beta}, \clause{\gamma} and
\clause{\delta} differ from the definition of \erlcode{rm\_fst/3}. The
rationale behind its design is to consider the accumulator \erlcode{A}
as a prefix of the original list, not as a reversed prefix as we did
before. That is why, instead of pushing \erlcode{J} on \erlcode{A} in
clause \(\delta\), \erlcode{join/2} is called so that \erlcode{[J]} is
appended \emph{at the end} of \erlcode{A}. Consequently, the result is
\erlcode{A}, instead of \erlcode{rev(A)}, in clause \(\beta\) and
\erlcode{join(A,L)}, instead of \erlcode{rev\_join(A,L)}, in clause
\(\gamma\). This solution is appealing on two grounds. Firstly, the
order of the items is not reversed, which makes it easier to bear them
in mind. Secondly, in the worst case, we avoid reversing two times the
items as we did in \erlcode{rm\_fst/3}: the items of the original
input were reversed into the accumulator and then the accumulator is
reversed to make the final result. Let us compute the delays
\(\comp{rm\_fst\_alt}{n,k} :=
\delay{rm\_fst\_alt(\(I\),\(L\))}\), where \(n\) is the length of
\(L\) and \(k\) is the location where \(I\) occurs in \(L\) (\(k = n\)
if absent), and \(\comp{rm\_fst\_\_}{n,m} :=
\delay{rm\_fst(\(I\),\(P\),\(Q\))}\), where \(n\) is the length of
\(P\) and \(m\) the length of \(Q\). There are three cases:
\begin{itemize}

  \item \(L\) is empty;

  \item \(L\) is not empty and \(I\) is absent;

  \item \(L\) is not empty and \(I\) occurs at position
    \(k\), counting the head of \(L\) as position \(0\).

\end{itemize}
In the first case, the execution trace is \(\alpha\beta\): the delay
is \(\abs{\alpha\beta} = 2\). Let us now suppose that \(I\) is absent
from \(L\), that is, \(k = n\).
\begin{itemize}

  \item clause \clause{\alpha} is used once;

  \item clause \clause{\delta} is used \(n\) times;

  \item clause \clause{\beta} is used once;

  \item clause \clause{\gamma} is unused
.
\end{itemize}
This execution trace is \(\alpha\delta^n\beta\), which must be
completed with the call \erlcode{join(A,L)}. We know that
\(\comp{join}{q} = q + 1\), where \(q\) is the length of \erlcode{A},
from equation \eqref{delay:join} \vpageref{delay:join}. Assuming that
\erlcode{L} contains \(p\) items, all this translates into the
following recurrent equations, where \(p,q \geqslant 0\).
\begin{align*}
\comp{rm\_fst\_alt}{p,n}  &\eqn{\alpha} 1 + \comp{rm\_fst\_\_}{p,0},
\qquad \comp{rm\_fst\_\_}{0,q} \eqn{\beta} 1,\\
\comp{rm\_fst\_\_}{p+1,q} &\eqn{\delta} 1 + \comp{join}{q} +
                             \comp{rm\_fst\_\_}{p,q+1}
                          = 1 + (q+1) + \comp{rm\_fst\_\_}{p,q+1}.
\end{align*}
Note that we only used the information about whether a clause is used
or not. Now we are going to use the fact that clause \clause{\delta}
is used \(n\) times. Indeed, these recurrence equations should remind
us of equations \eqref{eq:rev_join_p_q} of \erlcode{rev\_join/2}
\vpageref{eq:rev_join_p_q}, albeit the slight difference has important
consequences. We can use the same technique, called
\emph{telescoping}, on the equation drawn from clause \clause{\delta},
\(n\) times:
\begin{align*}
  && \boxed{\comp{rm\_fst\_\_}{n,m}} - \comp{rm\_fst\_\_}{n-1,m+1} &\eqn{\delta}
  2 + (m+0) = m + 2\\
+ && \comp{rm\_fst\_\_}{n-1,m+1} - \comp{rm\_fst\_\_}{n-2,m+2}
&\eqn{\delta} 2 + (m+1) = m + 3\\
+ &&&\;\;\vdots\\
+ &&\comp{rm\_fst\_\_}{1,m+(n-1)} - \boxed{\comp{rm\_fst\_\_}{0,m+n}} 
&\eqn{\delta} 2 + (m+(n-1)) = m + (n + 1)
\intertext{\rule{\linewidth}{0.4pt}}
\Rightarrow 
  &&\boxed{\comp{rm\_fst\_\_}{n,m}} - \boxed{\comp{rm\_fst\_\_}{0,m+n}}
  &= nm + \sum_{p=2}^{n+1}{p} = nm + \sum_{p=1}^{n}{(1+p)}\\
&&&= nm + \left(n + \sum_{p=1}^{n}{p}\right)\\
\Leftrightarrow &&\comp{rm\_fst\_\_}{n,m} - 1 &= nm + n + n(n+1)/2\\
\Leftrightarrow &&2 \cdot \comp{rm\_fst\_\_}{n,m}
  &= 2 + 2nm + 2n + n^2 + n\\
&&&= 2nm + n^2 + 3n + 2, \,\; \text{with} \;\, n > 0.
\intertext{Since that replacing \(n\) by \(0\) in the last formula
  leads to the expected \(\comp{rm\_fst\_\_}{0,m} \eqn{\beta} 1\), we
  can gather all the cases into just one formula:}
&&2 \cdot \comp{rm\_fst\_\_}{n,m} &= 2nm + n^2 + 3n + 2,\,\;
\text{where} \, n, m \geqslant 0.
\end{align*}
What we want finally is \(\comp{rm\_fst\_alt}{n,n}\) for \(n \geqslant 0\):
\begin{align*}
\comp{rm\_fst\_alt}{n,n} 
   &\eqn{\alpha} 1 + \comp{rm\_fst\_\_}{n,0} = 1 + (n^2 + 3n + 2)/2
   = (n^2 + 3n + 4)/2.\\
\comp{rm\_fst\_alt}{n,n}
   &\mathrel{\sim} \tfrac{1}{2} n^2,
\,\; \text{as} \;\, n \rightarrow \infty.
\end{align*}
The asymptotic delay of \erlcode{rm\_fst\_alt/2} is therefore
quadratic. Now let us suppose that \(I\) does occur in \(L\) at
position \(k\), counting the head of the list as being at position
\(0\). Then \(0 \leqslant k < n\) and
\begin{itemize}

  \item clause \clause{\alpha} is used once;

  \item clause \clause{\delta} is used \(k\) times;

  \item clause \clause{\gamma} is used once;

  \item clause \clause{\beta} is unused.

\end{itemize}
The execution trace is \(\alpha\delta^k\gamma\), to which must be
added the delays of the calls \erlcode{join(A,L)} and of the repeated
calls \erlcode{join(A,[J])}. We know that \(\comp{join}{q} = q + 1\),
where \(q\) is the length of \erlcode{A}. In clause \clause{\gamma},
this argument is \erlcode{A} and we know that it contains the first
\(k\) items of \(L\), because clause \clause{\delta} was used \(k\)
times. Assuming that \erlcode{L} contains \(p\) items, all this
translates into the following recurrent equations, where \(p,q
\geqslant 0\) and \(k\) is fixed.\label{delay:rm_fst_alt}
\begin{align*}
\comp{rm\_fst\_alt}{p,k}  &\eqn{\alpha} 1 + \comp{rm\_fst\_\_}{p,0},
\qquad
\comp{rm\_fst\_\_}{n-k,q} \eqn{\gamma} 1 + \comp{join}{q} = 2 + q,\\
\comp{rm\_fst\_\_}{p+1,q} &\eqn{\delta} 1 + \comp{join}{q} +
                             \comp{rm\_fst\_\_}{p,q+1}
                          = 2 + q + \comp{rm\_fst\_\_}{p,q+1}.
\end{align*}
Note that we only used the information about whether a clause is used
or not. Now we are going to use the fact that clause \clause{\delta}
is used \(k\) times, by telescoping the equation drawn from clause
\clause{\delta}, \(k\) times, with \(0 \leqslant k <
n\):
\begin{align*}
  && \boxed{\comp{rm\_fst\_\_}{n,m}} - \comp{rm\_fst\_\_}{n-1,m+1}
  &\eqn{\delta} 2 + (m+0) = m + 2\\
+ && \comp{rm\_fst\_\_}{n-1,m+1} - \comp{rm\_fst\_\_}{n-2,m+2}
&\eqn{\delta} 2 + (m+1) = m + 3\\
+ &&&\;\;\vdots\\
+ &&\comp{rm\_fst\_\_}{n-k+1,m+k-1} -
\boxed{\comp{rm\_fst\_\_}{n-k,m+k}} &\eqn{\delta} 2 + (m+(k-1)) = m +
(k + 1)
\intertext{\rule{\linewidth}{0.4pt}}
\Rightarrow 
  &&\boxed{\comp{rm\_fst\_\_}{n,m}} - \boxed{\comp{rm\_fst\_\_}{n-k,m+k}}
  &= km + \sum_{p=2}^{k+1}{p} = km + \sum_{p=1}^{k}{(1+p)}\\
&&&= km + \left(n + \sum_{p=1}^{k}{p}\right)\\
\Leftrightarrow &&\comp{rm\_fst\_\_}{n,m} - (2+(m+k)) &= km + k +
k(k+1)/2,\,\; \text{by} \,\; (\eqn{\gamma}),\\
\Leftrightarrow &&2 \cdot \comp{rm\_fst\_\_}{n,m}
  &= k^2 + (2m+5)k + 2m + 4.
\end{align*}
What we want finally is \(\comp{rm\_fst\_alt}{n,k}\):
\begin{align*}
\comp{rm\_fst\_alt}{n,k}
  &\eqn{\alpha} 1 + \comp{rm\_fst\_\_}{n,0}
   = 1 + (k^2 + 5k + 4)/2 = (k+2)(k+3)/2.\\
\comp{rm\_fst\_alt}{n,k} &\mathrel{\sim} \frac{1}{2}
k^2,\,\; \text{as} \;\, k \rightarrow \infty \,\; \text{and} \,\;
k < n.
\end{align*}
We can actually extend this asymptotic expression to the
  case \(k = n\), as the previous case analysis proved, so
\[
\comp{rm\_fst\_alt}{n,k} \mathrel{\sim} \tfrac{1}{2} k^2,\,\;
\text{as} \;\, k \rightarrow \infty \,\; \text{and} \,\; k
\leqslant n.
\]
In the worst case, \(k = n-1\), that is, the item \(I\) is at the end
of the list \(L\), so the delay is then \((n+1)(n+2)/2 = (n^2 + 3n +
2)/2\). Notice how this is exactly \(1\) less than the delay when
\(I\) is missing in \(L\), that is, \(\comp{rm\_fst\_alt}{n,n} = (n^2
+ 3n + 4)/2\). Therefore, the worst case happens when \(I\) is missing
from \(L\). There is no best case when \(I\) is absent from \(L\);
when present, it is \(k=0\), so the delay is \((0+2)(0+3)/2 =
3\). Thus the best case is when \(I\) is the head of \(L\). If the
list is empty, that is, \(n=0\), then there is only one case, which is
thus both the best and the worst, and the delay is \((0^2 + 3 \cdot 0
+ 4)/2 = 2\). For \(n > 0\),
\[
\best{rm\_fst\_alt}{0} = 2,\; \best{rm\_fst\_alt}{n} = 3,\;
\worst{rm\_fst\_alt}{n} = \frac{n^2 + 3n + 4}{2} \mathrel{\sim}
\frac{1}{2} n^2.
\]
This proves that the delay of \erlcode{rm\_fst\_alt/2} in the worst
case is asymptotically quadratic, as opposed to the linear behaviour
of \erlcode{rm\_fst/2}. Thus, despite the items being reversed twice
in the latter, the delay is much lower. Just like with
\erlcode{srev/1}, \vpageref{code:srev}, the root of the problem here
is that \erlcode{join/2} is moving the same items a number of times
which is not constant, actually proportional to their position in the
input list, and which, once compounded, yields a quadratic
term. Instead, when pushing the items to the accumulator and from it
to the result, we have moved each item exactly twice, which is a
(small) constant number of times, hence the linear term.

Assuming that the item \(I\) is in the list \(L\), what is the
expected delay of \erlcode{rm\_fst\_alt(\(I\),\(L\))}? This is the
arithmetic mean of the delays for all positions. Let us denote
\(\ave{rm\_fst\_alt}{n}\) this quantity, which is not an integer in
general. Mathematically, it is defined as follows:
\begin{align*}
\ave{rm\_fst\_alt}{n}
  &:= \frac{1}{n}\sum_{k=0}^{n-1}{\comp{rm\_fst\_alt}{n,k}}
   = \frac{1}{n}\sum_{k=0}^{n-1}{\frac{(k+2)(k+3)}{2}}\\
  &= \frac{1}{2n}\sum_{k=0}^{n-1}{(k^2 + 5k + 6)}
   = \frac{1}{2n}\left(\sum_{k=0}^{n-1}{k^2} + 5 \sum_{k=0}^{n-1}{k} +
     6n\!\right)\\
  &= \frac{1}{2n}\left(\sum_{k=0}^{n-1}{k^2} + 5 \cdot \frac{n(n-1)}{2} +
     6n\!\right).
\end{align*}
We need to find a closed expression of the sum of the first \(n\)
squares. We can use the telescoping technique as shown
\vpageref{tele_squares}, but on the series \((n^3)_{n>0}\). We start
with the equality \((n+1)^3 = n^3 + 3n^2 + 3n + 1\), hence \((n+1)^3 -
n^3 = 3n^2 + 3n + 1\). Then we can sum these differences:
\begin{align*}
  && (1+1)^3 - \boxed{1^3} &= 3 \cdot 1^2 + 3 \cdot 1 + 1\\
+ && (2+1)^3 - 2^3         &= 3 \cdot 2^2 + 3 \cdot 2 + 1\\
+ &&&\;\;\vdots\\
+ && \boxed{(n+1)^3} - n^3 &= 3n^2 + 3n + 1
\intertext{\rule{\linewidth}{0.4pt}}
\Rightarrow 
  &&\boxed{(n+1)^3} - \boxed{1^3}
  &= 3 \sum_{k=1}^{n}{k^2} + 3 \sum_{k=1}^{n}{k} + n\\
  && n^3 + 3n^2 + 3n
  &= 3 \sum_{k=1}^{n}{k^2} + 3 \cdot \frac{n(n+1)}{2} + n\\
\Leftrightarrow && \sum_{k=1}^{n}{k^2} &= \frac{n(n+1)(2n+1)}{6}.
\end{align*}
We can now resume or calculation of \(\ave{rm\_fst\_alt}{n}\):
\begin{align*}
\ave{rm\_fst\_alt}{n}
  &= \frac{1}{2n}\left(\frac{(n-1)n(2n-1)}{6} + 5 \cdot
     \frac{n(n-1)}{2} + 6n\!\right)\\
  &= \frac{1}{6}{n^2} + n + \frac{11}{6} \mathrel{\sim}
     \frac{1}{6}n^2,\,\; \text{as} \,\; n \rightarrow \infty.
\end{align*}
We found earlier that the worst case when the item is present is when
it is located at the end of the list and the delay is then
\[
\worst{rm\_fst\_alt}{n} = \tfrac{1}{2}{n^2} + \tfrac{3}{2}{n} + 1.
\]
Therefore, \(\ave{rm\_fst\_alt}{n} \mathrel{\sim} \tfrac{1}{3}
\worst{rm\_fst\_alt}{n}\), as \(n \rightarrow \infty\). That is to
say, asymptotically, the average delay is about 33\% the worst
delay. In other words, the worst does not happen very often.

\medskip

\paragraph{Last occurrence.}

Let us consider the definition of a function \erlcode{rm\_lst/2} such
that the call \erlcode{rm\_lst(\(I\),\(L\))} is rewritten to \(L\) if
\(I\) does not belong to the list \(L\), otherwise it is rewritten to
a list identical to \(L\) but without the last occurrence of
\(I\). This is our \emph{specification}. For instance, we expect the
following rewrites:\label{rm_lst_ex}
\begin{align*}
\erlcode{rm\_lst(3,[])}         & \twoheadrightarrow \erlcode{[]};\\
\erlcode{rm\_lst([],[])}        & \twoheadrightarrow \erlcode{[]};\\
\erlcode{rm\_lst([5,2],[3,[]])} & \twoheadrightarrow \erlcode{[3,[]]};\\
\erlcode{rm\_lst(4,[[],[1,2],4,[],4])}
 & \twoheadrightarrow \erlcode{[[],[1,2],4,[]]};\\
\erlcode{rm\_lst([],[4,[1,2],[],[],4])}
 & \twoheadrightarrow \erlcode{[4,[1,2],[],4]}.
\end{align*}
Let us go for a direct way to the solution. First of all, does this
problem reminds us of another? Filtering out the first occurrence
seems close. How then could we reuse \erlcode{rm\_fst/2} to solve our
current problem? The key is to see that, provided that the input list
is first reversed, applying \erlcode{rm\_fst/2} will filter out the
item which was the last occurrence in the original list, since the
last occurrence becomes the first due to the reversing. The last step
thus consists in reversing the result so the order of the items is
restored.\label{rm_lst}
\begin{verbatim}
rm_lst(I,L) -> rev(rm_fst(I,rev(L))).
\end{verbatim}
What is the delay of the calls to this function? Since we compose
three functions, the delays are simply added. Let us suppose that the
length of the input list \(L\) is \(n\). Let us compute the delay
\(\comp{rm\_lst}{n,k} := \delay{rm\_lst(\(I\),\(L\))}\), where
\(n\) is the length of \(L\) and \(k\) is the location where \(I\)
occurs in \(L\), \(k = n\) if absent. Since the first call is
\erlcode{rev(L)}, the definition of \erlcode{rm\_lst/2} leads to the
equation
\[
\comp{rm\_lst}{n,k} = 1 + \comp{rev}{?} + \comp{rm\_fst}{?,?} +
\comp{rev}{n}.
\]
We need to determine the length of the arguments to the composed
functions. Because the length of \erlcode{rev(\(L\))} is the same as
the length of \(L\), we know that the list argument of
\erlcode{rm\_fst/2} has length \(n\), so we have
\[
\comp{rm\_lst}{n,k} = 1 + \comp{rev}{?} + \comp{rm\_fst}{n,?} +
\comp{rev}{n}.
\]
Now we need to consider different cases. Indeed, depending on the
presence or absence of \(I\) in \(L\), the length of
\erlcode{rm\_fst(\(I\),rev(\(L\)))} is different: if present, it is
\(n-1\), otherwise it is \(n\). But there is another case to consider
first: the input list may be empty. In this case, \(n = k = 0\), but
this is not interpreted as \(I\) being the head of \(L\). We have
\[
\comp{rm\_lst}{0,0} = 1 + 2 + 1 + 2 = 6.
\]
Next, let us assume that \(I\) is absent, that is, \(k = n\). We found
\vpageref{comp_rm_fst}:
\[
\comp{rm\_fst}{n,k} = k + 1,\,\; \text{with} \,\; 0 \leqslant k
\leqslant n,
\]
where \(k\) is the position at which \(I\) occurs last in \(P\) in the
call \erlcode{rm\_fst(\(I\),\(P\))}, with the convention \(k=n\) if it
does not. Thus, \(\comp{rm\_fst}{n,n} = n + 1\) and
\begin{align*}
\comp{rm\_lst}{n,n} 
  &= 1 + \comp{rev}{n} + \comp{rm\_fst}{n,n} + \comp{rev}{n}\\
  &= 1 + (n+2) + (n+1) + (n+2)
   = 3n + 6,\,\; \text{with} \,\; n > 0.
\end{align*}
Because we just found that \(\comp{rm\_lst}{0,0} = 6\), we can extend
this formula to the case \(n=0\), that is, the empty list:
\[
\comp{rm\_lst}{n,n} = 3n + 6,\,\; \text{with} \,\; n \geqslant 0.
\]
Let us suppose now that \(I\) occurs last in \(L\) at position
\(k\). Then the length of \erlcode{rm\_fst(\(I\),rev(\(L\)))} is
\(n-1\). Furthermore, \(I\) occurs in \erlcode{rev(\(L\))} at position
\(n-k-1\), so \(\comp{rm\_fst}{n,n-k-1} = (n-k-1) + 1 = n-k\). We
deduce, for \(0 \leqslant k < n\),
\begin{align}
\comp{rm\_lst}{n,k} 
  &= 1 + \comp{rev}{n-1} + \comp{rm\_fst}{n,n-k-1} + \comp{rev}{n}\notag\\
  &= 1 + ((n-1)+2) + (n-k) + (n+2) = 3n - k + 4.\label{eq:rm_lst}
\end{align}
Let us note \(\best{rm\_lst}{n}\) and \(\worst{rm\_lst}{n}\) the
delays, respectively, in the best and worst cases. By definition, they
are, respectively, the minimum and the maximum delay, when \(n\) is
fixed and \(k\) varies, that is
\begin{align*}
\best{rm\_lst}{n}
  &:= \min_{0 \leqslant k \leqslant n}{\comp{rm\_lst}{n,k}}
   = \min_{0 \leqslant k < n}{\{3n+6, 3n-k+4\}}\\
  &= 3n - (n-1) + 4
   = 2n + 5 \mathrel{\sim} 2n,\,\; \text{as} \,\; n \rightarrow
     \infty;\\
\worst{rm\_lst}{n}
  &:= \max_{0 \leqslant k \leqslant n}{\comp{rm\_lst}{n,k}}
   = \max_{0 \leqslant k < n}{\{3n+6, 3n-k+4\}}\\
  &= 3n + 6 \mathrel{\sim} 3n,\,\; \text{as} \,\; n \rightarrow
     \infty.
\end{align*}
The best case is thus when \(k=n-1\), that is, when the item is the
last in the list, and the worst case happens when the item is
missing. Let us denote \(\ave{rm\_lst}{n}\) the average delay, which
is the arithmetic mean of the delays for all
positions. Mathematically, it is defined as follows:
\begin{align}
\ave{rm\_lst}{n}
  &= \frac{1}{n}\sum_{k=0}^{n-1}{\comp{rm\_lst}{n,k}}
   = \frac{1}{n}\sum_{k=0}^{n-1}{(3n-k+4)}
   = 3n+4 - \frac{1}{n}\sum_{k=0}^{n-1}{k}\notag\\
  &= 3n + 4 - \frac{n-1}{2}
   = \frac{5}{2}{n} + \frac{9}{2}
     \mathrel{\sim} \frac{5}{2}{n},\,\; \text{as} \,\;
     n \rightarrow \infty.\label{eq:ave_rm_lst}
\end{align}
When the item is present, the worst case is when it is the head:
\begin{equation*}
\worst{rm\_lst}{n}
  = \comp{rm\_lst}{n,0}
  = 3n+4 \mathrel{\sim} 3n,\,\; \text{as} \,\; n \rightarrow \infty.
\end{equation*}
This implies \(\ave{rm\_lst}{n} \mathrel{\sim} \tfrac{5}{6}
\worst{rm\_lst}{n}\), as \(n \rightarrow \infty\), which means that,
unfortunately, the average delay is asymptotically 83.3\% of the worst
delay. Contrast this with the average delay of \erlcode{rm\_fst/2},
equation \eqref{eq:ave_rm_fst} \vpageref{eq:ave_rm_fst}, which is only
50\% of the worst delay.

\medskip

\paragraph{Adding a sequential search.}

We found that the worst case of \erlcode{rm\_lst/2} happens when the
item is missing in the list of length \(n\) and the delay is then
\(3n+6\). This means that three complete traversals of the lists are
made, whilst only one would be enough to realise that the item is
indeed absent. Similarly, the best case is when the item occurs in the
last position in the list, in which case the delay is \(2n+5\), that
is, two complete traversals are performed, despite only one would be
necessary. This suggests that there may be some room for
improvement. Perhaps the first try consists in keeping the previous
algorithm while adding some shortcuts. For instance, we can delay the
composition of \erlcode{rev/1} and \erlcode{rm\_fst/2} until we
actually find an occurrence of the item; otherwise, we keep going
until the end of the list:
\begin{alltt}
rm_lst1(_,   []) \(\smashedrightarrow{\alpha}\) [];
rm_lst1(I,[I|L]) \(\smashedrightarrow{\beta}\) \textbf{rev(rm_fst(I,rev([I|L])));}
rm_lst1(I,[J|L]) \(\smashedrightarrow{\gamma}\) [J|rm_lst1(I,L)].
\end{alltt}
The part of the definition not in bold is the typical \emph{sequential
  search} we have seen at play in \erlcode{rm\_fst/2}: all the items
of a list are visited in turn until one of interest is
found. Importantly, the visited items are not discarded, but kept,
with the same relative order, in the control context of the last
clause. The body in bold is the action we want to perform in case we
find the item we sought. What is the delay of \erlcode{rm\_lst1/2}? As
usual in sequential searches, two cases are a priori relevant: either
the item is present or not.

Let us assume that the item occurs first at position \(f\) in the list
and last at position \(k\), counting the head as position
\(0\). Clause \clause{\gamma} is then used \(f\) times and clause
\clause{\beta} once. Then \erlcode{rev(rm\_fst(I,rev([I|L])))} is
rewritten into a value. The delay of reversing \erlcode{[I|L]} is
\((n-f)+2\). The last occurrence of \erlcode{I} in \erlcode{[I|L]} is
\(k-f\), so the first occurrence in \erlcode{rev([I|L])} is
\(((n-1)-f)-(k-f)=n-k-1\). We can check these expressions on a small
example in \fig~\vref{fig:rm_lst_ex}.
\begin{figure}
\centering
\includegraphics{rm_lst_ex}
\caption{Filtering out the last \textsf{b} in \textsf{eabdcbr}
\label{fig:rm_lst_ex}}
\end{figure}
We used \erlcode{e}, \erlcode{a}, \erlcode{b} etc. for the contents of
the list, instead of numbers, in order to avoid confusion with the
indexes. The item whose last occurrence is \(k\) is \erlcode{b}. We
already know that \(\comp{rm\_fst}{n,k} = k + 1\). Let
\(\comp{rm\_lst1}{n,f,k}\) be the delay when the item occurs first in
position \(f\) and last in position \(k\) in a list of length
\(n\). We have
\begin{align*}
\comp{rm\_lst1}{n,f,k} 
  &= f + 1 + \comp{rev}{n-f} + \comp{rm\_fst}{n-f,n-k-1}
     + \comp{rev}{n-f-1}\\
  &= f + 1 + ((n-f)+2) + ((n-k-1)+1)\\
  &\phantom{= f\;} + ((n-f-1)+2)
   = 3n - f - k + 4,\,\; \text{with} \,\; n > 0.
\end{align*}
We found earlier that the delay of \erlcode{rm\_lst/2}, when the item
was occurring last at position \(k\) in a list of length \(n\) was
\(3n - k + 4\). Thus, we have in this case \(\comp{rm\_lst1}{n,f,k} =
\comp{rm\_lst}{n,k} - f\). This was expected because the reduction in
delay due to \(f\) comes from the wrapping of a sequential search for
the item around \erlcode{rm\_lst/2}.

Let us suppose now that the item does not occur in the list. Then
clause \clause{\beta} is unused and we have a sequential search ending
in a failure, that is, \(\comp{rm\_lst1}{n,n,n} = n + 1\). (Using the
convention that an occurrence at position \(n\) means ``absent.'') If we
suppose now that the item does occur in the list, first at position
\(f\) and last at position \(k\), then, by definition, the best case
happens when \(\comp{rm\_lst1}{n,f,k}\) is minimum and the worst case
when it is maximum:
\begin{align*}
\best{rm\_lst1}{n}
  &:= \min_{0 \leqslant f \leqslant k \leqslant  n}{\comp{rm\_lst1}{n,f,k}}
   = \min_{0 \leqslant f \leqslant k < n}{\{n+1,3n - f - k + 4\}}\\
  &= \min\{n+1,3n-(n-1)-(n-1)+4\}
   = n+1.\\
\worst{rm\_lst1}{n}
  &:= \max_{0 \leqslant f \leqslant k \leqslant  n}{\comp{rm\_lst1}{n,f,k}}
   = \max_{0 \leqslant f \leqslant k < n}{\{n+1,3n - f - k + 4\}}\\
  &= \max\{n+1,3n - 0 - 0 + 4\}
   = 3n + 4.
\end{align*}
As a conclusion, the best case of \erlcode{rm\_lst1/2} happens when
the item is missing and the worst case when the item occurs
\emph{uniquely} as the head of the list (\(f=k=0\)). Notice that it is
not possible to compare these results with the delays of
\erlcode{rm\_lst/2} because the best and worst cases differ (the worst
case of \erlcode{rm\_lst/2} is the best case of \erlcode{rm\_lst1/2})
and
\[
\best{rm\_lst1}{n} < \best{rm\_lst}{n} < \worst{rm\_lst1}{n} <
\worst{rm\_lst}{n}.
\]
(It would have been possible to compare \erlcode{rm\_lst/2} and
\erlcode{rm\_lst1/2} in an absolute sense had the worst delay of one
been smaller than the best delay of the other.) Perhaps an analysis of
the average delay would enable a comparison.

Assuming that the item occurs at least once in the list, what is the
average delay of \erlcode{rm\_lst1/2}? The first occurrence position,
\(f\), ranges from \(0\) to \(n-1\), whilst the last occurrence
position \(k\) ranges from \(f\) to \(n-1\). The pairs \((f,k)\) can
be counted as follows: \((0,0) \dots (0,n-1)\), then \((1,1) \dots
(1,n-1)\), etc. until \((n-1,n-1)\). Thus the number of pairs with
\(f=0\) is \(n\), the number of pairs with \(f=1\) is \(n-1\) etc. up
to \(1\) when \(f=n-1\). This means that the total number of pairs
\((f,k)\) is \(1 + 2 + \dots + n = n(n+1)/2\). We can now lay out the
definition
\begin{align*}
\ave{rm\_lst1}{n}
  &:=
     \frac{2}{n(n+1)}\!\!\sum_{f=0}^{n-1}\sum_{k=f}^{n-1}{\comp{rm\_lst}{n,f,k}}
   = \frac{2}{n(n+1)}\!\!\sum_{f=0}^{n-1}\sum_{k=f}^{n-1}{(3n - f - k +
     4)}\\
  &= 3n + 4 - \frac{2}{n(n+1)}\sum_{f=0}^{n-1}\sum_{k=f}^{n-1}{(f+k)}\\
  &= 3n + 4
     - \frac{2}{n(n+1)}\sum_{f=0}^{n-1}\left(f(n-f) +
     \sum_{k=f}^{n-1}{k}\right)\\
  &= 3n + 4 - \frac{2}{n(n+1)}\sum_{f=0}^{n-1}\left(nf-f^2 +
     \frac{n(n-1)}{2} - \frac{f(f-1)}{2}\right)\\
  &= 3n + 4 - \frac{2}{n(n+1)} \cdot \frac{n^2(n-1)}{2}\\
  &\phantom{= 3}
     + \frac{2}{n(n+1)}\sum_{f=0}^{n-1}\left(\frac{3}{2}f^2
                               -\left(\frac{1}{2} + n\right)f\right)\\
  &= 3n + 4 - \frac{n(n-1)}{n+1} +
                               \frac{3}{n(n+1)}\sum_{f=0}^{n-1}{f^2}
   - \frac{2n+1}{n(n+1)}\sum_{f=0}^{n-1}{f}\\
  &= 3n + 4 - \frac{n(n-1)}{n+1} +
     \frac{3}{n(n+1)}\cdot\frac{(n-1)n(2n-1)}{6}\\
  &\phantom{= 3}
     - \frac{2n+1}{n(n+1)} \cdot \frac{n(n-1)}{2}\\
  &= 3n + 4 - \frac{n(n-1)}{n+1} + \frac{(n-1)(2n-1)}{2(n+1)}
     - \frac{(n-1)(2n+1)}{2(n+1)}\\
  &= 3n + 4 + \frac{(n-1)(-2n+2n-1-2n-1)}{2(n+1)}
   = 2n + 5.
\end{align*}
We found earlier that \(\ave{rm\_lst}{n} = \tfrac{5}{2}n +
\tfrac{9}{2}\), therefore \(\ave{rm\_lst1}{n} < \ave{rm\_lst}{n}\),
for \(n > 1\). In other words, \erlcode{rm\_lst1/2} is in average
faster than \erlcode{rm\_lst/2}. Finally, we remark that
\(\ave{rm\_lst1}{n} \mathrel{\sim} \tfrac{2}{3}\worst{rm\_lst1}{n}\),
as \(n \rightarrow \infty\), that is, the average delay of
\erlcode{rm\_lst1/2} is 66\% of the worst delay.

\medskip

\paragraph{Tail form.}

For the sake of further training, let us transform the definition of
\erlcode{rm\_lst/2} into tail form. The starting point consists in
copying the original definition and rename it \erlcode{rm\_lst\_tf/2}:
\begin{verbatim}
rm_lst_tf(I,L) -> rev(rm_fst(I,rev(L))).
\end{verbatim}
The first call to be performed is \erlcode{rev(L)}. The first thing to
check is whether \erlcode{rev/1} is in tail form or not. It is,
because we found the following, \vpageref{code:rev}:
\verbatiminput{rev.def} Secondly, we locate the control context of the
innermost call, that is, \erlcode{rev(L)}: it is
\erlcode{rev(rm\_fst(I,\textvisiblespace))}. It is handy to visualise
on the code the variables it contains by framing them, while the first
call is in bold:
\begin{alltt}
rm_lst_tf(I,L) -> rev(rm_fst(\fbox{I},\textbf{rev(L)})).
\end{alltt}
Now let us add a list accumulator to \erlcode{rev/1}, whose purpose is
to hold the variables present in the control context of the innermost
call. Here, there is only one variable, \erlcode{I}:
\begin{alltt}
rm_lst_tf(I,L)    -> rev(rm_fst(I,rev(L\textbf{,[I]}))).
rev(L\textbf{,A})          -> rev_join(L,[]).\hfill% A \emph{still unused}
rev_join(   [],Q) -> Q;
rev_join([I|P],Q) -> rev_join(P,[I|Q]).
\end{alltt}
Let us follow the clauses that rewrite the innermost call,
\erlcode{rev(L,[I])} and let us inscribe within a box the first body
which is not a recursive call:
\begin{alltt}
rm_lst_tf(I,L)    -> rev(rm_fst(I,rev(L,[I]))).
rev(L,A)          -> \fbox{rev_join(L,[])}.
rev_join(   [],Q) -> Q;
rev_join([I|P],Q) -> rev_join(P,[I|Q]).
\end{alltt}
Since the definition of \erlcode{rev/2} is made of a unique clause, we
can move the control context of \erlcode{rev(L,[I])} to apply to it,
for which we need to use the variable \erlcode{I} in the accumulator:
\begin{alltt}
rm_lst_tf(I,L)    -> \textbf{rev(L,[I])}.
rev(L,\textbf{[I]})        -> \textbf{rev(rm_fst(I,}rev_join(L,[])\textbf{))}.
rev_join(   [],Q) -> Q;
rev_join([I|P],Q) -> rev_join(P,[I|Q]).
\end{alltt}
We must start again the same operations on the new body where the
innermost call is \erlcode{rev\_join(L,[])} and whose control context
is \erlcode{rev(rm\_fst(I,\textvisiblespace))}. We must check whether
the call to perform first is in tail form or not. The definition of
\erlcode{rev\_join/2} is in tail form, so the next step is to add an
accumulator \erlcode{A} to \erlcode{rev\_join/2}, initialised with the
variables in the control context (here, only \erlcode{I}). This
accumulator is passed, unchanged to any recursive calls:
\begin{alltt}
rm_lst_tf(I,L)      -> rev(L,[I]).
rev(L,[I])          -> rev(rm_fst(I,rev_join(L,[]\textbf{,[I]}))).
rev_join(   [],Q\textbf{,A}) -> Q;\hfill% A \emph{yet unused}
rev_join([I|P],Q\textbf{,A}) -> rev_join(P,[I|Q]\textbf{,A}).\hfill% \emph{Unchanged}
\end{alltt}
Then we repeat the same steps, that is, we identify the first bodies
used to rewrite the innermost call, that is,
\erlcode{rev\_join(L,[],[I])}, which are not recursive (this is why it
is paramount that \erlcode{rev\_join/2} is already in tail form: its
bodies are either a call without control context or a value or an
arithmetic expression):
\begin{alltt}
rm_lst_tf(I,L)      -> rev(L,[I]).
rev(L,[I])          -> rev(rm_fst(I,rev_join(L,[],[I]))).
rev_join(   [],Q,A) -> \fbox{Q};
rev_join([I|P],Q,A) -> rev_join(P,[I|Q],A).
\end{alltt}
Next, we move the control context
\erlcode{rev(rm\_fst\_tf(I,\textvisiblespace))} to apply to the framed
bodies (here, only \erlcode{Q}), making sure that the variable
\erlcode{I} is bound with the help of the accumulator \erlcode{A}:
\begin{alltt}
rm_lst_tf(I,L)        -> rev(L,[I]).
rev(L,[I])            -> \textbf{rev_join(L,[],[I])}.
rev_join(   [],Q,\textbf{[I]}) -> \textbf{rev(rm_fst(I,}Q\textbf{))};
rev_join([I|P],Q,  A) -> rev_join(P,[I|Q],A).
\end{alltt}
The good news is that the only body which is not in tail form now is
made of only two calls instead of three originally. Now the innermost
call is \erlcode{rm\_fst(I,Q)}, so we must check first whether
\erlcode{rm\_fst/2} is in tail form. It is not, as we found
\vpageref{code:rm_fst_alpha}:
\begin{alltt}
rm_fst(_,   []) -> [];
rm_fst(I,[I|L]) -> L;
rm_fst(I,[J|L]) -> [J|rm_fst(I,L)].\hfill% \emph{Not in tail form}
\end{alltt}
Therefore, we must transform now the definition of \erlcode{rm\_fst/2}
into an equivalent one in tail form. If we don't, this is what
happens: the control context \erlcode{rev(\textvisiblespace)} is moved
so it applies to the bodies of \erlcode{rm\_fst/2} which contain no
calls. We have
\begin{alltt}
rm_lst_tf(I,L)        -> rev(L,[I]).
rev(L,[I])            -> rev_join(L,[],[I]).
rev_join(   [],Q,[I]) -> \textbf{rm_fst(I,Q)};
rev_join([I|P],Q,  A) -> rev_join(P,[I|Q],A).
rm_fst(_,   [])       -> \textbf{rev(}[]\textbf{)};\hfill% \emph{Wrong}
rm_fst(I,[I|L])       -> \textbf{rev(}L\textbf{)};\hfill% \emph{Wrong}
rm_fst(I,[J|L])       -> [J|rm_fst(I,L)].
\end{alltt}
This is wrong because the last clause of \erlcode{rm\_fst/2} is not in
tail form, so when no calls are needed, as in the two first clauses of
the original \erlcode{rm\_fst/2}, there is still work to do: the
control context \erlcode{[J|\textvisiblespace]} holds values
``waiting'' to be pushed. But, what we meant here was to
apply~\erlcode{rev/1} to the \emph{value} of \erlcode{rm\_fst(I,Q)},
that is, the final result, when nothing else is left to be computed by
\erlcode{rm\_fst/2}---hence the error. The version of
\erlcode{rm\_fst/2} in tail form was already encountered
\vpageref{code:rm_fst_tf} and named \erlcode{rm\_fst\_tf/2}:
\verbatiminput{rm_fst_tf.def} So we now simply change the call
\erlcode{rm\_fst(I,Q)} in the definition above of
\erlcode{rev\_join/3} into the call \erlcode{rm\_fst\_tf(I,Q)} and
resume our process:
\begin{alltt}
rm_lst_tf(I,L)        -> rev(L,[I]).
rev(L,[I])            -> rev_join(L,[],[I]).
rev_join(   [],Q,[I]) -> rev(\textbf{rm_fst_tf}(I,Q));\hfill% \emph{Renaming}
rev_join([I|P],Q,  A) -> rev_join(P,[I|Q],A).
\end{alltt}
Because \erlcode{rm\_fst\_tf/2} is defined by a single clause, we can
replace the call \erlcode{rm\_fst\_tf(I,Q)} by its definition:
\begin{alltt}
rm_lst_tf(I,L)        -> rev(L,[I]).
rev(L,[I])            -> rev_join(L,[],[I]).
rev_join(   [],Q,[I]) -> rev(\textbf{rm_fst(I,Q,[])});
rev_join([I|P],Q,  A) -> rev_join(P,[I|Q],A).
\end{alltt}
Since \erlcode{rm\_fst/3}~is also in tail form and since the control
context~\erlcode{rev(\textvisiblespace)} contains no variable, we can
repeat the same operation and move the control context so it applies
to the non\hyp{}recursive bodies of \erlcode{rm\_fst/3}:
\begin{alltt}
rm_lst_tf(I,L)        -> rev(L,[I]).
rev(L,[I])            -> rev_join(L,[],[I]).
rev_join(   [],Q,[I]) -> rm_fst(I,Q,[]);
rev_join([I|P],Q,  A) -> rev_join(P,[I|Q],A).
rm_fst(_,   [],A)     -> \textbf{rev(}rev(A)\textbf{)};\hfill% \emph{Added context}
rm_fst(I,[I|L],A)     -> \textbf{rev(}rev_join(A,L)\textbf{)};\hfill% \emph{Added context}
rm_fst(I,[J|L],A)     -> rm_fst(I,L,[J|A]).
\end{alltt}
The new definition of \erlcode{rm\_fst/3} is not in tail form, but we
can trust that composing twice~\erlcode{rev/1} on a list yields a list
equal to the original list, so \erlcode{rev(rev(A))}~is equivalent
to~\erlcode{A}. Moreover, a little figure on paper would convince us
that \erlcode{rev(rev\_join(A,L))}~is equivalent to
\erlcode{rev\_join(L,A)}, that is, these two expressions are always
rewritten to the same value. Finally, taking into account these
simplifications, we reach
\begin{alltt}
rm_lst_tf(I,L)        -> rev(L,[I]).
rev(L,[I])            -> rev_join(L,[],[I]).
rev_join(   [],Q,[I]) -> rm_fst(I,Q,[]);
rev_join([I|P],Q,  A) -> rev_join(P,[I|Q],A).
rm_fst(_,   [],A)     -> \textbf{A};\hfill% \emph{Algebraic simplification}
rm_fst(I,[I|L],A)     -> \textbf{rev_join(L,A)};\hfill% \emph{Idem}
rm_fst(I,[J|L],A)     -> rm_fst(I,L,[J|A]).
\end{alltt}
All these definitions are in tail form now. A quick inspection reveals
that \erlcode{rev/2}~is called only once and is defined by means of a
single clause, thus, we can replace its only call by its body:
\begin{alltt}
rm_lst_tf(I,L)        -> \textbf{rev_join(L,[],[I])}.
rev_join(   [],Q,[I]) -> \textbf{rm_fst(I,Q,[])};
rev_join([I|P],Q,  A) -> rev_join(P,[I|Q],A).
rm_fst(_,   [],A)     -> A;
rm_fst(I,[I|L],A)     -> rev_join(L,A);
rm_fst(I,[J|L],A)     -> rm_fst(I,L,[J|A]).
\end{alltt}
Note how we actually took care of replacing the call by the body
\emph{where the parameters are replaced by their corresponding
  arguments}. Moreover, consider how, in the first clause of
\erlcode{rev\_join/3}, \erlcode{Q}~becomes the input list of the new
\erlcode{rm\_fst/3}. There is still room for a tiny improvement: the
accumulator of \erlcode{rev\_join/3} named~\erlcode{A} always contains
exactly one item:~\erlcode{I}. So why not replace it by~\erlcode{I}
itself? Also, in order enhance legibility, we can rename
variable~\erlcode{L} in \erlcode{rm\_fst/3} into~\erlcode{Q} and
variable~\erlcode{I} in \erlcode{rev\_join/3}
into~\erlcode{J}. Finally, we obtain these simpler tail form
definitions:\label{code:rm_lst_tf}
\begin{alltt}
rm_lst_tf(I,L)      \(\smashedrightarrow{\alpha}\) rev_join(L,[],\textbf{I}).
rev_join(   [],Q,\textbf{I}) \(\smashedrightarrow{\beta}\) rm_fst(I,Q,[]);
rev_join([\textbf{J}|P],Q,\textbf{I}) \(\smashedrightarrow{\gamma}\) rev_join(P,[\textbf{J}|Q],\textbf{I}).
rm_fst(_,   [],A)   \(\smashedrightarrow{\delta}\) A;
rm_fst(I,[I|\textbf{Q}],A)   \(\smashedrightarrow{\epsilon}\) rev_join(\textbf{Q},A);
rm_fst(I,[J|\textbf{Q}],A)   \(\smashedrightarrow{\zeta}\) rm_fst(I,\textbf{Q},[J|A]).
\end{alltt}

\medskip

\paragraph{Delays.}

Let us compute the delay \(\comp{rm\_lst\_tf}{n,k} :=
\delay{rm\_lst\_tf(\(I\),\(L\))}\), where \(n\)~is the length of~\(L\)
and \(k\)~is the greatest location where \(I\)~occurs in~\(L\) (\(k =
n\) if absent). There are three cases to consider:
\begin{itemize}

  \item \(L\)~is empty;

  \item \(L\)~is not empty and \(I\)~is absent;

  \item \(L\)~is not empty and \(I\)~occurs at position~\(k\),
    counting the head of~\(L\) as position~\(0\).

\end{itemize}
If the list is empty, clause~\clause{\alpha} is used once,
clause~\clause{\beta} is used once and clause~\clause{\delta} is used
once.  In total, the delay is~\(3\). In other words:
\(\comp{rm\_lst\_tf}{0,0} = 3\).

\noindent If the list~\(L\) is not empty and \(I\)~is absent, that is,
\(k = n > 0\), then
\begin{itemize}

  \item clause~\clause{\alpha} is used once;

  \item clause~\clause{\gamma} is used \(n\)~times;

  \item clause~\clause{\beta} is used once;

  \item clause~\clause{\zeta} is used \(n\)~times;

  \item clause~\clause{\delta} is used once.

\end{itemize}
This execution trace is formally \(\alpha\gamma^n\beta\zeta^n\delta\).
In total, \(\comp{rm\_lst\_tf}{n,n} = 2n + 3\), with~\(n > 0\). We see
that this formula is valid even if~\(n=0\), so we can merge the two
first cases into one: \(\comp{rm\_lst\_tf}{n,n} = 2n + 3\), with~\(n
\geqslant 0\). If the list~\(L\) is not empty and the item~\(I\)
lastly occurs in~\(L\) at position~\(k\), with~\(0 \leqslant k < n\),
assuming that the head has position~\(0\), then
\begin{itemize}

  \item clause~\clause{\alpha} is used once,

  \item clause~\clause{\gamma} is used \(n\)~times,

  \item clause~\clause{\beta} is used once,

  \item clause~\clause{\zeta} is used \(n-k\)~times,

  \item clause~\clause{\epsilon} is used once.

\end{itemize}
This is \(\alpha\gamma^n\beta\zeta^{n-k}\epsilon\). The delay of the
call \erlcode{rev\_join(Q,A)} in the body of clause~\clause{\epsilon}
must be counted and added. Since \erlcode{Q}~contains \(k\)~items,
equation \eqref{delay:rev_join} \vpageref{delay:rev_join} states that
\(\comp{rev\_join}{k} = k + 1\), so the total is
\[
\comp{rm\_lst\_tf}{n,k} = 1 + n + 1 + (n-k) + 1 + (k+1) = 2n + 4,\,\;
\text{with} \,\; n > 0.
\]
Interestingly, the delays here do not depend on \(k\), that is, the
position of the last occurrence of the item. This is not the case for
\erlcode{rm\_lst/2}, as equation \eqref{eq:rm_lst}
\vpageref{eq:rm_lst} is
\[
\comp{rm\_lst}{n,k} = 3n - k + 4, \,\; \text{with} \,\; 0 \leqslant k
\leqslant n.
\]
The worst case for \erlcode{rm\_lst\_tf/2} happens when the item is
present \emph{anywhere} in the list, because \(2n + 3 < 2n + 4\). In
any case,
\[
\comp{rm\_lst\_tf}{n,k} \mathrel{\sim} 2n, \,\; \text{as} \,\; n
\rightarrow \infty \,\; \text{and} \,\; k < n.
\]
Note that there are only two possible delays, for any input. Since the
position of the item in the list does not impact the delay of
\erlcode{rm\_lst\_tf/2}, the average delay is the normal delay:
\[
\ave{rm\_lst\_tf}{n} = \comp{rm\_lst\_tf}{n,k} = 2n + 4.
\]
Equation \eqref{eq:ave_rm_lst} \vpageref{eq:ave_rm_lst} was
\(
\ave{rm\_lst}{n} = \tfrac{5}{2}{n} + \tfrac{9}{2},
\)
so it yields now
\[
\ave{rm\_lst\_tf}{n} \mathrel{\sim} \tfrac{4}{5}
\ave{rm\_lst}{n},\,\; \text{as} \,\; n \rightarrow \infty,
\]
which means that the average delay of \erlcode{rm\_lst\_tf/2} is~80\%
of the average delay of \erlcode{rm\_lst/2}---in other words, it
is~20\% smaller.

As we found above, the worst case of \erlcode{rm\_lst\_tf/2} when the
item is present does not depend on its position inside the list. This
in stark contrast with \erlcode{rm\_lst/2}, whose worst case is when
the item is the head of the list. The best case for both functions is
when the item is absent from the list. Let us note
\(\best{rm\_lst\_tf}{n}\) the delay in the best case for an input of
size~\(n\) and \(\worst{\rm\_lst\_tf}{n}\) the delay in the worst
case. We found that, for~\(n > 0\),
\begin{equation*}
  \best{rm\_lst\_tf}{n} = 2n + 3 \mathrel{\sim} 2n,
\quad
  \worst{rm\_lst\_tf}{n} = 2n + 4 \mathrel{\sim} 2n,\,\; \text{as}
  \,\; n \rightarrow \infty.
\end{equation*}
We can now precisely compare \erlcode{rm\_lst\_tf/2} and
\erlcode{rm\_lst/2} in terms of delay. We have, for~\(n > 0\),
\begin{align*}
\best{rm\_lst}{n}     &= 2n + 5, & \worst{rm\_lst}{n}     &= 3n + 6,\\
\best{rm\_lst\_tf}{n} &= 2n + 3, & \worst{rm\_lst\_tf}{n} &= 2n + 4.
\end{align*}
The keen reader may object now that the worst case
of \erlcode{rm\_lst/2} is \emph{not} the same as
for \erlcode{rm\_lst\_tf/2}, since, for the former, it is defined as
the item being the head of the list, whilst for the latter it is when
the item occurs \emph{anywhere} in the list. Indeed. To understand
why, imagine that the last occurrence of the item is not at the head
of the list, whereas it is one possible worst case
for \erlcode{rm\_lst\_tf/2} but not for
\erlcode{rm\_lst/2}. Therefore, \erlcode{rm\_lst/2} might outperform
\erlcode{rm\_lst\_tf/2} on this particular example. Nevertheless, we
can compare both functions globally because the worst delay
of \erlcode{rm\_lst\_tf/2} is smaller than the best delay
of \erlcode{rm\_lst/2}:
\[
\worst{rm\_lst\_tf}{n} < \best{rm\_lst}{n}.
\]
We can also compare directly the two functions on the
\emph{intersection} of their worst cases, that is, when the item we
look for is the head of the list. In this special case, then a
sensible difference is likely, especially for large values of~\(n\),
as
\[
\worst{rm\_lst\_tf}{n} \mathrel{\sim}
\tfrac{2}{3}{\worst{rm\_lst}{n}},\,\; \text{as} \,\; n \rightarrow
\infty.
\]
We can compare directly \(\best{rm\_lst}{n}\) and
\(\best{rm\_lst\_tf}{n}\), because these delays correspond to the same
case, that is, when the item is missing from the non\hyp{}empty list,
and we also find that \erlcode{rm\_lst\_tf/2}~is barely faster
than~\erlcode{rm\_lst/2}.

As a conclusion, the version in tail form is always more efficient
(between 0\%~and~\(33.3\%\), in average~\(20\%\)), although we said
earlier that it should not be surprising that it is in fact slower,
since the main purpose of being in tail form is to save call stack
space, not rewrites. The reason why \erlcode{rm\_lst\_tf/2} is both
faster and uses a small constant amount of call stack is because we
did not follow our method until the end and, instead, relied on the
following algebraic identities, used from left to right, for a
speedup:
\begin{align*}
\erlcode{rev(rev(\(P\)))} &\mathrel{\equiv} P,\\
\erlcode{rev(rev\_join(\(P\),\(Q\)))} 
                          &\mathrel{\equiv} \erlcode{rev\_join(\(Q\),\(P\))},
\end{align*}
for all lists \(P\)~and~\(Q\).

We could wonder now: what if we added a sequential search to
\erlcode{rm\_lst\_tf/2}? Here is the definition again, as found
\vpageref{code:rm_lst_tf}:
\begin{alltt}
rm_lst_tf(I,L)      -> rev_join(L,[],I).
rev_join(   [],Q,I) -> rm_fst(I,Q,[]);
rev_join([J|P],Q,I) -> rev_join(P,[J|Q],I).
rm_fst(_,   [],A)   -> A;
rm_fst(I,[I|Q],A)   -> rev_join(Q,A);
rm_fst(I,[J|Q],A)   -> rm_fst(I,Q,[J|A]).
\end{alltt}
We could save time by not reversing~\erlcode{L} by means of
\erlcode{rev\_join/3} and instead start with a sequential search for
item~\erlcode{I}. If it is found, then \emph{the rest of the list} is
worth reversing. This would be
\begin{alltt}
rm_lst2(_,   [])  \(\smashedrightarrow{\alpha}\) [];
rm_lst2(I,[I|L])  \(\smashedrightarrow{\beta}\) \textbf{rm_fst(I,rev_join(L,[I]),[])};
rm_lst2(I,[J|L])  \(\smashedrightarrow{\gamma}\) [J|rm_lst2(I,L)].
rm_fst(_,   [],A) \(\smashedrightarrow{\delta}\) A;
rm_fst(I,[I|Q],A) \(\smashedrightarrow{\epsilon}\) rev_join(Q,A);
rm_fst(I,[J|Q],A) \(\smashedrightarrow{\zeta}\) rm_fst(I,Q,[J|A]).
\end{alltt}
Again, the part of the definition not in bold is the typical scheme of
a sequential search, whilst the body in bold is a specific action
performed if the first occurrence of the sought item is found. Note
that we did not care of preserving the tail form, but only to reuse
\erlcode{rm\_fst/2}. What is the delay of \erlcode{rm\_lst\_tf/2}?  If
the item is absent from the list, the delay is that of a failed
sequential search in a list of length~\(n\), that is,~\(n+1\). If the
item occurs first at position~\(f\), the delay is
\begin{align*}
\comp{rm\_lst2}{n}
  &= f + \comp{rev\_join}{n-f-1} + \comp{rm\_fst}{n-f}.
\intertext{We already know that \(\comp{rev\_join}{n} = n + 1\), so}
\comp{rm\_lst2}{n}
  &= f + ((n-f-1)+1) + \comp{rm\_fst}{n-f}
   = n + \comp{rm\_fst}{n-f}.
\end{align*}
The execution trace of \erlcode{rm\_fst/2} in this context is
\(\zeta^{n-1-k}\epsilon\), followed by the trace of the call
\erlcode{rev\_join(Q,A)} in clause~\clause{\epsilon}, so
\[
\comp{rm\_fst}{n} = (n-1-k) + 1 + \comp{rev\_join}{k-f}
= n - k + ((k-f)+1) = n - f + 1,
\]
where the length of~\erlcode{Q} is~\(k-f\). Finally,
\(\comp{rm\_lst2}{n} = n + (n - k + 1) = 2n - k + 1\). Therefore, the
best and worst delays \emph{when the item is present} are
\(\best{rm\_lst2}{n} = 2n-(n-1)+1=n+2, \worst{rm\_lst2}{n} = 2n + 1\).
Assuming now that the item is absent from the list, the delay is
simply that of a failed sequential search, that is,
\(\comp{rm\_lst2}{n} = n + 1\). Which allows us to conclude that, in
any case, for~\(n \geqslant 0\),
\[
\best{rm\_lst2}{n} = n+1,\qquad \worst{rm\_lst2}{n} = 2n + 1.
\]
The average delay when the item occurs in the list is
\begin{align*}
\ave{rm\_lst2}{n}
  &:= \frac{1}{n}\sum_{k=0}^{n-1}{\comp{rm\_lst2}{k}}
   = \frac{1}{n}\sum_{k=0}^{n-1}{(2n - k + 1)}\\
  &= 2n + 1 - \frac{1}{n}\sum_{k=0}^{n-1}{k}
   = 2n + 1 - \frac{n-1}{2} = \frac{3}{2}{n} + \frac{3}{2}.
\end{align*}
Up to now, the findings can be related in the table of
\fig~\vref{fig:delays_rm_lst}, bearing in mind that two best cases or
two worst delays cannot always be compared because the configuration
of the input may be different.
\begin{figure}[b]
\centering
\includegraphics[bb=71 615 375 721]{rm_lst_partial}
\caption{Delays of different definitions for filtering out an item
\label{fig:delays_rm_lst}}
\end{figure}
We can deduce nevertheless that \erlcode{rm\_lst2/2} is always faster
than \erlcode{rm\_lst/2} and \erlcode{rm\_lst\_tf/2} because
\(\worst{rm\_lst2}{n} < \best{rm\_lst}{n}\) and \(\worst{rm\_lst2}{n}
< \best{rm\_lst\_tf}{n}\). Moreover, two average delays can always be
compared (they suppose here that the item does occur in the list),
thus \erlcode{rm\_lst2/2} is the fastest of all in average. Finally,
we want to insist again that ``average delay'' should not be confused
with ``delay of the average case,'' because it might be that there is no
average case, that is, a configuration of the input entailing a delay
which is the average delay. The average delay is defined as an
arithmetic computation the delays corresponding to all possible input
configurations, under some equi\-probability hypotheses.

\medskip

\paragraph{Tail form redux.}

Until now we have seen slightly different ways to convert a given
definition into tail form:
\begin{enumerate}

  \item from \erlcode{sum/1} to \erlcode{sum\_tf/1} and from
    \erlcode{fact/1} to \erlcode{fact\_tf/1};

  \item from \erlcode{join/2} to \erlcode{join\_tf/2};

  \item from \erlcode{rm\_fst/2} to \erlcode{rm\_fst\_tf/2};

  \item from \erlcode{rm\_lst/2} to \erlcode{rm\_lst\_tf/2}.

\end{enumerate}
All these transformations rely on an additional parameter but their
design seem to differ in the nature of this accumulator: it represents
either a partial result or a temporary list. The reason for the
dissimilarity actually lies in the use of dedicated improvements for
each case, but there is a common scheme for all these transformations
into tail form.

\medskip

\paragraph{Factorial.}

For instance, if we do not rely on associativity of the arithmetic
operators involved in \erlcode{sum/1} and \erlcode{fact/1}, that is,
addition and multiplication, we then need to keep the integers in a
list so that they can be processed in the proper, reversed,
order. First, let us recall the original definition of
\erlcode{fact/1}:
\begin{alltt}
fact(1)            \(\smashedrightarrow{\alpha}\) 1;
fact(N) when N > 1 \(\smashedrightarrow{\beta}\) N * fact(N-1).
\end{alltt}
and the example \vpageref{trace:fact_5}:

\begin{alltt}
fact(5) \(\xrightarrow{\beta}\) 5 * fact(4)
        \(\smashedrightarrow{\beta}\) 5 * (4 * fact(3))
        \(\smashedrightarrow{\beta}\) 5 * (4 * (3 * fact(2)))
        \(\smashedrightarrow{\beta}\) 5 * (4 * (3 * (2 * fact(1))))
        \(\smashedrightarrow{\alpha}\) 5 * (4 * (3 * (2 * (1))))    \(=\) 120\textrm{.}
\end{alltt}


\noindent We derived the following definition of \erlcode{fact\_tf/1}
in tail form, \vpageref{code:fact_tf_alpha}:
\verbatiminput{fact_tf.def} The order in which the multiplications are
carried out by~\erlcode{fact\_tf/2} is not the same as done
by~\erlcode{fact/1}, as it was assumed that multiplication is
associative. If we do want not rely on associativity, we need to
create and store the integers in a list and then multiply them, for
instance, by reusing~\erlcode{mult\_tf/1}, defined
\vpageref{code:mult_tf}. This way the multiplications will be
performed exactly in the same order as in \erlcode{fact/1} because
pushing the numbers on the accumulator actually reverses their order:
\begin{alltt}
fact_alt_tf(N) when N > 0 \(\smashedrightarrow{\alpha}\) fact_alt_tf(N-1,\textbf{[N]}).
fact_alt_tf(\textbf{1},A)          \(\smashedrightarrow{\beta}\) \textbf{mult_tf(A,1)};
fact_alt_tf(N,A)          \(\smashedrightarrow{\gamma}\) fact_alt_tf(N-1,\textbf{[N|A]}).
mult_alt_tf(   [],A)      \(\smashedrightarrow{\delta}\) A;
mult_alt_tf([N|L],A)      \(\smashedrightarrow{\epsilon}\) mult_alt_tf(L,\textbf{N*A}).\hfil% \emph{Changed}
\end{alltt}
Here, we have instead
\begin{tabbing}
\erlcode{fact\_alt\_tf(5)}\\
\erlcode{fact\_a} \= \(\smashedrightarrow{\alpha}\) \= \erlcode{mult\_alt\_tf([3,4,5],2*1)} \=\kill
\> \(\smashedrightarrow{\alpha}\) \> \erlcode{fact\_alt\_tf(4,[5])}\\
\> \(\smashedrightarrow{\gamma}\) \> \erlcode{fact\_alt\_tf(3,[4,5])}\\
\> \(\smashedrightarrow{\gamma}\) \> \erlcode{fact\_alt\_tf(2,[3,4,5])}\\
\> \(\smashedrightarrow{\gamma}\) \> \erlcode{fact\_alt\_tf(1,[2,3,4,5])}\\
\> \(\smashedrightarrow{\beta}\) \> \erlcode{mult\_alt\_tf([2,3,4,5],1)}\\
\> \(\smashedrightarrow{\epsilon}\) \> \erlcode{mult\_alt\_tf([3,4,5],2*1)}
\> \(=\) \erlcode{mult\_alt\_tf([3,4,5],2)}\\
\> \(\smashedrightarrow{\epsilon}\) \> \erlcode{mult\_alt\_tf([4,5],3*2)}
\> \(=\) \erlcode{mult\_alt\_tf([4,5],6)}\\
\> \(\smashedrightarrow{\epsilon}\) \> \erlcode{mult\_alt\_tf([5],4*6)}
\> \(=\) \erlcode{mult\_alt\_tf([5],24)}\\
\> \(\smashedrightarrow{\epsilon}\) \> \erlcode{mult\_alt\_tf([],5*24)}
\> \(=\) \erlcode{mult\_alt\_tf([],120)}\\
\> \(\smashedrightarrow{\delta}\) \> \erlcode{120}\textrm{.}
\end{tabbing}
We see that the multiplications are \(5 \cdot (4 \cdot (3 \cdot (2
\cdot 1)))\), just as they were when rewriting~\erlcode{fact(5)}, but
this faithfully ordered sequence comes at a cost: if the number is
\(n\), a list of length~\(n\) has to be created (\(n\)~rewrites)
before it is traversed (\(n\)~additional rewrites) to perform the
multiplications. Therefore, this solution incurs a~50\% slowdown,
compared to \erlcode{fact\_tf/1} (\(2n\)~versus~\(n+1\) exactly) and
requires additional memory to hold the temporary list. Nevertheless,
\erlcode{fact\_alt\_tf/1}~illustrates our point, that is, the control
context can be transformed into an operation upon a linear
accumulator. More precisely, there are three phases: (1)~a new
function is created, which calls an auxiliary function with an empty
list as initial accumulator; (2)~the values of the control context are
pushed onto the accumulator of the call; (3)~these values are popped
from the accumulator in the non\hyp{}recursive bodies and are applied
the original control context. In the case of
\erlcode{fact\_alt\_tf/1}, the first phase is done by
clause~\clause{\alpha}, the second by clause~\clause{\gamma} and the
third by clauses \clause{\beta},
\clause{\delta}~and~\clause{\epsilon}. If the control context is an
associative function or operator, then the accumulator can be
specialised, like we did with~\erlcode{fact\_tf/1}, and it may even
not need to be a list if it only holds one value all along.

\medskip

\paragraph{Joining two lists.}

Let us examine now \erlcode{join/2} and \erlcode{join\_tf/2}. We had
the following definitions:
\begin{alltt}
join(   [],Q)       \(\smashedrightarrow{\alpha}\) Q;
join([I|P],Q)       \(\smashedrightarrow{\beta}\) [I|join(P,Q)].

join_tf(P,Q)        \(\smashedrightarrow{\gamma}\) join(P,Q,[]).
join(   [],Q,   []) \(\smashedrightarrow{\delta}\) Q;
join(   [],Q,[I|A]) \(\smashedrightarrow{\epsilon}\) join([],[I|Q],A);
join([I|P],Q,    A) \(\smashedrightarrow{\zeta}\) join(P,Q,[I|A]).
\end{alltt}
This transformation illustrates perfectly the general method we
underline here: (1)~a linear accumulator, \erlcode{A}, was added in a
new clause \clause{\gamma}; (2)~the values in the control context
(here only~\erlcode{I} in clause~\clause{\beta}) were stored in it by
clause~\clause{\zeta}; (3)~and later popped out to be processed as in
the original control context in clause~\clause{\epsilon}, that is,
pushed on the partial result---which is~\erlcode{Q} because of
clause~\clause{\delta} that derives from clause~\clause{\alpha}.

\medskip

\paragraph{Filtering out the first occurrence.}

Let us consider in turn the transformation from \erlcode{rm\_fst/2} to
\erlcode{rm\_fst\_tf/2}. Here is the definition of the former, as
found \vpageref{code:rm_fst_alpha}:
\begin{alltt}
rm_fst(_,   []) \(\smashedrightarrow{\alpha}\) [];
rm_fst(I,[I|L]) \(\smashedrightarrow{\beta}\) L;
rm_fst(I,[J|L]) \(\smashedrightarrow{\gamma}\) [J|rm_fst(I,L)].
\end{alltt}
and of the latter \vpageref{code:rm_fst_tf}:
\begin{alltt}
rm_fst_tf(I,L)       \(\smashedrightarrow{\delta}\) rm_fst_tf(I,L,[]).
rm_fst_tf(_,   [],A) \(\smashedrightarrow{\epsilon}\) rev(A);
rm_fst_tf(I,[I|L],A) \(\smashedrightarrow{\zeta}\) rev_join(A,L);
rm_fst_tf(I,[J|L],A) \(\smashedrightarrow{\eta}\) rm_fst_tf(I,L,[J|A]).

rev_join(   [],Q) -> Q;
rev_join([I|P],Q) -> rev_join(P,[I|Q]).
\end{alltt}
We have here another illustration of our method, although it is a
little bit obscured: (1)~a list accumulator has been added in
clause~\clause{\delta}; (2)~the values in the control context
(only~\erlcode{J}) are pushed onto it in clause~\clause{\eta};
(3)~these values are popped from the accumulator and the original
control context they belonged to is applied to them in clauses
\clause{\epsilon}~and~\clause{\zeta}. The usage
of \erlcode{rev\_join/2} itself is a handy shortcut, because it was
already defined and it corresponds exactly to the last step we
mentioned: the values from the control context, stored in the
accumulator, have to be popped (this is the first phase of the
definition of \erlcode{rev\_join/2}) and applied to the partial result
as if in the original control context, that is, pushed on it (because
of clause~\clause{\beta}).

\medskip

\paragraph{Filtering out the last occurrence.}

The last illustrations we have to reconsider are the definitions of
\erlcode{rm\_lst/2} and \erlcode{rm\_lst\_tf/2}. The former is
\begin{alltt}
rm_lst(I,L) -> rev(rm_fst(I,rev(L))).
\end{alltt}
while the latter is found \vpageref{code:rm_lst_tf}:
\verbatiminput{rm_lst_tf.def} This transformation is longer than the
previous ones but it also underlines our general method. Two aspects
make it slightly different in appearance, though. Firstly, we applied
to the first two clauses of \erlcode{aux2/3} two algebraic equalities
to speed up the code:
\begin{align*}
\erlcode{rev(rev(A))}         &\mathrel{\equiv} \erlcode{A},\\
\erlcode{rev(rev\_join(A,L))} &\mathrel{\equiv}
\erlcode{rev\_join(L,A)}.
\end{align*}
Secondly, we realised that the accumulator of \erlcode{rev\_join/3}
only contained a single item, so it was transformed further into that
item. We also removed an auxiliary definition which was expanded
\emph{in situ}. Lastly, we renamed some functions and variables to
enhance the readability.

\medskip

\paragraph{Exercises.}
\label{ex:filtering_out}

\noindent [Answers \vpageref{ans:filtering_out}.]
\begin{enumerate}

  \item Derive an equivalent definition \erlcode{diff\_tf/1} in tail
    from \erlcode{diff/1}:
\begin{verbatim}
diff([M,N]) -> M - N;
diff([M|L]) -> M - diff(L).
\end{verbatim}
Next, make sure that the function never fails by ending instead with
an atom~\erlcode{list\_too\_short}. Compare the delays of
\erlcode{diff/1} and \erlcode{diff\_tf/1}.

  \item Transform the definition of \erlcode{srev/1} into tail form
    \erlcode{srev\_tf/1} and give its delay.
\verbatiminput{srev.def}
\verbatiminput{join_2.def}

  \item Another idea for filtering out the last occurrence of an item
    in a list consists in combining a sequential search with a
    membership test: if an occurrence is found, it is checked whether
    it is the last or not by performing another sequential search
    \emph{on the remaining items}. (This kind of tactic is commonly
    called in Computer Science a \emph{lookahead}.) If the item is
    found not to be the last, it is retained in the partial result and
    the main sequential search is resumed, otherwise it stops and the
    item is left out. Implement this idea as
    function~\erlcode{rm\_lst3/2} and compute its delay, in particular
    in its best and worst cases, as well as its average delay. Compare
    it to the prior equivalent functions of this section.

  \item Define a function \erlcode{drop/2} such that the call
    \erlcode{drop(\(L\),\(n\))} is rewritten in the list made of the
    items of list~\(L\) in the same order, except the items occurring
    every \(n\)~positions, the first item being counted as occurring
    at position~\(1\). For instance
    \begin{align*}
      \erlcode{drop([a,b,c,d,e,f,g,h],3)} &\twoheadrightarrow
      \erlcode{[a,b,d,e,g,h]};\\ \erlcode{drop([a,b,c,d,e,f,g,h],1)}
      &\twoheadrightarrow \erlcode{[]}.
    \end{align*}
   What do you think should be done when the period is~\(0\)? What
   about the empty list? What if the period is greater than the length
   of the list? Or negative? Justify your choices and make sure no
   special case is left out, that is, your definition is complete. Find
   the delay in the worst case.

  \item Check the following definitions and assess their correctness
    and completeness. If a definition is incorrect, provide a
    counter\hyp{}example and explain why, in your opinion, the
    programmer made that mistake. If a definition is overly complex or
    too long, improve upon it. Check also delay, number of pushes and
    memory usage. Which definitions are in tail form?
    \begin{enumerate}
      \item
\begin{verbatim}
rm_fst(I,   []) -> error;
rm_fst(I,  [I]) -> [];
rm_fst(A,  [I]) -> [I];
rm_fst(I,[I|L]) -> L;
rm_fst(A,[I|L]) -> [I|rm_fst(A,L)].
\end{verbatim}

\medskip

      \item
\begin{verbatim}
rm_fst(X,[X|T]) -> rm_fst(T);
rm_fst(X,[Y|T]) -> [Y|rm_fst(X,T)];
rm_fst(T,   []) -> [].
rm_fst([X|T])   -> [X|rm_fst(T)];
rm_fst(   [])   -> [].
\end{verbatim}

\medskip

      \item
\begin{verbatim}
rm_fst(_,   [])   -> [];
rm_fst(I,[I|L])   -> L;
rm_fst(A,[I|L])   -> rm_fst(A,L,[I]).
rm_fst(I,[I|L],C) -> rev_join(C,L);
rm_fst(A,[I|L],C) -> rm_fst(A,L,[I|C]);
rm_fst(_,   [],M) -> rev(M).
\end{verbatim}

\medskip

      \item
\begin{verbatim}
rm_fst(I,L)       -> rm_fst(I,L,[]).
rm_fst(I,[I|L],A) -> join(A,L);
rm_fst(I,[J|L],A) -> rm_fst(I,L,join(A,[J])).
\end{verbatim}

\medskip

      \item
\begin{verbatim}
rm_fst(I,  [I]) -> [];
rm_fst(_,  [I]) -> [I];
rm_fst(I,[I|L]) -> rm_fst(L);
rm_fst(J,[I|L]) -> [I|rm_fst(J,L)].
rm_fst(L)       -> L.
\end{verbatim}
    \end{enumerate}

  \item Define a function~\erlcode{rm\_all/2} such that
    \erlcode{rm\_all(\(I\),\(L\))}, where \(I\)~is any expression and
    \(L\)~an~expression rewritable to a list, is rewritten into a list
    containing the elements of~\(L\), except all occurrences of~\(I\),
    in the same order. For example, \erlcode{rm\_all(7,[7,7,1,7,2])}
    \(\twoheadrightarrow\) \erlcode{[1,2]}. Is there a worst case,
    that is, a set of inputs which maximise the number of rewrites?
    Find the delay and the number of pushes to reach the result. Give
    an equivalent definition~\erlcode{rm\_all\_tf/2} in tail form and
    answer the same questions.

\end{enumerate}
