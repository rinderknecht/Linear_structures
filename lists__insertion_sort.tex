%%-*-latex-*-

\chapter{Insertion Sort}
\label{chap:insertion_sort}

Let us start by the following observation. At any time, if we already
have a sorted list, it is easy to insert one more number at the right
place by comparing it with the first number, then, if necessary, the
second, the third etc. If it is greater than all the numbers in the
list, we must build a list where the number is present at the end. For
example, inserting~\erlcode{1} in \erlcode{[3,6]} requires comparing
\erlcode{1}~and~\erlcode{3} and, since~\erlcode{1~=<~3}, the result is
\erlcode{[\textbf{1},3,6]}---we don't need to compare~\erlcode{1}
to~\erlcode{6}. When inserting~\erlcode{4} in~\erlcode{[1,3,6]}, we
compare first \erlcode{4}~and~\erlcode{1}; because~\erlcode{4~>~1}, we
must compare \erlcode{4}~with~\erlcode{3}; since~\erlcode{4~>~3}, we
must further compare \erlcode{4}~and~\erlcode{6}; because
\erlcode{4~=<~6}, we know that the result is
\erlcode{[1,3,\textbf{4},6]}. Inserting~\erlcode{7} in~\erlcode{[3,6]}
results in comparing \erlcode{7}~with~\erlcode{3} and~\erlcode{6},
then concluding \erlcode{[3,6,\textbf{7}]}. Inserting a number in the
empty list results in no comparison and the result is a singleton
list. The algorithm, called \emph{insertion sort}, consists then in
inserting the numbers one by one in a list originally empty.

\medskip

\paragraph{One\hyp{}way insertion sort.}

The previous description shows that insertion sort requires two
functions: one to insert a number in a sorted list and another to
traverse the input and call the insertion function for each number
encountered. Let us focus first on \erlcode{insert/2}, such that
\erlcode{insert(\(I\),\(S\))} is rewritten in the sorted list
containing all the numbers in the sorted list~\(S\) plus~\(I\). As
usual, we reason by case analysis and prepare patterns matching the
empty list and the non\hyp{}empty list:
\begin{alltt}
insert(I,   []) -> \fbcode{[J|insert(I,S)]};
insert(I,[J|S]) -> \fbcode{[J|insert(I,S)]}.
\end{alltt}
The first clause is easy to complete: the result is the singleton
list~\erlcode{[I]} and no comparison is involved (we have only one
number at hand, anyway):
\begin{alltt}
insert(I,   []) -> \textbf{[I]};
insert(I,[J|S]) -> \fbcode{[J|insert(I,S)]}.
\end{alltt}
The second clause must be split because either \erlcode{I~=<~J} or
\erlcode{I~>~J}. Let us try
\begin{alltt}
insert(I,   [])             -> [I];
insert(I,[J|S]) \textbf{when I =< J} -> \fbcode{[J|insert(I,S)]};
insert(I,[J|S])             -> \fbcode{[J|insert(I,S)]}.
\end{alltt}
If \erlcode{I~=<~J}, then~\erlcode{I} must be before~\erlcode{J} in
the output, therefore we can, here too, conclude in one step:
\begin{alltt}
insert(I,   [])             -> [I];
insert(I,[J|S]) when I =< J -> \textbf{[I,J|S]};
insert(I,[J|S])             -> \fbcode{[J|insert(I,S)]}.
\end{alltt}
Finally, if~\erlcode{I~>~J}, we know that \erlcode{I}~must be
\emph{after}~\erlcode{J} in the output, so we must keep~\erlcode{J} at
its current relative position and insert~\erlcode{I} in~\erlcode{S}:
\begin{alltt}
insert(I,   [])             -> [I];
insert(I,[J|S]) when I =< J -> [I,J|S];
insert(I,[J|S])             -> \textbf{[J|insert(I,S)]}.
\end{alltt}
Focusing back on the second clause, we realise that we do nothing
with~\erlcode{S} and we copy as a whole \erlcode{[J|S]}, because
\erlcode{[I,J|S]}~is just the same as~\erlcode{[I|[J|S]]}. As we saw
earlier, this results in the creation of a node~(\erlcode{|}), while
it would be more economical to refer to the node~(\erlcode{|}) in
\erlcode{[J|S]}. This is where an \emph{alias} comes handy:
\begin{alltt}
insert(I,   [])               -> [I];
insert(I,\textbf{L=[J|S]}) when I =< J -> [I|\textbf{L}];
insert(I,[J|S])               -> [J|insert(I,S)].
\end{alltt}
We could go further and silence~\erlcode{S} in the aliased pattern:
\begin{alltt}
insert(I,   [])               -> [I];
insert(I,L=[J|\textbf{\_}]) when I =< J -> [I|L];
insert(I,[J|S])               -> [J|insert(I,S)].
\end{alltt}
But this improvement is not entirely satisfying. What if we test
\erlcode{I~>~J} first? This results in swapping the two last clauses:
\begin{alltt}
insert(I,     [])            -> [I];
insert(I,  [J|S]) \textbf{when I > J} -> [J|insert(I,S)];
insert(I,L=[J|_])            -> [I|L].
\end{alltt}
Nothing is done with~\erlcode{J} in the last clause, so we can silence
it and rename~\erlcode{L} into~\erlcode{S} to remind the reader that
it is a sorted list:
\begin{alltt}
insert(I,   [])            \(\smashedrightarrow{\gamma}\) [I];
insert(I,[J|S]) when I > J \(\smashedrightarrow{\delta}\) [J|insert(I,S)];
insert(I,    \textbf{S})            \(\smashedrightarrow{\epsilon}\) [I|\textbf{S}].
\end{alltt}
This is better because we use only two variables instead of three and
sharing is maximum without the need of an alias. Still, another look
reveals that clause~\clause{\gamma} is useless if~\erlcode{S} in
clause~\clause{\epsilon} is allowed to match the empty list. The final
word seems to be\label{code:insert}
\begin{alltt}
insert(I,[J|S]) when I > J \(\smashedrightarrow{\delta}\) [J|insert(I,S)];
insert(I,    S)            \(\smashedrightarrow{\epsilon}\) [I|S].
\end{alltt}
The remaining function is \erlcode{isort/1}, such that
\erlcode{isort(\(L\))} is rewritten into the sorted list
corresponding to list \(L\). The only task left consists in walking
the input list and inserting each number, that is, calling
\erlcode{insert/2}:
\begin{alltt}
isort(   []) \(\smashedrightarrow{\beta}\) [];
isort([I|L]) \(\smashedrightarrow{\gamma}\) insert(I,isort(L)).
\end{alltt}
Let us compute now the delay. At this point it is worth mentioning
that, traditionally, textbooks about program analysis assess the
delays of sorting algorithms by counting the number of comparisons,
not the number of function calls. Doing so allows one to compare with
the same measure different sorting algorithms, as long as they perform
comparisons. (There are sorting algorithms which do not rely on
comparisons.) A glimpse at the definition of \erlcode{insert/2}
convinces us that the delay depends on the result of \emph{all the
  comparisons} needed to insert a number. It is not possible to
capture this into a single, exact number, therefore we have instead to
think in terms of best, worst and average delays. How can we pick and
arrange \(n\)~integers into a list~\(L\) which leads to the minimum
delay for evaluating \erlcode{isort(\(L\))}? The first hint comes from
the observation that such an input would not exert
clause~\clause{\delta}, since it is recursive, whereas it would rely
on clause~\clause{\epsilon}, which contains no call at all. In other
words, each number to be inserted would be lower or equal than the
head of the current sorted list. The second hint is that \emph{the
  items are inserted in reverse order}, due to clause~\clause{\gamma},
that is, the last item is first inserted in the empty list, the
penultimate is then inserted, thus compared to the last number in the
input list (which is the first now). Since it is expected here to be
lower, it means that the penultimate number in the input is lower than
the last. Plainly, this means that \emph{the best case happens when
  the input list is already nondecreasingly sorted.}

Let us note \(\best{isort}{n}\) the delay in the best case for
\(n\)~items. Then the execution trace is
\(\gamma^n\beta\epsilon^n\). In total, \(\best{isort}{n} = 2n+1\).

The worst case must exert clause~\clause{\delta} as much as possible,
which implies that \emph{the worst case happens when the input is a
  list decreasingly sorted}. Note the precision of the vocabulary:
``decreasingly'' excludes the possibility of two numbers being equal,
contrary to ``non\hyp{}increasingly,'' which allows that to
happen. Let us note \(\worst{isort}{n}\) the delay in the worst case
of \(n\)~items to be sorted. It requires more care as the number of
times clause~\clause{\delta} is used is the length of its second
argument, which means that we cannot count separately the number of
rewrites by clauses \clause{\gamma}~and~\clause{\delta}. In this case,
it is wiser to write down the recurrence equations corresponding to
the program, the measure being the length of the list. Assuming
\(p\geqslant{}0\), we deduce that
\begin{align*}
  \worst{insert}{0}    &\eqn{\epsilon} 1;
& \worst{isort}{0}   &\eqn{\beta} 1;\\
  \worst{insert}{p+1}  &\eqn{\delta} 1 + \worst{insert}{p};
& \worst{isort}{p+1} &\eqn{\gamma} 1 + \worst{insert}{p} +
                                         \worst{isort}{p}.
\end{align*}
In equation~\((\eqn{\gamma})\), the term \(\worst{insert}{p}\) is
correct because the call \erlcode{isort(L)} evaluates in a list of the
same length as~\erlcode{L}, that is,~\(p\). Notice that we use
clause~\clause{\epsilon} only in the case of the empty list, because,
otherwise, we would use clause~\clause{\delta}. It is now easy to
deduce from equations \((\eqn{\delta})\)~and~\((\eqn{\epsilon})\) a
closed expression for \(\worst{insert}{n}\), where \(n\)~is the length
of the input list:
\[
\worst{insert}{n} = n + 1,\,\; \text{with} \,\; n \geqslant 0.
\]
Replacing this expression in the remaining equations leads to
\[
\worst{isort}{0}   \eqn{\beta} 1,\quad
\worst{isort}{p+1} \eqn{\gamma} 2 + p + \worst{isort}{p}.
\]
Summing both sides of~\((\eqn{\gamma})\) and simplifying (this is the
telescoping technique), we get
\begin{align*}
\sum_{p=0}^{n-1}{\worst{isort}{p+1}}
  &= \sum_{p=0}^{n-1}{2} + \sum_{p=0}^{n-1}{p}
     + \sum_{n=0}^{n-1}{\worst{isort}{p}},\\
\worst{isort}{n}
  &= 2n + n(n-1)/2 + \worst{isort}{0}
  = (n^2 + 3n + 2)/2\\ 
  &= \frac{1}{2}(n+1)(n+2) = \sum_{k=1}^{n+1}{k}
 \mathrel{\sim} \frac{1}{2} n^2,\;\, \text{as}
\,\; n \rightarrow \infty.
\end{align*}
In short, \erlcode{isort/1}~has a quadratic delay in the worst case,
which is when the input is sorted in decreasing order.

There is another way to find the result, by means of the execution
trace~\(T_w\) of the worst case. We expect it to be 
\[
T_w := \gamma^n\beta \cdot \epsilon
(\delta\epsilon)(\delta^2\epsilon)
\cdot\ldots\cdot (\delta^{n-1}\epsilon) =
\gamma^n\beta \prod_{k=0}^{n-1}\delta^k\epsilon.
\]
Thus, the length of the trace is
\[
\len{T_w} = \len{\gamma^n\beta} +
\sum_{k=0}^{n-1}\len{\gamma^k\epsilon} = n + 1 +
\sum_{k=0}^{n-1}{(k+1)} = \frac{1}{2}n^2 + \frac{3}{2}n + 1.
\]
We can do a rapid test to see if we made a mistake: Is \((n^2 + 3n +
2)/2\) always an integer? This question boils down to: Is \(n^2 + 3n\)
even? That is: Do \(n^2\) and \(3n\) have the same parity? The answer
is clearly yes, so \((n^2 + 3n + 2)/2\) is always an integer and we
cannot deduce anything. (But it was worth a try.)

The delay \(\comp{isort}{n}\) laying obviously between the best and
worst case delays, we deduce, for all~\(n\geqslant{}0\),
\[
2n + 1 \leqslant \comp{isort}{n} \leqslant \tfrac{1}{2}{n^2} +
\tfrac{3}{2}{n} + 1.
\]
This naturally raises the question as to whether the delay is more
often closer to the lower bound or to the upper bound, because the
former is linear whilst the latter is quadratic. (If the bounds were
both linear or both quadratic, the issue would perhaps be a bit less
critical when considering large values of~\(n\).) Let us assume that
all integers have equal probability to be present at a given location
in the input list and are pairwisely distinct. Let \(\ave{isort}{n}\)
be the average delay of \erlcode{isort(\(L\))}, where the list~\(L\)
is comprised of \(n\)~equiprobable distinct integers. Similarly, we
write \(\ave{insert}{n}\). More specifically, the issue will hinge on
the average number of times clause~\clause{\delta} is used for an
initial list of length~\(n\), therefore let us
call~\(\aveclause{\delta}{n}\) this important number. Let us recall
here \begin{alltt}
isort(   [])               \(\smashedrightarrow{\beta}\) [];
isort([I|L])               \(\smashedrightarrow{\gamma}\) insert(I,isort(L)).
insert(I,[J|S]) when I > J \(\smashedrightarrow{\delta}\) [J|insert(I,S)];
insert(I,    S)            \(\smashedrightarrow{\epsilon}\) [I|S].
\end{alltt}
 We deduce the following equations from
them:\label{def:ave_isort}
\begin{align*}
\ave{isort}{0}   &\eqn{\beta} 1;\quad
\ave{isort}{k+1} \eqn{\gamma} 1 + \ave{insert}{k} +
              \ave{isort}{k},\,\; \text{with} \;\, k \geqslant 0;\\
\ave{insert}{0} &= 1;\quad 
\ave{insert}{k} = 1 + \aveclause{\delta}{k},
\,\; \text{from clauses \clause{\delta} and \clause{\epsilon}}.
\end{align*}
The term~\(\ave{insert}{k}\) in equation~\((\eqn{\gamma})\) is
justified because the length of \erlcode{isort(\(L\))}, for all
lists~\(L\), equals the length of~\(L\)---here assumed to
be~\(k\). Summing over~\(k\) on both sides of~\((\eqn{\gamma})\), we
deduce
\begin{align*}
\sum_{k=0}^{n-1}{\ave{isort}{k+1}} &=
  \sum_{k=0}^{n-1}{1} + \sum_{k=0}^{n-1}{\ave{insert}{k}}
  + \sum_{k=0}^{n-1}{\ave{isort}{k}}.
\end{align*}
By cancelling identical terms on both sides, we draw
\[
\ave{isort}{n} = n + \!\sum_{k=0}^{n-1}{\!\!\ave{insert}{k}} +
                    \ave{isort}{0}
                 = 1 + n + \!\!\sum_{k=0}^{n-1}{\!\left(1 +
                      \aveclause{\delta}{k}\right)}
                 = 1 + 2n +\!\!\sum_{k=0}^{n-1}{\!\!\aveclause{\delta}{k}}.
\]
Let us determine now~\(\aveclause{\delta}{k}\) by first asking
ourselves how many ways are there to insert a number into a list of
length~\(k\). Clearly, there are~\(k+1\) possible locations: before
the first item, before the second, etc., before the last \emph{and
  after the last}. Secondly, assuming that the item is to be inserted
at position~\(j\) (the head of the list is at position~\(0\)), that
is, it is aimed at being placed between the \((j-1)\)th~and~\(j\)th
element, how many times clause~\clause{\delta} is needed? If inserted
at location~\(0\), clause~\clause{\delta} is not used. If
location~\(1\), then it is used once etc. until location~\(k\), that
is, after the last item, for which clause~\clause{\delta} is utilised
\(k\)~times. Thirdly, each insertion position is equally likely
because we posited the equal probability of occurrence for the
integers in the list. Therefore, we can average the total number of
times clause~\clause{\delta} is used for all insertion positions by
the number of such positions. In other words:\label{ave_insert}
\begin{equation}
\aveclause{\delta}{k} = \frac{1}{k+1}\sum_{j=0}^{k}{j}
                      = \frac{k}{2},\quad
\ave{insert}{k}       = 1 + \aveclause{\delta}{k}
                      = \frac{k}{2} + 1.\label{ave:insert}
\end{equation}
This value could have been expected by means of the following informal
reasoning. When considering the location to insert a number, since the
numbers in the list are equi\-probable, we should expect, in average,
half of them to be lower than the number inserted. So, in average, the
number of times clause~\clause{\delta} is used is~\(k/2\). We can now
substitute~\(\aveclause{\delta}{k}\) by its simple closed form in the
equation back and conclude\label{ave_isort}
\begin{equation}
\ave{isort}{n} = 1 + 2n + \sum_{k=0}^{n-1}{\frac{k}{2}}
               = \frac{1}{4}{n^2} + \frac{7}{4}{n} + 1
\mathrel{\sim} \frac{1}{4}n^2
                 \mathrel{\sim} \frac{1}{2} \worst{isort}{n}.
\label{ave:isort}
\end{equation}
This comes as a disappointment, though, because it means that the
average delay is quadratic, just as the worst case, even if the
multiplicative constant is twice as small, asymptotically, that is,
\emph{the average delay \(\ave{isort}{n}\) is 50\% of the worst delay
  \(\worst{isort}{n}\), for large values of \(n\).}

We shall see later more asymptotically efficient sort algorithms, so
it is worth wondering why insertion sort fails short to be a good
algorithm, at least for large lists of integers. First, let us recall
that items are inserted in reverse order because of
clause~\clause{\gamma}. Thus, when clause~\clause{\delta} applies, it
means that \erlcode{I}~was originally before~\erlcode{J} and it is
found now to be greater. After the rewrite, \erlcode{J}~is located
before~\erlcode{I}. All this means that \emph{one application of
  clause~\clause{\delta} removes one inversion from the input.} In
order to go faster, we need a sorting method that removes more than a
linear number of inversions per rewrite. In the next section, we shall
study such a faster method. As a corollary of this short analysis, the
average number of inversions in a random permutation of \(n\)~objects
is
\[
\sum_{k=0}^{n-1}{\aveclause{\delta}{k}} =
\sum_{k=0}^{n-1}{\frac{k}{2}} = \frac{n(n-1)}{4}.
\]

\smallskip

\paragraph{Tail form.}

At this point, after so much mathematics, it is important to check our
results. Another verification of the previously assessed delays can be
done by modifying the definition of \erlcode{isort/1} such that it
ends with the sorted list \emph{and} the number of performed
rewrites. As we saw before, this addition is easily done if the
original definition is already in tail form. Therefore, let us first
transform \erlcode{isort/1} into tail form. Starting from
\begin{alltt}
isort(   [])               \(\smashedrightarrow{\beta}\) [];
isort([I|L])               \(\smashedrightarrow{\gamma}\) insert(I,isort(L)).
insert(I,[J|S]) when I > J \(\smashedrightarrow{\delta}\) [J|insert(I,S)];
insert(I,    S)            \(\smashedrightarrow{\epsilon}\) [I|S].
\end{alltt}

we add a tuple accumulator to our functions and initialise it with the
empty tuple (new clause~\clause{\alpha}):
\begin{alltt}
\textbf{isort_tf(L)                  \(\smashedrightarrow{\alpha}\) isort(L,\{\}).}
isort(   [],\textbf{A})               \(\smashedrightarrow{\beta}\) [];\hfill% A \emph{unused yet}
isort([I|L],\textbf{A})               \(\smashedrightarrow{\gamma}\) insert(I,isort(L,\textbf{A}),\textbf{A}).
insert(I,[J|S],\textbf{A}) when I > J \(\smashedrightarrow{\delta}\) [J|insert(I,S,\textbf{A})];
insert(I,    S,\textbf{A})            \(\smashedrightarrow{\epsilon}\) [I|S].\hfill% A \emph{unused yet}
\end{alltt}
We can now inspect each clause and, depending on its body shape (that
is: expression in tail form, either with or without a call, or not in
tail form), some transformation is done. First, the body of
clause~\clause{\beta} is in tail form and does not contain any
function call. Thus, we transform it by applying an auxiliary
function \erlcode{appk/2}:
\begin{alltt}
isort(   [],A)               \(\smashedrightarrow{\beta}\) \textbf{appk(}[],\textbf{A)};
\end{alltt}
Next is clause~\clause{\gamma}, which is not in tail form. The first
call to be evaluated is \erlcode{isort(L,A)}, whose control context is
\erlcode{insert(I,\textvisiblespace,A)}. Let us keep the call whilst
saving into the accumulator~\erlcode{A} the variable~\erlcode{I}
needed to rebuild the control context later, in a new clause of
function \erlcode{appk/2}. This variable needs a priori to be tagged
by some unique atom, say~\erlcode{k1}:
\begin{alltt}
isort([I|L],A)               \(\smashedrightarrow{\gamma}\) isort(L,\textbf{\{k1,I,A\}}).
\textbf{appk(V,\{k1,I,A\})             \(\rightarrow\) insert(I,V,A).}
\end{alltt}
The following clause is~\clause{\delta}, which is not in tail
form. The sole call to be evaluated is \erlcode{insert(I,S,A)}, whose
control context is~\erlcode{[J|\textvisiblespace]}. Let us
associate~\erlcode{J} with a unique atom~\erlcode{k2}, then save both
of them in the accumulator~\erlcode{A} and, dually, add a clause to
\erlcode{appk/2} to reconstruct the erased control context:
\begin{alltt}
insert(I,[J|S],A) when I > J \(\smashedrightarrow{\delta}\) insert(I,S,\textbf{\{k2,J,A\}});
\textbf{appk(V,\{k2,J,A\})             \(\rightarrow\) appk([J|V],A);}
\end{alltt}
The last clause is~\clause{\epsilon}, which is in tail form and
contains no call, so we must apply its body to \erlcode{appk/2} in
order to check whether there are pending control contexts to rebuild:
\begin{alltt}
insert(I,    S,A)            \(\smashedrightarrow{\epsilon}\) \textbf{appk(}[I|S],\textbf{A)}.
\end{alltt}
In order to complete the transformation, we must add a clause
to \erlcode{appk/2} to process the case when the accumulator is empty,
so the final result is found. Finally, the resulting program is (last
step in bold)\label{isort_tf_appk}
\begin{alltt}
isort_tf(L)                  \(\smashedrightarrow{\alpha}\) isort(L,\{\}).
isort(   [],A)               \(\smashedrightarrow{\beta}\) appk([],A);
isort([I|L],A)               \(\smashedrightarrow{\gamma}\) isort(L,\{k1,I,A\}).
insert(I,[J|S],A) when I > J \(\smashedrightarrow{\delta}\) insert(I,S,\{k2,J,A\});
insert(I,    S,A)            \(\smashedrightarrow{\epsilon}\) appk([I|S],A).
\textbf{appk(V,      \{\})             \(\smashedrightarrow{\zeta}\) V;}
appk(V,\{k2,J,A\})             \(\smashedrightarrow{\eta}\) appk([J|V],A);
appk(V,\{k1,I,A\})             \(\smashedrightarrow{\theta}\) insert(I,V,A).
\end{alltt}
We can remark that the atom~\erlcode{k1} is not necessary in the
definition of \erlcode{isort\_tf/1}, since all other values in the
accumulator are tagged~\erlcode{k2}:
\begin{alltt}
isort_tf(L)                  \(\smashedrightarrow{\alpha}\) isort(L,\{\}).
isort(   [],A)               \(\smashedrightarrow{\beta}\) appk([],A);
isort([I|L],A)               \(\smashedrightarrow{\gamma}\) isort(L,\textbf{\{I,A\}}).\hfill% \emph{Here}
insert(I,[J|S],A) when I > J \(\smashedrightarrow{\delta}\) insert(I,S,\{k2,J,A\});
insert(I,    S,A)            \(\smashedrightarrow{\epsilon}\) appk([I|S],A).
appk(V,      \{\})             \(\smashedrightarrow{\zeta}\) V;
appk(V,\{k2,J,A\})             \(\smashedrightarrow{\eta}\) appk([J|V],A);
appk(V,   \textbf{\{I,A\}})             \(\smashedrightarrow{\theta}\) insert(I,V,A).\hfill% \emph{and here}
\end{alltt}
It becomes obvious now that \erlcode{isort/2} reverses its first
argument in the accumulator, which is initialised in
clause~\clause{\alpha} to the empty list. For instance,
\erlcode{isort([3,8,2],\{\})}~\(\smashedrightarrow{3}\)
\erlcode{appk([],\{2,\{8,\{3,\{\}\}\}\})}. Then it
calls \erlcode{appk/2} with the same arguments, in
clause~\clause{\beta}. Hence, we would like to conclude that
\[
\erlcode{isort(\(L\),[])} \mathrel{\equiv}
\erlcode{appk([],rev(\(L\)))},
\]
which would allow us to cut out the definition of \erlcode{isort/2}
entirely. The difficulty is that \(L\)~is not a list, but a tuple. Of
course, we could write a function whose special purpose would be to
reverse this particular accumulator, but this lacks a great deal of
generality. We must realise that, by opting for an accumulator
implemented as a tuple, instead of a list, we optimised too early in
terms of memory, which impedes now our effort to legitimately optimise
the delay. Let us remember this lesson and revert to a list
accumulator. When we are done with our delay optimisations, we shall
revert to tuples. Let us resume from the definition
\begin{alltt}
isort_tf(L)                  \(\smashedrightarrow{\alpha}\) isort(L,[]).
isort(   [],A)               \(\smashedrightarrow{\beta}\) appk([],A);
isort([I|L],A)               \(\smashedrightarrow{\gamma}\) isort(L,[I|A]).
insert(I,[J|S],A) when I > J \(\smashedrightarrow{\delta}\) insert(I,S,[\{k2,J\}|A]);
insert(I,    S,A)            \(\smashedrightarrow{\epsilon}\) appk([I|S],A).
appk(V,        [])           \(\smashedrightarrow{\zeta}\) V;
appk(V,[\{k2,J\}|A])           \(\smashedrightarrow{\eta}\) appk([J|V],A);
appk(V,     [I|A])           \(\smashedrightarrow{\theta}\) insert(I,V,A).
\end{alltt}
Now we can proceed and remove \erlcode{isort/2} as follows:
\begin{alltt}
isort_tf(L)                  \(\smashedrightarrow{\alpha}\) \textbf{appk([],rev(L))}.
insert(I,[J|S],A) when I > J \(\smashedrightarrow{\delta}\) insert(I,S,[\{k2,J\}|A]);
insert(I,    S,A)            \(\smashedrightarrow{\epsilon}\) appk([I|S],A).
appk(V,      \{\})             \(\smashedrightarrow{\zeta}\) V;
appk(V,\{k2,J,A\})             \(\smashedrightarrow{\eta}\) appk([J|V],A);
appk(V,   \{I,A\})             \(\smashedrightarrow{\theta}\) insert(I,V,A).
\end{alltt}
We expect that sorting a list or the same list reversed is the same:
\begin{align*}
\erlcode{isort\_tf(\(L\))} &\mathrel{\equiv}
\erlcode{isort\_tf(rev(\(L\)))}.
\intertext{By clause \clause{\alpha}, this implies}
\erlcode{isort\_tf(\(L\))} &\mathrel{\equiv}
\erlcode{appk([],rev(rev(\(L\))))},
\end{align*}
and since \(\erlcode{rev(rev(\(L\)))}~\mathrel{\equiv}~L\), we can
replace the body of clause~\clause{\alpha} simply by
\erlcode{appk([],\textbf{L})}. Note how clause~\clause{\theta} is
still useful. We have now
\begin{alltt}
isort_tf(L)                  \(\smashedrightarrow{\alpha}\) appk([],\textbf{L}).
insert(I,[J|S],A) when I > J \(\smashedrightarrow{\delta}\) insert(I,S,[\{k2,J\}|A]);
insert(I,    S,A)            \(\smashedrightarrow{\epsilon}\) appk([I|S],A).
appk(V,        [])           \(\smashedrightarrow{\zeta}\) V;
appk(V,[\{k2,J\}|A])           \(\smashedrightarrow{\eta}\) appk([J|V],A);
appk(V,     [I|A])           \(\smashedrightarrow{\theta}\) insert(I,V,A).
\end{alltt}
We can get a shorter program at the expense of more
comparisons. Remark that when clause~\clause{\eta} applies,
\erlcode{J}~is lower than the head of~\erlcode{V}, which exists
because this clause is only used to compute the bodies of clauses
\clause{\epsilon}~and~\clause{\eta}, where the first argument is not
the empty list. Therefore,
\erlcode{appl([J|V],A)}~\(\equiv\)~\erlcode{insert(J,V,A)}, because
clause~\clause{\epsilon} would apply. Accordingly, let us change
clause~\clause{\eta}:
\begin{alltt}
isort_tf(L)                  \(\smashedrightarrow{\alpha}\) appk([],L).
insert(I,[J|S],A) when I > J \(\smashedrightarrow{\delta}\) insert(I,S,[\{k2,J\}|A]);
insert(I,    S,A)            \(\smashedrightarrow{\epsilon}\) appk([I|S],A).
appk(V,        [])           \(\smashedrightarrow{\zeta}\) V;
appk(V,[\{k2,J\}|A])           \(\smashedrightarrow{\eta}\) \textbf{insert(J,V,A)};
appk(V,     [I|A])           \(\smashedrightarrow{\theta}\) insert(I,V,A).
\end{alltt}
We can see clearly now that \erlcode{appk/2} calls \erlcode{insert/3}
in the same way in clauses \clause{\eta}~and~\clause{\theta}, which
means that it is useless to tag~\erlcode{J} with~\erlcode{k2} and we
can get rid of clause~\clause{\theta}:
\begin{alltt}
isort_tf(L)                  \(\smashedrightarrow{\alpha}\) appk([],L).
insert(I,[J|S],A) when I > J \(\smashedrightarrow{\delta}\) insert(I,S,[\textbf{J}|A]);\hfill% \emph{Here}
insert(I,    S,A)            \(\smashedrightarrow{\epsilon}\) appk([I|S],A).
appk(V,   [])                \(\smashedrightarrow{\zeta}\) V;
appk(V,[\textbf{J}|A])                \(\smashedrightarrow{\eta}\) insert(J,V,A).\hfill% \emph{and here}
\end{alltt}
Perhaps it is clearer to get rid of \erlcode{appk/2} by integrating
its two operations in \erlcode{isort\_tf/1} and
\erlcode{insert/3}. Let us split clauses
\clause{\alpha}~and~\clause{\epsilon} to manifest the cases where,
respectively, \erlcode{L}~and~\erlcode{A} are empty:
\newlength\duparrow \settowidth\duparrow{\(\alpha\sb{0}\)}
\begin{alltt}
isort_tf(   [])                  \(\MyArrow{\duparrow}{\alpha\sb{0}}\) appk([],[]);
isort_tf([I|L])                  \(\MyArrow{\duparrow}{\alpha\sb{1}}\) appk([],[I|L]).
insert(I,[J|S],    A) when I > J \(\MyArrow{\duparrow}{\delta}\) insert(I,S,[J|A]);
insert(I,    S,[J|A])            \(\MyArrow{\duparrow}{\epsilon\sb{0}}\) appk([I|S],[J|A]);
insert(I,    S,   [])            \(\MyArrow{\duparrow}{\epsilon\sb{1}}\) appk([I|S],[]).
appk(V,   [])                    \(\MyArrow{\duparrow}{\zeta}\) V;
appk(V,[J|A])                    \(\MyArrow{\duparrow}{\eta}\) insert(J,V,A).
\end{alltt}
We can now replace the bodies of clauses
\clause{\alpha_0}~and~\clause{\epsilon_1} by their value, as given by
clause~\clause{\zeta}, and we can remove~\clause{\zeta}:
\begin{alltt}
isort_tf(   [])                  \(\MyArrow{\duparrow}{\alpha\sb{0}}\) \textbf{[]};
isort_tf([I|L])                  \(\MyArrow{\duparrow}{\alpha\sb{1}}\) appk([],[I|L]).
insert(I,[J|S],    A) when I > J \(\MyArrow{\duparrow}{\delta}\) insert(I,S,[J|A]);
insert(I,    S,[J|A])            \(\MyArrow{\duparrow}{\epsilon\sb{0}}\) appk([I|S],[J|A]);
insert(I,    S,   [])            \(\MyArrow{\duparrow}{\epsilon\sb{1}}\) \textbf{[I|S]}.
appk(V,[J|A])                    \(\MyArrow{\duparrow}{\eta}\) insert(J,V,A).
\end{alltt}
We saved one rewrite in case the input list is empty. Lastly, the
bodies of clauses \clause{\alpha_1}~and~\clause{\epsilon_0} can be
replaced by their value, as given by clause~\clause{\eta}, which can
be, finally, erased. We rename the accumulator~\erlcode{A} as
\erlcode{L}.\label{code:isort_tf}
\begin{alltt}
isort_tf(   [])                  \(\MyArrow{\duparrow}{\alpha\sb{0}}\) [];
isort_tf([I|L])                  \(\MyArrow{\duparrow}{\alpha\sb{1}}\) \textbf{insert(I,[],L)}.
insert(I,[J|S],    L) when I > J \(\MyArrow{\duparrow}{\delta}\) insert(I,S,[J|L]);
insert(I,    S,[J|L])            \(\MyArrow{\duparrow}{\epsilon\sb{0}}\) \textbf{insert(J,[I|S],L)};
insert(I,    S,   [])            \(\MyArrow{\duparrow}{\epsilon\sb{1}}\) [I|S].
\end{alltt}
It is important to remember that these last steps, relative to the
removal of tag~\erlcode{k2} and so forth, make sense only because, in
assessing the delay, we take into account only the number of function
calls, not the number of comparisons, which is now greater for not
using the control context~\erlcode{[J|\textvisiblespace]} in the
original clause~\clause{\delta} of \erlcode{insert/3}. In other words,
the items saved in the accumulator in the new clause~\clause{\delta}
have to be re\hyp{}inserted in clause~\clause{\epsilon_0}. Note also
that it makes no sense to use tuples instead of a list to accumulate
\emph{single} items because the memory consumption would be exactly
the same.

The same analysis used for assessing the delay of \erlcode{isort/1}
applies here as well, except that the items are inserted in their
original order. So when the items are sorted increasingly, the delay
is here \emph{maximum} (that is, clause~\clause{\delta} is used
maximally) and when it is sorted non\hyp{}increasingly, the delay is
\emph{minimum} (that is, clause~\clause{\delta} is never used).
\label{best_worst_swapped}
\emph{If items are not repeated, the best case of \erlcode{isort/1} is
  the worst case of \erlcode{isort\_tf/1} and the worst case
  of \erlcode{isort/1} is the best case of \erlcode{isort\_tf/1}.}

This is true because ``nondecreasing'' means the same as ``increasing''
when there is no repetition and ``non\hyp{}increasing'' means
``decreasing'' under the same assumption. In order to find the lowest
delay of the final version of \erlcode{isort\_tf/1}, it is helpful
to get first a better understanding of the computational process by
unfolding a simple example like sorting \erlcode{[4,3,2,1]}, which is
a list sorted in decreasing order:
\begin{alltt}
\erlcode{isort\_tf([4,3,2,1])} \(\MyArrow{\duparrow}{\alpha\sb{1}}\) \erlcode{insert(4,\ \ \ \ \ [],[3,2,1])}
                    \(\MyArrow{\duparrow}{\epsilon\sb{0}}\) \erlcode{insert(3,\ \ \ \ [4],\ \ [2,1])}
                    \(\MyArrow{\duparrow}{\epsilon\sb{0}}\) \erlcode{insert(2,\ \ [3,4],\ \ \ \ [1])}
                    \(\MyArrow{\duparrow}{\epsilon\sb{0}}\) \erlcode{insert(1,[2,3,4],\ \ \ \ \ [])}
                    \(\MyArrow{\duparrow}{\epsilon\sb{1}}\) \erlcode{[1,2,3,4]}\textrm{.}
\end{alltt}
Let us note \(\best{isort\_tf}{n}\) the best case of \(n\)~items to be
sorted. We have \(\best{isort\_tf}{0} = 1\) by
clause~\clause{\alpha_0}. Let us assume next that \(n>0\). Then
\begin{itemize}

  \item clause~\clause{\alpha\sb{1}} is used once;

  \item clause~\clause{\delta} is not used, since we assume here that
    the items are already sorted non\hyp{}increasingly;

  \item clause~\clause{\epsilon\sb{0}} is used once for each item in
    its third argument, which, by clause~\clause{\alpha\sb{1}}, means
    all items except the first, that is \(n-1\)~times;

  \item clause~\clause{\epsilon\sb{1}} is used once.

\end{itemize}
In sum, the execution trace is \(\alpha_1\epsilon_0^{n-1}\epsilon_1\).
The total delay is then \(\best{isort\_tf}{n} = 1 + (n-1) + 1 = n+1\),
if~\(n>0\). Since we found that \(\best{isort\_tf}{0} = 1 = 0 + 1\),
we can extend the previous formula to~\(n=0\). This result can be
related directly to \(\worst{isort}{n} = (n^2 + 3n + 2)/2\), because
the best case of \erlcode{isort\_tf/1} corresponds to the worst case
of \erlcode{isort/1} when integers are not repeated. We can further
reckon that this minimum delay for \erlcode{isort\_tf/1} is also an
absolute minimum for a sorting algorithm when the input is sorted
non\hyp{}increasingly, because it is simply the delay needed to
reverse the input. If we follow the growth of the second and third
arguments of \erlcode{insert(4,[],[3,2,1])}, we recognise the
behaviour of the second and first argument, respectively, of
\erlcode{rev\_join/2}, for which \(\comp{rev\_join}{n} = n + 1\) (see
equation \eqref{delay:rev_join} \vpageref{delay:rev_join}). Given a
list~\(L\) sorted non\hyp{}increasingly, we have
\[
\erlcode{isort(\(L\))} \mathrel{\equiv} \erlcode{rev(\(L\))}.
\]
Assuming now the worst case, let us note \(\worst{isort\_tf}{n}\) the
delay of \erlcode{isort\_tf(\(L\))} where the list~\(L\) contains
\(n\)~items in increasing order. For the empty list, the execution
trace is~\(\alpha_0\). For singletons, for example, \erlcode{[5]}, it
is~\(\alpha_1\epsilon_1\). To understand the general case \(n>1\), we
can try
\begin{alltt}
\erlcode{isort\_tf([1,2,3,4])} \(\MyArrow{\duparrow}{\alpha\sb{1}}\) \erlcode{insert(1,\ \ \ \ \ [],[2,3,4])}
                    \(\MyArrow{\duparrow}{\epsilon\sb{0}}\) \erlcode{insert(2,\ \ \ \ [1],\ \ [3,4])}
                    \(\MyArrow{\duparrow}{\delta}\) \erlcode{insert(2,\ \ \ \ \ [],[1,3,4])}
                    \(\MyArrow{\duparrow}{\epsilon\sb{0}}\) \erlcode{insert(1,\ \ \ \ [2],\ \ [3,4])}
                    \(\MyArrow{\duparrow}{\epsilon\sb{0}}\) \erlcode{insert(3,\ \ [1,2],\ \ \ \ [4])}
                    \(\MyArrow{\duparrow}{\delta}\) \erlcode{insert(3,\ \ \ \ [2],\ \ [1,4])}
                    \(\MyArrow{\duparrow}{\delta}\) \erlcode{insert(3,\ \ \ \ \ [],[2,1,4])}
                    \(\MyArrow{\duparrow}{\epsilon\sb{0}}\) \erlcode{insert(2,\ \ \ \ [3],\ \ [1,4])}
                    \(\MyArrow{\duparrow}{\epsilon\sb{0}}\) \erlcode{insert(1,\ \ [2,3],\ \ \ \ [4])}
                    \(\MyArrow{\duparrow}{\epsilon\sb{0}}\) \erlcode{insert(4,[1,2,3],\ \ \ \ \ [])}
                    \(\MyArrow{\duparrow}{\delta}\) \erlcode{insert(4,\ \ [2,3],\ \ \ \ [1])}
                    \(\MyArrow{\duparrow}{\delta}\) \erlcode{insert(4,\ \ \ \ [3],\ \ [2,1])}
                    \(\MyArrow{\duparrow}{\delta}\) \erlcode{insert(4,\ \ \ \ \ [],[3,2,1])}
                    \(\MyArrow{\duparrow}{\epsilon\sb{0}}\) \erlcode{insert(3,\ \ \ \ [4],\ \ [2,1])}
                    \(\MyArrow{\duparrow}{\epsilon\sb{0}}\) \erlcode{insert(2,\ \ [3,4],\ \ \ \ [1])}
                    \(\MyArrow{\duparrow}{\epsilon\sb{0}}\) \erlcode{insert(1,[2,3,4],\ \ \ \ \ [])}
                    \(\MyArrow{\duparrow}{\epsilon\sb{1}}\) \erlcode{[1,2,3,4]}\textrm{.}
\end{alltt}
Notice the interplay of clauses
\clause{\delta}~and~\clause{\epsilon\sb{0}}. A series of application
of clause~\clause{\delta} ends with the second argument to be the
empty list. This is because the effect of clause~\clause{\delta} is to
save the contents of this argument by reversing it on top of the third
argument. In other words, in the worst case, clause~\clause{\delta} is
equivalent to
\begin{alltt}
insert(I,[J|S],L) when I > J \(\rightarrow\)
\hfill insert(I,\textbf{[]},\textbf{rev_join(S,[J|L])});
\end{alltt}
A sequence of~\clause{\delta} is followed by a series
of~\clause{\epsilon\sb{0}} \emph{of same length}, followed by another
\clause{\epsilon\sb{0}}~or~\clause{\epsilon\sb{1}}. The reason is that
clause~\clause{\epsilon\sb{0}} restores on top of the second argument
the items saved previously by clause~\clause{\delta}. Then, if there
are some items left in the last argument (to be sorted), one more
application of clause~\clause{\epsilon\sb{0}} is required, otherwise
the program ends with clause \clause{\epsilon\sb{1}}, that is, the
execution trace when \(n>1\) is
\[
\alpha_1\prod_{p=0}^{n-2}{\left(\delta^p\epsilon_0^{p+1}\right)} \cdot
\delta^{n-1}\epsilon_0^{n-1} \cdot \epsilon_1
= \alpha_1\prod_{p=0}^{n-2}{\left((\delta\epsilon_0)^p\epsilon_0\right)}
\cdot (\delta\epsilon_0)^{n-1} \cdot \epsilon_1.
\]
This observation is the key for finding the exact delay in the worst
case as it hints at counting the rewrite steps of
clause~\clause{\delta} and of clause~\clause{\epsilon\sb{0}}
\emph{together}, as evinced in the right\hyp{}hand side of the
equality. We can now directly derive the worst delay:
\begin{align*}
\worst{isort\_tf}{n}
  &= \left|\alpha_1\prod_{p=0}^{n-2}{\left((\delta\epsilon_0)^p\epsilon_0\right)}
     \cdot (\delta\epsilon_0)^{n-1} \cdot \epsilon_1\right|\\
  &= \abs{\alpha_1} +
     \left|\prod_{p=0}^{n-2}{\left((\delta\epsilon_0)^p\epsilon_0\right)}\right|
     + \left|(\delta\epsilon_0)^{n-1}\right| + \abs{\epsilon_1}\\
  &= 1 + \sum_{p=0}^{n-2}{\left|(\delta\epsilon_0)^p\epsilon_0\right|}
     + (n-1)\abs{\delta\epsilon_0} + 1\\
\worst{isort\_tf}{n}
  &= 1 + \sum_{p=0}^{n-2}{(2p+1)} + 2(n-1) + 1 = n^2 + 1.
\end{align*}
Since the worst case of \erlcode{isort\_tf/1} and
\erlcode{isort/1} are identical, we can compare their delays in this
case, for~\(n\geqslant{}0\),
\[
\worst{isort}{n}     = (n^2 + 3n + 2)/2,\qquad
\worst{isort\_tf}{n} = n^2 + 1.
\]
Let us relate now the best and worst cases for both \erlcode{isort/1}
and \erlcode{isort\_tf/1}, we have, for~\(n>3\),
\[
\best{isort\_tf}{n} < \best{isort}{n} < \worst{isort}{n} <
\worst{isort\_tf}{n}.
\]
If we note \(\comp{isort}{n}\) the delay of \erlcode{isort/1} on an
input of length~\(n\), these inequalities are equivalent to say
\[
\best{isort\_tf}{n} < \comp{isort}{n} < \worst{isort\_tf}{n}.
\]
This is the best we can do because we only have the obvious
inequalities
\[
\best{isort\_tf}{n} \leqslant \comp{isort\_tf}{n} 
\leqslant \worst{isort\_tf}{n},
\]
which do not allow us to compare \(\comp{isort}{n}\) and
\(\comp{isort\_tf}{n}\). In order to obtain a stronger result, we need
an average delay analysis so we can tell apart \erlcode{isort\_tf/1}
from \erlcode{isort/1}. Indeed, it might be that, for a given input
list of length~\(n\), most configurations of the input lead to a delay
for \erlcode{isort\_tf/1} which is actually lower than for
\erlcode{isort/1}. Let us note \(\ave{isort\_tf}{n}\) the average
number of rewrites needed to compute \erlcode{isort\_tf}{(\(L\))},
where the length of list~\(L\) is~\(n\). Similarly, we note
\(\ave{insert}{p,q}\) for the average delay of the call
\erlcode{insert(\(I\),\(P\),\(Q\))}, where list~\(P\) has length~\(p\)
and list~\(Q\) has length~\(q\). The short story is this: because the
numbers are random, the average number of times clause~\clause{\delta}
is used is~\(p/2\). Since clause \clause{\epsilon_0} purpose, as
observed before, is to put back the numbers previously moved by
clause~\clause{\delta}, we expect, in average, the same number,
\(p/2\), plus~\(1\), because clause~\clause{\epsilon_0} also prepares
the possible following use of clause \clause{\delta}. In other words,
the difference with the longest execution trace defining
\(\worst{isort}{n}\) is that the subsequences~\(\delta\epsilon_0\) are
expected to be 50\%~shorter in average, so the execution trace is, in
average,
\[
\alpha_1\prod_{p=0}^{n-2}{\left((\delta\epsilon_0)^{p/2}\epsilon_0\right)}
\cdot (\delta\epsilon_0)^{(n-1)/2} \cdot \epsilon_1,
\]
from which we deduce the average delay for~\(n>1\):
\[
\ave{isort\_tf}{n} = 1 + \sum_{p=0}^{n-2}{\left(2\cdot\frac{p}{2}+1\right)} 
                       + \left(2\cdot\frac{n-1}{2}\right) + 1
                     = \frac{1}{2}{n^2} + \frac{1}{2}{n} + 1.
\]
Elegantly, this formula extends to cope with~\(n=0,1\) and we can
compare now \(\ave{isort\_tf}{n}\) to \(\ave{isort}{n}\),
for~\(n\geqslant{}0\):
\[
\ave{isort\_tf}{n}
  = \tfrac{1}{2}{n^2} + \tfrac{1}{2}{n} + 1
  \mathrel{\sim} \tfrac{1}{2}{n^2} \mathrel{\sim} 2 \cdot
\ave{isort}{n}, \,\; \text{as} \,\; n \rightarrow \infty.
\]
In other words, \erlcode{isort\_tf/1}, in spite of being optimised, is
nevertheless 50\%~slower than the original function, \emph{in average
  for large values of~\(n\)}.

\medskip

\paragraph{Adding a counter.}

Anyway, it is always a good idea to test this kind of result also by
means of a little experiment which consists simply in modifying the
source code so the function returns its normal result paired with the
number of function calls that have been needed to find it. Since
\erlcode{isort\_tf/1} is, by definition, in tail form, this
transformation is straightforward as it consists merely in adding a
parameter (the call counter) to all function calls and incrementing it
each time an arrow is crossed (note that we append to the function
names the string~\erlcode{\_c}):
\begin{alltt}
isort_tf_c(   [])              -> \textbf{\{[],1\}};
isort_tf_c([I|L])              -> insert(I,[],L,\textbf{1}).
insert(I,[J|S],L,\textbf{C}) when I > J -> insert(I,S,[J|L],\textbf{C+1});
insert(I,S,[J|L],\textbf{C})            -> insert(J,[I|S],L,\textbf{C+1});
insert(I,S,   [],\textbf{C})            -> \textbf{\{[I|S],C+1\}}.
\end{alltt}
We can deduce in the same manner a version of \erlcode{isort/1}
which computes the delay as well by recalling the result of the
automatic transformation of \erlcode{isort/1} into tail form,
\emph{before any improvement}, as shown \vpageref{isort_tf_appk}:
\begin{alltt}
isort_tf(L)                  \(\smashedrightarrow{\alpha}\) isort(L,\{\}).
isort(   [],A)               \(\smashedrightarrow{\beta}\) appk([],A);
isort([I|L],A)               \(\smashedrightarrow{\gamma}\) isort(L,\{k1,I,A\}).
insert(I,[J|S],A) when I > J \(\smashedrightarrow{\delta}\) insert(I,S,\{k2,J,A\});
insert(I,    S,A)            \(\smashedrightarrow{\epsilon}\) appk([I|S],A).
appk(V,        [])           \(\smashedrightarrow{\zeta}\) V;
appk(V,\{k2,J,A\})             \(\smashedrightarrow{\eta}\) appk([J|V],A);
appk(V,\{k1,I,A\})             \(\smashedrightarrow{\theta}\) insert(I,V,A).
\end{alltt}
Then we can add the counters and be cautious so as not to count the
rewrites by \erlcode{appk/2}, which have no equivalent in the original
definition:\label{code:isort_c}
\begin{alltt}
isort_c(L)                     \(\smashedrightarrow{\alpha}\) isort_c(L,[],\textbf{0}).
isort_c(   [],A,\textbf{C})             \(\smashedrightarrow{\beta}\) appk([],A,\textbf{C+1});
isort_c([I|L],A,\textbf{C})             \(\smashedrightarrow{\gamma}\) isort_c(L,\{k1,I,A\},\textbf{C+1}). 
insert(I,[J|S],A,\textbf{C}) when I > J \(\smashedrightarrow{\delta}\) insert(I,S,\{k2,J,A\},\textbf{C+1});
insert(I,    S,A,\textbf{C})            \(\smashedrightarrow{\epsilon}\) appk([I|S],A,\textbf{C+1}).
appk(V,        [],\textbf{C})           \(\smashedrightarrow{\zeta}\) \textbf{\{V,C\}};
appk(V,\{k2,J,A\},\textbf{C})             \(\smashedrightarrow{\eta}\) appk([J|V],A,\textbf{C});
appk(V,\{k1,I,A\},\textbf{C})             \(\smashedrightarrow{\theta}\) insert(I,V,A,\textbf{C}).
\end{alltt}

\smallskip

\paragraph{Two\hyp{}way insertion sort.}

Let us consider again \erlcode{isort\_tf/1}, as defined
\vpageref{code:isort_tf}:
\begin{alltt}
isort_tf(   [])                  \(\MyArrow{\duparrow}{\alpha\sb{0}}\) [];
isort_tf([I|L])                  \(\MyArrow{\duparrow}{\alpha\sb{1}}\) insert(I,[],L).
insert(I,[J|S],    L) when I > J \(\MyArrow{\duparrow}{\delta}\) insert(I,S,[J|L]);
insert(I,    S,[J|L])            \(\MyArrow{\duparrow}{\epsilon\sb{0}}\) insert(J,[I|S],L);
insert(I,    S,   [])            \(\MyArrow{\duparrow}{\epsilon\sb{1}}\) [I|S].
\end{alltt}
A source of inefficiency stems from clause~\clause{\delta} moving
items from the output list, that is, the second argument of
\erlcode{insert/3} in clause~\clause{\epsilon_0}, to the input list,
that is, the third argument. After the item to be inserted has found
its place, these items will be \emph{moved back} to the output
list. One simple way to avoid this back and forth consists in keeping
an auxiliary list onto which push and pop items while searching for
the right location for insertion. This way, the input list will always
strictly decrease. The technical difference is that we have two lists
in which to insert. The first list holds the smaller items, sorted
non\hyp{}increasingly, and the second one contains greater items,
sorted nondecreasingly. The current sorted list would therefore be
obtained by reversing the first list onto the second one. Technically,
the first list is called a \emph{reversed prefix} of the current
output and the second is a \emph{suffix}. For example,
\erlcode{[5,3,1]} and \erlcode{[6,8,9]}, making the current
\erlcode{[1,3,5,6,8,9]}. Conceptually, this couple of lists implements
one list with constant\hyp{}time access to two successive, inner
items. Alternatively, it can be thought as one list with a moving
finger. Consider\label{code:unbalanced_i2w}
\begin{alltt}
i2w(L)                              \(\smashedrightarrow{\zeta}\) i2w([],[],L).
i2w(   [],    Q,     [])            \(\smashedrightarrow{\alpha}\) Q;
i2w([I|P],    Q,     [])            \(\smashedrightarrow{\beta}\) i2w(    P,[I|Q],[]);
i2w(    P,[J|Q],L=[K|_]) when J < K \(\smashedrightarrow{\gamma}\) i2w([J|P],    Q, L);
i2w([I|P],    Q,L=[K|_]) when K < I \(\smashedrightarrow{\delta}\) i2w(    P,[I|Q], L);
i2w(    P,    Q,  [K|R])            \(\smashedrightarrow{\epsilon}\) i2w(    P,[K|Q], R).
\end{alltt}
Notice how this is a definition that comes out naturally in tail form
because it is a refinement of a definition itself in tail
form. Clauses \clause{\gamma}~and~\clause{\delta} are in charge of
finding the right position for insertion, whilst
clause~\clause{\epsilon} inserts the item. There are two possible
places for that, because we have two lists and we chose to implement
the insertion by pushing the item on top of the second list, that is,
on the suffix. It would have been also correct to push it on the
reversed prefix:
\begin{alltt}
i2w(    P,    Q,  [K|R]) \(\smashedrightarrow{\epsilon}\) i2w([K|P],   Q, R)
\end{alltt}
When the insertions are over, we must construct the final sorted list
with the reversed prefix and the suffix. This situation is taken care
of by clauses \clause{\alpha}~and~\clause{\beta}, because their third
argument is the empty list (no more items left to be inserted). We
could call \erlcode{rev\_join/2}, as defined on
page~\pageref{code:rev_join}:
\verbatiminput{rev_join.def} since it is exactly what we need, but,
for didactic purposes, we prefer to keep the definition of
\erlcode{i2w/3} self\hyp{}contained, whence clauses
\clause{\alpha}~and~\clause{\beta}. Note that, because
clause~\clause{\epsilon} grows the suffix of the result, we must
reverse the first argument on top of the second to compute the
result. Let us unroll a short example.
\begin{alltt}
\erlcode{i2w([2,3,1,4])} \(\smashedrightarrow{\zeta}\) \erlcode{i2w(\ \ \ \ \ [],\ \ \ \ \ \ \ [],[2,3,1,4])}
               \(\smashedrightarrow{\epsilon}\) \erlcode{i2w(\ \ \ \ \ [],\ \ \ \ \ \ [2],\ \ [3,1,4])}
               \(\smashedrightarrow{\gamma}\) \erlcode{i2w(\ \ \ \ [2],\ \ \ \ \ \ \ [],\ \ [3,1,4])}
               \(\smashedrightarrow{\epsilon}\) \erlcode{i2w(\ \ \ \ [2],\ \ \ \ \ \ [3],\ \ \ \ [1,4])}
               \(\smashedrightarrow{\delta}\) \erlcode{i2w(\ \ \ \ \ [],\ \ \ \ [2,3],\ \ \ \ [1,4])}
               \(\smashedrightarrow{\epsilon}\) \erlcode{i2w(\ \ \ \ \ [],\ \ [1,2,3],\ \ \ \ \ \ [4])}
               \(\smashedrightarrow{\gamma}\) \erlcode{i2w(\ \ \ \ [1],\ \ \ \ [2,3],\ \ \ \ \ \ [4])}
               \(\smashedrightarrow{\gamma}\) \erlcode{i2w(\ \ [2,1],\ \ \ \ \ \ [3],\ \ \ \ \ \ [4])}
               \(\smashedrightarrow{\gamma}\) \erlcode{i2w([3,2,1],\ \ \ \ \ \ \ [],\ \ \ \ \ \ [4])}
               \(\smashedrightarrow{\epsilon}\) \erlcode{i2w([3,2,1],\ \ \ \ \ \ [4],\ \ \ \ \ \ \ [])}
               \(\smashedrightarrow{\beta}\) \erlcode{i2w(\ \ [2,1],\ \ \ \ [3,4],\ \ \ \ \ \ \ [])}
               \(\smashedrightarrow{\beta}\) \erlcode{i2w(\ \ \ \ [1],\ \ [2,3,4],\ \ \ \ \ \ \ [])}
               \(\smashedrightarrow{\beta}\) \erlcode{i2w(\ \ \ \ \ [],[1,2,3,4],\ \ \ \ \ \ \ [])}
               \(\smashedrightarrow{\alpha}\) \erlcode{[1,2,3,4]}\textrm{.}
\end{alltt}
In short, the execution trace is \((\zeta)(\epsilon)(\gamma\epsilon)
(\delta\epsilon)(\gamma^3\epsilon)(\beta^3\alpha)\). Note how the
number of times clause~\clause{\beta} is used is the number of items
left on the left list. Clause~\clause{\alpha} is used exactly once.

Let us find the delay in the best, worst and average delays for an
input list of \(n\)~items, that is, counting the function calls
instead of the comparisons. The best case will exert minimally clauses
\clause{\gamma}~and~\clause{\delta} and this minimum number of calls
turns out to be zero when the two comparisons are false. The first
item inserted does not use clauses
\clause{\gamma}~and~\clause{\delta}, but only
clause~\clause{\epsilon}, so, right after, the reversed prefix is
empty and the suffix contains this item. If we want to insert the
second item without moving the contents of the lists, and go straight
to use clause~\clause{\epsilon}, the second item must be smaller than
the first. Based on the same argument, the third item must be smaller
than the second etc. In the end, this means that \emph{the input, in
  the best case, is a list sorted non\hyp{}increasingly.} This is the
same best case as for \erlcode{isort\_tf/1}. The last steps consisting
in the reversal of the prefix, such prefix being empty, we do not even
use clause~\clause{\beta} at all---only clause~\clause{\alpha}
once. In other words, if we note \(\best{i2w}{n}\) the delay when the
input is a list of \(n\)~items in non\hyp{}increasing order, then,
because the execution trace is~\(\zeta\epsilon^n\alpha\), we have
\[
\best{i2w}{n} = n + 2.
\]
Let us assume that the input list is noted
\erlcode{[\(a_0, a_1, \dots, a_{n-1}\)]}. The worst case must
exert clauses \clause{\gamma}~and~\clause{\delta}, on the one hand,
and clauses \clause{\alpha}~and~\clause{\beta}, on the other hand,
maximally. Let us focus first on maximising the use of
\clause{\gamma}~and~\clause{\delta}. Since \(a_0\)~is the first item,
it is always pushed on the suffix list by
clause~\clause{\epsilon}. The second item,~\(a_1\), in order to travel
the furthest, has to be inserted below~\(a_0\). By doing so,
clause~\clause{\gamma} is used once and then~\clause{\epsilon},
therefore, as a result, \(a_0\)~is on the left (the reversed prefix,
which is the first argument) and \(a_1\)~on the right (the suffix,
which is the second argument). In other words: we have
\erlcode{[\(a_0\)]}~and~\erlcode{[\(a_1\)]}. Because of this symmetry,
in pursuit of the worst case, we can now move either
\(a_0\)~or~\(a_1\) to the facing list, that is, choose either to set
\(a_2<a_0\) or \(a_1<a_2\).
\begin{itemize}

  \item If \(a_2<a_0\), clause~\clause{\delta} is used once, then
    clause~\clause{\epsilon}. As a result, we have the configuration
    \erlcode{[]} and \erlcode{[\(a_2\),\(a_0\),\(a_1\)]}. This
    translates as \(a_2< a_0< a_1\). The fourth item,~\(a_3\), must
    be inserted at the bottom of the right list, which must be first
    reversed on top of the left list by clause~\clause{\gamma}: we
    then have \erlcode{[\(a_1\),\(a_0\),\(a_2\)]} and
    \erlcode{[\(a_3\)]}. In other words: \( a_2< a_0< a_1< a_3\).
    Finally, the left list is reversed on top of the second by
    clause~\clause{\beta} and clause~\clause{\alpha} is last. The
    execution trace is
    \((\zeta)(\epsilon)(\gamma\epsilon)(\delta\epsilon)(\gamma^3\epsilon)
    (\beta^3\alpha)\), whose length is~\(14\).

  \item If \(a_1<a_2\), we would have \erlcode{[\(a_1\),\(a_0\)]}
    and \erlcode{[\(a_2\)]}, then the lists \erlcode{[]} and
    \erlcode{[\(a_3\),\(a_0\),\(a_1\),\(a_2\)]}, that is, \(a_3<
    a_0< a_1< a_2\). The complete execution trace:
    \((\zeta)(\epsilon)(\gamma\epsilon)(\gamma\epsilon)(\delta^2\epsilon)
    (\alpha)\). The length of this trace is~\(10\), which is shorter
    than the previous trace.
\end{itemize}
As a conclusion, the choice \(a_2<a_0\) leads to a worse case. But
what if the input list contains an odd number~\(n\) of items? To guess
what happens, let us insert~\(a_4\) assuming either \(a_1<a_2\) or
\(a_2<a_0\).
\begin{itemize}

  \item If \(a_2 < a_0\), we should move out all the items of the left
    list, so we draw the configuration \erlcode{[\(a_4\)]} and
    \erlcode{[\(a_2\),\(a_0\),\(a_1\),\(a_3\)]}: \(a_4< a_2< a_0< a_1<
    a_3\), corresponding to the execution trace
    \((\zeta)(\epsilon)(\gamma\epsilon)(\delta\epsilon)(\gamma^3\epsilon)
    (\delta^3\epsilon)(\beta\alpha)\), whose length is~\(16\). 

  \item If \(a_1 < a_2\), we want to insert~\(a_4\) at the bottom of
    the second list, thus obtaining
    \erlcode{[\(a_2\),\(a_1\),\(a_0\),\(a_3\)]} and
    \erlcode{[\(a_4\)]}: \(a_3 < a_0 < a_1 < a_2 < a_4\),
    corresponding to the execution trace \((\zeta)(\epsilon)
    (\gamma\epsilon)(\gamma\epsilon)(\delta^2\epsilon)(\gamma^4\epsilon)
    (\beta^4\alpha)\), whose length is~\(19\). It is
    perhaps better visualised by means of oriented edges, revealing a
    spiral in \fig~\vref{fig:spiral}.
    \begin{figure}
      \centering
      \includegraphics[bb=71 671 174 715]{a3a0a1a2a4}
      \caption{Worst case for \erlcode{i2w/1} if \(n=5\) (\(a_1 < a_2\))
               \label{fig:spiral}}
    \end{figure}


\end{itemize}
Therefore, it seems that when the number of items is odd, having
\(a_1<a_2\) leads to the worst delay, whilst \(a_2<a_0\) leads to
the worst delay when the number of items is even. Let us determine
these delays for any~\(n\) and determine which is the greater. Let us
note \(\worst{\(a_1\!\!<\!\!a_2\)}{2p+1}\) the former delay and
\(\worst{\(a_2\!\!<\!\!a_0\)}{2p}\) the latter.
\begin{itemize}

  \item If \(n=2p+1\) and \(a_1<a_2\), then the execution trace is
    \[
    (\zeta)(\epsilon)(\gamma\epsilon)(\gamma\epsilon)
    (\delta^2\epsilon)(\gamma^4\epsilon)(\delta^4\epsilon) \ldots
    (\gamma^{2p-2}\epsilon)(\delta^{2p-2}\epsilon)(\gamma^{2p}\epsilon)
    (\beta^{2p}\alpha),
    \]
    as a run with~\(p=3\) suggests:
\settowidth\duparrow{\(\gamma^6\)}
\begin{tabbing}
\erlcode{i2w([\(a_0\),\(a_1\),\(a_2\),\(a_3\),\(a_4\)])} \=
\(\MyArrow{\duparrow}{\zeta}\) \=\kill
\erlcode{i2w([\(a_0\),\(a_1\),\(a_2\),\(a_3\),\(a_4\),\(a_5\),\(a_6\)])}\\
\> \(\MyArrow{\duparrow}{\zeta}\)
\> \erlcode{i2w([],[],[\(a_0\),\(a_1\),\(a_2\),\(a_3\),\(a_4\),\(a_5\),\(a_6\)])}\\
\> \(\MyArrow{\duparrow}{\epsilon}\)
\> \erlcode{i2w([],[\(a_0\)],[\(a_1\),\(a_2\),\(a_3\),\(a_4\),\(a_5\),\(a_6\)])}\\
\> \(\MyArrow{\duparrow}{\gamma}\)
\> \erlcode{i2w([\(a_0\)],[],[\(a_1\),\(a_2\),\(a_3\),\(a_4\),\(a_5\),\(a_6\)])}\\
\> \(\MyArrow{\duparrow}{\epsilon}\)
\> \erlcode{i2w([\(a_0\)],[\(a_1\)],[\(a_2\),\(a_3\),\(a_4\),\(a_5\),\(a_6\)])}\\
\> \(\MyArrow{\duparrow}{\gamma}\)
\> \erlcode{i2w([\(a_1\),\(a_0\)],[],[\(a_2\),\(a_3\),\(a_4\),\(a_5\),\(a_6\)])}\\
\> \(\MyArrow{\duparrow}{\epsilon}\)
\> \erlcode{i2w([\(a_1\),\(a_0\)],[\(a_2\)],[\(a_3\),\(a_4\),\(a_5\),\(a_6\)])}\\
\> \(\MyArrow{\duparrow}{\delta^2}\)
\> \erlcode{i2w([],[\(a_0\),\(a_1\),\(a_2\)],[\(a_3\),\(a_4\),\(a_5\),\(a_6\)])}\\
\> \(\MyArrow{\duparrow}{\epsilon}\)
\> \erlcode{i2w([],[\(a_3\),\(a_0\),\(a_1\),\(a_2\)],[\(a_4\),\(a_5\),\(a_6\)])}\\
\> \(\MyArrow{\duparrow}{\gamma^4}\)
\> \erlcode{i2w([\(a_2\),\(a_1\),\(a_0\),\(a_3\)],[],[\(a_4\),\(a_5\),\(a_6\)])}\\
\> \(\MyArrow{\duparrow}{\epsilon}\)
\> \erlcode{i2w([\(a_2\),\(a_1\),\(a_0\),\(a_3\)],[\(a_4\)],[\(a_5\),\(a_6\)])}\\
\> \(\MyArrow{\duparrow}{\delta^4}\)
\> \erlcode{i2w([],[\(a_3\),\(a_0\),\(a_1\),\(a_2\),\(a_4\)],[\(a_5\),\(a_6\)])}\\
\> \(\MyArrow{\duparrow}{\epsilon}\)
\> \erlcode{i2w([],[\(a_5\),\(a_3\),\(a_0\),\(a_1\),\(a_2\),\(a_4\)],[\(a_6\)])}\\
\> \(\MyArrow{\duparrow}{\gamma^6}\)
\> \erlcode{i2w([\(a_4\),\(a_2\),\(a_1\),\(a_0\),\(a_3\),\(a_5\)],[],[\(a_6\)])}\\
\> \(\MyArrow{\duparrow}{\epsilon}\)
\> \erlcode{i2w([\(a_4\),\(a_2\),\(a_1\),\(a_0\),\(a_3\),\(a_5\)],[\(a_6\)],[])}\textrm{.}
\end{tabbing}

    \noindent If we omit clauses~\clause{\zeta}, \clause{\epsilon},
    \clause{\alpha}~and~\clause{\beta}, we can see a pattern emerge
    from the sub\hyp{}trace
    \((\gamma^2\delta^2)(\gamma^4\delta^4)(\gamma^6\delta^6) \ldots
    (\gamma^{2p-2}\delta^{2p-2})(\gamma^{2p})\). Clause~\clause{\epsilon}
    is used \(n\)~times because it inserts the item in the right
    place. So the total delay is
    \begin{align*}
      \worst{\(a_1\!\!<\!\!a_2\)}{2p+1}
        &= \abs{\zeta} + \abs{\epsilon^{2p+1}}
           + \sum_{k=1}^{p-1}{\left(\abs{\gamma^{2k}} + \abs{\delta^{2k}}\right)}
           + \abs{\gamma^{2p}} + \abs{\beta^{2p}\alpha}\\
        &= 1 + (2p + 1) + \sum_{k=1}^{p-1}{2(2k)} + (2p) + (2p + 1)
         = 2p^2 + 4p + 3.
    \end{align*}

  \item If \(n=2p\), and \(a_2<a_0\), then the execution trace is
    \[
      (\zeta)(\epsilon)(\gamma\epsilon)(\delta\epsilon)
      (\gamma^3\epsilon)(\delta^3\epsilon)
      \ldots (\gamma^{2p-1}\epsilon)(\beta^{2p-1}\alpha),
    \]
    as the following run with~\(p=3\) suggests (first difference with
    the previous case is in bold):%XXXXXXXXXXX
\begin{alltt}
\erlcode{i2w([\(a\sb{0}\),\(a\sb{1}\),\(a\sb{2}\),\(a\sb{3}\),\(a\sb{4}\),\(a\sb{5}\)])}
\(\MyArrow{\duparrow}{}\)\,\erlcode{i2w(\ \ \ \ \ \ \ \ \ \ \ \ [],\ \ \ \ \ \ \ \ \ \ \ \ \ \ [],[\(a\sb{0}\!\),\(a\sb{1}\!\),\(a\sb{2}\!\),\(a\sb{3}\!\),\(a\sb{4}\!\),\(a\sb{5}\!\)])}
\(\MyArrow{\duparrow}{\epsilon}\)\,\erlcode{i2w(\ \ \ \ \ \ \ \ \ \ \ \ [],\ \ \ \ \ \ \ \ \ \ \ \ [\(a\sb{0}\)],\ [\(a\sb{1}\),\(a\sb{2}\),\(a\sb{3}\),\(a\sb{4}\),\(a\sb{5}\)])}
\(\MyArrow{\duparrow}{\gamma}\)\,\erlcode{i2w(\ \ \ \ \ \ \ \ \ \ [\(a\sb{0}\)],\ \ \ \ \ \ \ \ \ \ \ \ \ \ [],\ [\(a\sb{1}\),\(a\sb{2}\),\(a\sb{3}\),\(a\sb{4}\),\(a\sb{5}\)])}
\(\MyArrow{\duparrow}{\epsilon}\)\,\erlcode{i2w(\ \ \ \ \ \ \ \ \ \ [\(a\sb{0}\)],\ \ \ \ \ \ \ \ \ \ \ \ [\(a\sb{1}\)],\ \ \ \ [\(a\sb{2}\),\(a\sb{3}\),\(a\sb{4}\),\(a\sb{5}\)])}
\(\MyArrow{\duparrow}{\delta}\)\,\(\boldsymbol{\erlcode{i2w(\ \ \ \ \ \ \ \ \ \ \ \ [],\ \ \ \ \ \ \ \ [\(a\sb{0}\),\(a\sb{1}\)],\ \ \ [\(a\sb{2}\),\(a\sb{3}\),\(a\sb{4}\),\(a\sb{5}\)])}}\)
\(\MyArrow{\duparrow}{\epsilon}\)\,\erlcode{i2w(\ \ \ \ \ \ \ \ \ \ \ \ [],\ \ \ \ \ \ [\(a\sb{2}\),\(a\sb{0}\),\(a\sb{1}\)],\ \ \ \ \ \ \ [\(a\sb{3}\),\(a\sb{4}\),\(a\sb{5}\)])}
\(\MyArrow{\duparrow}{\gamma\sp{3}}\)\,\erlcode{i2w(\ \ \ \ [\(a\sb{1}\),\(a\sb{0}\),\(a\sb{2}\)],\ \ \ \ \ \ \ \ \ \ \ \ \ \ [],\ \ \ \ \ \ \ [\(a\sb{3}\),\(a\sb{4}\),\(a\sb{5}\)])}
\(\MyArrow{\duparrow}{\epsilon}\)\,\erlcode{i2w(\ \ \ \ [\(a\sb{1}\),\(a\sb{0}\),\(a\sb{2}\)],\ \ \ \ \ \ \ \ \ \ \ \ [\(a\sb{3}\)],\ \ \ \ \ \ \ \ \ \ [\(a\sb{4}\),\(a\sb{5}\)])}
\(\MyArrow{\duparrow}{\delta\sp{3}}\)\,\erlcode{i2w(\ \ \ \ \ \ \ \ \ \ \ \ [],\ \ \ [\(a\sb{2}\),\(a\sb{0}\),\(a\sb{1}\),\(a\sb{3}\)],\ \ \ \ \ \ \ \ \ \ [\(a\sb{4}\),\(a\sb{5}\)])}
\(\MyArrow{\duparrow}{\epsilon}\)\,\erlcode{i2w(\ \ \ \ \ \ \ \ \ \ \ \ [],[\(a\sb{4}\),\(a\sb{2}\),\(a\sb{0}\),\(a\sb{1}\),\(a\sb{3}\)],\ \ \ \ \ \ \ \ \ \ \ \ \ [\(a\sb{5}\)])}
\(\MyArrow{\duparrow}{\gamma\sp{5}}\)\,\erlcode{i2w([\(a\sb{3}\!\),\(a\sb{1}\!\),\(a\sb{0}\!\),\(a\sb{2}\!\),\(a\sb{4}\!\)],\ \ \ \ \ \ \ \ \ \ \ \ \ \;[],\ \ \ \ \ \ \ \ \ \ \ \ \ [\(a\sb{5}\)])}
\(\MyArrow{\duparrow}{\epsilon}\)\,\erlcode{i2w([\(a\sb{3}\!\),\(a\sb{1}\!\),\(a\sb{0}\!\),\(a\sb{2}\!\),\(a\sb{4}\!\)],\ \ \ \ \ \ \ \ \ \ \ \;[\(a\sb{5}\)],\ \ \ \ \ \ \ \ \ \ \ \ \ \ \ [])}
\end{alltt}
If we omit clauses~\clause{\zeta}, \clause{\epsilon},
\clause{\alpha}~and~\clause{\beta}, we can see a pattern emerge from
the sub\hyp{}trace
\((\gamma^1\delta^1)(\gamma^3\delta^3)(\gamma^5\delta^5) \ldots
(\gamma^{2p-3}\delta^{2p-3})(\gamma^{2p-1})\). Clause~\clause{\epsilon}
is used \(n\)~times because it inserts the item in the right place. So
the total delay is
\begin{align*}
\worst{\(a_2\!\!<\!\!a_0\)}{2p}
   &= \abs{\zeta} + \abs{\epsilon^{2p}}
           + \sum_{k=1}^{p-1}{\left(\abs{\gamma^{2k-1}} + \abs{\delta^{2k-1}}\right)}
           + \abs{\gamma^{2p-1}} + \abs{\beta^{2p-1}\alpha}\\
   &= 1 + (2p) + \sum_{k=1}^{p-1}{2(2k-1)} + (2p-1) + ((2p - 1) + 1)\\
   &= 2p^2 + 2p + 2.
\end{align*}
\end{itemize}
These formulas hold for all \(p\geqslant{}0\). We can now conclude
this section about the worst case of \erlcode{i2w/1}:
\begin{itemize}

  \item If \(n = 2p\), the worst case happens when the items satisfy
    the total order \(a_{2p} < a_{2p-2} < \dots < a_0 < a_1 < a_3 <
    \dots < a_{2p-3} < a_{2p-1}\) and \(\worst{i2w}{2p} = 2p^2 + 2p +
    2\), that is, \(\worst{i2w}{n} = \frac{1}{2}{n^2} + n + 2\).

  \item If \(n = 2p+1\), they satisfy \(a_{2p-1} < a_{2p-3} < \dots <
    a_3 < a_0 < a_1 < a_2 < \dots < a_{2p-2} < a_{2p}\) and
    \(\worst{i2w}{2p+1} = 2p^2 + 4p + 3\), that is, \(\worst{i2w}{n} =
    \frac{1}{2}{n^2} + n + \frac{3}{2}\).

\end{itemize}
These two delays are asymptotically equivalent to the worst delay of
\erlcode{isort/1}, which is quadratic:
\[
\worst{i2w}{n} \mathrel{\sim} \frac{1}{2}{n^2} \mathrel{\sim}
\worst{isort}{n},\,\; \text{as} \,\; n \rightarrow \infty.
\]
Despite our efforts, the two\hyp{}way insertion sort does not beat the
one\hyp{}way insertion sort in the worst case. Could it be better in
average, though?

Let us opt for a direct approach, based on an enumeration, which
avoids recurrence equations. Let \(\ave{i2w}{n}\) be the average delay
of \erlcode{i2w(\(L\))}, where \(L\)~is a list of \(n\)~distinct
integers uniformly chosen at random. Let
\(\ave{\(\gamma\delta\epsilon\)}{n}\) be the average number of times
clauses~\clause{\gamma}, \clause{\delta}~and~\clause{\epsilon} are
used, that is, it represents the average delay required to insert a
random number into two random lists of \(n\)~numbers in total. Let
\(\ave{\(\alpha\beta\)}{n}\) the average number of times clauses
\clause{\alpha}~and~\clause{\beta} are used, that is, it denotes the
average delay for reversing the final reversed prefix. This
decomposition of delays comes from the observation that the evaluation
first makes one step in order to set the initial values of the two
lists. Then each item from the input list is inserted. Finally, the
left list is reversed on top of the second. %We have,
\[
\ave{i2w}{0} := 2;\quad
\ave{i2w}{n}
  := 1 + \sum_{k=0}^{n-1}{\ave{\(\gamma\delta\epsilon\)}{k}}
       + \frac{1}{n}\sum_{k=0}^{n-1}{\ave{\(\alpha\beta\)}{k}},\quad
       \text{for} \,\; n > 0.
\]
The summation bounds of \(\ave{\(\alpha\beta\)}{k}\) deserve some
precise justification. Summing from \(0\)~to~\(n-1\) means that we
know that the result of a random insertion may lead to configurations
where the left list contains a number of items ranging from
\(0\)~to~\(n-1\). Actually, it means more than that, as it supposes
that each post\hyp{}insertion configuration is unique. Examining
clause~\clause{\epsilon} reminds us that we always insert on the right
list. Is this going to tip the scale in favour of the right list? Or
does that mean that, since we could also choose to insert on the left
list, the problem is perfectly symmetric and, therefore, it does not
matter in the end?
\begin{figure}[h]
\centering
\includegraphics{two-way_unbalanced}
\caption{Unbalanced two\hyp{}way insertions of integers \(a\), \(b\)
  and \(c\)
\label{fig:two-way_unbalanced}}
\end{figure}

If intuition deserts us, let us turn to
\fig~\vref{fig:two-way_unbalanced} where a small, albeit not too
small, example is fully detailed. We see how all permutations of three
objects are generated once.  A longer example would give us more
confidence in concluding that, after an insertion in a configuration
whose two lists contain a total of \(k\)~numbers, all configurations
of \(k+1\)~numbers are present exactly once, except the one where the
second list is empty, which is missing. This last point had to be
expected because the insertion always takes place on the second
list. Therefore the summation bounds of \(\ave{\(\alpha\beta\)}{k}\)
must be \(0\)~and~\(n-1\) (but not~\(n\) because the second list
cannot be empty by construction) and the sum must be averaged by~\(n\)
because the result of \(n\)~insertions on an empty configuration leads
to \(n\)~unique configurations of \(n\)~numbers. Let us resume our
calculations by noting that \(\ave{\(\alpha\beta\)}{k} = k + 1\) and
\(\ave{\(\gamma\delta\epsilon\)}{0} = 1\):
\begin{align*}
\ave{i2w}{n} &=
  1 + \left(\ave{\(\gamma\delta\epsilon\)}{0} +
  \sum_{k=1}^{n-1}{\ave{\(\gamma\delta\epsilon\)}{k}}\right)
  + \frac{1}{n}\sum_{k=0}^{n-1}{(k+1)}\\
 &= 1 + 1
    + \sum_{k=1}^{n-1}{\ave{\(\gamma\delta\epsilon\)}{k}}
    + \frac{1}{n}\sum_{k=1}^{n}{k}
  = \frac{1}{2}{n} + \frac{5}{2} +
    \sum_{k=1}^{n-1}{\ave{\(\gamma\delta\epsilon\)}{k}}.
\end{align*}
Let us note \(\ave{\(\gamma\delta\epsilon\)}{p,q}\) the average delay
for inserting one number in a configuration where the left list is
made of \(p\)~numbers and the right list contains \(q\)~numbers, all
of them being unique. Given the total number~\(k\) of numbers in the
two lists, all the possible values for \(p\)~and~\(q\) must satisfy
\(p+q=k\), with \(p\geqslant{}0\) and \(q>0\) (the second list can
never be empty). So we have
\begin{align*}
\ave{\(\gamma\delta\epsilon\)}{k}
  &:= \frac{1}{k}
              \sum_{p+q=k}^{q>0}{\ave{\(\gamma\delta\epsilon\)}{p,q}}
   = \frac{1}{k}\sum_{q=1}^{k}{\ave{\(\gamma\delta\epsilon\)}{k-q,q}}.
\end{align*}
The reason for the division by~\(k\) is that there are \(k\)~possible
configurations with a total of \(k\)~numbers shared over the two
lists. In order to compute \(\ave{\(\gamma\delta\epsilon\)}{p,q}\),
given a configuration with \(p\)~and~\(q\) numbers, we have to
enumerate all the places between two numbers where we can insert
another one. Bear in mind that no number is actually slipped in, since
insertion always occurs on the top of the second list by
clause~\clause{\epsilon}; what we really mean is that these places are
possible locations for insertion in the total order corresponding to
the configuration at hand.

For example, if the two lists are \erlcode{[6,4,0]} and
\erlcode{[8,9]}, target locations for an insertion are evinced by a
place\hyp{}holder \erlcode{\textvisiblespace} in the left list as
\erlcode{[6,\textvisiblespace,4,0]},
\erlcode{[6,4,\textvisiblespace,0]} and
\erlcode{[6,4,0,\textvisiblespace]}; in the right list, we have the
possibilities \erlcode{[\textvisiblespace,8,9]},
\erlcode{[8,\textvisiblespace,9]} and
\erlcode{[8,9,\textvisiblespace]}. There are \(p+q+1\) such locations
in general: bottom of the second list up to its top, accounting for
\(q+1\), then bottom of the first list up to the place after the first
item (if any), accounting for~\(p\). The location on top of the first
list cannot be accounted for since it corresponds to the same position
in the total order as the top of the second list; in the previous
example, there can only be one location for
\erlcode{\textvisiblespace} in \(\erlcode{0}< \erlcode{4}<
\erlcode{6}< \erlcode{\textvisiblespace}< \erlcode{8}< \erlcode{9}\),
which consists in having \erlcode{[6,4,0]} and
\erlcode{[\textvisiblespace,8,9]}, \emph{not}
\erlcode{[\textvisiblespace,6,4,0]} and \erlcode{[8,9]}.

The cumulated delay \(\ave{\(\delta\epsilon\)}{p}\) of targeting all
the valid locations in a first list of length~\(p\) is
\(\sum_{j=1}^{p}{(j+1)}\), where the index~\(j\) ranges over the
locations, \(1\)~meaning ``after the item at position \(1\),'' the
summed~\(j\) is the number of times clause \clause{\delta} is used and
the summed~\(1\) accounts for the use of
clause~\clause{\epsilon}. Similarly, the cumulated delay
\(\ave{\(\gamma\epsilon\)}{q}\) of targeting all the valid locations
in the second list of length~\(q\) is \(\sum_{j=0}^{q}{(j+1)}\),
using~\(j\) times clause~\clause{\gamma} instead
of~\clause{\delta}. Thus, we have\label{A_pq}
\begin{align*}
\ave{\(\gamma\delta\epsilon\)}{p,q}
  &:= \frac{\ave{\(\delta\epsilon\)}{p} +
     \ave{\(\gamma\epsilon\)}{q}}{p+q+1}
   = \frac{1}{p+q+1}{\left(\sum_{j=1}^{p}{(j+1)}
     + \sum_{j=0}^{q}{(j+1)}\!\right)}\\
  &= \frac{1}{p+q+1}{\left(\sum_{j=2}^{p+1}{j} +
     \sum_{j=1}^{q+1}{j}\!\right)}
   = \frac{p^2 + 3p + q^2 + 3q + 2}{2p + 2q + 2}.
\intertext{Notice the complete symmetry of the expression:
  \(\ave{\(\gamma\delta\epsilon\)}{p,q} =
  \ave{\(\gamma\delta\epsilon\)}{q,p}\). Anyway,}
\ave{\(\gamma\delta\epsilon\)}{k-q,q}
  &= \frac{(k-q)^2 + 3(k-q) + q^2 + 3q + 2}{2(k-q) + 2q + 2}
  = \frac{q^2}{k+1} - \frac{kq}{k+1} + \frac{k+2}{2}.
\intertext{We can now replace
  \(\ave{\(\gamma\delta\epsilon\)}{k-q,q}\) in the definition of
  \(\ave{\(\gamma\delta\epsilon\)}{k}\):}
\ave{\(\gamma\delta\epsilon\)}{k}
  &= \frac{1}{k}\left(\frac{1}{k+1}\sum_{q=1}^{k}{q^2} -
\frac{k}{k+1}\sum_{q=1}^{k}{q} +
\frac{k+2}{2}\sum_{p=1}^{k}{1}\!\right)
  = \frac{1}{3}{k} + \frac{7}{6}.
\end{align*}
Finally, we can substitute \(\ave{\(\gamma\delta\epsilon\)}{k}\) for
its newly found value in the definition of \(\ave{i2w}{n}\):
\begin{equation*}
\ave{i2w}{0} = 2;\quad
\ave{i2w}{n}
  = \frac{1}{2}{n} + \frac{5}{2} + \frac{1}{3}\sum_{k=1}^{n-1}{k}
     + \frac{7}{6}\sum_{k=1}^{n-1}{1}
   = \frac{1}{6}{n^2} + \frac{3}{2}{n} + \frac{4}{3}
   \mathrel{\sim} \frac{1}{6}{n^2}.
\end{equation*}
The bad news is that the average delay of \erlcode{i2w/1} is
quadratic, just as the other versions of insertion sort, but the good
news is that the coefficient is smaller. Compare with
\[
\ave{isort}{n}     = \tfrac{1}{4}{n^2} + \tfrac{7}{4}{n} + 1
\,\; \text{and} \,\;
\ave{isort\_tf}{n} = \tfrac{1}{2}{n^2} + \tfrac{1}{2}{n} + 1.
\]
Another remark is that the average delay becomes closer to~33\% of the
worst case as~\(n\) grows, since we proved
\[
\worst{i2w}{n} \mathrel{\sim} \tfrac{1}{2}{n^2},\;\, \text{as}
\,\; n \rightarrow \infty.
\]

\medskip

\paragraph{Balanced two\hyp{}way insertion sort.}

Is there a way to improve on the previous two\hyp{}way insertion? One
source of inefficiency stems from the fact that we wanted to retain
the tail form, therefore an insertion in one of the two lists
required, in general, to move some items to the other list. This
results often in unbalanced lists, so subsequent insertions in the
longest list will incur a greater delay. Let us design a
\emph{balanced two\hyp{}way insertion}, where the lengths of the two
lists differ by one at most and where we use the control context to
hold the items during the one\hyp{}way insertion phase in each of the
two lists. To understand how this could work, let us imagine that we
already have a configuration with two lists of same length. If we
insert a number in the first list, the first list will jut over the
second and, symmetrically, if we insert in the second, the second will
stand higher than the first. In order to have some regularity and a
shorter definition later, let us assume that, in this case, we want
the second list to always be taller after the insertion. This enables
dealing with only two situations instead of three, which are (1)~the
two lists have same length, (2)~the second list is longer by one. Let
us assume now that we insert a number when the second list is longer
than the first one. Then, if the integer has to be inserted in the
first list, the resulting lists will have equal lengths, which means
we go back to case (1), otherwise, we move the top of the second list
to the top of the first list in addition to the insertion itself, and
we go back to case (1) as well. If we are in case~(1) and the
insertion takes place in the second stack, no rebalancing has to be
done; otherwise, the top of the first stack is moved to the top of the
second: in both events, we are in case~(2). But what if the item to
insert has to be simply pushed on top of one of the two lists? If the
two lists have same length, let us push the number on top of the
second one, otherwise it means that the second exceeds the first by
one, so it is best to push it on the first list, so, as a result, the
lists have equal lengths and we are back to case (1).

To implement this design, we shall need an additional parameter, which
represents the difference in length between the lists. Since we opt
for never having the first list jut over the second, this parameter
will range just over two values exactly, more precisely, it will
toggle between these two values: after an insertion, if the two lists
had equal lengths, then the second one will be taller, otherwise, they
will be of equal length. Usually, this type of parameter should be
bound to two axioms, with explicit names telling the story, like
\erlcode{balanced} and \erlcode{unbalanced}, but here it is simple and
quite meaningful to give a numeric interpretation of this parameter as
the length of the second list minus the length of the first, so this
difference is an integer whose value is either
\erlcode{0}~or~\erlcode{1}. Because the first list is the reversed
prefix and the second is the suffix of the current sorted list, we
need two one\hyp{}way insertions: \erlcode{ins\_dn(\(I\),\(S\))}
inserts the number~\(I\) into the list~\(S\) which is sorted
non\hyp{}increasingly; \erlcode{ins\_up(\(I\),\(S\))} inserts~\(I\)
into~\(S\) sorted non\hyp{}decreasingly. These two functions are
easily defined, as the latter is none other than \erlcode{insert/2}
\vpageref{code:insert} in disguise:
\begin{alltt}
insert(I,[J|S]) when I > J -> [J|insert(I,S)];
insert(I,    S)            -> [I|S].

ins_up(I,[J|S]) when I > J \(\smashedrightarrow{\iota}\) [J|ins_up(I,S)];
ins_up(I,    S)            \(\smashedrightarrow{\kappa}\) [I|S].
\end{alltt}
and the former is derived by inverting the comparison in the guard:
\begin{alltt}
ins_dn(I,[J|S]) when J > I \(\smashedrightarrow{\lambda}\) [J|ins_dn(I,S)];
ins_dn(I,    S)            \(\smashedrightarrow{\mu}\) [I|S].
\end{alltt}
Now we add the difference parameter, which is initialised
with~\erlcode{0} because the two lists are initially empty:
\begin{alltt}
i2wb(L) -> i2wb([],[],L,\textbf{0}).
\end{alltt}
Function \erlcode{i2wb/3} \vpageref{code:unbalanced_i2w} consists of
two groups of clauses: clauses from~\clause{\gamma}
to~\clause{\epsilon} perform the comparisons, moving around the
numbers until the insertion, properly speaking, takes place (a simple
push on the right list); clauses \clause{\alpha}~and~\clause{\beta}
perform the final steps consisting in reversing the first list on top
of the second and returning it as the result. The definition of
\erlcode{i2wb/4} will contain one more group of rules, stemming from
the splitting of the group dedicated to comparisons into two
subgroups, depending on the difference of lengths of the two lists,
named~\erlcode{D} in the following code schema:
\begin{alltt}
i2wb(   [],    Q,   [],_)            \(\smashedrightarrow{\alpha}\) Q;
i2wb([I|P],    Q,   [],D)            \(\smashedrightarrow{\beta}\) i2wb(P,[I|Q],[],D);

i2wb(    P,[J|Q],[K|R],\textbf{0}) when J < K \(\smashedrightarrow{\gamma}\) \fbcode{HHHHHHHHHHHHH};
i2wb([I|P],    Q,[K|R],\textbf{0}) when K < I \(\smashedrightarrow{\delta}\) \fbcode{HHHHHHHHHHHHH};
i2wb(    P,    Q,[K|R],\textbf{0})            \(\smashedrightarrow{\epsilon}\) \fbcode{HHHHHHHHHHHHH};

i2wb(    P,[J|Q],[K|R],\textbf{1}) when J < K \(\smashedrightarrow{\zeta}\) \fbcode{HHHHHHHHHHHHH};
i2wb([I|P],    Q,[K|R],\textbf{1}) when K < I \(\smashedrightarrow{\eta}\) \fbcode{HHHHHHHHHHHHH};
i2wb(    P,    Q,[K|R],\textbf{1})            \(\smashedrightarrow{\theta}\) \fbcode{HHHHHHHHHHHHH}.
\end{alltt}
Clauses~\clause{\gamma}, \clause{\delta}~and~\clause{\epsilon} are
respectively dual of~\clause{\zeta},
\clause{\eta}~and~\clause{\theta}, although details due to the
difference in length will lead to small differences in the bodies. Let
us start with the patterns matching the case when both lists have the
same length, that is, \erlcode{D}~is~\erlcode{0}. Let us consider
clause~\clause{\gamma}. It matches the case when the item to
insert,~\erlcode{K}, belongs to the second list \erlcode{[J|Q]}, so
\erlcode{ins\_up/2} must be called. Since the second list will exceed
the first by one, the recursive call must be done with a difference
of~\erlcode{1}. Notice how~\erlcode{J} is already at the right place
and does no need to be included in the call to \erlcode{ins\_up/2}.
\begin{alltt}
i2wb(P,[J|Q],[K|R],0) when J < K \(\smashedrightarrow{\gamma}\)
\hfill i2wb(P,[J|ins_up(K,Q)],R,\textbf{1});
\end{alltt}
At this point it may appear more obviously that our new design is
based both on one\hyp{}way insertion and unbalanced, two\hyp{}way
insertion. Clause~\clause{\delta} processes the situation in
which~\erlcode{K} has to be inserted in the first list. Because we
decided to always have the second list of greater or equal length than
the first, we need, in addition to a one\hyp{}way insertion into a
non\hyp{}increasing list, to move the top of the first
list,~\erlcode{I}, to the second list. The recursive call is
subsequentially performed with a difference of length equal to
\erlcode{1}.
\begin{alltt}
i2wb([I|P],Q,[K|R],0) when K < I \(\smashedrightarrow{\delta}\)
\hfill i2wb(ins_dn(K,P),[I|Q],R,\textbf{1});
\end{alltt}
Clause~\clause{\epsilon} handles the case when~\erlcode{K} lays
between the two lists. Following the same design principle to favour
the second list in case of equal lengths, we push it on~\erlcode{Q}
and call recursively with a difference of~\erlcode{1}.
\begin{alltt}
i2wb(P,Q,[K|R],0) \(\smashedrightarrow{\epsilon}\) i2wb(P,[K|Q],R,\textbf{1});
\end{alltt}
Let us turn to the dual clauses where the difference of lengths
is~\erlcode{1} instead of~\erlcode{0}. The difference between clauses
\clause{\gamma}~and~\clause{\zeta} is that the latter must reinstate
the balance by moving the top of the second list, \erlcode{J}~to the
top of the first,~\erlcode{P}. The recursive call is performed,
consistently, with a difference of~\erlcode{0}, that is, the lists are
balanced after insertion.
\begin{alltt}
i2wb(P,[J|Q],[K|R],1) when J < K \(\smashedrightarrow{\zeta}\)
\hfill i2wb([J|P],ins_up(K,Q),R,\textbf{0});
\end{alltt}
The difference between clauses \clause{\delta}~and~\clause{\eta} is
that we do not move the topmost item of the first list because the
second list is already taller. Therefore, we insert~\erlcode{K} in it,
and that's it. The recursive call indicates that the lists are now
balanced.
\begin{alltt}
i2wb([I|P],Q,[K|R],1) when K < I \(\smashedrightarrow{\eta}\)
\hfill i2wb([I|ins_dn(K,P)],Q,R,\textbf{0});
\end{alltt}
Finally, the difference between clauses
\clause{\epsilon}~and~\clause{\theta} is that we insert~\erlcode{K} on
the first list because it is shorter by one, achieving thus the
balance.
\begin{alltt}
i2wb(P,Q,[K|R],1) \(\smashedrightarrow{\theta}\) i2wb([K|P],Q,R,\textbf{0}).
\end{alltt}
As a summary, here is the full definition of
\erlcode{i2wb/4}\label{code:i2wb}:
\begin{alltt}
i2wb(   [],    Q,   [],_)            \(\smashedrightarrow{\alpha}\) Q;
i2wb([I|P],    Q,   [],D)            \(\smashedrightarrow{\beta}\) i2wb(P,[I|Q],[],D);
i2wb(    P,[J|Q],[K|R],0) when J < K \(\smashedrightarrow{\gamma}\)
\hfill i2wb(P,[J|ins_up(K,Q)],R,1);
i2wb([I|P],    Q,[K|R],0) when K < I \(\smashedrightarrow{\delta}\)
\hfill i2wb(ins_dn(K,P),[I|Q],R,1);
i2wb(    P,    Q,[K|R],0)            \(\smashedrightarrow{\epsilon}\) i2wb(P,[K|Q],R,1);
i2wb(    P,[J|Q],[K|R],1) when J < K \(\smashedrightarrow{\zeta}\)
\hfill i2wb([J|P],ins_up(K,Q),R,0);
i2wb([I|P],    Q,[K|R],1) when K < I \(\smashedrightarrow{\eta}\)
\hfill i2wb([I|ins_dn(K,P)],Q,R,0);
i2wb(    P,    Q,[K|R],1)            \(\smashedrightarrow{\theta}\) i2wb([K|P],Q,R,0).
\end{alltt}
Let us consider in \figs~\vrefrange{fig:balanced_ex}{fig:balanced_ex2}
all the different configurations for the two lists after the insertion
of~\(b\), \(c\)~and~\(d\).
\begin{figure}[t]
\centering
\includegraphics{balanced_ex}
\caption{Balanced two\hyp{}way insertions of \(b\) and
  \(c\) in \erlcode{[],[\(a\)]}\label{fig:balanced_ex}
}
\end{figure}
\begin{figure}[b]
\centering
\includegraphics{balanced_ex2}
\caption{Balanced two\hyp{}way insertion of \(d\) in
  \erlcode{[\(a\)],[\(b\),\(c\)]}\label{fig:balanced_ex2}}
\end{figure}
Let us wonder now what are the best, worst and average delays of
\erlcode{i2wb/1}. Let us assume that we have the input
\erlcode{[\(a_0, a_1, a_2, a_3, a_4\)]} and we want it to minimise the
rewrites, which means not to use clauses~\clause{\gamma},
\clause{\delta}, \clause{\zeta}~and~\clause{\eta}; also, the usage of
clause~\clause{\beta} should be minimum. The latter clause is not an
issue because it reverses the first list and, by design, the second
list has the same length as the first, or exceeds it at most by one
number. A simple diagram with the two lists initially empty suffices
to convince us that the numbers must go alternatively to the right and
then to the left, leading, for example, to \erlcode{[\(a_3 , a_1\)]}
and \erlcode{[\(a_4, a_2, a_0\)]}. This is perhaps better visualised
by means of oriented edges revealing a whirlpool in
\fig~\vref{fig:whirlpool}, to be contrasted with the spiral in
\fig~\vref{fig:spiral} for \erlcode{i2w/1}.
\begin{figure}
\centering
\includegraphics[bb=71 671 174 715]{a1a3a4a2a0}
\caption{Best case for \erlcode{i2wb/1} if \(n=5\)
\label{fig:whirlpool}}
\end{figure}

The clause definining \erlcode{i2w/1} has to be used first. Then each
number is inserted, alternatively by means of clause
\clause{\epsilon}~and~\clause{\theta}. Finally, the first list is
reversed by clauses \clause{\alpha}~and~\clause{\beta}, so the
question hinges on determining the length of this list in the best
case. By design, if the total number of items is even, then the two
lists will end up containing, before using clause~\clause{\beta},
exactly half of them, because the lists have the same length. If the
total is odd, the first list contains the quotient of this number when
halved. Technically, let us note \(\best{i2wb}{n}\) the delay of any
call \erlcode{i2wb(\(L\))}, where the list~\(L\) contains
\(n\)~numbers.
\begin{align*}
\best{i2wb}{2p}   &:= 1 +     2p + p = 3p + 1,\\
\best{i2wb}{2p+1} &:= 1 + (2p+1) + p = 3p + 2,
                      \,\; \text{with} \,\; p \geqslant 0. 
\intertext{Another, more compact, way to put it is: for~\(n\geqslant{}0\),}
\best{i2wb}{n} &= 1 + n + \floor{n/2}
\mathrel{\sim} \tfrac{3}{2}{n},\,\; \text{as} \,\; n \rightarrow
\infty.
\end{align*}
where \(\lfloor{x}\rfloor\), pronounced ``floor \(x\),'' is the
greatest integer smaller than~\(x\). The equivalence is correct
because \(n/2-1 < \floor{n/2} \leqslant n/2\).

The worst case occurs when insertions are repeatedly performed at the
bottom of the longest list. This case is left to the reader as an
exercise.

Let us focus on the average delay. Let us suppose first the easiest
case in which \(n\)~is even. It is the easiest because, by design,
both lists will have the same length after the insertions are
done. Technically, this means that there exists an integer~\(p\) such
that \(n=2p\). Then we have
\[
\ave{i2wb}{0} := 2;\quad
\ave{i2wb}{2p}
  := 1 + \sum_{k=0}^{2p-1}{\ave{}{k}}
              + \ave{\(\alpha\beta\)}{p},
              \,\; \text{for} \,\; p \geqslant 0,
\]
where \(\ave{}{k}\) is the average number of rewrites to insert a
random number into a random configuration of two lists whose total
number of items is~\(k\) and \(\ave{\(\alpha\beta\)}{p}\) is the
number of rewrites to reverse \(p\)~numbers from the first list to the
second and finally return the second list. The term
\(\ave{\(\alpha\beta\)}{p}\) comes from the fact that, because
\(n=2p\), there will be \(p\)~numbers left after all the insertions
are over. Obviously, we have
\[
\ave{\(\alpha\beta\)}{p} := p + 1.
\]
The determination of \(\ave{}{k}\) requires the consideration of only
two cases: either \(k\)~is even or it is odd. When analysing the
average delay of \erlcode{i2w/1}, there were much more configurations
to take into account because not all the insertions lead to balanced
lists. If \(k\)~is even, then there exists an integer~\(j\) such that
\(k=2j\) and
\[
\ave{}{2j} = \ave{}{j,j},
\]
where \(\ave{}{j,j}\) is the average number of rewrites to insert a
random number into a configuration of two lists of length \(j\). We
already computed \(\ave{}{p,q}\) \vpageref{A_pq} where we called it
\(\ave{\(\gamma\delta\epsilon\)}{p,q}\) (we omit here the names of the
clauses because there are too many involved):
\[
\ave{}{p,q} = \frac{p^2 + 3p + q^2 + 3q + 2}{2p + 2q + 2}.
\]
So we draw, by substitution and simplification,
\[
\ave{}{2j} = \frac{j^2 + 3j + 1}{2j+1} = \frac{1}{2}{j} - \frac{1}{4}
\cdot \frac{1}{2j+1} + \frac{5}{4}.
\]
The remaining case is \(k\)~being odd, that is, \(k=2j+1\). By design,
we know that the first list will hold \(j\)~numbers whilst the
second~\(j+1\):
\[
\ave{}{2j+1} = \ave{}{j,j+1} = \tfrac{1}{2}{j} + \tfrac{3}{2}.
\]
Let us come back to our initial goal:
\begin{align*}
\ave{i2wb}{2p}
  &= 1 + \sum_{k=0}^{2p-1}{\ave{}{k}} + (p+1).
\intertext{In order to use \(\ave{}{2j}\) and \(\ave{}{2j+1}\), we
  must group the terms \(\ave{}{k}\) pairwise, so we
  explicitly see that we sum successive even and odd numbers (\(k=2j\)
  and \(k=2j+1\)):}
\ave{i2wb}{2p}
  &= 2 + p + \sum_{j=0}^{p-1}{(\ave{}{2j} + \ave{}{2j+1})}
   = 2 + p + \sum_{j=0}^{p-1}{\ave{}{2j}}
           + \sum_{j=0}^{p-1}{\ave{}{2j+1}}.
\intertext{Now we can make use of the values we found above for
  \(\ave{}{2j}\) and \(\ave{}{2j+1}\):}
\ave{i2wb}{2p}
  &= 2 + p + \sum_{j=0}^{p-1}{\left(\frac{1}{2}{j} - \frac{1}{4}
               \cdot \frac{1}{2j+1} + \frac{5}{4}\right)}
           + \sum_{j=0}^{p-1}{\left(\frac{1}{2}{j} +
             \frac{3}{2}\right)}\\
  &= 2 + p + \left(\frac{1}{2}\sum_{j=0}^{p-1}{j} -
             \frac{1}{4}\sum_{j=0}^{p-1}{\frac{1}{2j+1}} +
             \frac{5}{4}{p}\right)
           + \left(\frac{1}{2}\sum_{j=0}^{p-1}{j} +
             \frac{3}{2}{p}\right)\\
  &= \frac{1}{2}{p^2} + \frac{13}{4}{p} + 2 -
             \frac{1}{4}\sum_{j=0}^{p-1}{\frac{1}{2j+1}}.
\end{align*}
We need to find the value of this sum. By definition, the \(n\)th
\emph{harmonic number}, noted \(H_n\) for all \(n>0\), is
\[
H_n := \sum_{j=1}^{n}{\frac{1}{j}}.\,\; \text{So, obviously,}
\,\; H_{2p} = \sum_{j=1}^{2p}{\frac{1}{j}}.
\]
Let us group in \(H_{2p}\), on the one hand, the inverses with an odd
numerator and, on the other hand, the inverses with an even numerator:
\[
H_{2p} = \sum_{j=0}^{p-1}{\frac{1}{2j+1}} + \sum_{j=1}^{p}{\frac{1}{2j}}
      = \sum_{j=0}^{p-1}{\frac{1}{2j+1}} + \frac{1}{2}{H_{p}}.
\]
We can express now our sum in terms of the harmonic series:
\[
\sum_{j=0}^{p-1}{\frac{1}{2j+1}} = H_{2p} - \frac{1}{2}{H_{p}}.
\]
We can now replace it in the latest equation defining
\(\ave{i2wb}{2p}\):
\begin{align*}
\ave{i2wb}{2p}
  &= \frac{1}{2}{p^2} + \frac{13}{4}{p} + 2 -
     \frac{1}{4}\Big(H_{2p} - \frac{1}{2}{H_{p}}\Big)\\
  &= \frac{1}{2}{p^2} + \frac{13}{4}{p} + \frac{1}{8}{H_p} 
     - \frac{1}{4}{H_{2p}} + 2.
\end{align*}
We can check this formula for \(p=0\), looking forward to the expected
result~\(2\). For the formula to work, we must extend~\(H_p\) so that
\(H_0 = 0\). Then we indeed deduce \(\ave{i2wb}{0} = 2\). The
remaining task consists in finding \(\ave{i2wb}{2p+1}\). Resuming
the same line of thoughts, we deduce
\begin{align}
\ave{i2wb}{2p+1}
  &= 1 + \sum_{k=0}^{2p}{\ave{}{k}}
       + \ave{\(\alpha\beta\)}{p},
         \,\; \text{for} \,\; p \geqslant 0,\notag
\intertext{because we know that there will be \(p\)~numbers left in
  the first list after the insertions are over. Instead of doing
  calculations almost similar to the previous ones, it is wise to
  reuse them:}
\ave{i2wb}{2p+1}
  &= 1 + \left(\sum_{k=0}^{2p-1}{\ave{}{k}} + \ave{}{2p}\right) 
       + \ave{\(\alpha\beta\)}{p}
   = \ave{i2wb}{2p} + \ave{}{2p}\notag\\
  &= \left(\frac{1}{2}{p^2} + \frac{13}{4}{p} + \frac{1}{8}{H_p} 
     - \frac{1}{4}{H_{2p}} + 2\right)
     \!+\! 
     \left(\frac{1}{2}{p} - \frac{1}{4} \!\cdot\! \frac{1}{2p+1} +
     \frac{5}{4}\right)\notag\\
  &= \frac{1}{2}{p^2} + \frac{15}{4}{p} + \frac{1}{8}{H_p} -
     \frac{1}{4}{H_{2p+1}} + \frac{13}{4}.\notag
\intertext{Let us try \(p=0\): \(\ave{i2wb}{1}
  = -\frac{1}{4}{H_1} + \frac{13}{4} = 3\), which is
  correct. Summarising, for all \(p \geqslant 0\):}
\ave{i2wb}{2p}
  &= \frac{1}{2}{p^2} + \frac{13}{4}{p} + \frac{1}{8}{H_p} 
     - \frac{1}{4}{H_{2p}} + 2,\label{eq:A_i2wb_2p}\\
\ave{i2wb}{2p+1}
  &= \frac{1}{2}{p^2} + \frac{15}{4}{p} + \frac{1}{8}{H_p} -
     \frac{1}{4}{H_{2p+1}} + \frac{13}{4}.\label{eq:A_i2wb_2p+1}
\end{align}
Let us determine the asymptotic behaviour of~\(H_n\). Perhaps the
easiest way consists in proving that, for all nonzero real
numbers~\(x\),
\[
1 + x < e^x,
\]
where \(e\) is an irrational constant whose approximate value is
\[
e = 2.718281828459045\ldots
\]
This is easily done by defining a real function~\(\phi\) such that
\(\phi(x) = e^x - x - 1\), and demonstrating that its values increase
from a positive value. The derivative is \(\phi'(x) = e^x - 1\), so
\(\phi'(x) > 0\) for all \(x > 0\). Since \(\phi(0) = 0\), the
inequality above follows. The next step makes use of it to find a
lower bound for~\(H_n\). As a special case, for all integers
\(i>0\),
\begin{align*}
1 + \frac{1}{i} &< e^{1/i}.
\intertext{Since both sides of the inequality are positive, we can
  multiply all its instances for~\(i\) ranging from~\(1\) to~\(n\):}
\prod_{i=1}^{n}{\left(1 + \frac{1}{i}\right)}
&<
\prod_{i=1}^{n}{e^{1/i}}.
\intertext{Equivalently and using the notation \(\exp(x) :=
  e^x\):}
n+1 = \frac{(n+1)!}{n!} = \prod_{i=1}^{n}{\frac{i+1}{i}}
&<
\exp\left(\sum_{i=1}^{n}{\frac{1}{i}}\right) = \exp(H_n).
\intertext{Let \(\ln\) be the inverse function of~\(\exp\). The
  number~\(\ln{p}\) is called the \emph{natural logarithm} (or
  \emph{Napierian logarithm}) of \(p>0\). It is an increasing
  function, so we can apply it to both sides of the inequation above
  and}
\ln(n+1) &< H_n.
\intertext{An upper bound of~\(H_n\) can be similarly obtained by
  replacing~\(x\) by~\(-x\) in the original inequation \(1 + x
  \leqslant e^x\), since it is valid for all \(x~\neq~0\), which
  yields \(1 - x~\leqslant e^{-x}\). Following now the same track as
  before, we set \(x = 1/i\), for \(i>1\), and find, for \(n>1\),
  the equivalent inequalities:}
1 - \frac{1}{i} &< e^{-1/i},\\
\frac{1}{n} = \frac{(n-1)!}{n!} = \prod_{i=2}^{n}{\frac{i-1}{i}}
  &< \exp\left(-\sum_{i=2}^{n}{\frac{1}{i}}\right)
= \frac{1}{\exp\left(H_n-1\right)},\\
\exp\left(H_n-1\right) &< n,\\
H_n &< 1 + \ln n.
\intertext{Therefore, we established the following inequalities for
  \(n>1\):}
\ln(n+1) < H_n &< 1 + \ln n.
\end{align*}
We can now proceed with bounding \(\ave{i2wb}{2p}\) and
\(\ave{i2wb}{2p+1}\):
\begin{align*}
\ln(p+1) - 2\ln(2p)
&< 8 \cdot \ave{i2wb}{2p} - 4{p^2} - 26{p} - 14\\
&\hphantom{8 \cdot \ave{i2wb}{2p} - 4{p^2} - 26p} 
< \ln p - 2\ln(2p+1) + 3;\\
\ln(p+1) - 2\ln(2p+1)
&< 8 \cdot \ave{i2wb}{2p+1} - 4{p^2} - 30{p} - 24\\
&\hphantom{8 \cdot \ave{i2wb}{2p+1} - 4{p^2} - 30{p}}
< \ln{p} - 2\ln(2p+2) + 3.
\end{align*}
Setting \(n=2p\) and \(n=2p+1\) leads, respectively, to the
following bounds:
\begin{align*}
-2\ln n + \ln(n+2) + 4 &< \varphi(n) < -2\ln(n+1) + \ln n + 7,\\
-2\ln n + \ln(n+1)     &< \varphi(n) < -2\ln(n+1) + \ln(n-1) + 3,
\end{align*}
where \(\varphi(n) := 8 \cdot \ave{i2wb}{n} - n^2 - 13n - 10 + \ln
2\). We retain the minimum of the lower bounds and the maximum of the
upper bounds of \(\varphi(n)\) so, for all \(n>0\),
\begin{equation*}
\ln(n+1) - 2\ln n < \varphi(n) < -2\ln(n+1) + \ln n + 7.
\end{equation*}
We can weaken the bounds a little bit with \(\ln n < \ln(n+1)\) and
simplify into
\begin{equation*}
0 < 8 \cdot \ave{i2wb}{n} - n^2 - 13n + \ln n - 10 + \ln 2 < 7.
\end{equation*}
In other words, for any \(n \in \mathbb{N}\), there exists
\(\epsilon_n\) such that \(0 < \epsilon_n < 7/8\) and
\begin{equation}
\ave{i2wb}{n} = \tfrac{1}{8}(n^2 + 13n - \ln n + 10 - \ln 2) +
\epsilon_n.\label{ineq:i2wb}
\end{equation}
Now we need bounds on~\(\ln n\). Let us start by proving that, for any
\(x\geqslant 0, x^2 < e^x\). Let \(\psi(x) = e^x - x^2\). We have
\(\psi'(x) = e^x - 2x > 0\) for all~\(x\) (see above how we dealt with
\(1 + x < e^x\)) and \(\psi(0) = 1\), thus \(\psi(x)>0\) for
\(x\geqslant{}0\). Next, we can replace~\(x\) by~\(\sqrt{y}\), when
\(x>1\), yielding the inequalities
\begin{align}
1 < y < e^{\sqrt{y}} & \Rightarrow\,\; 0 < \ln y < \sqrt{y}
\;\,\Rightarrow\;\, 0 < (\ln y)/y < 1/\sqrt{\smash[b]{y}}\notag\\
                   & \Rightarrow\;\, (\ln y)/y \mathrel{\sim} 0.
\label{ineq:ln}
\end{align}
We conclude
\[
\ave{i2wb}{n} \mathrel{\sim} \tfrac{1}{8}{n^2},\,\; \text{as} \,\; n
\rightarrow \infty.
\]
This is the best, that is, lowest, asymptotic average delay we have
found so far and we conjecture that this is the best we can hope for
with two\hyp{}way insertion sort. Of course, the bad news is that we
could not beat the quadratic delay in itself and, therefore, we shall
study more efficient sorting algorithms in a later section.

\medskip

\paragraph{Proving correctness.}
\label{par:insertion_correctness}

Let us recall the original version of \erlcode{isort/1}:
\verbatiminput{isort.erl} (We renamed~\erlcode{S} into~\erlcode{T} in
the second clause of \erlcode{insert/2} to make the following proof
easier to understand.) On page~\pageref{correctness} we introduced
informally the concept of correctness as being the property of some
function to compute what it is expected to compute and not to compute
anything it is not expected to. Therefore, we should define precisely
what it means for \erlcode{isort/1} to be correct. (In the following,
we shall assume that ``sorted list'' means ``nondecreasingly sorted
list.'' The same proof can be done for non\hyp{}increasingly sorted
lists.) The function at hand is supposed to implement a sorting
algorithm, so we could say that we want to make sure that the result
is a sorted list and that it contains exactly the same items as the
input. Let us make this statement more precise and reckon by induction
on the length of the list.

First, let us check that \erlcode{isort([])} is a sorted list: the
first clause tells us that the result is~\erlcode{[]}, as expected.

Now, let us posit the \emph{induction hypothesis} that
\erlcode{isort/1} is correct for all lists of a given length~\(n>0\)
and let us endeavour to prove the same for all lists of
length~\(n+1\). If we succeed, the \emph{induction principle} would
allow us to conclude that \erlcode{isort/1} is correct for all inputs.

Let us suppose that we have an arbitrary list~\(L\) of
length~\({n>0}\) and an arbitrary item~\(I\), comparable to all items
in~\(L\). Then, the second clause defining \erlcode{isort/1} implies
\erlcode{insert(\(I\),isort(\(L\)))}. The induction hypothesis applies
to the call \erlcode{isort(\(L\))}, so it is correct; hence we only
need to establish that the call to \erlcode{insert/2} is correct. But
we have not defined yet what it precisely means for \erlcode{insert/2}
to be correct. Let us say that it signifies that, given a sorted
list~\(T\) and an item~\(I\) comparable to all items in~\(T\), the
call \erlcode{insert(\(I\),\(T\))} results in a sorted list
containing~\(I\) and all the items of~\(T\).

We have to do a proof by \emph{complete induction} on this function,
based on the length of~\(T\). The second clause defining
\erlcode{insert/2} entails that \erlcode{insert(\(I\),[])} evaluates
in the expected result \erlcode{[\(I\)]}: this is the base case. The
complete induction hypothesis for \erlcode{insert/2} is that
\erlcode{insert(\(I\),\(T\))} is correct for all lists~\(T\) whose
length is lower or equal than a given \(n>0\) and for all items~\(I\)
comparable with all the items in~\(T\).

Let us prove that \erlcode{insert(\(I\),\(U\))} is correct, where the
arbitrary sorted list~\(U\) has length~\(n+1\). If we succeed, the
induction principle would imply that \erlcode{insert/1} is correct for
all inputs. Let us suppose given an arbitrary list~\(T\) of length
\(n>0\) and a totally comparable item \(I\). Because \(T\)~is not
empty, it has a first item~\(J\) and an immediate sub\hyp{}list~\(S\),
that is, \(T=\erlcode{[\(J\)|\(S\)]}\). Item~\(I\) is comparable
to~\(J\), so two cases are possible: either \(I~\leqslant~J\) or
\(I>J\). If the former, the second clause applies and ends with
\erlcode{[\(I\)|\(T\)]}. This list has length~\(n+1\), it contains all
the items it should, it is sorted because \(I<J\) and~\(T\) being
sorted entail that~\(I\) is lower than all the items in~\(T\). In the
latter case, that is, \(I>J\), the first clause applies and rewrites
the call into \erlcode{[\(J\)|insert(\(I\),\(S\))]}. The complete
induction hypothesis holds for the call \erlcode{insert(\(I\),\(S\))}
because~\(S\) has length~\(n-1\). (This is why we needed the induction
to be complete: the hypothesis has to apply to the case~\(n-1\). The
correctness proof on \erlcode{isort/1} alone does not require it.)
Therefore, its value is the correct expected sorted list of
length~\(n\). Since \(I>J\) and \(J\)~is the smallest item of~\(T\),
\(J\)~has to be the smallest item of the result, which it is, being
the first one in \erlcode{[\(J\)|insert(\(I\),\(S\))]}. This closes
the second case.

Finally, the complete induction principle entails that
\erlcode{insert/2} is correct, which allows us to apply the induction
principle to \erlcode{isort/1} and conclude that it is correct as
well. \textsc{qed}

The keen reader may have remarked that the proof is incomplete,
because we did not clearly established that the resulting list is a
permutation of the original list. That is because we wanted to convey
the main line of thought. Notice also how to one recursive call in the
code corresponds one application of the induction hypothesis in the
proof. This is not a coincidence and, on the contrary, exemplifies a
duality between proof theory and programming language theory.

\medskip

\paragraph{Exercises.} (\emph{More playful variations on two\hyp{}way
  insertion sort})
\label{ex:insertion_sort}

\noindent [See answers \vpageref{ans:insertion_sort}.]

\begin{enumerate}

  \item \label{ins:ex:1} When designing \erlcode{i2w/3},
    \vpageref{code:unbalanced_i2w}, we chose to always push the item
    to be inserted,~\erlcode{K}, in the right list in
    clause~\clause{\epsilon}. Let us modify slightly this tactic and
    push on the left when the left list is empty (in bold in the
    following definition):
\begin{alltt}
i2w_a(L) -> i2w_a([],[],L).

i2w_a(   [],    Q,     []) -> Q;
i2w_a([I|P],    Q,     []) -> i2w_a(    P,[I|Q],[]);
i2w_a(    P,[J|Q],L=[K|_]) when J < K 
                           -> i2w_a([J|P],    Q, L);
\textbf{i2w_a(   [],    Q,  [K|R]) -> i2w_a(  [K],    Q, R);}
i2w_a([I|P],    Q,L=[K|_]) when K < I
                           -> i2w_a(    P,[I|Q], L);
i2w_a(    P,    Q,  [K|R]) -> i2w_a(    P,[K|Q], R).
\end{alltt}
Find the delay in the best and worst cases for an input list of \(n\)
items. Prove that the average delay is
\[
\ave{i2w\_a}{n} = \frac{1}{6}{n^2} + \frac{3}{2}{n} - H_n +
\frac{10}{3}.
\]
From which value of \(n\) do you have \(\ave{i2w\_a}{n} <
\ave{i2w}{n}\)? Can you explain in simple terms why the delay is
slightly lesser?

\noindent (\emph{Hint.} Write down the examples similar to the ones
found in \fig~\vref{fig:two-way_unbalanced} and observe that the
difference with \erlcode{i2w/1} is that the configuration with an
empty left list is replaced with a configuration with a singleton left
list, that is, \(\ave{}{0,k}\) is replaced with \(\ave{}{1,k-1}\) in the
definition of \(\ave{}{k}\).)

  \item In clause~\clause{\epsilon} of \erlcode{i2w/3}, the
    number~\erlcode{K} is inserted on the right list. Let us insert it
    instead on the left list (change in bold):
\begin{alltt}
i2w_b(L) -> i2w_b([],[],L).

i2w_b(   [],    Q,     []) -> Q;
i2w_b([I|P],    Q,     []) -> i2w_b(    P,[I|Q],[]);
i2w_b(    P,[J|Q],L=[K|_]) when J < K 
                           -> i2w_b([J|P],    Q, L);
i2w_b([I|P],    Q,L=[K|_]) when K < I
                           -> i2w_b(    P,[I|Q], L);
i2w_b(    P,    Q,  [K|R]) -> i2w_b(\textbf{[K|P]},    \textbf{Q}, R).
\end{alltt}
    Find the best and worst cases of \erlcode{i2w\_b/1}. Prove that
    the average delay is
    \[
      \ave{i2w\_b}{n} = \frac{1}{6}{n^2} + \frac{3}{2}{n} + 
      \frac{7}{3}.
    \] 
    How does that compare with \erlcode{i2w/1}? Explain why.

  \item When designing \erlcode{i2wb/1}, we decided, in case the
    number of items in a configuration is odd, to always maintain the
    right list longer by one. What if we allowed the left list to be
    longer by one as well? Extend the definition of \erlcode{i2wb/4}
    so to cope with a difference in length ranging over~\(-1\),
    \(0\)~and~\(1\). Find the best and worst cases. Find the average
    delay. Do you expect it to be lower? Try guessing by drawing some
    examples and building an argument and then do your calculations.

\end{enumerate}
