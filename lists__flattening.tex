%%-*-latex-*-

\chapter{Flattening}

Let us undertake the task of designing a function~\erlcode{flat/1}
such that the call~\erlcode{flat(\(L\))}, where \(L\)~is an arbitrary
list, is rewritten into a list containing the same items as~\(L\), in
the same order, except empty lists and all inner pushes~(\erlcode{|})
have been removed. If \(L\)~contains no list, then
\(\erlcode{flat(\(L\))} \mathrel{\equiv} L\). Let us review some more
examples to grasp the concept.
\begin{align*}
\erlcode{flat([])}             & \twoheadrightarrow \erlcode{[]};\\
\erlcode{flat([3,[]])}         & \twoheadrightarrow \erlcode{[3]};\\
\erlcode{flat([[],[[]]])}      & \twoheadrightarrow \erlcode{[]};\\
\erlcode{flat([5,foo,[3,[]]])} & \twoheadrightarrow \erlcode{[5,foo,3]};\\
\erlcode{flat([[],[1,[2,[]],4],[],x])}
                               & \twoheadrightarrow \erlcode{[1,2,4,x]}.
\end{align*}

\medskip

\paragraph{Direct approach.}

Let us try a direct approach. A list is either empty or not:
\begin{alltt}
flat(   []) -> \fbcode{HHHHHHHH};
flat([I|L]) -> \fbcode{HHHHHHHH}.
\end{alltt}
Then we must discriminate on the kind of head~\erlcode{I} because if
it is not a list, for example, an integer or an atom, we keep it as it
is in the result, otherwise, we must flatten it as well. Function
\erlcode{is\_a\_list/1} \vpageref{code:is_a_list} shows how to
distinguish a list from other kind of values. We can adapt the same
idea here:
\begin{alltt}
flat(       []) -> \fbcode{HHHHHHHH};
flat(   [\textbf{[]}|L]) -> \fbcode{HHHHHHHH};
flat([\textbf{[I|M]}|L]) -> \fbcode{HHHHHHHH};
flat(    [I|L]) -> \fbcode{HHHHHHHH}.\hfill% I \emph{not a list}
\end{alltt}
It is easy to guess the first and second bodies:
\begin{alltt}
flat(       []) -> \textbf{[]};
flat(   [[]|L]) -> \textbf{flat(L)};\hfill% \emph{Skipping} []
flat([[I|M]|L]) -> \fbcode{HHHHHHHH};
flat(    [I|L]) -> \fbcode{HHHHHHHH}.
\end{alltt}
The last clause is also easy to complete, because we know that
\erlcode{I} is not a list, so it can remain at the same place in the
result, whilst the tail~\erlcode{L} is flattened:
\begin{alltt}
flat(       []) -> [];
flat(   [[]|L]) -> flat(L);
flat([[I|M]|L]) -> \fbcode{HHHHHHHH};
flat(    [I|L]) -> \textbf{[I|flat(L)]}.\hfill% \emph{Keeping} I
\end{alltt}
The third body requires more pondering. Perhaps the trap to avoid is
to think that \erlcode{I}~is not a list. Actually, we can know nothing
of the type of~\erlcode{I}, although we know that
\erlcode{M}~and~\erlcode{L} are lists (perhaps not flat). Different
courses of action can be pursued. First, let us try to flatten
\erlcode{[I|M]}~and~\erlcode{L} separately and later join the two
resulting lists. Let us rename the definition to avoid confusion and
tag each clause:\label{code:sflat_par_alpha}
\begin{alltt}
sflat(       []) \(\smashedrightarrow{\alpha}\) [];
sflat(   [[]|L]) \(\smashedrightarrow{\beta}\) sflat(L);
sflat([[I|M]|L]) \(\smashedrightarrow{\gamma}\) \textbf{join(sflat([I|M]),sflat(L))};
sflat(    [I|L]) \(\smashedrightarrow{\delta}\) [I|sflat(L)].
\end{alltt}
This choice is interesting because it leaves open, in theory, the
possibility to compute in parallel the two recursive calls in clause
\clause{\gamma}, but the outermost call to \erlcode{join/2} reminds us
of the definition of \erlcode{srev/1}:
\verbatiminput{srev.def} which was proved \vpageref{delay:srev} to be
quadratic in the length of the input list: the delay
of \erlcode{join/2} depends solely on the length of its first argument
and the recursive call~\erlcode{srev(L)} is precisely the first
argument to \erlcode{join/2}, hence incurring the quadratic cost
because the same items are joined a number of times proportional to
the length. So, perhaps, the similarity does not run deep because, in
the case of \erlcode{sflat/1}, the first argument of \erlcode{join/2}
is the recursive call~\erlcode{sflat([I|M])}, which is not the tail of
the list~\erlcode{[[I|M]|L]} but its head.

Is this enough a difference to not fear a quadratic delay? Let us try
an example that exerts all the clauses of the definition. In the
following, the call to be rewritten next is
underlined.\label{trace:sflat_abc}
\begin{tabbing}
\underline{\erlcode{sflat([[],[[a],b],c])}}\\
\erlcode{sflat([[]} \= \(\smashedrightarrow{\beta}\) \=\kill
\> \(\xrightarrow{\beta}\) \> \underline{\erlcode{sflat([[[a],b],c])}}\\
\> \(\smashedrightarrow{\gamma}\) \> \erlcode{join(\underline{sflat([[a],b])},sflat([c]))}\\
\> \(\smashedrightarrow{\gamma}\)
\> \erlcode{join(join(\underline{sflat([a])},sflat([b])),sflat([c]))}\\
\> \(\smashedrightarrow{\delta}\)
\> \erlcode{join(join([a|\underline{sflat([])}],sflat([b])),sflat([c]))}\\
\> \(\smashedrightarrow{\alpha}\)
\> \erlcode{join(join([a|[]],sflat([b])),sflat([c]))}\\
\> \(=\)
\> \erlcode{join(join([a],\underline{sflat([b])}),sflat([c]))}\\
\> \(\smashedrightarrow{\delta}\)
\> \erlcode{join(join([a],[b|\underline{sflat([])}]),sflat([c]))}\\
\> \(\smashedrightarrow{\alpha}\)
\> \erlcode{join(join([a],[b|[]]),sflat([c]))}\\
\> \(=\)
\> \erlcode{join(join([a],[b]),\underline{sflat([c])})}\\
\> \(\smashedrightarrow{\delta}\)
\> \erlcode{join(join([a],[b]),[c|\underline{sflat([])}])}\\
\> \(\smashedrightarrow{\alpha}\)
\> \erlcode{join(join([a],[b]),[c|[]])}\\
\> \(=\)
\> \erlcode{join(join([a],[b]),[c])}\\
\> \;\vdots\\
\> \(\rightarrow\) \> \erlcode{[a,b,c]}\textrm{.}
\end{tabbing}
Notice that it was possible to choose to rewrite the call to
\erlcode{join/2} when considering
\erlcode{join(join([a],[b]),sflat([c]))}, but we preferred to rewrite
the call to \erlcode{sflat/1} because this allows us to didactically
distinguish the computations of\erlcode{join/2} from those
of \erlcode{sflat/1}. The abstract syntax tree of
\erlcode{join(join([a],[b]),[c])} in \fig~\vref{fig:sflat} brings to
the fore the deep similarity with \erlcode{srev/1} as shown in
\fig~\vref{fig:srev_iota}. The key point to notice is how a leftmost
branch, that is, a series of connected edges, of \erlcode{join}~nodes
can be produced, leading to a quadratic behaviour because the same
items on the leftmost subtrees are processed several times until the
root is reached.
\begin{figure}[h]
\centering
\includegraphics[bb=71 651 174 721]{sflat}%
\caption{Abstract syntax tree of \erlcode{join(join([a],[b]),[c])}
\label{fig:sflat}}
\end{figure}

%\medskip

\paragraph{Lifting.}

There is another approach to the flattening of a list, which consists
in lifting item~\erlcode{I} in clause~\clause{\gamma} one level up
among the embedded lists, thus approaching little by little a flat
list. Given a list of values written without the
symbol~``\erlcode{|}'', the \emph{embedding level} of an item is the
number of symbols~``\erlcode{[}'' to its left, including the outermost
  one, minus the number of symbols~``\erlcode{]}'' encountered. The
embedding levels of items \erlcode{a}, \erlcode{b}~and~\erlcode{c}
in~\erlcode{[[],[[a,[b]]],c]} are, respectively, \(4-1=3\), \(5-1=4\)
and \(5-4=1\). This is visualised better this way (opening square
brackets in bold and item underlined):
\begin{equation*}
\underbrace{\erlcode{\textbf{[[}],\textbf{[[}}}_{\textbf{4}-1}\erlcode{\underline{a},[b]]],c]}\qquad
\underbrace{\erlcode{\textbf{[[}],\textbf{[[}a,\textbf{[}}}_{\textbf{5}-1}\erlcode{\underline{b}]]],c]}\qquad
\underbrace{\erlcode{\textbf{[[}],\textbf{[[}a,\textbf{[}b]]],}}_{\textbf{5}-4}\erlcode{\underline{c}]}
\end{equation*}
(Note that this example is different from the one we used to test
\erlcode{sflat/1}.) But \erlcode{[a|[b|[]]]} should be written
\erlcode{[a,b]} first before applying the previous definition,
otherwise, \erlcode{b} would have a embedding level of~\(2\) instead
of the correct level~\(1\). This is what was meant by ``Given a list
written \emph{without the symbol \emph{`\erlcode{|}'}},[...]''

The definition allows items to be lists as well and, for instance, the
embedding levels of~\erlcode{[]}, \erlcode{[a,[b]]}~and~\erlcode{[b]}
are, respectively, \(1-0=1\), \(3-1=2\) and \(4-1=3\):
\begin{equation*}
\underbrace{\erlcode{\textbf{[}}}_{\textbf{1}-0}\erlcode{\underline{[]},[[a,[b]]],c]}\quad\;\;
\underbrace{\erlcode{\textbf{[[}],\textbf{[}}}_{\textbf{3}-1}\erlcode{\underline{[a,[b]]}],c]}\quad\;\;
\underbrace{\erlcode{\textbf{[[}],\textbf{[[}a,}}_{\textbf{4}-1}\erlcode{\underline{[b]}]],c]}
\end{equation*}
By \emph{lifting} we meant that the embedding level of
item~\erlcode{I} was reduced, more precisely it is decremented by
one. The embedding level of an item can be alternatively seen on the
abstract syntax tree as the number of left\hyp{}oriented edges in the
path from the root to the node containing the item in question. See,
for example, \fig~\vref{fig:embedding} where left\hyp{}oriented edges
are in bold and each of them corresponds to one level of embedding on
a path up to the root.
\begin{figure}
\centering
\includegraphics[bb=73 591 174 721]{embedding}
\caption{Embedding levels of~\(1\) in bold in
  \erlcode{[[],[[a,[b]]],c]}\label{fig:embedding}}
\end{figure}
How to lift exactly one item up one level? The general scheme is shown
in \fig~\vref{fig:flat_lifting}, where it is evident that the
embedding level of~\erlcode{I} is~\(3\) and becomes~\(2\) after the
rewrite, whilst the levels of \erlcode{M}~and~\erlcode{L} remain
unchanged---respectively, \(1\)~and~\(0\). It is worth noting that the
representation of \Erlang expressions as abstract syntax trees enables
a natural extension of the concept of embedding level to
sub\hyp{}lists, like~\erlcode{L}: a sub\hyp{}list has the same
embedding level as the embedding list because the edge from a list to
its immediate sub\hyp{}list is, by construction,
right\hyp{}oriented. When a list is an item of another list, that is,
it is graphically connected downwardly by a left\hyp{}oriented edge,
it is said to be \emph{embedded}, and it is not a sub\hyp{}list. For
instance, in the value~\erlcode{[[]]}, the list~\erlcode{[]} is
embedded, while the implicit \erlcode{[]} terminating the whole list,
visible if we write instead \erlcode{[[]|\textbf{[]}]}, is a
sub\hyp{}list. The definition corresponding to the lifting technique,
translated from \fig~\vref{fig:flat_lifting},
is\label{code:flat_lifting}
\begin{alltt}
flat(       []) \(\smashedrightarrow{\alpha}\) [];
flat(   [[]|L]) \(\smashedrightarrow{\beta}\) flat(L);
flat([[I|M]|L]) \(\smashedrightarrow{\gamma}\) \textbf{flat([I,M|L])};\hfill% I \emph{lifted one level up}
flat(    [I|L]) \(\smashedrightarrow{\delta}\) [I|flat(L)].
\end{alltt}
\begin{figure}
\centering
\subfloat[Before]{\includegraphics[bb=71 653 108 721]{flat_embedded}}
\qquad
\subfloat[After]{\includegraphics[bb=72 653 106 721]{flat_lifted}}
\caption{Lifting~\erlcode{I} only one level up
\label{fig:flat_lifting}}
\end{figure}
By repeating clause~\clause{\eta}, the first item of the first item
etc. of~\erlcode{I} (following the leftmost path from the root to a
leaf) will end up first of the outermost list and will be, in turn,
matched against the empty list (clause~\clause{\zeta}) or a
non\hyp{}list (clause~\clause{\theta}). Let us run the same example
again:\label{trace:flat_abc}
\begin{tabbing}
\erlcode{flat([[],[[a,[b]]],c])} \= \(\smashedrightarrow{\beta}\) \= \erlcode{flat([[[a,[b]]],c])}\\
\> \(\smashedrightarrow{\gamma}\) \> \erlcode{flat([[a,[b]],[],c])}\\
\> \(\smashedrightarrow{\gamma}\) \> \erlcode{flat([a,[[b]],[],c])}\\
\> \(\smashedrightarrow{\delta}\) \> \erlcode{[a|flat([[[b]],[],c])]}\\
\> \(\smashedrightarrow{\gamma}\) \> \erlcode{[a|flat([[b],[],[],c])]}\\
\> \(\smashedrightarrow{\gamma}\) \> \erlcode{[a|flat([b,[],[],[],c])]}\\
\> \(\smashedrightarrow{\delta}\) \> \erlcode{[a|[b|flat([[],[],[],c])]]}\\
\> \(=\)                    \> \erlcode{[a,b|flat([[],[],[],c])]}\\
\> \(\smashedrightarrow{\beta}\)    \> \erlcode{[a,b|flat([[],[],c])]}\\
\> \(\smashedrightarrow{\beta}\)  \> \erlcode{[a,b|flat([[],c])]}\\
\> \(\smashedrightarrow{\beta}\)  \> \erlcode{[a,b|flat([c])]}\\
\> \(\smashedrightarrow{\delta}\) \> \erlcode{[a,b|[c|flat([])]]}\\
\> \(=\)                    \> \erlcode{[a,b,c|flat([])]}\\
\> \(\smashedrightarrow{\alpha}\) \> \erlcode{[a,b,c|[]]}\\
\> \(=\)                    \> \erlcode{[a,b,c]}\textrm{.}
\end{tabbing}
These rewrites should also be seen in two dimensions, that is, on the
abstract syntax trees shown in
\figs~\vrefrange{fig:flat_ast1}{fig:flat_ast2}, where the original
empty list is inscribed in a dotted circled to distinguish it from the
empty lists used to build non\hyp{}empty lists, and the root of the
subtree being rewritten next is boxed as usual (here,
only~\erlcode{flat}).
\begin{figure}
\centering
\includegraphics[bb=71 566 263 720]{flat_ast_0}
\includegraphics[bb=71 608 284 721]{flat_ast_1}
\includegraphics[bb=71 587 302 720]{flat_ast_2}
\caption{\erlcode{flat([[],[[a,[b]]],c])} \(\twoheadrightarrow\)
  \erlcode{[a|flat([[b],[],[],c])]} \label{fig:flat_ast1}}
\end{figure}


\begin{figure}[t]
\centering
\includegraphics[bb=71 566 367 721]{flat_ast_3}
\includegraphics[bb=71 606 375 721]{flat_ast_4}
\caption{\erlcode{[a|flat([b,[],[],[],c])]} \(\twoheadrightarrow\)
  \erlcode{[a,b,c]} \label{fig:flat_ast2}}
\end{figure}

\medskip

\paragraph{Delays.}

Let us now compare \erlcode{sflat/1} and \erlcode{flat/1} in terms of
the number of rewrites to reach a value---in other words, their
delays. We usually take as measure of a list its length, but this
approach does not work here. For instance, \erlcode{[[1,w],6]} and
\erlcode{[a,b]} have the same length, but the delay of their
flattening differ. Examining how many times each clause is used gives
us a clue about the right measures needed to assess the delay for the
whole definition. Starting with \erlcode{sflat(\(L\))}, where \(L\)~is
an arbitrary list, and recalling the definition
\vpageref{code:sflat_par_alpha} \begin{alltt}
sflat(       []) \(\smashedrightarrow{\alpha}\) [];
sflat(   [[]|L]) \(\smashedrightarrow{\beta}\) sflat(L);
sflat([[I|M]|L]) \(\smashedrightarrow{\gamma}\) join(sflat([I|M]),sflat(L));
sflat(    [I|L]) \(\smashedrightarrow{\delta}\) [I|sflat(L)].
\end{alltt}
 it comes
that
\begin{itemize}

  \item clause~\clause{\beta} is used once for each empty list
    originally in the input;

  \item clause~\clause{\alpha} is used once when the end of the input
    is reached \emph{and} once for each empty list generated by the
    call~\erlcode{sflat(L)} in clause~\clause{\gamma};

  \item clause~\clause{\delta} is used once for each item which is not
    a list;

  \item clause~\clause{\gamma} is used once for each non\hyp{}empty
    embedded list.

\end{itemize}
We need to add the delay of calling~\erlcode{join/2} in
clause~\clause{\gamma} but, first, let us realise that we now know the
parameters which the delay depends upon:
\begin{enumerate}

  \item the number of items which are not lists, that is,
    \erlcode{len(sflat(\(L\)))};

  \item the number of non\hyp{}empty lists embedded in the input,
    say~\(\mathcal{N}\);

  \item the number of original empty lists, say~\(\mathcal{E}\).

\end{enumerate}
Then we can reformulate the above analysis in the following terms:
\begin{itemize}

  \item clause~\clause{\beta} is used \(\mathcal{E}\)~times;

  \item clause~\clause{\alpha} is used \(1 + \mathcal{N}\)~times;

  \item clause~\clause{\delta} is used
    \erlcode{len(sflat(\(L\)))}~times;

  \item clause~\clause{\gamma} is used \(\mathcal{N}\)~times.

\end{itemize}
So the delay due to the clauses of \erlcode{sflat/1} alone is \(1 +
\erlcode{len(sflat(\(L\)))} + \mathcal{E} + 2 \cdot
\mathcal{N}\). Hence, for instance, in the case of
\erlcode{sflat([[],[[a],b],c])}, we should expect \(1 + 3 + 1 + 2
\cdot 2 = 9\) rewrites---which is correct, according to our previous
rewrites \vpageref{trace:sflat_abc}. Now we must add the delay of
calling~\erlcode{join/2}. We already know \vpageref{delay:join} that
it is the length of its first argument plus one, formally:
\(\comp{join}{n} = n + 1\). Clause~\clause{\gamma} shows that
\erlcode{join/2}~is called on all flattened non\hyp{}empty embedded
lists, so the total delay of these calls is the number of
non\hyp{}list items in all embedded lists, plus the number of embedded
lists (this is due to the~``\(+ 1\)'' in~\(n+1\)). The latter number
is none other than~\(\mathcal{N}\) but the former is perhaps more
obscure.

For a better understanding, let us visualise in
\fig~\vref{fig:flat_example} the abstract syntax tree of some list
like \erlcode{[[],[[a],b],c]}. The circled nodes corresponds to
\emph{list constructors} (also known as \emph{pushes}, symbolised
by~(\erlcode{|})) of embedded lists. Next to them is the number of
non\hyp{}list items in the tree rooted at the node. Hence \(1\)
because of the unique item~\erlcode{a} (clause~\clause{\epsilon})
and~\(2\) because of \erlcode{b}~and~\erlcode{a} \emph{again}
(\clause{\zeta}). It is convenient to visualise these numbers going up
from the leaves and being added at circled nodes. Let us
note~\(\mathcal{P}\) the number we try to understand: it is the sum of
all the numbers at the circled nodes, in our example, it is thus \(1 +
2 = 3\). So the total delay of \erlcode{sflat(\(L\))} is
\[
1 + n + \mathcal{E} + 3 \cdot \mathcal{N} + \mathcal{P},
\]
where the length of \erlcode{sflat(\(L\))} is~\(n\). In our example,
\(\mathcal{P} = 3\), \(\mathcal{E} = 1\) (one empty list in the
input), \(\mathcal{N} = 2\) (two non\hyp{}empty embedded lists, at the
circled nodes) and the length of \erlcode{sflat([[],[[a],b],c])}
is~\(3\). So, the total delay, including the rewrites by
\erlcode{join/2} is \( 1 + 3 + 1 + 3 \cdot 2 + 3 = 14\). Indeed, we
found that the delay for \erlcode{sflat/1} calls alone was~\(9\), and
\(14\)~is consistent with the fact that \(5\)~additional steps are
required to evaluate \erlcode{join(join([a],[b]),[c])}.
\begin{figure}
\centering
\includegraphics[bb=71 631 192 721]{flat_example}%
\caption{Abstract syntax tree of \erlcode{[[],[[a],b],c]}
\label{fig:flat_example}}
\end{figure}
Unfortunately, it is almost as difficult to express
algorithmically~\(\mathcal{P}\) as it is to define \erlcode{sflat/1},
so it is certainly worth finding another, more intuitive, point of
view.

The idea may rise from the observation made in passing above, when
computing~\(\mathcal{P}\) on the example of
\fig~\vref{fig:flat_example}: a circled node was annotated with
\clause{\zeta} because it counts~\erlcode{a} \emph{twice}. In other
words, the atom item~\erlcode{a} is counted for one at each circled
node because it is included in two non\hyp{}empty embedded lists, one
of which is included in the other: \erlcode{a}~is embedded
in~\erlcode{[a]}, which is, in turn, embedded
in~\erlcode{[[a],b]}. Therefore, if we fix the size of the output,
that is, the number of non\hyp{}list items, we can wonder how to
arrange these items so as to maximise the delay. This is an example of
worst case analysis which is based on the output size instead of the
input size.  When the delay depends on the result of the call, it is
said to be \emph{output\hyp{}dependent}. If we want to count as many
times as possible the same items, we should put all of them in the
deeper leftmost subtree. That is, we set the input
\[
L = \erlcode{[}\underbrace{\erlcode{[...[}}_{\mathcal{N} \;
        \text{times}}I_0, I_1, \dots, I_{n-1}
      \underbrace{\erlcode{]...]}}_{\mathcal{N} \;
    \text{times}}\erlcode{]},
\]
where the~\(I_i\) are non\hyp{}list items. Here, the number of items
which are non\hyp{}lists, called~\(\mathcal{N}\), is none other than
the embedding level of the non\hyp{}list items~\(I\), minus one. Then
\(\mathcal{P} = n \cdot \mathcal{N}\). Consider
\fig~\vref{fig:sflat_iota}, where (1)~\Fig~\vref{fig:sflat_ast} is the
input in the worst case, whose embedded lists are annotated with the
number of non\hyp{}list items they contain, as in
\fig~\vref{fig:flat_example}; (2)~\Fig~\vref{fig:sflat_join} displays
the abstract syntax tree after all calls to \erlcode{sflat/1} on the
input have been rewritten, hinting at the total number of steps
required to finish the computation: \(\mathcal{N} \cdot (n+1)\). (Let
us not forget that this is not the way \Erlang computes, so this
presentation is set out as a didactic mean only.) and
(3)~\Fig~\vref{fig:sflat_final} shows the final result. Of course, it
is possible to increase the number of rewrites by inserting empty
lists everywhere in the input, as this triggers clause~\clause{\beta}
but this case is already accounted for. As a conclusion, the delay in
the worst case of \erlcode{sflat(\(L\))} is
\[
\worst{sflat}{n,\mathcal{N}} = 1 + n + 2 \cdot \mathcal{N} +
\mathcal{N} \cdot (1+n) = (\mathcal{N} + 1)(n + 3) - 2,
\]
where \(n\)~is the length of \erlcode{sflat(\(L\))} and
\(\mathcal{N}\)~is the number of embedded non\hyp{}lists. All the
terms involved in expressing the delay are now intuitive. For example
\erlcode{sflat([[[[[a,b]]]]])} has a delay of \((4+1)(2+3)-2 = 23\)
rewrites.
\begin{figure}
\centering
\subfloat[Input \(L\) to \erlcode{sflat/1}\label{fig:sflat_ast}]{%
  \includegraphics[bb=71 586 154 721]{sflat_input}%
}
\quad
\subfloat[Partial result\label{fig:sflat_join}]{%
  \includegraphics[bb=71 591 147 721]{sflat_partial}%
}
\qquad
\subfloat[Result\label{fig:sflat_final}]{%
  \includegraphics[bb=71 651 127 721]{sflat_final}%
}
\caption{\erlcode{sflat([\(\ldots\)[\(I_0, I_1, \dots,
        I_{n-1}\)]\(\ldots\)])} \(\twoheadrightarrow\) \erlcode{[\(I_0,
      I_1, \dots, I_{n-1}\)]}
\label{fig:sflat_iota}}
\end{figure}

Let us consider now the delay of \erlcode{flat(\(L\))}, where \(L\)~is
an arbitrary list. We settled for the following definition:
\begin{alltt}
flat(       []) \(\smashedrightarrow{\alpha}\) [];
flat(   [[]|L]) \(\smashedrightarrow{\beta}\) flat(L);
flat([[I|M]|L]) \(\smashedrightarrow{\gamma}\) flat([I,M|L]);
flat(    [I|L]) \(\smashedrightarrow{\delta}\) [I|flat(L)].
\end{alltt}

We draw that
\begin{itemize}

  \item clause~\clause{\alpha} is used once when the end of the
    input is reached;

  \item clause~\clause{\gamma} is used once for each item of all the
    embedded lists;

  \item clause~\clause{\beta} is used once for each empty list
    originally in the input \emph{and} once for each empty
    list~\erlcode{M} by clause~\clause{\gamma} (if \erlcode{I}~is
    empty, it has already been accounted for as an empty list
    originally present);

  \item clause~\clause{\delta} is used once for each item which is not
    a list.

\end{itemize}
Therefore, the delay is
\[
1 + \mathcal{L} + (\mathcal{E} + \mathcal{N}) + n.
\]
where \(\mathcal{L}\)~is the sum of the lengths of all the
sub\hyp{}lists (clause~\clause{\gamma}), \(\mathcal{E}\)~is the number
of empty lists originally in the input (clause~\clause{\beta}),
\(\mathcal{N}\)~is the number of non\hyp{}empty embedded lists
(clause~\clause{\beta}) and \(n\)~is the number of non\hyp{}list items
(clause~\clause{\delta}). For example,
\begin{align*}
\delay{flat([[[[[a,b]]]]])}    &= 1 + 5 + (0+4) + 2 = 12,\\
\delay{flat([[],[[a],b],c])}   &= 1 + 3 + (1+2) + 3 = 10,\\
\delay{flat([[],[[a,[b]]],c])} &= 1 + 4 + (1+3) + 3 = 12.
\end{align*}
These are the expected values and the last case was laid out in
details \vpageref{trace:flat_abc}.

How does~\erlcode{flat/1} compare with~\erlcode{sflat/1} in general?
The delay of calling \erlcode{sflat/1} is \(1 + n + \mathcal{E} + 3
\cdot \mathcal{N} + \mathcal{P}\), so the question boils down to
comparing \(2 \cdot \mathcal{N} + \mathcal{P}\)
with~\(\mathcal{L}\). Let us consider a few examples:
\begin{itemize}

  \item Let the input be~\erlcode{[a,b]}. Then~\(\mathcal{N} = 0\),
    \(\mathcal{P} = 0\) and~\(\mathcal{L} = 0\), so \(2 \cdot 0 + 0 =
    0\); that is, the delays are equal.

  \item Let the input be~\erlcode{[[a,b]]}. Then \(\mathcal{N} = 1\),
    \(\mathcal{P} = 2\) and~\(\mathcal{L} = 2\), so \(2 \cdot 1 + 2 >
    2\); that is, \erlcode{flat/1}~is faster than~\erlcode{sflat/1}.

  \item Let the input be~\erlcode{[[[],[],[]]]}. Then \(\mathcal{N} =
    1\), \(\mathcal{P} = 0\) and~\(\mathcal{L} = 3\), so \(2 \cdot 1 +
    0 < 3\); that is, \erlcode{sflat/1}~is faster
    than~\erlcode{flat/1}.

\end{itemize}
Therefore, both functions are not comparable in general. Nevertheless,
if we have no empty list in the input, that is, if~\(\mathcal{E} =
0\), then~\(\mathcal{P} \geqslant \mathcal{L}\), because the length of
each embedded list is lower than or equal to the number of
non\hyp{}list items at greater or equal embedding levels (it is equal
if there is no further embedded list). Therefore, summing up all these
inequalities, we draw that if~\(\mathcal{E} = 0\), then
\erlcode{flat/1}~is faster than~\erlcode{sflat/1}.

\medskip

\paragraph{Slight improvement.}

Is it possible to improve the definition of \erlcode{flat/1}?  Upon
close examination of the example given in
\figs~\vrefrange{fig:flat_ast1}{fig:flat_ast2}, it becomes apparent
that, for each non\hyp{}empty embedded list an empty list is produced
when lifting the last item in them (\erlcode{M}~is an empty list
then). These additional empty lists have to be removed by means of
clause~\clause{\beta}. It is easy to avoid introducing empty lists
which need to be eliminated next: let us just add a clause matching
the case of an embedded list with one item, that is, a singleton
list:\label{code:flat_opt}
\begin{alltt}
flat_opt(       []) -> [];
flat_opt(   [[]|L]) -> flat_opt(L);
\textbf{flat\_opt(  [[I]|L]) -> flat_opt([I|L]);}\hfill% \emph{Improvement}
flat_opt([[I|M]|L]) -> flat_opt([I,M|L]);\hfill% M \(\neq\) []
flat_opt(    [I|L]) -> [I|flat_opt(L)].
\end{alltt}
What is the delay of this improved definition? Simply, the
term~\(\mathcal{N}\) vanishes:
\[
1 + \mathcal{L} + \mathcal{E} + n.
\]
As a consequence, we now have the delays
\begin{align*}
\delay{flat\_opt([[[[[a,b]]]]])}    &= 1 + 5 + 0 + 2 = 8,\\
\delay{flat\_opt([[],[[a],b],c])}   &= 1 + 3 + 1 + 3 = 8,\\
\delay{flat\_opt([[],[[a,[b]]],c])} &= 1 + 4 + 1 + 3 = 9.
\end{align*}
The improvement goes beyond a shorter delay: the delay itself can be
expressed with one notion less---namely here, the number of embedded
lists, noted~\(\mathcal{N}\). This is important too.

\medskip

\paragraph{Exercises.}

\begin{enumerate}

  \item Reconsider the delay of \erlcode{sflat/1} if we augment the
    definition of \erlcode{join/2} with a special case (in bold):
\begin{alltt}
\textbf{join(    P,[]) -> P;}
join(   [], Q) -> Q;
join([E|P], Q) -> [E|join(P,Q)].
\end{alltt}

  \item Check that the following definition is equivalent
    to \erlcode{flat/1} and find its delay.
\begin{verbatim}
flat_bis(L)          -> flat_bis(L,[],[]).
flat_bis(   [],[],B) -> rev(B);
flat_bis(   [], A,B) -> flat_bis(A,   [],    B);
flat_bis(  [I], A,B) -> flat_bis(I,    A,    B);
flat_bis([I|L], A,B) -> flat_bis(I,[L|A],    B);
flat_bis(    I, A,B) -> flat_bis(A,   [],[I|B]).
\end{verbatim}

\end{enumerate}
