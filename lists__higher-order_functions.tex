%%-*-latex-*-

\chapter{Higher-Order Functions}
\label{chap:higher-order_functions}

\paragraph{Polymorphic sorting.}

There is an aspect of \erlcode{isort/1} and \erlcode{isort\_tf/1}
which deserves a second thought. We described \vpageref{polymorphism}
a very useful property of many \Erlang functions, called
\emph{polymorphism}, which is the ability to process values of any
kind in the same, uniform manner. For example, reversing a list does
not depend on the nature of the items it contains---it is a purely
structural algorithm. By contrast, our definitions of
\erlcode{isort/1} and \erlcode{isort\_tf/1} rely on the usage of the
predefined comparison operator~(\erlcode{<}) in guards. This implies
that all items in the list must be pairwise comparable---for example,
they can be integers. But what if we want to sort other kinds of
values, like lists? Consider the very practical need to sort a set of
bills: each bill can be represented by a list of prices rounded to the
closest integer and the set in question by a list itself; we would
then want to sort by insertion the bills by nondecreasing total
amounts. If we set on writing a version of \erlcode{isort/1} tailored
to work only with items which are lists of integers, we are
duplicating code and we would have to write a different instance every
time a different kind of values to be sorted presents itself. As a
consequence, what is needed here is polymorphism on \emph{function
  parameters}, more precisely, the possibility of a function to be a
value, thus to be used as an argument. \Erlang provides this facility
in a natural way and many functional languages do as well. In our
case, we need the caller of \erlcode{isort/1} to provide an additional
parameter which is a comparison function between the items. Then
\erlcode{isort/2} would make use of this caller\hyp{}defined
comparison, instead of always applying the default
operator~(\erlcode{<}) which works only (or mostly) on integers. Here
is again the definition of \erlcode{isort/1}: \begin{alltt}
isort(   [])               \(\smashedrightarrow{\beta}\) [];
isort([I|L])               \(\smashedrightarrow{\gamma}\) insert(I,isort(L)).
insert(I,[J|S]) when I > J \(\smashedrightarrow{\delta}\) [J|insert(I,S)];
insert(I,    S)            \(\smashedrightarrow{\epsilon}\) [I|S].
\end{alltt}

Then, our first attempt at modification leads us straightforwardly to
\begin{alltt}
isortf(\textbf{\_},   [])               \(\smashedrightarrow{\beta}\) [];
isortf(\textbf{F},[I|L])               \(\smashedrightarrow{\gamma}\) insert(\textbf{F},I,isortf(\textbf{F},L)).
insert(\textbf{F},I,[J|S]) when \textbf{F(I,J)} \(\smashedrightarrow{\delta}\) [J|insert(\textbf{F},I,S)];
insert(\textbf{\_},I,    S)             \(\smashedrightarrow{\epsilon}\) [I|S].
\end{alltt}
But the \Erlang compiler would reject this program because the
language does not allow a user\hyp{}defined function to be called in a
guard. The rationale is that the call \erlcode{F(I,J)} above may not
terminate and the semantics, that is, the expected behaviour of
\Erlang constructs, is that pattern matching with guards always
ends. Because it is impossible to automatically check whether a
function call terminates on all inputs (this problem is equivalent to
the famous \emph{Turing halting problem}, which is
\emph{undecidable}), the compiler does not even try and prefers to
reject all guards made of function calls. Thus we must move the call
\erlcode{F(I,J)} inside the body of the same clause, which begets the
question as how to merge clauses \clause{\delta}~and~\clause{\epsilon}
into a new clause~\clause{\delta\sb{0}}. A simple way out is to create
another function, \erlcode{triage/4}, whose task is to take the result
of the comparison and proceed with the rest of the computation. Of
course, this means that \erlcode{triage/4} must also receive all
necessary information to carry on:
\newlength\dblarrow\settowidth\dblarrow{\(\delta\sb{0}\)}
\begin{alltt}
isortf(_,   [])        \(\MyArrow{\dblarrow}{\beta}\) [];
isortf(F,[I|L])        \(\MyArrow{\dblarrow}{\gamma}\) insert(F,I,isortf(F,L)).
insert(F,I,[J|S])      \(\MyArrow{\dblarrow}{\delta\sb{0}}\) triage(F,I,[J|S],F(I,J)).
triage(F,I,[J|S],\fbcode{true}) \(\MyArrow{\dblarrow}{\zeta}\) [J|insert(F,I,S)];
triage(F,I,[J|S],\fbcode{true}) \(\MyArrow{\dblarrow}{\eta}\) [I|S].
\end{alltt}
The empty boxes must be filled with the result of a comparison. In our
case, we want a comparison with two possible outputs, depending on the
first argument argument being lower or greater than the second. By
definition, the result of \erlcode{I~>~J} is the atom~\erlcode{true}
if the value of~\erlcode{I} is greater than the value of~\erlcode{J}
and~\erlcode{false} otherwise. Let us follow the same convention
for~\erlcode{F} and impose that the value of \erlcode{F(I,J)} is the
atom~\erlcode{true} if~\erlcode{I} is greater than \erlcode{J} and
\erlcode{false} otherwise. It is even better to rename the
parameter~\erlcode{F} into something more intuitive according to its
behaviour, like~\erlcode{Gt} (\emph{\textbf{G}reater \textbf{t}han}):
\begin{alltt}
triage(Gt,I,[J|S],\textbf{true})    \(\MyArrow{\dblarrow}{\zeta}\) [J|insert(Gt,I,S)];
triage(Gt,I,[J|S],\textbf{false})   \(\MyArrow{\dblarrow}{\eta}\) [I|S].
\end{alltt}
We notice that clause~\clause{\eta} makes no use of~\erlcode{J}, which
means we actually lose an item. What went wrong and when? The mistake
came from not realising that clause~\clause{\epsilon} covered two
cases, \erlcode{S}~is empty or not, therefore we should have untangled
these two cases before merging clause~\clause{\epsilon} with
clause~\clause{\delta}, because in~\clause{\delta} we have the pattern
\erlcode{[J|S]}, that is, the non\hyp{}empty case. Let us rewind and
split~\clause{\epsilon} into
\clause{\epsilon\sb{0}}~and~\clause{\epsilon\sb{1}}:
\begin{alltt}
isortf( _,   [])               \(\MyArrow{\dblarrow}{\beta}\) [];
isortf(Gt,[I|L])\hfill\(\MyArrow{\dblarrow}{\gamma}\) insert(Gt,I,isortf(Gt,L)).
insert(Gt,I,[J|S]) when Gt(I,J) \(\MyArrow{\dblarrow}{\delta}\) [J|insert(Gt,I,S)];
insert(\textbf{Gt},I,\textbf{[J|S]})              \(\MyArrow{\dblarrow}{\epsilon\sb{0}}\) [I|\textbf{[J|S]}];
insert( _,I,   \textbf{[]})              \(\MyArrow{\dblarrow}{\epsilon\sb{1}}\) \textbf{[I]}.
\end{alltt}
Notice that we did not write
\begin{alltt} 
insert(_,I,   [])               \(\MyArrow{\dblarrow}{\epsilon\sb{1}}\) [I];
insert(_,I,    S)               \(\MyArrow{\dblarrow}{\epsilon\sb{0}}\) [I|S].
\end{alltt}
even though it would have been correct, because we had in mind the
fusion with clause~\clause{\delta}, so we needed to make the pattern
\erlcode{[J|S]} conspicuous in~\clause{\epsilon\sb{0}}. For even more
clarity, we made apparent the parameter~\erlcode{Gt}: the heads of
clauses \clause{\delta}~and~\clause{\epsilon\sb{0}} are now identical
and ready to merge into a new clause~\clause{\delta\sb{0}}:
\begin{alltt}
isortf( _,   [])         \(\MyArrow{\dblarrow}{\beta}\) [];
isortf(Gt,[I|L])         \(\MyArrow{\dblarrow}{\gamma}\) insert(Gt,I,isortf(Gt,L)).
insert(Gt,I,[J|S])       \(\MyArrow{\dblarrow}{\delta\sb{0}}\) triage(Gt,I,[J|S],Gt(I,J)).
insert( _,I,   [])       \(\MyArrow{\dblarrow}{\epsilon\sb{1}}\) [I].
triage(Gt,I,[J|S],true)  \(\MyArrow{\dblarrow}{\zeta}\) [J|insert(Gt,I,S)];
triage( _,I,[J|S],false) \(\MyArrow{\dblarrow}{\eta}\) [I|[J|S]].
\end{alltt}
We can improve a little bit clause~\clause{\eta} by not distinguishing
\erlcode{J}~and~\erlcode{S}:
\begin{alltt}
triage( _,I,    \textbf{S},false) \(\MyArrow{\dblarrow}{\eta}\) [I|\textbf{S}].
\end{alltt}
This transformation is correct because \erlcode{S}~is never
empty. Instead of using an auxiliary function like \erlcode{triage/4},
which takes many arguments and serves no purpose other than performing
a test on the value of \erlcode{Gt(I,J)} and proceed accordingly, we
can make good use of a special \Erlang construct based on the new
keywords \erlcode{case}, \erlcode{of}~and~\erlcode{end}:
\begin{alltt}
isortf( _,   [])   \(\MyArrow{\dblarrow}{\beta}\) [];
isortf(Gt,[I|L])   \(\MyArrow{\dblarrow}{\gamma}\) insert(Gt,I,isortf(Gt,L)).
insert(Gt,I,[J|S]) \(\MyArrow{\dblarrow}{\delta\sb{0}}\) \textbf{case Gt(I,J) of
                         true  \(\smashedrightarrow{\zeta}\) [J|insert(Gt,I,S)];
                         false \(\smashedrightarrow{\eta}\) [I|[J|S]]
                       end;}
insert( _,I,   []) \(\MyArrow{\dblarrow}{\epsilon\sb{1}}\) [I].
\end{alltt}
We can decrease the memory usage again in the clause~\clause{\eta}
(case~\erlcode{false}), this time by means of an alias for the pattern
\erlcode{[J|S]}, so the best version of the code is
\begin{alltt}
isortf( _,   [])     \(\MyArrow{\dblarrow}{\beta}\) [];
isortf(Gt,[I|L])     \(\MyArrow{\dblarrow}{\gamma}\) insert(Gt,I,isortf(Gt,L)).
insert(Gt,I,\textbf{Q=}[J|S]) \(\MyArrow{\dblarrow}{\delta\sb{0}}\) case Gt(I,J) of
                           true  \(\smashedrightarrow{\zeta}\) [J|insert(Gt,I,S)];
                           false \(\smashedrightarrow{\eta}\) [I|\textbf{Q}]
                         end;
insert( _,I,     []) \(\MyArrow{\dblarrow}{\epsilon\sb{1}}\) [I].
\end{alltt}
Before we keep on studying \erlcode{isortf/2} and \erlcode{insert/3},
let us put to good use the \erlcode{case} construct of \Erlang to
improve the definition of \erlcode{i2wb/4}, \vpageref{code:i2wb}. We
always determine the delay of a function by counting the number of
rewrites (in other words, the number of function calls) it requires to
compute its result. Of course, this is not the whole story. For
instance, comparisons do take some real time to be performed, although
they are individually fast when applied to integers. Another reason
why we ignored them up to now is that we always included these integer
comparisons into guards, so they were not taken into account, just as
the very real time taken to match an argument against a pattern in the
head has been ignored all along. We may improve the legibility of our
programs by factoring some matches on integers \emph{outside the
  heads}. A speed\hyp{}up is not guaranteed because we do not suppose
anything about how the compiler processes the heads. Here is a better
\erlcode{isort2wb/4}:
\begin{alltt}
isort2wb(L) -> isort2wb([],[],L,0).

isort2wb(   [],    Q,   [],_)            -> Q;
isort2wb([I|P],    Q,   [],D)            ->
\hfill isort2wb(P,[I|Q],[],D);
isort2wb(    P,[J|Q],[K|R],D) when J < K ->
  \textbf{case} D \textbf{of}
    1 -> isort2wb([J|P],    ins_up(K,Q),R,0);
    0 -> isort2wb(    P,[J|ins_up(K,Q)],R,1)
  \textbf{end};
isort2wb([I|P],    Q,[K|R],D) when K < I ->
  \textbf{case} D \textbf{of}
    1 -> isort2wb([I|ins_down(K,P)],    Q,R,0);
    0 -> isort2wb(    ins_down(K,P),[I|Q],R,1)
  \textbf{end};
isort2wb(    P,    Q,[K|R],D)            ->
  \textbf{case} D \textbf{of}
    1 -> isort2wb([K|P],    Q,R,0);
    0 -> isort2wb(    P,[K|Q],R,1)
  \textbf{end}.
\end{alltt}
Let us resume now our examination of \erlcode{isortf/1} by determining
its average delay. It is important \emph{not} to count the arrows in a
\erlcode{case} construct because these arrows are certainly not
compiled as a function call but as a conditional. Of course, the
expression between the \erlcode{case} and \erlcode{of} keywords may
contain function calls and these must be accounted for. In the
function under consideration, the relevant call is \erlcode{Gt(I,J)},
where the function \erlcode{fun Gt/2} is a parameter. Because of this
polymorphism, the delay of the call cannot be determined statically,
that is, by solely examining the definition, and the delay of
\erlcode{isortf/2} depends on the delay of its parameter~\erlcode{Gt},
whose delay is a priori unknown because an infinite number of
functional arguments are possible. Let us suppose that this delay is
constant and let us call it \(\comp{Gt}{}\). The only difference, as
far as delays are concerned, between \erlcode{isortf/1} and
\erlcode{isort/1}, is the call \erlcode{Gt(I,J)}. We can make an
analysis similar to the one we did on page~\pageref{def:ave_isort}:
\[
\ave{isortf}{0}   \eqn{\beta} 1;\quad
\ave{isortf}{k+1} \eqn{\gamma} 1 + \ave{insert}{k} +
  \ave{isortf}{k},\,\; \text{with} \;\, k \geqslant 0.
\]
Telescoping both sides in a sum yields, for~\({n \geqslant 0}\),
\[
\ave{isortf}{n} = 1 + n + \sum_{k=0}^{n-1}{\ave{insert}{k}}.
\]
The delay for using once clause~\clause{\delta\sb{0}} does not include
any delay due to \clause{\zeta}~nor~\clause{\eta} because these are
not clauses but cases, therefore it amounts to \(\comp{Gt}{} +
1\). There are \(k+1\)~ways to insert an item into a list of
\(k\)~items. For an insertion before the item at location~\(j\), the
head of the list being at location~\(0\), it thus takes \((\comp{Gt}{}
+ 1)j\)~rewrites. The insertion at the end of the list, that is, at
position~\(k+1\), is special, because the insertion, properly
speaking, is performed by clause~\clause{\epsilon\sb{1}} instead
of~\clause{\eta}, which means that \erlcode{Gt}~is not called for the
last rewrite. So the delay for inserting after the last item
is~\((\comp{Gt}{} + 1)k + 1\). Now we can form
\begin{align*}
\ave{insert}{k}
  &= \frac{1}{k+1}\left(\sum_{j=1}^{k}{((\comp{Gt}{} + 1)j)} +
     ((\comp{Gt}{} + 1)k + 1)\!\!\right)\\
  &= (\comp{Gt}{} + 1)\left(\frac{1}{2}{k} + 1\right) -
     \frac{\comp{Gt}{}}{k+1}.
\intertext{We can check the case~\(\comp{Gt}{}=0\), which coincides
  with equation~\eqref{ave:insert} \vpageref{ave:insert}:}
\ave{insert}{k}
  &= \frac{k}{2} + 1,\,\; \text{if} \,\; \comp{Gt}{} = 0.
\intertext{Let us resume:}
\sum_{k=0}^{n-1}{\ave{insert}{k}}
  &= \frac{1}{2}{(\comp{Gt}{} + 1)} \sum_{k=0}^{n-1}{k}
     + (\comp{Gt}{} + 1)n - \comp{Gt}{}
     \sum_{k=0}^{n-1}\frac{1}{k+1}\\
  &= \frac{1}{4}{(\comp{Gt}{} + 1)n^2}
     + \frac{3}{4}{(\comp{Gt}{} + 1)n} - \comp{Gt}{} H_n.
\end{align*}
With the convention \(H_0 = 0\), we finally conclude,
for~\({n\geqslant 0}\),
\begin{align}
\ave{isortf}{n} 
  &= 1 + n
     + \left(\frac{1}{4}{(\comp{Gt}{} + 1)n^2}
     + \frac{3}{4}{(\comp{Gt}{} + 1)n} - \comp{Gt}{} H_n\!\right)\notag\\
  &= \frac{1}{4}{(\comp{Gt}{} + 1)n^2}
     + \frac{1}{4}{(3\comp{Gt}{}+7)n} - \comp{Gt}{} H_n +
     1.\label{delay:i_sortf}
\intertext{A quick check again consists in plugging \(\comp{Gt}{} =
  0\) into the formula and it indeed becomes identical to
  equation~\eqref{ave:isort} \vpageref{ave:isort}:}
\ave{isort}{n} &= \frac{1}{4}{n^2} + \frac{7}{4}{n} + 1,\,\;
\text{if} \,\; \comp{Gt}{} = 0.\notag
\end{align}

How would we call \erlcode{isortf/2} so the resulting value is the
same as calling \erlcode{isort/1}? First, we need a comparison
function which behaves exactly like the operator~(\erlcode{>}):
\begin{verbatim}
gt_int(I,J) when I > J -> true;
gt_int(_,_)            -> false.
\end{verbatim}
If we try now to form the call
\begin{center}
\erlcode{isortf(gt\_int,[5,3,1,4,2])},
\end{center}
we find that an error occurs at run\hyp{}time because
\erlcode{gt\_int} is an atom, not a function. That is why \Erlang
provides a special syntax for denoting functions used as values:
\begin{center}
\erlcode{isortf(\textbf{fun} gt\_int\textbf{/2},[5,3,1,4,2])}.
\end{center}
Notice the new keyword~\erlcode{fun} and the usual indication of the
number of arguments the function is expected to operate on (here,
two).

What would happen if we passed as a parameter the function
\erlcode{lt\_int/2} defined as follows?
\begin{verbatim}
lt_int(I,J) when I < J -> true;
lt_int(I,J)            -> false.
\end{verbatim}
The consequence is that the result is sorted non\hyp{}increasingly and
all we had to do was to change the comparison function, \emph{not the
  sorting function itself.}

It may seem a burden to have to name even simple comparison functions
like \erlcode{lt\_int/2}, which is none other than the predefined
operator~(\erlcode{<}). Fortunately, \Erlang provides a way to define
functions without giving them a name. The rule consists in using the
\erlcode{fun}~keyword together with the \erlcode{end}~keyword and put
the usual definition in\hyp{}between, without a function
name. Reconsider for example the previous calls but using such
anonymous functions (sometimes called \emph{lambdas}):
\begin{center}
\erlcode{isortf(\textbf{fun}(I,J) -> I > J \textbf{end},[5,3,1,4,2])}
\end{center}
results in \erlcode{[1,2,3,4,5]} and
\begin{center}
\erlcode{isortf(\textbf{fun}(I,J) -> I < J \textbf{end},[5,3,1,4,2])}
\end{center}
results in \erlcode{[5,4,3,2,1]}. The delay for these two comparison
functions is~\(1\), so the average delay of the function call
\erlcode{isortf(fun(I,J) -> I > J end,\(L\))}, where \(L\)~is a list
of \(n\)~integers, is obtained by setting \(\comp{Gt}{} = 1\) in the
previously found equation \eqref{delay:i_sortf}
\vpageref{delay:i_sortf}:
\begin{align*}
\ave{isortf}{n} 
  &= \frac{1}{4}{(\comp{Gt}{} + 1)n^2}
     + \frac{1}{4}{(3\comp{Gt}{}+7)n} - \comp{Gt}{} H_n + 1\\
  &= \frac{1}{2}{n^2} + \frac{5}{2}{n} - H_n + 1,\,\; \text{if} \,\;
     \comp{Gt}{} = 1.
\end{align*}
Let us now use \erlcode{isortf/2} to sort lists of lists of
integers, according to the sum of the integers in each list---this is
the practical application of sorting bills we previously mentioned. As
the example of sorting in non\hyp{}increasing order hints at, we only
need here to write how to compare two lists of integers by means of
the \erlcode{sum/1} function, whose several definitions we have
already reviewed. Here is a simple one
\vpageref{code:sum_tf}:\label{code:sum_tf_repeated}
\verbatiminput{sum_tf.def}
We have \(\comp{sum\_tf}{n} = n + 2\). Now we can define the
comparison function \erlcode{gt\_bill/2}, based upon the operator
(\erlcode{>}):
\begin{alltt}
gt_bill(P,Q) -> sum_tf(P) > sum_tf(Q).
\end{alltt}
Notice in passing that the predefined \Erlang comparison
operator~(\erlcode{>}) results in either the atom
\erlcode{true}~or~\erlcode{false}, so there is no need to use a
\erlcode{case} construct. Then we can sort our bills by simply calling
\begin{center}
\erlcode{isortf(fun gt\_bill/2,[[1,5,2,9],[7],[2,5,11],[4,3]])}
\end{center}
or, simply,
\begin{alltt}
      isortf(\textbf{fun}(P,Q) -> sum_tf(P) < sum_tf(Q) \textbf{end},
             [[1,5,2,9],[7],[2,5,11],[4,3]]).
\end{alltt}
(By the way, do you expect~\erlcode{[7]} to appear before or after
\erlcode{[4,3]} in the answer? What would you have to modify so the
relative order of these two lists is reversed?) It is just as easy to
sort the bills in non\hyp{}increasing order. This great easiness in
passing around functions as any other kind of values is what justifies
the adjective \emph{functional} for a language like \Erlang and many
others. A function taking another function as a parameter is said to
be a \emph{higher\hyp{}order function}.

\medskip

\paragraph{Sorted association lists.}
\label{sorted_association_lists}

There is something that we could improve in the previous definition of
\erlcode{isortf/2}. Sorting by comparison, as the worst case of
insertion sort demonstrates eloquently, may imply that some items are
compared more than once with others. It may be that the comparison has
a small delay but, compounded over many uses, it leads to a
significant delay. In the case of sorting bills, it is more efficient
to compute all the total amounts first and then only use these amounts
during the sort process, because comparing one integer to another is
much faster than recomputing the sum of many integers in a list. So,
what is sorted is a list of pairs whose first component, called the
\emph{key}, is a simple and small representative of the second
component, called the \emph{value} (improperly, as keys are \Erlang
values as well, but such is the traditional nomenclature). This data
structure is sometimes called an \emph{association list}. Only the key
is used for sorting, not the value, therefore, if the key is an
integer, the comparison on the key is likely to be faster than on the
values. The only penalty is that all the keys must be precomputed in a
first pass over the initial data and they must be stripped from the
final result in an additional post\-processing. This time we shall
design these first and last passes in the most general fashion by
parameterisation upon the computation~\erlcode{Mk} of the keys:
\begin{verbatim}
% Preprocessing
mk_keys( _,        [])  -> [];
mk_keys(Mk,[V|Values])  -> [{Mk(V),V}|mk_keys(Mk,Values)].

% Postprocessing
rm_keys(            []) -> [];
rm_keys([{_,V}|KeyVal]) -> [V|rm_keys(KeyVal)].
\end{verbatim}
The delay of \erlcode{mk\_keys/2} depends on the delay
of~\erlcode{Mk}. The delay of \erlcode{rm\_keys(\(L\))} is~\(n+1\) if
\(L\)~contains \(n\)~pairs key\hyp{}value. Now we can sort by calling
\erlcode{isortf/2} with a comparison on two keys and with the function
to build the keys, \erlcode{sum\_tf/1}. For instance:
\begin{verbatim}
 rm_keys(isortf(fun({K1,_},{K2,_}) -> K1 > K2 end,
                mk_keys(fun sum_tf/1,
                        [[1,5,2,9],[7],[2,5,11],[4,3]])))
\end{verbatim}
It is very important to notice that we did not need to redefine
\erlcode{isortf/2}. Actually, \erlcode{isortf/2}, \erlcode{mk\_keys/2}
and \erlcode{rm\_keys/1} would very well constitute a library by
grouping their definitions in the same module. The client, that is,
the user of the library, would then provide the comparison function
fitted to her data to be sorted and the function making the keys. This
\emph{modularisation} is enabled by polymorphism and higher\hyp{}order
functions. Finally, let us determine the general average delay of
\begin{alltt}
      rm\_keys(isortf(fun(\{K1,\_\},\{K2,\_\}) -> K1 > K2 end,
                     mk\_keys(fun sum\_tf/1,\(L\))))
\end{alltt}
where \(L=\erlcode{[}L_0,L_1,\dots,L_{n-1}\erlcode{]}\) is a list of
\(n\)~lists, each of them containing \(l_i\)~integers and, in total,
\(p\)~integers. Since the call is the composition of several calls,
let us decompose it as the sum of the following delays:
\[
\comp{mk\_keys}{n,p} + \ave{isortf}{n} + \comp{rm\_keys}{n}.
\]
Furthermore, we have
\begin{align*}
\comp{mk\_keys}{n,p}
  &= (n+1) + \sum_{i=0}^{n-1}{(\comp{sum\_tf}{l_i} + 2)} = 3n + p + 1,\\
\ave{isortf}{n}
  &= \frac{1}{2}{n^2} + \frac{5}{2}{n} - H_n + 1,\quad
\comp{rm\_keys}{n}
  = n + 1.
\end{align*}
Therefore, the total average delay is
\[
\frac{1}{2}{n^2} + \frac{13}{2}{n} - H_n + p + 3.
\]
This delay is quadratic in the number of sub\hyp{}lists but linear in
the total number of integers, so the \emph{maximum average delay}
(sometimes called \emph{max\hyp{}mean delay}) happens, for a fixed
value of~\(p\), when \(n\)~is maximum, that is, when \(n=p\)---in
other words, when each sub\hyp{}list is a singleton. Dually, the
\emph{minimum average delay} (called \emph{min\hyp{}mean delay})
happens when \(n=1\), that is, when there is only one sub\hyp{}list.

As a last example proving the versatility of our program, let us sort
lists by their non\hyp{}increasing lengths:
\begin{verbatim}
 rm_keys(isortf(fun({K1,_},{K2,_}) -> K1 < K2 end,
                mk_keys(fun len_tf/1,
                        [[1,5,2,9],[7],[2,5,11],[4,3]])))
\end{verbatim}
where \erlcode{len\_tf/1} is defined as\label{code:len_tf}
\verbatiminput{len_tf.def}
The result is \erlcode{[[1,5,2,9],[2,5,11],[4,3],[7]]}. Notice that
\erlcode{[4,3]} occurs before~\erlcode{[7]} because the former is
longer.

Let us specialise further \erlcode{isortf/1}. Here is the definition
again:
\begin{alltt}
isortf( _,   [])     \(\MyArrow{\dblarrow}{\beta}\) [];
isortf(Gt,[I|L])     \(\MyArrow{\dblarrow}{\gamma}\) insert(Gt,I,isortf(Gt,L)).
insert(Gt,I,Q=[J|S]) \(\MyArrow{\dblarrow}{\delta\sb{0}}\) case Gt(I,J) of
                           true  \(\smashedrightarrow{\zeta}\) [J|insert(Gt,I,S)];
                           false \(\smashedrightarrow{\eta}\) [I|Q]
                         end;
insert( _,I,     []) \(\MyArrow{\dblarrow}{\epsilon\sb{1}}\) [I].
\end{alltt}
It is clear that if items are repeated in the input list, the call
\erlcode{Gt(I,J)} is expected to be rewritten to~\erlcode{false} at
least once, therefore duplicates are kept by clause~\clause{\eta} and
their relative order is preserved, that is, the sorting algorithm is
\emph{stable}. What if we do not want to preserve such duplicates in
the output? We need to rewrite the definition to support this
choice. The choice itself, that is, to keep or not to keep, would
naturally be implemented as an additional functional parameter,
say~\erlcode{Eq}. Also, we would need to change our assumption about
the values of the comparison: we need three cases, so the equality
case is explicit. Let us modify the variable~\erlcode{Gt} to reflect
this increase in detail and call it more generally~\erlcode{Cmp}
(\emph{compare}). The corresponding arguments should be values amongst
the user\hyp{}defined axioms~\erlcode{lt} (\emph{lower than}),
\erlcode{gt} (\emph{greater than}) and~\erlcode{eq} (\emph{equal}). We
have
\begin{alltt}
isortf(  _, _,   [])     \(\MyArrow{\dblarrow}{\beta}\) [];
isortf(Cmp,\textbf{Eq},[I|L])     \(\MyArrow{\dblarrow}{\gamma}\)
  insert(Cmp,\textbf{Eq},I,isortf(Cmp,\textbf{Eq},L)).
insert(Cmp,\textbf{Eq},I,Q=[J|S]) \(\MyArrow{\dblarrow}{\delta\sb{0}}\)
  case Cmp(I,J) of
    gt  \(\smashedrightarrow{\zeta}\) [J|insert(Cmp,\textbf{Eq},I,S)];
    lt  \(\smashedrightarrow{\eta}\) [I|Q];
    eq  \(\smashedrightarrow{\theta}\) \textbf{Eq(I,Q)}\hfill% \emph{New case}
  end;
insert(  _, _,I,     []) \(\MyArrow{\dblarrow}{\epsilon\sb{1}}\) [I].
\end{alltt}
Now, let us say that we want to sort nondecreasingly a list of
integers and retain possible redundant numbers, just as the previous
version allowed. We have (novelty in bold):
\begin{alltt}
isortf(fun(I,J) -> I>J end,\textbf{fun(I,Q) -> [I|Q] end},[5,3,1,4])
\end{alltt}
which results in \erlcode{[1,3,4,5]}. If we do not want the numbers
repeated, we form instead the call
\begin{alltt}
isortf(fun(I,J) -> I>J end,fun(_,Q) ->     \textbf{Q} end,[5,3,1,4])
\end{alltt}
resulting in \erlcode{[1,3,4,5]}. In passing, this technique allows us
to solve the problem of removing duplicates in a list of items for
which there is a total order. However, if only successive duplicates
have to be removed from a list, the function \erlcode{compress/1} on
page~\pageref{code:compress}:
\verbatiminput{compress.def} is more efficient because its growth is
linear in the input size.

We would be remiss if we do not mention that a higher\hyp{}order
function is not only a function whose at least one parameter is a
function, but it also can be a function whose calls evaluate in a
function. This kind of function is said to be \emph{curried}, as an
homage to the logician Haskell Curry. The possibility was already
there when we introduced the keywords \erlcode{fun}~and~\erlcode{end},
because they allow us to define an anonymous function and use it just
like another value, so nothing impeded us from using such a functional
value as the result of a named function, like in the following
function mathematically composing two functions:
\begin{verbatim}
compose(F,G) -> (fun(X) -> F(G(X)) end).
\end{verbatim}
Actually, the parentheses around the functional value are useless if
we remember that the keywords \erlcode{fun}~and~\erlcode{end} play the
role of parentheses when the anonymous function is not called:
\begin{verbatim}
compose(F,G) -> fun(X) -> F(G(X)) end.
\end{verbatim}
The higher\hyp{}order function \erlcode{compose/2} can be used to
compute the composition of two other functions, the result being a
function, of course.

\medskip

\paragraph{Maps.}
\label{maps}

We may want to compute a function which sums the images of a
list~\(L\) of integers by a given function~\(f\). In mathematical
notation, the final result would be expressed as
\[
\sum_{k \in L}{f(k)}.
\]
In order to implement this in \Erlang, we must proceed in two steps:
firstly, we need a higher\hyp{}order function which computes the
images of the items of a list by a function; secondly, we need a
function summing the integers of a list. We already have the latter,
known from page~\pageref{code:sum_tf_repeated} as
\erlcode{sum\_tf/1}. The former is traditionally called
\erlcode{map/2}, such that the call \erlcode{map(\(F\),\(L\))} applies
function~\(F\) to all the items of list~\(L\) and evaluates into the
list of the results. That is,
\[
\erlcode{map(\(F\),[\(I_0\),\(I_1\),\(\ldots\),\(I_{n-1}\)])}
\mathrel{\equiv}
\erlcode{[\(F\)(\(I_0\)),\(F\)(\(I_1\)),\(\ldots\),\(F\)(\(I_{n-1}\))]}.
\]
With this goal in mind, it is straightforward to define \erlcode{map/2}:
\begin{verbatim}
map(_,   []) -> [];
map(F,[I|L]) -> [F(I)|map(F,L)].
\end{verbatim}
The function we were looking for is now compactly defined as the
composition of \erlcode{map/2} and \erlcode{sum\_tf/1} as follows:
\begin{verbatim}
sum_f(F) -> fun(L) -> sum_tf(map(F,L)) end.
\end{verbatim}
For instance, the function call
\begin{center}
\erlcode{sum\_f(fun(X) -> X*X end)}
\end{center}
denotes the function which sums the squares of the numbers in a list
to be provided. It is equivalent to the value
\begin{center}
\erlcode{fun(L) -> sum\_tf(map(fun(X) -> X*X end,L)) end.}
\end{center}
It is possible to call this function just after is has been computed,
but \emph{parentheses must be added around a function being called
  when it is anonymous}. For instance, see the bold typefaces in
\begin{center}
\erlcode{\textbf{(}sum\_f(fun(X) -> X*X end)\textbf{)}([1,2,3])}.
\end{center}
The function \erlcode{map/2} is often used because it captures a
common operation on lists. For example, \vpageref{code:push}, we
defined
\begin{alltt}
push(_,          []) -> [];
push(I,[Perm|Perms]) -> [[I|Perm]|push(I,Perms)].
\end{alltt}
It is equivalent to
\begin{alltt}
push(I,Perms) -> map(fun(Perm) -> [I|Perm] end,Perms).
\end{alltt}
This style leads to clearer programs as it shows the underlying
recursive computation without having to read or write a definition for
it. In other words, using a higher\hyp{}order function like
\erlcode{map/2} allows us to identify a common recursive pattern and
let the programmer focus instead on the specific processing of the
items. We shall encounter other examples in the next sections but,
before we move on, imagine we typed instead
\begin{alltt}
push(I,Perms) -> map(fun(\textbf{Perms}) -> [I|\textbf{Perms}] end,Perms).
\end{alltt}
The \Erlang compiler would issue the following warning:
\begin{center}
\emph{\texttt{Warning: variable 'Perms' shadowed in
    'fun'.}}\label{shadowing}
\end{center}
What happens is that the parameter \textbf{\erlcode{Perms}} (in bold)
``hides'' the parameter \erlcode{Perms} of \erlcode{push/2} in the
sense that, in the body of the anonymous function, any occurrence
of~\erlcode{Perms} refers to \textbf{\erlcode{Perms}}. In this case,
it is not an error, but the compiler designers worried about
programmers walking on the shadowy side of the street. For example,
\begin{alltt}
push(I,Perms) -> map(fun(I) -> [I|I] end,Perms).\hfill% \emph{Capture}
\end{alltt}
is definitely wrong because the two variables~\erlcode{I} in
\erlcode{[I|I]}, which is the body of the anonymous function, are the
parameter of the anonymous function. A faulty shadowing is called a
\emph{capture}. Here, the parameter~\erlcode{I} bound by
\erlcode{push(I,Perms)} has been captured to mean instead the
parameter of \erlcode{fun(I)}. As a guideline, it is best to avoid
shadowing a parameter by another, as the \Erlang compiler reminds us
for our own sake. Note that
\begin{center}
\erlcode{sum\_tf(fun(L) -> L*L end)}
\end{center}
is fine because it is equivalent to
\begin{center}
\erlcode{fun(L) -> sum\_tf(map(fun(L) -> L*L end,L)) end}
\end{center}
which is a correct shadowing.

\medskip

\paragraph{Folds.}

Some other useful and frequently recursive patterns can be
conveniently reified into some other higher\hyp{}order
functions. Consider a function which traverses completely a list while
processing an accumulator depending on the current visited item. In
the end, the result is the final value of the accumulator, or else
another function is called to finalise it. A simple example is a
function computing the length of a list found \vpageref{code:len_tf}:
\verbatiminput{len_tf.def}
In this case, the accumulator is an integer and the operation on it
consists in incrementing it, whatever the current item is. Another
function reverses a list, \vpageref{code:rev}:
\verbatiminput{rev.def} 
Here, the accumulator is a list and the operation on it consists in
pushing the current item on top of it. Let us abstract separately
these two concerns in a higher\hyp{}order function
\begin{enumerate}

  \item which takes as input the function creating a new accumulator
    from the current item and the previous accumulator and

  \item which applies it successively to all the items of a parameter
    list.

\end{enumerate}
One famous function doing exactly this is called \erlcode{foldl/3} in
\Erlang, which stands for ``fold left,'' because once the new
accumulator for some item has been computed, the prefix of the list up
to it can be folded back, as if the list were written down on a sheet
of paper, because it is no longer useful. So the name should be better
read as ``fold from left to right'' or \emph{rightward fold}. We want
\[
\erlcode{foldl(\(F\),\(A\),[\(I_0\),\(I_1\),\(\ldots\),\(I_{n-1}\)])}
\mathrel{\equiv}
\erlcode{\(F\)(\(I_{n-1}\),\(\ldots\),\(F\)(\(I_1\),\(F\)(\(I_0\),\(A\)))\(\ldots\))},
\]
where \(A\)~is the initial value of the
accumulator. \Fig~\vref{fig:foldl} shows the corresponding abstract
syntax trees.
\begin{figure}
\centering
\subfloat[List \erlcode{L}]{%
  \includegraphics[bb=71 650 148 721]{a_list}
}
\qquad
\subfloat[Value of \erlcode{foldl(F,A,L)}]{%
  \includegraphics[bb=69 651 164 721]{foldl}
}
\caption{The result of \erlcode{foldl/3} on a non\hyp{}empty list
\label{fig:foldl}}
\end{figure}
The following definition implements the desired effect:
\begin{verbatim}
foldl(_,A,   []) -> A;
foldl(F,A,[I|L]) -> foldl(F,F(I,A),L).
\end{verbatim}
Now we can rewrite new definitions of \erlcode{len\_tf/1} and
\erlcode{rev/1}:
\begin{verbatim}
lenl(L) -> foldl(fun(_,A) ->   A+1 end, 0,L).
revl(L) -> foldl(fun(I,A) -> [I|A] end,[],L).
\end{verbatim}
Function \erlcode{foldl/3} is not in tail form because of the embedded
call \erlcode{F(I,A)}, but only a constant amount of control stack is
used for the recursion of \erlcode{foldl/3} itself (one node). In our
two examples, \erlcode{F}~is in tail form, therefore these new
definitions are \emph{almost} in tail form and can stand against the
originals. More definitions almost in tail form are
\begin{verbatim}
suml([N|L])    -> foldl(fun(I,A) ->      I+A end, N,L).
rev_joinl(P,Q) -> foldl(fun(I,A) ->    [I|A] end, Q,P).
rev_map(F,L)   -> foldl(fun(I,A) -> [F(I)|A] end,[],L).
\end{verbatim}
Again, the reason why these definitions are not exactly in tail form
is due to the call \erlcode{F(I,A)} in the definition of
\erlcode{foldl/3}, \emph{not} because of the functional arguments
\erlcode{fun(I,A) -> ... end} in the calls to \erlcode{foldl/3}: they
are not function calls but function definitions. The main advantage of
using \erlcode{foldl/3} is that it allows the programmer to focus
exclusively on the processing of the accumulator, whilst
\erlcode{foldl/3} itself provides the ride for free. Moreover, we can
easily compare different functions defined by means of
\erlcode{foldl/3}.

When the accumulator is a list on which values are pushed, the result
is in reverse order with respect to the input. That is why
\erlcode{rev\_map/2}, above, is not \erlcode{map/2}. The former is to
be preferred over the latter if the order of the items is not
relevant, because \erlcode{map/2} requires a control stack as long as
the input list. This leads us quite naturally to introduce another
higher\hyp{}order function: \erlcode{foldr/3}, meaning ``fold from
right to left,'' or \emph{leftward fold}. We expect
\[
\erlcode{foldr(\(F\),\(A\),[\(I_0\),\(I_1\),\(\ldots\),\(I_{n-1}\)])}
\mathrel{\equiv}
\erlcode{\(F\)(\(I_0\),\(F\)(\(I_1\),\(\ldots\),\(F\)(\(I_{n-1}\),\(A\)))\(\ldots\))}.
\]
\Fig~\vref{fig:foldr} shows the corresponding abstract syntax trees.
\begin{figure}
\centering
\subfloat[List \erlcode{L}]{%
  \includegraphics[bb=71 650 148 721]{a_list}
}
\qquad
\subfloat[Value of \erlcode{foldr(F,A,L)}]{%
  \includegraphics[bb=62 651 155 721]{foldr}
}
\caption{The result of \erlcode{foldr/3} on a non\hyp{}empty list
\label{fig:foldr}}
\end{figure}
We achieve this behaviour with the following definition:
\begin{verbatim}
foldr(_,A,   []) -> A;
foldr(F,A,[I|L]) -> F(I,foldr(F,A,L)).
\end{verbatim}
This definition, like \erlcode{foldl/3}, is not in tail form but,
unlike \erlcode{foldl/3}, it requires a control stack as long as the
input list. This is due to \emph{the recursive call of
  \erlcode{foldl/3} being in tail position}, that is, if its arguments
were in tail form, the recursive call would be in tail form as
well. In the case of \erlcode{foldr/3}, it is the recursive call
itself which is not in tail position, so, even if the functional
parameter~\erlcode{F} were in tail form, the amount of control stack
used by the call to \erlcode{foldr/3} would be proportional to the
length of the input list.

With the help of \erlcode{foldr/3}, we can redefine \erlcode{map/2}
and \erlcode{join/2} as
\begin{verbatim}
mapr(F,L)  -> foldr(fun(I,A) -> [F(I)|A] end,[],L).
joinr(P,Q) -> foldr(fun(I,A) ->    [I|A] end, P,Q).
\end{verbatim}
Compare \erlcode{rev\_joinl/2}, defined above, with \erlcode{joinr/2}:
the role of the accumulator and of the input list have been exchanged,
as well as \erlcode{foldl/3} and \erlcode{foldr/3}. It is also
possible to define
\begin{alltt}
lenr(L)     -> foldr(fun(_,A) -> 1+A end, 0,L).\hfill% \emph{To avoid}
sumr([N|L]) -> foldr(fun(I,A) -> I+A end, N,L).\hfill% \emph{To avoid}
isortr(L)   -> foldr(fun insert/2,       [],L).\hfill% \emph{To avoid}
\end{alltt}
but that would be unwise because \erlcode{foldr/3} does not use a
bounded amount of control stack, contrary to \erlcode{foldl/3}. In the
case of \erlcode{isort/1}, it is best to call \erlcode{foldl/3}
instead because the order of insertion does not matter in average
(although it swaps the best and worst cases if numbers are not
repeated, as seen \vpageref{best_worst_swapped}). This leads us to
formulate some guidelines about the transformation into tail form. We
already know that a definition in tail form is worth having or even
necessary if the maximum size of the control stack is smaller than
some input recursively traversed in its greatest extension. No
speed\hyp{}up should be expected a priori from turning a definition
not in tail form into one in tail form---although this may happen
sometimes. Usually,
\begin{itemize}

  \item it is preferable, if possible, to use \erlcode{foldl/3}
    instead of \erlcode{foldr/3} because, provided the functional
    parameter is in tail form, the call will use a small limited
    amount of control stack (if the parameter is not in tail form, at
    least \erlcode{foldl/3} won't burden further the control stack,
    contrary to \erlcode{foldr/3});

  \item when writing our own recursion, that is, without resorting to
    folds, it is best to have it in tail form if the accumulator is an
    integer, otherwise, the maximum size of the control stack may need
    to be proportional to the size of the input, despite the output
    being a single integer. (Contrast \erlcode{sum/1} and
    \erlcode{sum\_tf/2}, as well as \erlcode{len/1} and
    \erlcode{len\_tf/1}.)

\end{itemize}
Independently of stack allocation, there can be a significant
difference in delay when using one fold instead of the other. Take for
example the following two calls computing the same value:
\begin{center}
\erlcode{foldl(fun join/2,[],L)}
\(\mathrel{\equiv}\)
\erlcode{foldr(fun join/2,[],L)}.
\end{center}
The first one will be slower than the second. See exercise
\vpageref{ex:delay}.

What cannot be programmed with folds? As the defining properties show,
folds traverse the input list in its entirety, hence there is no way
to get off the bus while it is running. For instance,
\erlcode{rm\_fst/2} \vpageref{code:rm_fst_alpha}, \begin{alltt}
rm_fst(_,   []) \(\smashedrightarrow{\alpha}\) [];
rm_fst(I,[I|L]) \(\smashedrightarrow{\beta}\) L;
rm_fst(I,[J|L]) \(\smashedrightarrow{\gamma}\) [J|rm_fst(I,L)].
\end{alltt}

cannot be implemented by means of a fold because there are two ends to
the function calls: either the item has not been found and we ran
afoul the end of the list in clause~\clause{\alpha}, or it has been
found somewhere inside the list in clause~\clause{\beta}. However, in
theory, if we accept a full traversal of the list for every call, then
\erlcode{rm\_fst/2} can be programmed by means of a rightward
fold. The usual technique is to have an accumulator which is either an
atom meaning ``not found'' or a pair with an atom meaning ``found''
and the current built list. If the result is ``not found'' then we
just return the original list. The following function makes a stronger
case because it is really impossible to express by means of a fold,
even inefficiently. It is the ``complete tail''
function:\label{code:ctail} \verbatiminput{ctail.def} In general, a
function~\(F\) can be equivalently expressed by a call to a fold if,
and only if, for all lists \(P\)~and~\(Q\), for all item~\(I\), we
have\label{prop:fold}
\begin{equation}
\erlcode{\(F\)(\(P\))} \mathrel{\equiv} \erlcode{\(F\)(\(Q\))}
\Rightarrow
\erlcode{\(F\)([\(I\)|\(P\)])}
\mathrel{\equiv}
\erlcode{\(F\)([\(I\)|\(Q\)])}.\label{eq:fold_theory}
\end{equation}
We have \(\erlcode{ctail([])} \mathrel{\equiv} \erlcode{ctail([2])}\)
but \(\erlcode{ctail([1])} \mathrel{\not\equiv}
\erlcode{ctail([1,2])}\).

One positive side\hyp{}effect of using maps and folds is that they
sometimes allow the programmer to recognise some compositions that can
be optimised by means of some algebraic transformation of the abstract
syntax tree. As an example of a simple equation, we have, for all
functions \(F\)~and~\(G\):
\[
\erlcode{map(\(F\),map(\(G\),\(L\)))}
\mathrel{\equiv}
\erlcode{map(compose(\(F\),\(G\)),\(L\))}.
\]
Without counting in the delays of \(F\)~and~\(G\), the left\hyp{}hand
side induces the delay \(2n+2\), if \(L\)~contains \(n\)~items,
whereas the right\hyp{}hand side incurs the delay \(n+2\), so it is
much preferable to the former. Another interesting equation is
\begin{equation}
\erlcode{foldl(\(F\),\(A\),\(L\))}
\mathrel{\equiv}
\erlcode{foldr(\(F\),\(A\),\(L\))}\label{eq:foldlr}
\end{equation}
\emph{if \(F\)~is associative and symmetric.} Let us prove it. The
first clause of the definition of \erlcode{foldl/3} and the first
clause of the definition of \erlcode{foldr/3} imply that, for all
\(F\)~and~\(A\),
\[
\erlcode{foldl(\(F\),\(A\),[])}
\mathrel{\equiv} A
\mathrel{\equiv}
\erlcode{foldr(\(F\),\(A\),[])}.
\]
For non\hyp{}empty lists, this equation means:
\[
F(I_{n-1},\ldots,F(I_1,F(I_0,A))\ldots)
\mathrel{\equiv}
F(I_0,F(I_1,\ldots,F(I_{n-1},A))\ldots).
\]
Although the ellipses in the previous equation are intuitive, they are
not a valid foundation for a rigorous mathematical argument. By
definition, we have
\begin{align*}
\erlcode{foldl(\(F\),\(A\),[\(I\)|\(L\)])}
  &\mathrel{\equiv}
   \erlcode{foldl(\(F\),\(F\)(\(I\),\(A\)),\(L\))}.
\intertext{Dually, by definition, we also have}
\erlcode{foldr(\(F\),\(A\),[\(I\)|\(L\)])}
  &\mathrel{\equiv}
   \erlcode{\(F\)(\(I\),foldr(\(F\),\(A\),\(L\)))}.
\intertext{The original equation would thus be established for all
  lists if we prove}
\erlcode{foldl(\(F\),\(F\)(\(I\),\(A\)),\(L\))}
  &\mathrel{\equiv}
   \erlcode{\(F\)(\(I\),foldr(\(F\),\(A\),\(L\)))}.
\end{align*}
Let us call this conjecture~\(\mathcal{P}\) and prove it by
\emph{structural induction}. This principle states that, given a
finite data structure~\(S\), a property~\(\mathcal{P}\) to be proved
about it, then
\begin{enumerate}

  \item if \(\mathcal{P}\)~is provable for all the atomic~\(S\), that
    is, configurations of~\(S\) that cannot be decomposed;

  \item if, assuming that \(\mathcal{P}\)~is proved for all immediate
    substructures of~\(S\), then \(\mathcal{P}(S)\)~is proved;

  \item then \(\mathcal{P}(S)\)~is proved for \emph{all}~\(S\).

\end{enumerate}
The data structure~\(S\) being a list here, actually named~\(L\),
there is a unique atomic list: the empty list. So we must first prove
\(\mathcal{P}(\erlcode{[]})\). The first clause of the definition of
\erlcode{foldl/3} and the first clause of the definition of
\erlcode{foldr/3} imply that, for all \(F\)~and~\(A\),
\[
\erlcode{foldl(\(F\),\(F\)(\(I\),\(A\)),[])}
\mathrel{\equiv} \erlcode{\(F\)(\(I\),\(A\))}
\mathrel{\equiv}
\erlcode{\(F\)(\(I\),foldr(\(F\),\(A\),[]))},
\]
which is \(\mathcal{P}(\erlcode{[]})\). Next, let us consider a
non\hyp{}empty list \erlcode{[\(J\)|\(L\)]}. What are its immediate
substructures? By construction of lists, there is only one immediate
sub\hyp{}list of \erlcode{[\(I\)|\(L\)]}, namely~\(L\). Therefore, let
us assume \(\mathcal{P}(L)\) for a given list \(L\) and suppose that
\(F\)~is associative and symmetric (This is called the
\emph{structural induction hypothesis}) and let us prove
\(\mathcal{P}(\erlcode{[\(J\)|\(L\)]})\) for all~\(J\). For~\(F\) to
be associative means that, for all values~\(X\), \(Y\)~and~\(Z\), we
have
\begin{align*}
F(X,F(Y,Z)) &\mathrel{\equiv} F(F(X,Y),Z).
\intertext{The symmetry of \(F\) means that, for all \(X\) and \(Y\),
  we have}
F(X,Y) &\mathrel{\equiv} F(Y,X).
\end{align*}
Let us start with the left\hyp{}hand side of
\(\mathcal{P}(\erlcode{[\(J\)|\(L\)]})\):
\begin{align*}
&\erlcode{foldl(\(F\),\(F\)(\(I\),\(A\)),[\(J\)|\(L\)])}\\
  &\phantom{X}\mathrel{\equiv}
   \erlcode{foldl(\(F\),\(F\)(\(J\),\(F\)(\(I\),\(A\))),\(L\))},
  &&\text{by definition of \erlcode{foldl/3},}\\
  &\phantom{X}\mathrel{\equiv}
   \erlcode{foldl(\(F\),\(F\)(\(F\)(\(J\),\(I\)),\(A\)),\(L\))},
  &&\text{by associativity of \(F\),}\\
  &\phantom{X}\mathrel{\equiv}
   \erlcode{foldl(\(F\),\(F\)(\(F\)(\(I\),\(J\)),\(A\)),\(L\))},
  &&\text{by symmetry of \(F\),}\\
  &\phantom{X}\mathrel{\equiv}
   \erlcode{\(F\)(\(F\)(\(I\),\(J\)),foldr(\(F\),\(A\),\(L\)))}
  &&\text{by induction hypothesis \(\mathcal{P}(L)\),}\\
  &\phantom{X}\mathrel{\equiv}
   \erlcode{\(F\)(\(I\),\(F\)(\(J\),foldr(\(F\),\(A\),\(L\))))}
  &&\text{by associativity of \(F\),}\\
  &\phantom{X}\mathrel{\equiv}
   \erlcode{\(F\)(\(I\),foldr(\(F\),\(A\),[\(J\)|\(L\)]))},
  &&\text{by definition of \erlcode{foldr/3}.}
\end{align*}
This proves \(\mathcal{P}(\erlcode{[\(J\)|\(L\)]})\). The principle of
structural induction then implies that \(\mathcal{P}(L)\) is proved
for all lists~\(L\), hence the original equation \eqref{eq:foldlr}
\vpageref{eq:foldlr}. The previous derivation suggests a variation in
the definition of \erlcode{foldl/3}:
\begin{alltt}
foldl_alt(_,A,   []) -> A;
foldl_alt(F,A,[I|L]) -> foldl_alt(F,F(\textbf{A,I}),L).
\end{alltt}
The difference lies in the order of the parameters of~\erlcode{F}. We
would then have to prove a slightly different conjecture:
\[
\erlcode{foldl\_alt(\(F\),\(F\)(\(A\),\(I\)),\(L\))}
\mathrel{\equiv}
\erlcode{\(F\)(\(I\),foldr(\(F\),\(A\),\(L\)))}.
\]
The previous derivation would read now as follows:
\begin{align*}
&\erlcode{foldl\_alt(\(F\),\(F\)(\(A\),\(I\)),[\(J\)|\(L\)])}\\
  &\phantom{}\mathrel{\equiv}
   \erlcode{foldl\_alt(\(F\),\(F\)(\(F\)(\(A\),\(I\)),\(J\)),\(L\))},
  &&\text{by definition,}\\
  &\phantom{}\mathrel{\equiv}
   \erlcode{fold\_alt(\(F\),\(F\)(\(A\),\(F\)(\(I\),\(J\))),\(L\))}
  &&\text{by associativity of \(F\),}\\
  &\phantom{}\mathrel{\equiv}
   \erlcode{\(F\)(\(F\)(\(I\),\(J\)),foldr(\(F\),\(A\),\(L\)))},
  &&\text{by induction hypothesis \(\mathcal{P}(L)\),}\\
  &\phantom{}\mathrel{\equiv}
   \erlcode{\(F\)(\(I\),\(F\)(\(J\),foldr(\(F\),\(A\),\(L\))))}
  &&\text{by associativity of \(F\),}\\
  &\phantom{}\mathrel{\equiv}
   \erlcode{\(F\)(\(I\),foldr(\(F\),\(A\),[\(J\)|\(L\)]))},
  &&\text{by definition of \erlcode{foldr/3}.}
\end{align*}
We see that the symmetry of~\(F\) is not required anymore but in only
one remaining place: when the list is empty. Indeed, we have
\begin{align*}
\erlcode{foldl\_alt(\(F\),\(F\)(\(A\),\(I\)),[])}
  &\mathrel{\equiv} \erlcode{\(F\)(\(A\),\(I\))},
  &&\text{by definition.}\\
\erlcode{\(F\)(\(I\),foldr(\(F\),\(A\),[]))}
  &\mathrel{\equiv} \erlcode{\(F\)(\(I\),\(A\))},
  &&\text{by definition of \erlcode{foldr/3}.}
\end{align*}
Therefore, in order to prove the variant conjecture about
\erlcode{foldl\_alt/3} and \erlcode{foldr/3}, we must have
\[
\erlcode{\(F\)(\(A\),\(I\))}
\mathrel{\equiv}
\erlcode{\(F\)(\(I\),\(A\))},
\]
that is,~\(A\), which is the initial value of the accumulator, must be
symmetric with all items~\(I\) through~\(F\). This is not a clear
improvement over the first theorem about \erlcode{foldl/3}, which
required all pairs of successive items to be symmetric. Nevertheless,
there is an interesting special case, which is when \(A\)~is a neutral
element for~\(F\), that is, for all~\(I\),
\[
\erlcode{\(F\)(\(A\),\(I\))}
\mathrel{\equiv}
\erlcode{\(F\)(\(I\),\(A\))} \mathrel{\equiv} A.
\]
Then symmetry altogether is no more required. Therefore,
\erlcode{foldl\_alt/3} is preferable over \erlcode{foldl/3}, because
it provides more opportunities when transforming applications of
\erlcode{foldr/3}. But, since the standard library of \Erlang offers
the definition \erlcode{foldl/3}, we shall stick to it. The standard
library of \OCaml, however, proposes the function
\erlcode{fold\_left}, which corresponds to \erlcode{foldl\_alt/3}.

Anyway, theorem \eqref{eq:foldlr} allows us to transform immediately
some calls to \erlcode{foldr/3}, which requires an amount of control
stack at least proportional to the size of the input list, into calls
to \erlcode{foldl/3}, whose parameter~\(F\) is the only function
possibly not using a small and constant amount of control stack (if it
does, the gain is immediate and obvious). This is why the following
definitions are equivalent:
\begin{verbatim}
lenl(L) -> foldl(fun(_,A) -> A+1 end,0,L).
lenr(L) -> foldr(fun(_,A) -> A+1 end,0,L).
\end{verbatim}
Proving that the following two definitions are equivalent happens to
be a bit trickier:
\begin{verbatim}
suml([N|L]) -> foldl(fun(I,A) -> I+A end,N,L).
sumr([N|L]) -> foldr(fun(I,A) -> I+A end,N,L).
\end{verbatim}
The reason is that the first item of the list serves as the initial
value of the accumulator in both cases, despite the order of traversal
of the list being reversed (rightward versus leftward). Actually, we
already proved that they were equivalent \vpageref{proof_sum}. It is
much more obvious to see that the following definitions are
equivalent:
\begin{verbatim}
sum_tf(L=[_|_]) -> foldl(fun(I,A) -> I+A end,0,L).
sum(L=[_|_])    -> foldr(fun(I,A) -> I+A end,0,L).
\end{verbatim}
only because addition is associative and symmetric.

\medskip

\paragraph{Mappings without association lists.}

In order to illustrate further the expressive power of
higher\hyp{}order functions, let us muse about a small, albeit
unlikely, example. We mentioned \vpageref{sorted_association_lists}
association lists being a collection of pairs key\hyp{}value,
straightforwardly implemented with lists, for instance,
\erlcode{[\{a,0\},\{b,1\},\{a,5\}]}. A \emph{mapping} is an
association list which is searched based on the first component of the
pairs. Typically, we have
\begin{alltt}
find(_,       []) -> absent;
find(I,[\{I,V\}|L]) -> V;\hfill% \emph{Associated value found}
find(I,    [_|L]) -> find(I,L).\hfill% \emph{Keep searching}
\end{alltt}
Notice that if a key is repeated, only the first pair will be
considered, for instance, \erlcode{find(a,[\{a,0\},\{b,1\},\{a,5\}])}
evaluates to~\erlcode{0}, not~\erlcode{5}. These pairs are called
\emph{bindings}. Let us assume that we want to present formally what a
mapping is but without relying upon any particular programming
language. In this case, we must count on mathematics to convey the
concept, more precisely, on mathematical functions. We would say that
a mapping~\(M\) is a function from some finite set of
values~\(\mathcal{K}\) to some finite set of
values~\(\mathcal{V}\). Therefore, what was previously the conjunction
of a data type (a list) and a lookup function (\erlcode{find/2}) is
now a single function, representing the mapping \emph{and} the lookup
at the same time. A binding \(x \mapsto y\) is just another notation
for the pair~\((x, y)\), where \(x \in \mathcal{K}\) and \(y \in
\mathcal{V}\). We need now to express how a mapping is updated, that
is, how a mapping is extended with new bindings. With a list, this is
simply done by pushing a new pair but, without a list, we would say
that an update is a function itself, taking a mapping and a binding as
arguments and returning a new mapping. \emph{An update is thus a
  higher\hyp{}order function.}  Let the function~\((\oplus)\) be such
that \(M \mathrel{\oplus} x_1 \mapsto y\) is the \emph{update} of the
mapping~\(M\) by the binding \(x_1 \mapsto y\), as defined by
\begin{equation*}
(M \mathrel{\oplus} x_1 \mapsto y)(x_2) :=
\begin{cases}
  y      & \text{if} \; x_1 = x_2,\\
  M(x_2) & \text{otherwise.}
\end{cases}
\end{equation*}
We can check that we return the value associated to the first key
matching the input, as expected. The empty mapping would be a special
function returning a special symbol meaning ``not found,'' like
\(M_\varnothing(x) = \bot\), for all~\(x\). The mapping containing the
binding \((1,5)\) would be \(M_\varnothing \mathrel{\oplus} 1 \mapsto
5\). This is very abstract and independent of any programming
language, whilst being totally precise. If now the need arises to show
how this definition can be programmed, this is when functional
languages can shine. An update would be directly written in \Erlang as
\begin{verbatim}
update(M,{X1,Y}) -> fun(X2) -> case X2 of X1 -> Y; 
                                           _ -> M(X2) 
                               end
                    end.
\end{verbatim}
The correspondence with the formal definition is almost immediate,
there is no need to introduce a data structure and its
interpretation. The empty mapping is simply
\begin{verbatim}
empty(_) -> absent.
\end{verbatim}
For example, the mapping as list \erlcode{[\{a,0\},\{b,1\},\{a,5\}]}
can be modelled with higher\hyp{}order functions only as
\begin{center}
\erlcode{update(update(update(fun empty/1,\{a,5\}),\{b,1\}),\{a,0\})}.
\end{center}
Perhaps what needs to be learnt from all this is that lists in
functional languages, despite having a distinctive syntax and being
used pervasively, are not a fundamental data type: functions are.

\medskip

\paragraph{Functional encoding of tuples.}

Let us start by abstracting the tuple into its essence and, because in
a functional language, functions are the main feature, we should ask
ourselves what is \emph{done} with something we think of as a
tuple. Actually, we rushed because we should have realised first that
all tuples can actually be expressed in terms of the empty tuple and
\emph{pairs}, that is, tuples with exactly two components. For
instance, \erlcode{\{5,foo,\{fun(X) -> X*X end\}\}} can be rewritten
as \erlcode{\{5,\{foo,\{fun(X) -> X*X end,\{\}\}\}\}}. So let us
rephrase the question in terms of pairs only. Basically, a pair is
constructed (or \emph{injected}) and matched, that is, deconstructed (or
\emph{projected}). In the latter, two components can result: either
the first component of the pair or the second. This analysis leads to
the conclusion that the functional encoding of pairs requires three
functions: one for making, \erlcode{mk\_pair/2}, and two for unmaking,
\erlcode{fst/1} and \erlcode{snd/1}. Once a pair is built, it is
represented as a function, therefore functions extracting the
components take as an argument another function denoting the pair,
thus they are higher\hyp{}order functions. Consider
\begin{alltt}
mk_pair(X,Y) \(\smashedrightarrow{\alpha}\) fun(Pr) \(\smashedrightarrow{\beta}\) Pr(X,Y) end.\hfill%Pr \emph{is a projection}
fst(P) \(\smashedrightarrow{\gamma}\) P(fun(X,_) \(\smashedrightarrow{\delta}\) X end).\hfill% P \emph{denotes a pair}
snd(P) \(\smashedrightarrow{\epsilon}\) P(fun(_,Y) \(\smashedrightarrow{\zeta}\) Y end).
\end{alltt}
We have the following expected behaviour:
\begin{alltt}
fst(mk_pair(3,5)) \(\smashedrightarrow{\alpha}\) fst(fun(Pr) \(\smashedrightarrow{\beta}\) Pr(3,5))
                  \(\smashedrightarrow{\gamma}\) (fun(Pr) \(\!\smashedrightarrow{\beta}\!\) Pr(3,5))(fun(X,_) \(\!\smashedrightarrow{\delta}\!\) X end)
                  \(\smashedrightarrow{\beta}\) (fun(X,_) \(\smashedrightarrow{\delta}\) X end)(3,5)
                  \(\smashedrightarrow{\delta}\) 3.
\end{alltt}
To proof the versatility of this encoding, let us define a function
\erlcode{add/1} which adds the components of the pair passed to it:
\begin{alltt}
add(P) \(\smashedrightarrow{\eta}\) fst(P) + snd(P).
\end{alltt}
A call to \erlcode{add/1} would unravel as follows, assuming that
arguments are evaluated rightward:
\begin{alltt}
add(mk_pair(3,5))
         \(\smashedrightarrow{\alpha}\) add(fun(Pr) \(\smashedrightarrow{\beta}\) Pr(3,5) end)
         \(\smashedrightarrow{\eta}\)   fst(fun(Pr) \(\smashedrightarrow{\beta}\) Pr(3,5) end)
            + snd(fun(Pr) \(\smashedrightarrow{\beta}\) Pr(3,5) end)
         \(\smashedrightarrow{\gamma}\)   (fun(Pr) \(\smashedrightarrow{\beta}\) Pr(3,5) end)(fun(X,_) \(\smashedrightarrow{\delta}\) X end)
            + snd(fun(Pr) \(\smashedrightarrow{\beta}\) Pr(3,5) end)
         \(\smashedrightarrow{\beta}\)   (fun(X,_) \(\smashedrightarrow{\delta}\) X end)(3,5)
            + snd(fun(Pr) \(\smashedrightarrow{\beta}\) Pr(3,5) end)
         \(\smashedrightarrow{\delta}\) 3 + snd(fun(Pr) \(\smashedrightarrow{\beta}\) Pr(3,5) end)
         \(\smashedrightarrow{\epsilon}\) 3 + (fun(Pr) \(\smashedrightarrow{\beta}\) Pr(3,5) end)(fun(_,Y) \(\smashedrightarrow{\zeta}\) Y end)
         \(\smashedrightarrow{\beta}\) 3 + (fun(_,Y) \(\smashedrightarrow{\zeta}\) Y end)(3,5)
         \(\smashedrightarrow{\zeta}\) 3 + 5 \(=\) 8.
\end{alltt}
This is very nicely done but the keen reader may feel cheated, though,
because we could have simply defined \erlcode{add/2} as
\begin{verbatim}
add(X,Y) -> X + Y.
\end{verbatim}
Indeed, this critique is valid. The ability of functions to receive
various arguments at once amounts to them receiving \emph{one} tuple
exactly, whose components are these various values. Therefore, we have
to retry and make sure that our functions are \emph{nullary} or
\emph{unary}, that is, take zero or one argument. This is achieved by
taking the one value as argument and rewrite the call into a function
which will take in turn the next value as argument etc. This
translation is called \emph{currying}.
\begin{alltt}
mk_pair(X) \(\smashedrightarrow{\alpha}\) fun(Y) \(\smashedrightarrow{\beta}\) fun(Pr) \(\smashedrightarrow{\gamma}\) (Pr(X))(Y) end end.
fst(P) \(\smashedrightarrow{\delta}\) P(fun(X) \(\smashedrightarrow{\epsilon}\) fun(_) \(\smashedrightarrow{\zeta}\) X end end).
snd(P) \(\smashedrightarrow{\eta}\) P(fun(_) \(\smashedrightarrow{\theta}\) fun(Y) \(\smashedrightarrow{\iota}\) Y end end).
add(P) \(\smashedrightarrow{\kappa}\) fst(P) + snd(P).
\end{alltt}
Remember that writing \erlcode{fun(X) -> fun(P) -> ...} is the same as
\erlcode{fun(X) -> (fun(P) -> ...)} Also, the parentheses around
\erlcode{Pr(X)} are necessary in \Erlang because this call is in the
place of a function being called itself. Now we can run the example
again and the call now unfolds as follows, supposing that arguments
are computed rightward:
\begin{alltt}
add((mk_pair(3))(5))
  \(\smashedrightarrow{\alpha}\) add((fun(Y) \(\smashedrightarrow{\beta}\) fun(Pr) \(\smashedrightarrow{\gamma}\) (Pr(3))(Y) end end)(5))
  \(\smashedrightarrow{\beta}\) add(fun(Pr) \(\smashedrightarrow{\gamma}\) (Pr(3))(5) end)
  \(\smashedrightarrow{\kappa}\)   fst(fun(Pr) \(\smashedrightarrow{\gamma}\) (Pr(3))(5) end)
     + snd(fun(Pr) \(\smashedrightarrow{\gamma}\) (Pr(3))(5) end)
  \(\smashedrightarrow{\delta}\)   (fun(Pr)\(\smashedrightarrow{\gamma}\)(Pr(3))(5) end)(fun(X)\(\smashedrightarrow{\epsilon}\)fun(_)\(\smashedrightarrow{\zeta}\)X end end)
     + snd(fun(Pr) \(\smashedrightarrow{\gamma}\) (Pr(3))(5) end)
  \(\smashedrightarrow{\gamma}\)   ((fun(X) \(\smashedrightarrow{\epsilon}\) fun(_) \(\smashedrightarrow{\zeta}\) X end end)(3))(5)
     + snd(fun(Pr) \(\smashedrightarrow{\gamma}\) (Pr(3))(5) end)
  \(\smashedrightarrow{\epsilon}\) (fun(_) \(\smashedrightarrow{\zeta}\) 3 end)(5) + snd(fun(Pr) \(\smashedrightarrow{\gamma}\) (Pr(3))(5) end)
  \(\smashedrightarrow{\zeta}\) 3 + snd(fun(Pr) \(\smashedrightarrow{\gamma}\) (Pr(3))(5) end)
  \(\smashedrightarrow{\eta}\) 3
     + (fun(Pr)\(\smashedrightarrow{\gamma}\)(Pr(3))(5) end)(fun(_)\(\smashedrightarrow{\theta}\)fun(Y)\(\smashedrightarrow{\iota}\)Y end end)
  \(\smashedrightarrow{\gamma}\) 3 + ((fun(_) \(\smashedrightarrow{\theta}\) fun(Y) \(\smashedrightarrow{\iota}\) Y end end)(3))(5)
  \(\smashedrightarrow{\theta}\) 3 + (fun(Y) \(\smashedrightarrow{\iota}\) Y end)(5)
  \(\smashedrightarrow{\iota}\) 3 + 5 \(=\) 8.
\end{alltt}
Of course, this encoding is usually not worth doing because the number
of function calls is much greater than using a data
structure.

\medskip

\paragraph{Functional encoding of lists.}

To gain more insight into the nature of lists as a data structures, we
can encode lists only with higher\hyp{}order functions. In the former
view, a list is an infrastructure, a kind of inert container for data
and functions are expected to operate on it. In the latter, it is a
composition of functions containing data as arguments and waiting to
be called to \emph{do} something with them. The difference between
both points of view is not an imagined dichotomy between data and
functions, which is blurred in object\hyp{}oriented languages too, but
the fact that higher\hyp{}order functions \emph{alone} can make up a
list.

Since we already know how to encode pairs with higher\hyp{}order
functions, a first approach to encoding lists with functions simply
consists in encoding them with pairs. Abstractly, a list can either be
empty or constructed by pushing an item into another list, so all we
need is to translate these two concepts. The empty list can readily be
represented by the empty tuple~\erlcode{\{\}} and pushing becomes
pairing:
\begin{verbatim}
push(I,L) -> {I,L}.
\end{verbatim}
This encoding was introduced \vpageref{tuples_vs_list} to save memory
on linear accumulators. Here, we want to go one step further and get
rid of the pairs themselves by means of their functional
interpretation seen above, so \erlcode{push/2} becomes a renaming of
\erlcode{mk\_pair/2}:
\begin{alltt}
push(I,L) -> fun(Pr) -> Pr(I,L) end.\hfill% \emph{See} mk_pair/2
\end{alltt}
To understand the status of the empty list, we must switch to consider
projections on lists. These are usually called \emph{head} and
\emph{tail}, as we saw long ago \vpageref{chap:lists}. We implement
them as the original versions of \erlcode{fst/2} and \erlcode{snd/2},
where~\erlcode{L}, \erlcode{H}~and~\erlcode{T} denote, respectively,
an encoding of a list, a head and a tail:
\begin{alltt}
head(L) -> L(fun(H,_) -> H end).\hfill% \emph{See} fst/2
tail(L) -> L(fun(_,T) -> T end).\hfill% \emph{See} snd/2
\end{alltt}
Let us now think \emph{how} the empty list is used. It is a list such
that any projection of its putative contents fails, that is,
projecting the first component (the head) fails, as well as projecting
the second component (the tail). A trick consists in defining
\begin{alltt}
empty() -> fail.\hfill% \emph{The atom} fail \emph{is arbitrary}
\end{alltt}
The point is that \erlcode{empty/0} is nullary, so calling it with an
argument fails, as in \erlcode{head(fun empty/0)}. For example, the
list \erlcode{[a,b,c]} is encoded as
\begin{center}
\erlcode{push(a,push(b,push(c,fun empty/0))).}
\end{center}
This solution relies on the \emph{arity}, that is, the number of
parameters, of \erlcode{empty/0} to lead to failure. This failure is
consistent with the way classic lists, that is, lists as data structures,
are used: \erlcode{tail([\_|L]) -> L} and the call \erlcode{tail([])}
fails to match a clause. The limit of this encoding is that, being
based on functions, the encoded lists cannot be matched by the heads
of the clauses. For example, the function \erlcode{ctail/1}
\vpageref{code:ctail} \verbatiminput{ctail.def} cannot not be encoded
because we would need a way to check whether an encoded list is empty
without crashing the program if it is not. If we prefer the caller to
be gently informed of the problem instead, that is, we want the
definition of the projections to be complete, we could allow
\erlcode{empty/1} to take a projection which is then discarded:
\begin{verbatim}
empty(_) -> fail.
\end{verbatim}
We would have the rewrite \erlcode{head(fun empty/1) \(\rightarrow\)
  fail}, which is not a failure insofar the run\hyp{}time system is
concerned, but is interpreted by the application as a \emph{logical}
failure. Of course, it becomes the burden of the caller to check
whether the atom~\erlcode{fail} is returned and the burden of the list
maker to make sure not to push this atom in the encoded list,
otherwise a caller would confuse the empty list with a regular item.

\medskip

\paragraph{Local recursion.}

Many functions need some other auxiliary functions to carry out
subtasks. For example, consider
\verbatiminput{len_tf.def} where \erlcode{len/2} is the auxiliary
function, supposed not to be used by any other function. To forbid its
usage outside the scope of the module, it is simply omitted in the
\erlcode{-export} clause at the beginning. But it still could be
called from within the module it is defined. How could we avoid this
as well? This is where anonymous functions comes handy:
\begin{alltt}
len_tf(L) -> 
  Len = fun(   [],N) -> N;
           ([_|L],N) -> \textbf{Len}(L,N+1)\hfill% \emph{Does not compile}
        end,
  Len(L,0).
\end{alltt}
This limits the visibility of the anonymous function bound to
variable~\erlcode{Len} to the body of \erlcode{len\_tf/1}, which is
exactly what we wanted. The problem here is that this definition is
rejected by the \Erlang compiler because the binding
construct~(\erlcode{=}) does not make the variable on its
left\hyp{}hand side visible to the right\hyp{}side, hence
\erlcode{Len}~is unknown in the call \erlcode{Len(L,N+1)}. In some
other functional languages, there is a specific construct to allow
recursion on local definitions, for instance, \erlcode{let~rec} in
\OCaml, but the following hypotyposis is nevertheless theoretically
relevant. The original problem becomes another one: how can we define
anonymous \emph{recursive} functions? A workaround is to pass an
additional function parameter, which is used in stead of the recursive
call:
\begin{alltt}
len1(L) -> Len = fun(_,   [],N) -> N;
                    (\textbf{F},[_|L],N) -> \textbf{F}(\textbf{F},L,N+1)
                 end,
           Len(\textbf{Len},L,0).
\end{alltt}
Notice that we renamed \erlcode{len\_tf/1} into \erlcode{len1/1}
because we are going to envisage several variants. Moreover, the
anonymous function is not equivalent to \erlcode{len/2} because it
takes three arguments. Also, the compiler emits the following warning
(we removed the line number):
\begin{center}
\emph{\texttt{Warning: variable 'L' shadowed in 'fun'.}}
\end{center}
We have seen this before, \vpageref{shadowing}. Here, the shadowing is
harmless, because inside the anonymous function denoted
by~\erlcode{Len}, the original value of~\erlcode{L}, that is, the
argument of \erlcode{len1/1}, is not needed. Nevertheless, for the
sake of tranquillity, a simple renaming will get rid of the warning:
\begin{alltt}
len1(L) -> Len = fun(_,   [],N) -> N;
                    (F,[_|\textbf{P}],N) -> F(F,\textbf{P},N+1)\hfill% \emph{Renaming}
                 end,
           Len(Len,L,0).
\end{alltt}
We can alter this definition by currying the anonymous function and
renaming it so \erlcode{Len}~now is equivalent to \erlcode{fun len/2}:
\begin{alltt}
len2(L) -> H = \textbf{fun(F) ->} fun(   [],N) -> N;
                            ([_|P],N) -> \textbf{(F(F))}(P,N+1)
                         end
               \textbf{end},
           Len = H(H),\hfill% \emph{Equivalent to} fun len/2
           Len(L,0).
\end{alltt}
Let us define a function \erlcode{u/1} which auto\hyp{}applies its
functional argument and let us make use of it in stead of
\erlcode{F(F)}:
\begin{alltt}
u(F) -> fun(X,Y) -> (F(F))(X,Y) end.\hfill% \emph{Self\hyp{}application}

len3(L) -> H = fun(F) -> fun(   [],N) -> N;
                            ([_|P],N) -> \textbf{(u(F))}(P,N+1)
                         end
               end,
           (H(H))(L,0).\hfill% \emph{Expanded} Len
\end{alltt}
Let us replace now \erlcode{u(F)} by~\erlcode{F}. This transformation
does not preserve the semantics of~\erlcode{H}, so let us rename the
resulting function~\erlcode{G} and we redefine~\erlcode{H} to be
equivalent to its prior instance:
\begin{alltt}
len3(L) -> G = fun(F) -> fun(   [],N) -> N;
                            ([_|P],N) -> F(P,N+1)
                         end
               end,
           \textbf{H = fun(F) -> G(u(F)) end},
           (H(H))(L,0).
\end{alltt}
The interesting point is that the anonymous function referred to by
variable~\erlcode{G} is very similar to~\erlcode{Len} at the
beginning. (It may sound paradoxical to speak of anonymous functions
with names, but, in \Erlang, variables and function names are two
distinct syntactic categories, so there is no contradiction in terms.)
Here it is again:
\begin{alltt}
len_tf(L) -> 
  Len = fun(   [],N) -> N;
           ([_|L],N) -> Len(L,N+1)\hfill% \emph{Unfortunately invalid}
        end,
  Len(L,0).
\end{alltt}
The difference is that \erlcode{G}~abstracts over~\erlcode{F} instead
of having a (problematic) recursive call. Let us expand back the call
\erlcode{u(F)} and getting rid of~\erlcode{u/1} altogether:
\begin{alltt}
len4(L) ->
  G = fun(F) -> fun(   [],N) -> N;
                   ([_|P],N) -> F(P,N+1)
                end
      end,
  H = fun(F) -> G(\textbf{fun(X,Y) -> (F(F))(X,Y) end}) end,
  (H(H))(L,0).
\end{alltt}
To gain some generality, we can extract the assignments to
\erlcode{H}~and~\erlcode{Len}, put them into a new function
\erlcode{y/1} and expand~\erlcode{Len} in place:
\begin{alltt}
\textbf{y(G) -> H=fun(F) -> G(fun(X,Y)->(F(F))(X,Y) end) end, H(H).}

len5(L) -> G = fun(F) -> fun(   [],N) -> N;
                            ([_|P],N) -> F(P,N+1)
                         end
               end,
           (y(G))(L,0).
\end{alltt}
By putting the definition of function \erlcode{y/1} into a dedicated
module, we can now easily define recursive anonymous functions. There
is a limitation, though, which is that \erlcode{y/1} is tied to the
arity of~\erlcode{F}. For instance, we cannot use it for the
factorial:
\begin{alltt}
fact(N) -> G = fun(F) -> fun(0) -> 1;
                            (N) -> N * F(N-1)
                         end
               end,
           (y(G))(L,0).\hfill% Arity mismatch
\end{alltt}
Therefore, if we really want a general scheme, we should work with
fully curried functions, so all functions are unary:
\begin{alltt}
y(G) -> H = fun(F) -> G(fun(\textbf{X}) -> (F(F))(\textbf{X}) end) end, H(H).

len6(L) -> G=fun(F) -> fun(N) -> fun(   []) -> N;
                                    ([_|P]) -> (F(N+1))(P)
                                 end
                       end
             end,
           ((y2(G))(0))(L).
\end{alltt}
Notice that we swapped the order of the list and the integer, since
there is no pattern matching to be done on the latter. The grammar of
\Erlang obliges us to put parentheses around every function call
resulting in a function being immediately called, so calling fully
curried functions with all their arguments, like
\erlcode{((y2(G))(0))(L)}, ends up being a bit fastidious, although a
good text editor can help us in paring properly the parentheses.

\emph{The theoretical point of this derivation is that we can write
  recursive functions without relying on recursion at all,} since even
\erlcode{y/1}~is not recursive. In fact, nothing special is required
as long as we have function calls unrestricted by typing rules. Some
strongly and statically typed languages like \OCaml reject the
definition of~\erlcode{y/1} above precisely because of this, but other
valid, albeit more complex, definitions are possible. (In the case of
\OCaml, the switch \erlcode{-rectypes} allows us to compile the one
above, though.) If we grant ourselves the use of recursion, which we
never banned, we can actually write a simpler definition
of~\erlcode{y/1}, named \erlcode{y\_\_/1}:
\begin{alltt}
y__(F) -> fun(X) -> (F(y__(F)))(X) end.\hfill% \emph{Recursive}
\end{alltt}
This definition is actually very easy to come by, as it relies on the
computational equivalence, for all~\erlcode{X},
\[
\erlcode{(y\_\_(F))(X)} \mathrel{\equiv} \erlcode{(F(y\_\_(F)))(X)},
\]
If we assume the mathematical property \(\forall x.f(x) = g(x)
\Rightarrow f = g\), the previous equivalence would yield
\[
\erlcode{y\_\_(F)} \mathrel{\equiv} \erlcode{F(y\_\_(F))},
\]
which, by definition, shows that \erlcode{y\_\_(F)} is a fixpoint of
\erlcode{F}. Beware that
\begin{alltt}
y__(F) -> F(y__(F)).\hfill% \emph{Infinite loop}
\end{alltt}
does not work because the call \erlcode{y\_\_(\(F\))} would
\emph{immediately} start evaluating the call \erlcode{y\_\_(\(F\))} in
the body, therefore never ending. Some functional programming
languages have a different rewrite strategy than \Erlang and do not
always start by evaluating the argument in a function call, making
this definition directly workable. Another example:
\begin{verbatim}
fact(N) -> F = fun(F) -> fun(A) -> fun(0) -> A;
                                      (M) -> (F(A*M))(M-1)
                                   end
                         end
               end,
           ((y__(F))(1))(N).
\end{verbatim}
The technique we developed in the previous lines can be used to reduce
the amount of control stack in some functions. For example, consider
\verbatiminput{join_2.def} Note how the parameter~\erlcode{Q} is
threaded until the first argument is empty. This means that a
reference to the original~\erlcode{Q} is duplicated at each rewrite
until the last step, because the definition is not in tail form. In
order to avoid this, we could use a recursive anonymous function which
captures~\erlcode{Q} not as a parameter but as part of the (embedding)
scope:
\begin{alltt}
join(P,\textbf{Q}) -> G = fun(F) -> fun(   []) -> \textbf{Q};\hfill% Q \emph{in scope}
                              ([I|R]) -> [I|F(R)]
                           end
                 end,
             (y\_\_(G))(P).
\end{alltt}
This transformation is called \emph{lambda\hyp{}dropping} and its
inverse \emph{lambda\hyp{}lifting}. The function \erlcode{y\_\_/1} is
called the \emph{Y fixpoint combinator}.

We sometimes may want to define two mutually recursive anonymous
functions. Consider the following example, which is in practice
utterly useless and inefficient, but simple enough to illustrate our
point.
\begin{verbatim}
even(0) -> true;
even(N) -> odd(N-1).

odd(0) -> false;
odd(N) -> even(N-1).
\end{verbatim}
Let us say that we do not want \erlcode{even/1} to be callable from
any other function but \erlcode{odd/1}. This means that we want the
following pattern:
\begin{alltt}
odd(I) -> Even = fun(\fbcode{Odd}) -> \fbcode{fun(Odd)} end,
          Odd  = fun(\fbcode{Odd}) -> \fbcode{fun(Odd)} end,
          Odd(I).
\end{alltt}
where \erlcode{Even} and \erlcode{Odd} depend on each other. As the
canvas is laid out, \erlcode{Even} cannot call \erlcode{Odd} and
\erlcode{Odd} can only call \erlcode{Even}. The technique to allow
mutual recursion consists in abstracting the first function over the
second, that is, \erlcode{Even} becomes a function whose parameter is
a function destined to be used as \erlcode{Odd}:
\begin{alltt}
odd(I) -> Even = fun(\textbf{Odd}) -> fun(0) -> true;
                                (N) -> \textbf{Odd}(N-1)
                             end
                 end,
          Odd  = fun(\fbcode{Odd}) -> \fbcode{fun(Odd)} end,
          Odd(I).
\end{alltt}
This breaks the (lexical) dependence of \erlcode{Even} on
\erlcode{Odd}. The next step is more tricky. We can start naively,
though, and let the problem come to the fore by itself:
\begin{alltt}
odd(I) -> Even = fun(\textbf{Odd}) -> fun(0) -> true;
                                (N) -> \textbf{Odd}(N-1)
                             end
                 end,
          Odd  = fun(0) -> false;
                    (N) -> (Even(\fbox{Odd}))(N-1)
                 end,
          Odd(I).
\end{alltt}
The problem is not unheard of and we already know how to define an
anonymous recursive function by abstracting over the recursive call
and passing the resulting function to a fixpoint combinator:
\begin{alltt}
odd(I) -> Even = fun(Odd) -> fun(0) -> true;
                                (N) -> Odd(N-1)
                             end
                 end,
          Odd  = \textbf{y(fun(F) ->} fun(0) -> false;
                                (N) -> (Even(\textbf{F}))(N-1)
                             end
                   \textbf{end)},
          Odd(I).
\end{alltt}

\medskip

\paragraph{Exercises.}
\label{ex:higher-order_functions}

\noindent [See answers page~\pageref{ans:higher-order_functions}.]
\begin{enumerate}

  \item Write a definition for the function \erlcode{max\_f/3} such
    that the call \erlcode{max\_f(\(F\),\(A\),\(B\))} is the maximum
    value of the integer~\(F(X)\) when integer~\(X\) ranges between
    integers \(A\)~and~\(B\), with~\(A \leqslant B\).

  \item Write a definition for the function \erlcode{max\_x/3} such
    that the call \erlcode{max\_x(\(F\),\(A\),\(B\))} is the smallest
    integer~\(X\) ranging between integers \(A\)~and~\(B\), with~\(A
    \leqslant B\), such that the integer \(F(X)\)~is maximum.

\end{enumerate}

