%%-*-latex-*-

\chapter{Answers to Exercises}
%\addcontentsline{toc}{chapter}{Answers to Exercises}

%\setcounter{chapter}{1}

%\setcounter{section}{2}
\section*{Joining and Reversing}
\label{ans:joining_and_reversing}

\noindent [See questions \vpageref{ex:joining_and_reversing}.] 

\medskip

\paragraph{Question 1.}
An auxiliary function \erlcode{len\_\_/2} is needed to hold the
current length from the beginning of the list. Accordingly, it has to
be initialised to~\erlcode{0} in the first call.
\begin{alltt}
len\_tf(L)       \(\smashedrightarrow{\alpha}\) len__(L,0).
len__(    [],N) \(\smashedrightarrow{\beta}\) N;
len__([_|L]),N) \(\smashedrightarrow{\gamma}\) len__(L,N+1).
\end{alltt}
Clause~\clause{\alpha} is used only once, at the start,
clause~\clause{\beta}, at the very end, and clause~\clause{\gamma} is
used for each item in the list, that is, \(n\)~times. In total:
\(n+2\)~rewrites.

\medskip

\paragraph{Question 2.}

Here is how to find the penultimate item in a list:
\begin{alltt}
penul([I,_]) -> I;\hfill% \emph{Use a comma}
penul([_|L]) -> penul(L).\hfill% \emph{Use a vertical bar}
\end{alltt}
We have to make sure that the calls \erlcode{penul([])} and
\erlcode{penul([\(I\)])}, for any~\(I\), fail, whilst all the others
succeed. When faced to choose among an infinite set of test values,
select those which seems extreme cases. Here, this method is not
useful, though, because the function is fully polymorphic in the type
of the items, that is, it does not depend on the nature of the lists
elements. This definition is in tail form because each body is in tail
form: the first because it is a value, the second because it is a call
not embedded in another call and not containing other calls. The
number of rewrites to reach a value is number of items before and
including the penultimate, that is, \(n-1\)~if the list contains
\(n\)~items.

\medskip

\paragraph{Question 3.}

Here a way to repeat the first item, while ignoring the others:
\begin{verbatim}
rep_fst([I|L])  -> [I|repeat(I,L)].
repeat(_,   []) -> [];
repeat(I,[_|L]) -> [I|repeat(I,L)].
\end{verbatim}
Notice that these are \emph{not} definitions in tail form, because the
calls \erlcode{repeat(I,L)} are embedded into a list in the first and
last clauses. A bad design would be to first compute the length of the
list and then construct a list of this length containing the first
item, because it would require two complete traversals of the list
instead of one, thus about 50\%~more time, since the number of
rewrites here is~\(n+2\) for an input of \(n\)~items.

The following version uses temporarily more memory but is shorter.
The number of rewrites to find a value is also~\(n+1\).
\begin{alltt}
rep_fst(     []) -> [];
rep_fst(    [I]) -> [I];
rep_fst([I,_|L]) -> [I|rep_fst([I|L])].\hfill% \emph{Two} | \emph{instead of 1}
\end{alltt}

\medskip

\paragraph{Question 4.}

The first approach that perhaps should spring up is to reuse
\erlcode{rep\_fst/1} and the list reversing function \erlcode{rev/1}:
\begin{verbatim}
rep_lst(L) -> rep_fst(rev(L)).

rev(L)            -> rev_join(L,[]).
rev_join(   [],Q) -> Q;
rev_join([I|P],Q) -> rev_join(P,[I|Q]).
\end{verbatim}
The number of rewrites to compute a call is~\(2n+3\). Another method
consists in finding the last item and repeating:
\begin{verbatim}
rep_lst([]) -> [];
rep_lst(L) -> rep_fst(last(L)).

last(  [I]) -> I;
last([_|L]) -> last(L).
\end{verbatim}
The number of rewrites to reach the result is~\(2n+2\). What should be
avoided here is the temptation of computing the length of the list
with the intent of rebuilding one of same length, filled with the last
item. This would lead to something in the following lines:
\begin{verbatim}
rep_lst([I|L])      -> len_and_make([I|L],0).

len_and_make(  [I],N) -> make(N+1,I);
len_and_make([_|L],N) -> len_and_make(L,N+1).

make(0,_)             -> [];
make(N,I)             -> [I|make(N-1,I)].
\end{verbatim}
The number of rewrites is \(1 + n + (n+1) = 2n+2\) but this program is
really too long for solving such a simple problem, so previous
approaches must be preferred instead.

\medskip

\paragraph{Question 5.}

Let us label the rewrite rules with some Greek letters to distinguish
them:
\begin{alltt}
rev_bis(   []) \(\smashedrightarrow{\alpha}\) [];
rev_bis([I|L]) \(\smashedrightarrow{\beta}\) join(rev_bis(L),[I],[]).

join(   [],Q,   []) \(\smashedrightarrow{\gamma}\) Q;
join(   [],Q,[E|A]) \(\smashedrightarrow{\delta}\) join([],[E|Q],A);
join([E|P],Q,    A) \(\smashedrightarrow{\epsilon}\) join(P,Q,[E|A]).
\end{alltt}
We deduce the trace of \erlcode{rev\_bis([3,2,1])} in flat
representation as follows (we skip the abstract syntax trees):
\begin{tabbing}
\erlcode{rev\_} \= \(\smashedrightarrow{\beta}\) \=\kill
\erlcode{rev\_bis([3,2,1])}\\
\> \(\smashedrightarrow{\beta}\)
\> \erlcode{join(rev\_bis([2,1]),[3],[])}\\
\> \(\smashedrightarrow{\beta}\)
\> \erlcode{join(join(rev\_bis([1]),[2],[]),[3],[])}\\
\> \(\smashedrightarrow{\beta}\)
\> \erlcode{join(join(join(rev\_bis([]),[1],[]),[2],[]),[3],[])}\\
\> \(\smashedrightarrow{\alpha}\)
\> \erlcode{join(join(join([],[1],[]),[2],[]),[3],[])}\\
\> \(\smashedrightarrow{\gamma}\)
\> \erlcode{join(join([1],[2],[]),[3],[])}\\
\> \(\smashedrightarrow{\epsilon}\)
\> \erlcode{join(join([],[2],[1]),[3],[])}\\
\> \(\smashedrightarrow{\delta}\)
\> \erlcode{join(join([],[1,2],[]),[3],p[)}\\
\> \(\smashedrightarrow{\gamma}\)
\> \erlcode{join([1,2],[3],[])}\\
\> \(\smashedrightarrow{\epsilon}\)
\> \erlcode{join([2],[3],[1])}\\
\> \(\smashedrightarrow{\epsilon}\)
\> \erlcode{join([],[3],[2,1])}\\
\> \(\smashedrightarrow{\delta}\)
\> \erlcode{join([],[2,3],[1])}\\
\> \(\smashedrightarrow{\delta}\)
\> \erlcode{join([],[1,2,3],[])}\\
\> \(\smashedrightarrow{\gamma}\)
\> \erlcode{[1,2,3]}\textrm{.}
\end{tabbing}

\section*{Delay}
\label{ans:delay}

\noindent [See questions \vpageref{ex:delay}.]

\medskip

\paragraph{Question 1.}

We are given
\begin{alltt}
rev_ter(   []) \(\smashedrightarrow{\alpha}\) [];
rev_ter([I|L]) \(\smashedrightarrow{\beta}\) join_tf(rev_ter(L),[I]).

join_tf(P,Q)        \(\smashedrightarrow{\gamma}\) join(P,Q,[]).
join(   [],Q,   []) \(\smashedrightarrow{\delta}\) Q;
join(   [],Q,[I|A]) \(\smashedrightarrow{\epsilon}\) join([],[I|Q],A);
join([I|P],Q,    A) \(\smashedrightarrow{\zeta}\) join(P,Q,[I|A]).
\end{alltt}
Let \(\comp{rev\_ter}{n}\) be the delay of the call
\erlcode{rev\_ter(\(L\))}, where the list~\(L\) holds \(n\)~items. The
definitions yield the equations
\begin{align*}
\comp{rev\_ter}{0}   &\eqn{\alpha} 1,\quad
\comp{rev\_ter}{n+1} \eqn{\beta} 1 + \comp{rev\_ter}{n}
                                    + \comp{join\_tf}{n}.
\intertext{We proved earlier that \(\comp{join\_tf}{n} = 2n + 2\),
  therefore}
\comp{rev\_ter}{n+1} &= 2n + 3 + \comp{rev\_ter}{n}\\
\sum_{k=0}^{n-1}{\comp{rev\_ter}{k+1}}
   &= \sum_{k=0}^{n-1}{(2k+3)} + \sum_{k=0}^{n-1}{\comp{rev\_ter}{k}}\\
\comp{rev\_ter}{n}
  &= 2 \cdot \frac{n(n-1)}{2} + 3n + \comp{rev\_ter}{0}
  \eqn{\alpha} n^2 + 2n + 1 = (n+1)^2.
\end{align*}

\medskip

\paragraph{Question 2.} Skipped.

\medskip

\paragraph{Question 3.}
We are given
\begin{verbatim}
join_3a(P,Q,R) -> join(P,join(Q,R)).
join_3b(P,Q,R) -> join(join(P,Q),R).
\end{verbatim}
The proved earlier \vpageref{delay:join} that
\(\comp{join}{n,m}=n+1\), where \(n\)~and~\(m\) are, respectively, the
length of the first and second lists. We straightforwardly deduce
\begin{align*}
\comp{join\_3a}{p,q,r}
  &= 1 + \comp{join}{q,r} + \comp{join}{p,q+r}
   = 1 + (q+1) + (p+1) = p + q + 3;\\
\comp{join\_3b}{p,q,r}
  &= 1 + \comp{join}{p,q} + \comp{join}{p+q,r}
   = 1 + (p+1) + (p+q+1) = 2p + q + 3\\
  &= \comp{join\_3a}{p,q,r} + p \geqslant \comp{join\_3a}{p,q,r}.
\end{align*}
Therefore, never use the call
\erlcode{join(join(\(P\),\(Q\)),\(R\))}.


\section*{Filtering Out}
\label{ans:filtering_out}

\noindent [See questions \vpageref{ex:filtering_out}.]

\medskip

\paragraph{Question 1.} Skipped.

\medskip

\paragraph{Question 2.}

The conceptual difference between \erlcode{rm\_fst/2} and the sought
\erlcode{rm\_all/2} is simply that the latter must not stop after
finding the first occurrence, but instead traverse the whole list to
find more occurrences---and ignore them in the reconstructed
list. Hence
\begin{verbatim}
rm_all(_,   []) -> [];
rm_all(I,[I|L]) -> rm_all(I,L);
rm_all(J,[I|L]) -> [I|rm_all(J,L)].
\end{verbatim}
Following the previous remark, there is no worst case and there is no
need to rely on equations to compute the delay: \(\comp{rm\_all}{n} =
n+1\). In the same vein, the number of pushes if the number of found
items. A version in tail form requires a list accumulator in which to
save the visited items which are not the one sought after, and,
therefore, it needs a final reversal:
\begin{verbatim}
rm_all_tf(I,L) -> rm_all_tf__(I,L,[]).

rm_all_tf__(_,   [],A) -> rev(A);
rm_all_tf__(I,[I|L],A) -> rm_all_tf__(I,L,A);
rm_all_tf__(I,[J|L],A) -> rm_all_tf__(I,L,[J|A]).
\end{verbatim}

\medskip

\paragraph{Question 3.}

\begin{alltt}
rm_lst3(_,   []) \(\smashedrightarrow{\alpha}\) [];
rm_lst3(I,[I|P]) \(\smashedrightarrow{\beta}\) aux3(I,P,P);
rm_lst3(I,[J|P]) \(\smashedrightarrow{\gamma}\) [J|rm_lst3(I,P)].
aux3(_,   [],P)  \(\smashedrightarrow{\delta}\) P;
aux3(I,[I|_],P)  \(\smashedrightarrow{\epsilon}\) [I|rm_lst3(I,P)];
aux3(I,[_|Q],P)  \(\smashedrightarrow{\zeta}\) aux3(I,Q,P).
\end{alltt}
If the item is missing, the sequential search fails as usual with a
delay of \(n+1\). Otherwise, let us name \(x_1 < x_2 < \dots < x_p\)
the indices of all the occurrences of the item in the list. Then, the
trace is
\begin{equation*}
\gamma^{x_1} 
\cdot
\prod_{k=2}^{p}(\beta\zeta^{x_k-x_{k-1}-1})(\epsilon\gamma^{x_k-x_{k-1}-1})
\cdot
(\beta\zeta^{n-x_p-1}\delta).
\end{equation*}
The length of the trace is therefore
\begin{equation*}
x_1 + 2\sum_{k=2}^{p}(x_k-x_{k-1}) + n - x_p + 1 = n + x_p - x_1 + 1.
\end{equation*}
oIn other words, if the position of the first occurrence is
noted~\(f\) and the position of the last is~\(k\), we find that
\(\comp{rm\_lst3}{n,f,k} = n + k - f + 1\). We deduce that the best
delay happens when \(k-f+1=p\), that is, when all the occurrences are
consecutive: we have \(\best{rm\_lst3}{n,p} = n + p\). The worst case
occurs when~\({f=0}\) and~\({k=n-1}\), that is, when there is at least
one occurrence at the head and one at the end: \(\worst{rm\_lst3}{n} =
n + (n-1) + 1 = 2n\). We can check that when the list only contains
occurrences of the item, best and worst delays concur in~\(2n\). The
average delay when the item is present:
\begin{align*}
\ave{rm\_lst3}{n} 
  &:=
   \frac{2}{n(n+1)}\!\!\sum_{f=0}^{n-1}\sum_{k=f}^{n-1}{\comp{rm\_lst3}{n,f,k}}
   = \frac{2}{n(n+1)}\!\!\sum_{f=0}^{n-1}\sum_{k=f}^{n-1}{(n-f+k+1)}\\
  &= n + 1 + \frac{2}{n(n+1)}\sum_{f=0}^{n-1}\left(-f(n-f) +
   \sum_{k=f}^{n-1}{k}\right)\\
  &= n + 1 
     + \frac{2}{n(n+1)}\sum_{f=0}^{n-1}\left(f^2-nf+
         \left(\sum_{k=0}^{n-1}{k} -
         \sum_{k=0}^{f-1}{k}\right)\right)\\
  &= n + 1 
     + \frac{2}{n(n+1)}\sum_{f=0}^{n-1}\left(f^2 -nf + 
       \left(\frac{n(n-1)}{2} - \frac{f(f-1)}{2}\right)\right)\\
  &= n + 1 + \frac{2}{n(n+1)} \cdot \frac{n^2(n-1)}{2}\\
  &\phantom{= n\;}
     + \frac{2}{n(n+1)}\sum_{f=0}^{n-1}{\left(\frac{1}{2}{f^2} -
       \frac{2n-1}{2}{f}\right)}\\
  &= n + 1 + \frac{n(n-1)}{n+1}\\
  &\phantom{= n\;}
     + \frac{1}{n(n+1)}\sum_{f=0}^{n-1}{f^2}
     - \frac{2}{n(n+1)} \cdot \frac{2n-1}{2}\sum_{f=0}^{n-1}{f}\\
  &= n + 1 + \frac{n(n-1)}{n+1}\\
  &\phantom{= n\;}
     + \frac{1}{n(n+1)} \cdot \frac{(n-1)n(2n-1)}{6}
     - \frac{2n-1}{n(n+1)} \cdot \frac{n(n-1)}{2}\\
  &= \frac{4}{3}{n} + \frac{2}{3}.
\end{align*}
The complete table is now found in
\fig~\vref{fig:complete_table_rm_lst}.
\begin{figure}[b]
\centering
\includegraphics[bb=71 603 375 721]{rm_lst_table}
\caption{Complete table of delays for variations on ``filtering out''
\label{fig:complete_table_rm_lst}}
\end{figure}
On a last note, clause \clause{\delta} allows the output to share with
the input the longest common suffix \emph{if the item occurs in the
  list}. (This was not the case with other versions based on
reversals.) The code can thus be improved so the sharing is maximum in
any case. How?

\smallskip

\noindent We can compare \(\worst{rm\_lst2}{n}\) and
\(\worst{rm\_lst3}{n}\) because the worst case of the former happens
when the item occurs \emph{anywhere} in the list, whilst the worst
case of the latter is when the item occurs uniquely at the end of the
list. Indeed, this means that for any input list of length~\(n\),
\(\worst{rm\_lst3}{n} < \worst{rm\_lst2}{n}\). Since both best cases
happen when the item is absent, we conclude that \(\comp{rm\_lst3}{n}
< \comp{rm\_lst2}{n}\). Furthermore, we observe that
\(\worst{rm\_lst3}{n} < \best{rm\_lst}{n}\) and \(\worst{rm\_lst3}{n}
< \best{rm\_lst\_tf}{n}\), so \(\comp{rm\_lst3}{n} <
\comp{rm\_lst}{n}\) and \(\comp{rm\_lst\_tf}{n}\). The only way to
compare \erlcode{rm\_lst3/2} to \erlcode{rm\_lst1/2} is to consider
their exact delays. We have, for all~\({f \leqslant k}\),
\[
k < n \Rightarrow \comp{rm\_lst3}{n,f,k} < \comp{rm\_lst1}{n,f,k}.
\]
Therefore, the fastest function is \erlcode{rm\_lst3/2}.

Note: Using the \erlcode{case} construct presented in
section~\vref{chap:higher-order_functions}, we can write more clearly
the code above as
\begin{alltt}
rm_lst3(_,   []) -> [];
rm_lst3(I,[I|P]) -> \textbf{case} mem(I,P) \textbf{of}
                      true  -> [I|rm_lst3(I,P)];
                      false -> P
                    \textbf{end};
rm_lst3(I,[J|P]) -> [J|rm_lst3(I,P)].
mem(_,   [])     -> false;
mem(I,[I|\_])     -> true;
mem(I,[\_|P])     -> mem(I,P).
\end{alltt}

\medskip

\paragraph{Question 4.}

Reading the definition with~\({n=0}\), that is, a null period, leads
to ``..., except the items occurring every \(0\)~positions.''
Therefore, the output ought to be the same as the input. If~\(n\) is
greater than the length of the list, the items at these extra
positions are simply ignored because they don't exist. Interpreting
the definition for the case of \(n < 0\) is inconclusive, so let us
decide, for instance, that this case is the same as \(n=0\). No case
has been left out in our analysis. We need a counter to record the
current position, thus initially set to~\erlcode{1}, and, when the
counter matches the period, it is reset to~\erlcode{1}. For the sake
of versatility, we shall set the accumulator as first argument,
instead of the last:
\begin{alltt}
drop(L,P) -> drop([],1,L,P).

drop(A,_,   [],_) \(\smashedrightarrow{\alpha}\) rev(A);
drop(A,P,[_|L],P) \(\smashedrightarrow{\beta}\) drop(A,1,L,P);
drop(A,N,[I|L],P) \(\smashedrightarrow{\gamma}\) drop([I|A],N+1,L,P).
\end{alltt}
The delay occurs when the delay of the call \erlcode{rev(A)} is
maximum, that is, when the accumulator~\erlcode{A} is the
longest. This is achieved when the minimum number of items have been
dropped (skipped), that is , when clause~\clause{\beta} is never
used. This means that no item is dropped and the output matches the
input, so the accumulator at the end is simply the input reversed. The
design analysis reminds us that this happens when the period is either
negative or greater than the length of the list. Moreover,
\(\comp{drop}{n} = 1 + n + 1 + (n+2) = 2n+4\), because \(\comp{rev}{n}
= n+2\).

\medskip

\paragraph{Question 5.}

We are given
\begin{verbatim}
diff([M,N]) -> M - N;
diff([M|L]) -> M - diff(L).
\end{verbatim}
This definition only fails on the empty list. Also, the first clause
is in tail form but not the second, because the call \erlcode{diff(L)}
is an argument to the operator~(\erlcode{-}). If we add an integer
accumulator, we end in trouble:
\begin{alltt}
diff([]) -> list_too_short;
diff( L) -> diff__(L,0).

diff__([M,N],A) -> (M-N)-A;\hfill% \emph{Wrong}
diff__([M|L],A) -> diff__(L,M-A).\hfill% \emph{Wrong}
\end{alltt}
because subtraction is not associative: \(x-(y-z) \neq (x-y)-z\). A
more general method is to be preferred then: the accumulator is a list
in which the numbers are stored until we can begin subtracting
them. This is equivalent to reverse the input and then performing the
delayed series of subtractions:
\begin{alltt}
diff_tf([]) -> list_too_short;
diff_tf( L) -> rev_diff(L,[]).

rev_diff(   [],A) -> diff__(A,0);\hfill% \emph{Prepare to subtract}
rev_diff([I|L],A) -> rev_diff(L,[I|A]).\hfill% \emph{Reversal}

diff__(   [],N) -> N;
diff__([I|L],N) -> diff__(L,I-N).
\end{alltt}
The delay of \erlcode{diff/1} is~\(n-1\), when the input contains
\(n\)~numbers. The delay of \erlcode{diff\_tf/1} is
\(\comp{diff\_tf}{n} = 1 + (n+1) + (n+1) = 2n + 3\).

\medskip

\paragraph{Question 6.}

We are given two definitions not in tail form:
\begin{verbatim}
srev(   []) -> [];
srev([I|L]) -> join(srev(L),[I]).

join(   [],Q) -> Q;
join([I|P],Q) -> [I|join(P,Q)].
\end{verbatim}
We have
\begin{alltt}
srev_tf([]) -> [];
srev_tf( L) -> srev__(L,[]).

srev__(   [],[I|A]) -> join__([],[I],[],A);
srev__([I|L],    A) -> srev__(L,[I|A]).

join__(   [],Q,   [],   []) -> Q;
join__(   [],Q,   [],[I|B]) -> join__(Q,[I],[],B);
join__(   [],Q,[I|A],    B) -> join__([],[I|Q],A,B);
join__([I|P],Q,    A,    B) -> join__(P,Q,[I|A],B).
\end{alltt}

\section*{Delay and Tail Form Revisited}
\label{ans:delay_and_tail_form_revisited}

\noindent [See questions \vpageref{ex:delay_and_tail_form_revisited}.]

\medskip

\paragraph{Question 1.}
\begin{alltt}
split(L,N) when N > 0 -> split([],L,N).

split(A,P=[\_|\_],0) -> \{rev(A),P\};
split(A,  [I|L],N) -> split([I|A],L,N-1).

% \emph{Reminder}
%
rev(L) -> rev_join(L,[]).
rev_join(   [],Q) -> Q;
rev_join([I|P],Q) -> rev_join(P,[I|Q]).
\end{alltt}

\medskip

\paragraph{Question 2.}
\begin{alltt}
rot(L,N) when N < 0 -> aux(len(L));
rot(L,N)            -> rot([],L,N rem len(L)).

aux(Len) -> rot([],L,Len + N rem Len).

rot([],    L,0) -> L;
rot( A,    L,0) -> join(L,rev(A));
rot( A,[I|L],N) -> rot([I|A],L,N-1).

% \emph{Reminder}
%
rev(L) -> rev_join(L,[]).

rev_join(   [],Q) -> Q;
rev_join([I|P],Q) -> rev_join(P,[I|Q]).

join(   [],Q) -> Q;
join([I|P],Q) -> [I|join(P,Q)].
\end{alltt}

\section*{Queues}
\label{ans:queues}

\noindent [See questions \vpageref{ex:queues}.]

\medskip

\paragraph{Question 1.}
\verbatiminput{r.erl}

\medskip

\paragraph{Question 2.} Skipped.

\section*{Aliasing and Tail\hyp{}Call Optimisation}
\label{ans:aliasing_and_tail-call_optimisation}

\noindent [See question
  \vpageref{ex:aliasing_and_tail-call_optimisation}.]

The only sharing between the input and the output lies in the items of
the list. This was expected because when building a list in reverse we
cannot reuse any part of the original superstructure.

\section*{Persistence and Backtracking}
\label{ans:persistence_and_backtracking}

\noindent [See questions \vpageref{ex:persistence_and_backtracking}.]

\medskip

\paragraph{Question 1.} Skipped.

\medskip

\paragraph{Question 2.} Skipped.

\medskip

\paragraph{Question 3.}

\begin{itemize}

  \item {[\erlcode{version/1}]} Skipped.

  \item {[\erlcode{set/3}]} Either the index is in bounds or not:
\begin{alltt}
set(E,I,\{\{L,U\},D,S\}) when L =< I,I =< U -> \fbcode{aux(E,I,A)}.
\end{alltt}
  Note the conjunction of tests in the guard, meaning here ``when
  \erlcode{L~=<~I} \emph{and} \erlcode{I~=<~U}.'' The same effect can
  be achieved with an auxiliary function, like
\begin{alltt}
set(E,I,A=\{\{L,_\},_,_\}) when L =< I -> aux(E,I,A);
aux(E,I,  \{\{L,U\},D,S\}) when I =< U -> \fbcode{aux(E,I,A)}.
\end{alltt}
  The case ``empty'' is easy because it means that the index is
  necessary out of bounds, according to the specification. This is the
  same as having a non\hyp{}empty array but the index is out of
  bounds:
\begin{alltt}
set(E,I,\{\{L,U\},D,S\}) when L =< I,I =< U -> \fbcode{aux(E,I,A)};
set(_,_,          _)                    -> \textbf{out}.
\end{alltt}
  We can see now that we can remove the first clause:
\begin{alltt}
set(E,I,\{\{L,U\},D,S\}) when L =< I,I =< U -> \fbcode{aux(E,I,A)};
set(_,_,          _)                    -> out.
\end{alltt}
  We need now an auxiliary function \erlcode{set\_\_/3} such that the
  function call \erlcode{set\_\_(\(E\),\(I\),\(S\))} rewrites into the
  the updated~\(S\).
\begin{alltt}
set(E,I,\{B=\{L,U\},D,S\}) when L =< I,I =< U ->
\hfill\{B,D,\textbf{set__(E,I,S)}\};
set(_,_,          _)                    -> out.
set__(E,I,         \{\})                  -> \fbcode{aux(E,I,A)};
set__(E,I,\{set,E,J,S\})                  -> \fbcode{aux(E,I,A)}.
\end{alltt}
  The first clause of \erlcode{set\_\_/3} corresponds to an out of
  bound access:
\begin{alltt}
set__(_,_,         \{\})                  -> \textbf{out};
set__(E,I,\{set,E,J,S\})                  -> \fbcode{aux(E,I,A)}.
\end{alltt}
  We realise now that we actually do not need to know the details of
  the first update because what only matters is that there is at least
  one. Then we add the assignment:
\begin{alltt}
set__(_,_,\{\})                  -> out;
set__(E,I, \textbf{S})                  -> \textbf{\{set,E,I,S\}}.
\end{alltt}


  \item {[\erlcode{get/2}]} The beginning of the analysis is identical
    to the analysis of \erlcode{set/3} until:
\begin{alltt}
get(\{\{L,U\},_,S\},I) when L =< I,I =< U -> get__(S,I);
get(          _,_)                    -> out.
get__(         \{\},_)                  -> out;
get__(\{set,E,J,S\},I)                  -> \fbcode{aux(E,I,A)}.
\end{alltt}
  The remaining clause needs its head to be refined in two cases:
  either~\erlcode{I} has been found in the current assignment or not:
\begin{alltt}
get__(         \{\},_)                  -> out;
get__(\{set,E,I,A\},I)                  -> \fbcode{aux(E,I,A)}.
get__(\{set,E,J,A\},I)                  -> \fbcode{aux(E,I,A)}.
\end{alltt}
  If found, the end with the current content of the cell; otherwise,
  we need to check the previous assignments, that is, perform a
  recursive call on the rest of the updates:
\begin{alltt}
get__(         \{\},_)                  -> out;
get__(\{set,E,I,_\},I)                  -> \textbf{E};
get__(\{set,_,_,A\},I)                  -> \textbf{get\_\_(A,I)}.
\end{alltt}

  \item {[\erlcode{nth/2}]} This function can simply be expressed in
    terms of \erlcode{get/2}:
\begin{verbatim}
nth(A={{L,_},_,_},I) -> get(A,L+I-1).
\end{verbatim}

  \item {[\erlcode{mem/2}]} Skipped.
    
  \item {[\erlcode{inv/1}]} Skipped.

  \item {[\erlcode{unset/2}]}
\begin{alltt}
unset(\{B=\{L,U\},D,S\},I) when L =< I,I =< U ->
\hfill\{B,D,unset__(S,I)\};
unset__(         \{\},_) -> out;\hfill% \emph{Absent}
unset__(\{set,_,I,A\},I) -> A;\hfill% \emph{Found}
unset__(\{set,E,J,A\},I) -> \{set,E,J,unset__(A,I)\}.
\end{alltt}

\end{itemize}


\section*{Permutations and Sorting}
\label{ans:permutations_and_sorting}

\noindent [See question \vpageref{ex:permutations_and_sorting}.]
The straight application of our algorithm leads to the following code.

\verbatiminput{p.erl}

\noindent The clauses of \erlcode{appk/2} matching the tags
\erlcode{k1}, \erlcode{k2}~and~\erlcode{k4} have isomorphic bodies, so
they can be merged into one clause. To keep track of the fusion, we
call the new tag~\erlcode{k124} in the following last version.

\verbatiminput{q.erl}


\section*{Insertion Sort}
\label{ans:insertion_sort}

\noindent [See questions \vpageref{ex:insertion_sort}.]

\medskip

\paragraph{Question 1.}
Let us recall the code and label the arrows as follows:
\begin{alltt}
i2w_a(L) -> i2w_a([],[],L).

i2w_a(   [],    Q,     []) \(\smashedrightarrow{\alpha}\) Q;
i2w_a([I|P],    Q,     []) \(\smashedrightarrow{\beta}\) i2w_a(    P,[I|Q],[]);
i2w_a(    P,[J|Q],L=[K|_]) when J < K 
                            \(\smashedrightarrow{\gamma}\) i2w_a([J|P],    Q, L);
\textbf{i2w_a(   [],    Q,  [K|R]) \(\smashedrightarrow{\delta}\) i2w_a(  [K],    Q, R);}
i2w_a([I|P],    Q,L=[K|_]) when K < I
                           \(\smashedrightarrow{\epsilon}\) i2w_a(    P,[I|Q], L);
i2w_a(    P,    Q,  [K|R]) \(\smashedrightarrow{\zeta}\) i2w_a(    P,[K|Q], R).
\end{alltt}
Clause~\clause{\delta} (in bold) leads to a slightly more balanced
scheme than \erlcode{isort2w/1}, insofar as when the left list is
empty, the item under consideration will be pushed in it instead of
the right list. Let us start with a small example, shown in
\fig.~\vref{fig:i2w_a}.
\begin{figure}
\centering
\includegraphics{i2w_a}
\caption{Slightly balanced two\hyp{}way insertions with
  \erlcode{i2w\_a/1}\label{fig:i2w_a}}
\end{figure}
Note how the configuration \erlcode{[],[\(c\),\(a\),\(b\)]} is not
reached anymore, as it is maximally unbalanced. Therefore, we should
expect in average a little improvement over the original function
\erlcode{isort2w/1}. We shall skip the best and worst cases and
instead focus on the average delay as follows. Let us
note~\(\ave{}{p,q}\) the average delay for inserting one number in a
configuration where the left list is made of \(p\) numbers and the
right list contains \(q\)~numbers, all of them being unique. This
function was \vpageref{A_pq}:
\[
\ave{\(\gamma\delta\epsilon\)}{k-q,q}
  = \frac{1}{k+1}{q^2} - \frac{k}{k+1}{q} + \frac{k+2}{2}.
\]
Let~\(\ave{}{n}\) represent the average delay required to insert a
random number into two random lists of \(n\)~numbers in total. This
function is different. As hinted, the difference with the analysis of
\erlcode{isort2w/1} consists in \emph{not} having to sum the
term~\(\ave{}{0,k}\) to compute~\(\ave{}{n}\) and, in its stead,
use~\(\ave{}{1,k-1}\), this change corresponding to the added
clause. We have
\begin{align*}
  \ave{}{0} &:= 1,\quad
  \ave{}{1} := \frac{3}{2},\\
  \ave{}{k} &:= \frac{1}{k}
     \left(\sum_{p+q=k}^{p,q>0}{\ave{}{p,q}} +
     \ave{}{1,k-1}\right)
     = \frac{1}{k}\left(\sum_{q=1}^{k-1}{\ave{}{k-q,q}} +
     \ave{}{1,k-1}\right).
\end{align*}
The values of \(\ave{}{0}\)~and~\(\ave{}{1}\) come from examining
\fig~\vref{fig:i2w_a}. Because
\[
\ave{}{1,k-1} = \frac{k^2 + k + 4}{2k+2},
\]
we resume
\begin{align*}
  \ave{}{k}
    &= \frac{1}{k}\left(\frac{1}{k+1}\sum_{q=1}^{k-1}{q^2} 
       - \frac{k}{k+1}\sum_{q=1}^{k-1}{q}
       + \frac{k+2}{2}\sum_{q=1}^{k-1}{1}
       + \frac{k^2 + k + 4}{2k+2}\right)\\
    &= \frac{2k^3 + 9k^2 + k + 6}{6k(k+1)}
     = \frac{k}{3} + \frac{1}{k} - \frac{2}{k+1} 
       + \frac{7}{6}.
 \end{align*}
 Let~\(\ave{\(\alpha\beta\)}{k}\) be the number of rewrites to reverse
 \(k\)~numbers from the first list to the second and finally return
 the second list. We have
 \begin{align*}
   \ave{\(\alpha\beta\)}{k} &:= k + 1;\quad
   \ave{i2w\_a}{0} := 1,\\
   \ave{i2w\_a}{n} &:= 1 + \sum_{k=0}^{n-1}{\ave{}{k}}
        + \frac{1}{n}\left(\sum_{k=1}^{n-1}{\ave{\(\alpha\beta\)}{k}}
        + 2\right)\\
     &= 1 + \left(\ave{}{0} + \ave{}{1} +
        \sum_{k=2}^{n-1}{\ave{}{k}}\right)
        + \frac{1}{n}\left(\sum_{k=1}^{n-1}{(k+1)} + 2\right)\\
     &= \frac{7}{2} + \sum_{k=2}^{n-1}{\left(\frac{k}{3}
        + \frac{1}{k} - \frac{2}{k+1} + \frac{7}{6}\right)}
        + \frac{1}{n}\left(\sum_{k=1}^{n}{k} - 1\right) +
        \frac{2}{n}\\
     &= \frac{7}{2} + \sum_{k=1}^{n-1}{\left(\frac{k}{3}
        + \frac{1}{k} - \frac{2}{k+1} + \frac{7}{6}\right)}
        - \left(\frac{1}{3}
        + \frac{1}{1} - \frac{2}{1+1} + \frac{7}{6}\right)\\
     &\phantom{= \frac{7}{2} \;}
        + \frac{1}{n} \cdot \frac{n(n+1)}{2} - \frac{1}{n} +
          \frac{2}{n}\\
     &= \left(\frac{1}{3}\sum_{k=1}^{n-1}{k}
        + \sum_{k=1}^{n-1}{\frac{1}{k}} 
        - 2 \sum_{k=1}^{n-1}{\frac{1}{k+1}}
        + \frac{7}{6}{(n-1)}\right)
        + \frac{n}{2} + \frac{1}{n} + \frac{5}{2}\\
     &= \frac{1}{3} \cdot \frac{n(n-1)}{2}
        + \sum_{k=1}^{n}{\frac{1}{k}}
        - 2 \sum_{k=2}^{n}{\frac{1}{k}}
        + \frac{5}{3}{n} + \frac{4}{3}\\
     &= \frac{1}{6}{n^2} + \frac{3}{2}{n} + H_n - 2(H_n - 1) +
        \frac{4}{3}
      = \frac{1}{6}{n^2} + \frac{3}{2}{n} - H_n + \frac{10}{3}.
\end{align*}
Furthermore,
\[
\ave{i2w\_a}{n} = \frac{1}{6}{n^2} + \frac{3}{2}{n} - H_n + \frac{10}{3}
  < \frac{1}{6}{n^2} + \frac{3}{2}{n} + \frac{4}{3} = \ave{isort2w}{n}
\Leftrightarrow 2 < H_n.
\]
Moreover, since \((H_n)_n\) is increasing and \(H_3 = 11/6\) and \(H_4
= 25/12\),
\[
H_3 < 2 < H_4 \leqslant H_n,
\]
so~\(\ave{i2w\_a}{n} < \ave{isort2w}{n} \Leftrightarrow 3 < n\). This
means that if four or more items are to be sorted, it is faster to
call \erlcode{i2w\_a/1} than \erlcode{isort2w/1}.


\section*{Higher\hyp{}Order Functions}
\label{ans:higher-order_functions}

\noindent [See questions \vpageref{ex:higher-order_functions}.]

\verbatiminput{ho.erl}

\section*{Merge Sort}
\label{ans:merge_sort}

\noindent [See questions \vpageref{ex:merge_sort}.]

\medskip

\paragraph{Question 1.}

It is not necessary to redo all the calculations. The shortcut reveals
itself if we start with assessing the difference between
\(\cdelay{merge}{m,n}\), which is the number of comparisons, and
\(\comp{merge}{m,n}\), which is the number of function calls. Let us
recall the \Erlang definition of \erlcode{merge/2}:
\begin{alltt}
merge(   [],      Q)            \(\smashedrightarrow{\alpha}\) Q;
merge(    P,     [])            \(\smashedrightarrow{\beta}\) P;
merge([I|P],Q=[J|_]) when I < J \(\smashedrightarrow{\gamma}\) [I|merge(P,Q)];
merge(    P,  [J|Q])            \(\smashedrightarrow{\delta}\) [J|merge(P,Q)].
\end{alltt}
 In order to only count the number of comparisons,
we must not count clauses \clause{\alpha}~and~\clause{\beta} in the
delay: \(\cdelay{merge}{m,n} = \comp{merge}{m,n} - 1\). Thus
\begin{align*}
\cbest{merge}{m,n}  &= \best{merge}{m,n} - 1 = \min\{m,n\};\\
\cworst{merge}{m,m} &= \worst{merge}{m,n} - 1 = m + n - 1;\\
\cave{merge}{m,m}   &= \ave{merge}{m,n} - 1 = mn/(m+1) + mn/(n+1).
\end{align*}
The shape of equation \eqref{eq:delay_power_2}
\vpageref{eq:delay_power_2} is the same, except it applies now to the
number of comparisons:
\begin{align*}
\Dcomp(2^p) &= 2^{p-1}\sum_{k=0}^{p-1}{\frac{1}{2^k}\cdelay{merge}{2^k,2^k}}
 = 2^{p-1}\sum_{k=0}^{p-1}{\frac{1}{2^k}\left(\comp{merge}{2^k,2^k}-1\right)}\\
 &= 2^{p-1}\sum_{k=0}^{p-1}{\frac{1}{2^k}\comp{merge}{2^k,2^k}}
   - 2^{p-1}\sum_{k=0}^{p-1}{\frac{1}{2^k}}
 = \Dideal(2^p) - 2^{p-1}\left(2 - \frac{1}{2^{p-1}}\right)\\
 &= \Dideal(2^p) - 2^p + 1.
\end{align*}
Therefore
\begin{align*}
\Bcomp(2^p) &= \Bideal(2^p) - 2^p + 1 = p2^{p-1};\\
\Wcomp(2^p) &= \Wideal(2^p) - 2^p + 1 = p2^p - 2^p + 1;\\
\Acomp(2^p) &= \Aideal(2^p) - 2^p + 1
             = p2^p - 2^p\sum_{k=0}^{p-1}\frac{1}{2^k+1}.
\end{align*}
For the general case, let us reconsider equation \eqref{eq:msort_gen}
\vpageref{eq:msort_gen}:
\begin{equation*}
\Dideal(n)
  = \sum_{i=0}^{r}{\Dideal(2^{e_i})}
    +
    \sum_{i=1}^{r}{\comp{merge}{2^{e_i},2^{e_{i-1}}+\dots+2^{e_0}}}.
\end{equation*}
Let us take into account only the number of comparisons:
\begin{align*}
\Dcomp(n)
  &= \sum_{i=0}^{r}{\Dcomp(2^{e_i})}
    +
    \sum_{i=1}^{r}{\cdelay{merge}{2^{e_i},2^{e_{i-1}}+\dots+2^{e_0}}}\\
  &= \sum_{i=0}^{r}{(\Dideal(2^{e_i}) - 2^{e_i} + 1)}
    +
    \sum_{i=1}^{r}{\left(\comp{merge}{2^{e_i},2^{e_{i-1}}+\dots+2^{e_0}}
      - 1\right)}\\
  &= \sum_{i=0}^{r}{\Dideal(2^{e_i})} - \sum_{i=0}^{r}{2^{e_i}}
     + (r+1)
     +  \sum_{i=1}^{r}{\comp{merge}{2^{e_i},2^{e_{i-1}}+\dots+2^{e_0}}}
     - r,\\
\Dcomp(n)
  &= \Dideal(n) - n + 1.
\end{align*}
It is easy to derive the best, worst and average number of
comparisons:
\[
\begin{array}{r@{\,}c@{\,}c@{\,}c@{\,}l}
  \frac{1}{2}{n\lg n} - \frac{3}{2}{n} + 1 & \leqslant & \Bcomp(n)
     & \leqslant & \frac{1}{2}{n \lg n} + n - 1,\\
  n\lg n - 4n + 3 & \leqslant & \Wcomp(n)
                  & \leqslant & n\lg n + n - 1,\\
  n\lg n - (\alpha + 3)n + \frac{3}{2} & \leqslant & \Acomp(n)
                             & \leqslant & n\lg n + n - 1.
\end{array}
\]
We have \(S(n) \leqslant \overline{\Wideal}(n)\), therefore, \(S(n)
\leqslant n \lg n + n - 1\).

\medskip

\paragraph{Question 2.}

If \(n=2^p-1\), then the binary representation of \(n\) is only made
of \(1\)\hyp{}bits. In other words, if we use the variant
representation \(n=2^{e_r} + \dots + 2^{e_1} + 2^{e_0} > 0\), with
\(e_r > \dots > e_1 > e_0 \geqslant 0\) and \({r \geqslant 0}\), it
means that \({e_i=i}\). Let us start by recalling the delays when
\({n=2^p}\). Equality \eqref{eq:best_power} \vpageref{eq:best_power}
is: \(\Bideal(2^p) = p2^{p-1} + 2^p -
1\). Equality~\eqref{eq:worst_power} \vpageref{eq:worst_power}:
\(\Wideal(2^p) = p2^p\). Reusing equation~\eqref{eq:Bideal}
\vpageref{eq:Bideal}, defining~\(\Bideal(n)\) in the general, with the
additional constraint \({e_i=i}\), becomes
\begin{align*}
\Bideal(n)
  &=   \sum_{i=0}^{r}{\Bideal(2^{e_i})}
     + r + \sum_{i=1}^{r}\sum_{j=0}^{i-1}{2^{e_j}}
   = \sum_{i=0}^{r}{\Bideal(2^i)}
     + r + \sum_{i=1}^{r}\sum_{j=0}^{i-1}{2^j}\\
  &= \sum_{i=0}^{r}{(i2^{i-1} + 2^i - 1)}
     + r + \sum_{i=1}^{r}{(2^i - 1)}\\
  &= \sum_{i=1}^{r}{i2^{i-1}} + 2^{r+2} - r - 4.
\end{align*}
A closed form for the remaining sum can be reached by the
\emph{perturbation technique}, which consists in summing one more term
and rewriting the new sum in terms of the original sum. Let \(S_{r} :=
\sum_{i=1}^{r}{i2^{i-1}}\). Then
\begin{align*}
S_{r} + (r+1)2^r
  &= \sum_{i=1}^{r+1}{i2^{i-1}} = \sum_{i=0}^{r}{(i+1)2^{i}}
   = \sum_{i=0}^{r}{i2^{i}} + \sum_{i=0}^{r}{2^{i}}\\
  &= 2\sum_{i=1}^{r}{i2^{i-1}} + (2^{r+1} - 1)
   = 2S_{r} + 2^{r+1} - 1,
\end{align*}
yielding \(S_{r} = (r+1)2^r - 2^{r+1} + 1 = (r-1)2^r + 1\). Then
\begin{equation*}
\Bideal(n) = ((r-1)2^r + 1) + 2^{r+2} - r - 4
           = (r+3)(2^r - 1).
\end{equation*}
We assumed \({e_i=i}\), that is, the bits of~\(n\) are all~\(1\),
which implies that there exists~\(p\) such that \(n=2^p-1\), hence
\(\nu_n := r+1 = p\). We conclude:
\begin{equation*}
\Bideal(2^p-1) = (p+2)(2^{p-1} -1).
\end{equation*}
We may remark that \(\Bideal(2^p) = \Bideal(2^p-1) + p + 1\).
Furthermore, reusing the result \(\Bcomp(n) = \Bideal(n) - n + 1\) of
Question~1, we would deduce
\begin{equation}
\Bcomp(2^p) = \Bcomp(2^p-1) + p.\label{eq:Bcomp}
\end{equation}
Let us turn now to equation \eqref{eq:Wideal} \vpageref{eq:Wideal}
defining \(\Wideal(n)\):
\begin{align*}
\Wideal(n)
  &=   \sum_{i=0}^{r}{\Wideal(2^{e_i})}
     + \sum_{i=1}^{r}\sum_{j=0}^{i}{2^{e_j}}
   =   \sum_{i=0}^{r}{e_i2^{e_i}}
     + \sum_{i=1}^{r}\sum_{j=0}^{i}{2^{e_j}}\\
  &=   \sum_{i=0}^{r}{i2^{i}}
     + \sum_{i=1}^{r}\sum_{j=0}^{i}{2^{j}}
   = 2S_r + \sum_{i=1}^{r}(2^{i+1} - 1),\\
%  &= (r-1)2^{r+1} + 2 + \sum_{i=0}^{r-1}{2^{i+2}} - r
%   = (r-1)2^{r+1} + 2 + 4(2^r - 1) - r\\
\Wideal(2^p-1)
  &= (r+1)2^{r+1} - r - 2
   = p2^p - p - 1.
\end{align*}
Remarkably, we stumble upon a familiar pattern again:
\begin{equation}
\Wcomp(2^p) = \Wcomp(2^p-1) + p.\label{eq:Wcomp}
\end{equation}

\medskip

\paragraph{Question 3.}

Let us take \(a=1\), \(b=n\) and \(f(x) = \lg x\). The theorem implies
% \lg (n-1)! = \sum_{k=1}^{n-1}{\lg k} \leqslant
\[
\int_{1}^{n}{\!\!\lg x} \,dx \leqslant \sum_{k=2}^{n}{\lg k} = \lg n!
\leqslant S(n).
\]
Since
\[
\int_{1}^{n}{\!\!\lg x}\,dx = \frac{1}{\ln 2}[x\ln x - x]^{n}_1 = n\lg n -
\frac{n}{\ln 2} + \frac{1}{\ln 2},
\]
we conclude, together with the result of Question~\ref{ex:merge_sort:1}:
\[
n\lg n - \frac{n}{\ln 2} + \frac{1}{\ln 2}
\leqslant S(n) \leqslant n\lg n + n - 1.
\]

\medskip

\paragraph{Question 4.}

We found earlier \(n\floor{\lg n}/2 + 2n - 2^{\floor{\lg n}+1}
\leqslant \Bideal(n)\). We want to find the largest~\(\lambda'\) such
that
\[
\tfrac{1}{2}n\lg n + \lambda' n
\leqslant \tfrac{1}{2}n\floor{\lg n} + 2n - 2^{\floor{\lg n}+1}.
\]
Equivalently, \(n\lg n + 2\lambda' n \leqslant n\floor{\lg n} + 4n -
2^{\floor{\lg n}+2}\), which is an instance of \(n\lg n + \lambda n
\leqslant n\floor{\lg n} + \psi n - \omega 2^{\floor{\lg n}}\), with
\(\lambda = 2\lambda'\), \(\psi = 4\)~and~\(\omega = 4\). Therefore,
let us work out the general case where \(\psi\)~and~\(\omega\) are
positive integers. We set \(n=q2^p\), where \(p \in \mathbb{N}, q \in
\mathbb{R}\) and \(1 \leqslant q < 2\). Thus \(\floor{\lg n} = p\) and
the previous inequality is equivalent to \(q2^p(\lg q + p) + \lambda
q2^p \leqslant pq2^p + \psi q2^p - \omega 2^p\), which, in turn, is
equivalent to \(q\lg q + (\lambda - \psi) q + \omega \leqslant 0\).
Let us define \(f(q) := q\lg q + (\lambda - \psi) q + \omega\). We
have \(f'(q) = \lg q + 1/\ln 2 + \lambda - \psi\) and \(f'(q) = 0
\Leftrightarrow q = 1/e/2^{\lambda - \psi}\). Furthermore,
\(\lim_{q\rightarrow 0^{+}}{f(q)} = \omega^{-}\) and \(\lim_{q
  \rightarrow +\infty}{f(q)} = +\infty\). This means that \(f\)~can be
continuously extended by setting \(f(0) := \lambda - \psi\), that it
decreases to its minimum \(f(1/e/2^{\lambda-\psi})\) and then
increases without a bound. Let us allow \({q=2}\) and consider what
happens at the bounds of the interval \([1,2]\): \(f(1) \leqslant 0
\Leftrightarrow \lambda - \psi \leqslant -\omega\) and \(f(2)
\leqslant 0 \Leftrightarrow \lambda - \psi \leqslant -\omega/2 - 1\).
Either \(-\omega \leqslant -\omega/2 - 1\) or else the converse
holds. In other words, either \(\omega \geqslant 2\) or~\(\omega
\leqslant 2\). Since the bounds we are interested in have~\(\omega
\geqslant 2\), we shall only consider the former case, from which we
deduce that \(f(1) \leqslant 0 \Rightarrow f(2) \leqslant 0\). Let us
then set \(\lambda - \psi := -\omega\) and prove that \(f(q) \leqslant
0\) then holds for \(q \in [1,2]\). We have now \(f(q) = q\lg q -
\omega q + \omega = q\lg q +(1-q)\omega \leqslant 0 \Leftrightarrow
q\lg q \leqslant (q-1)\omega\). Since \(1 \leqslant q\), this is
equivalent to \(q\lg q/(q-1) \leqslant \omega\). Let us define \(g(q)
:= q\lg q/(q-1)\). Then \(g'(q) = (-(\ln 2)\lg q + q - 1)/(q-1)^2/\ln
2\). The denominator is positive, so we turn our attention to the
numerator by setting \(h(q) := -(\ln 2)\lg q + q - 1\), yielding the
derivative \(h'(q) = 1 - 1/q\), which is positive because \({q
  \geqslant 1}\). So \(h\)~increases and, since \({h(1) = 0}\), we
have \(g'(q) \geqslant 0\) for \({q \geqslant 1}\). Thus
\(g\)~increases and reaches its maximum in \([1,2]\) at the upper
bound: \(g(2) = 2\). Since we set \(\omega \geqslant 2\), we proved
that \(f(q) \leqslant 0\), for all~\({q \in [1,2]}\). Because we
defined \(\lambda - \psi := -\omega\) and \(\lambda = 2\lambda'\), we
have the answer \(\lambda' = (\psi - \omega)/2\). In the example
above, \(\psi = 4\) and \(\omega = 4\), we can check that \(\omega
\geqslant 2\) and thus \(\lambda'= 0\). In other words:
\[
\tfrac{1}{2}n\lg n
\leqslant \tfrac{1}{2}n\floor{\lg n} + 2n - 2^{\floor{\lg n}+1}
\leqslant \Bideal(n).
\]
Let us apply the same theorem to the following remaining lower
bounds. First, \(n\floor{\lg n} + 2n - 2^{\floor{\lg n}+2} + 2
\leqslant \Wideal(n)\). Inequation \(n\lg n + \lambda n + 2 \leqslant
n\floor{\lg n} + 2n - 2^{\floor{\lg n}+2} + 2\) is equivalent to
\(n\lg n + \lambda n \leqslant n\floor{\lg n} + 2n - 2^{\floor{\lg
    n}+2}\), for which we want to maximise~\(\lambda\). We have \(\psi
= 2\) and \(\omega = 4 \geqslant 2\). Thus \(\lambda = \psi - \omega =
-2\) and
\[
n\lg n - 2n + 2 \leqslant n\floor{\lg n} + 2n - 2^{\floor{\lg n}+2} + 2
\leqslant \Wideal(n).
\]
Finally, \(n\floor{\lg n} + (3 - \alpha)n - 2^{\floor{\lg n}+2} + 1/2
\leqslant \Aideal(n)\). Inequation \(n\lg n + \lambda n + 1/2
\leqslant n\floor{\lg n} + (3 - \alpha)n - 2^{\floor{\lg n}+2} + 1/2\)
is equivalent to \(n\lg n + \lambda n \leqslant n\floor{\lg n} + (3 -
\alpha)n - 2^{\floor{\lg n}+2}\), for which we want to maximise
\(\lambda\). We have \(\psi = 3 - \alpha\) and \(\omega = 4 \geqslant
2\). Thus \(\lambda = \psi - \omega = - 1 - \alpha\) and
\[
n\lg n - (1+\alpha)n + 1/2
\leqslant n\floor{\lg n} + (3 - \alpha)n - 2^{\floor{\lg n}+2} + 1/2
\leqslant \Aideal(n).
\]
Let us now modify our previous line of argumentation to tackle upper
bounds. In other words, given \(\psi\)~and~\(\omega\), we want to find
the \emph{minimum}~\(\lambda\) such that \(n\floor{\lg n} + \psi n -
\omega 2^{\floor{\lg n}} \leqslant n\lg n + \lambda n\). We derive
\(f(q) \geqslant 0\), with \(f(q) := q\lg q + (\lambda - \psi) q +
\omega\). From~\(\omega \geqslant 2\) we deduce \(f(2) \geqslant 0
\Rightarrow f(1) \geqslant 0\). Let us set \(\lambda - \psi :=
-\omega/2 - 1\) and prove that \(f(q) \geqslant 0\) for \(q \in
[1,2]\). We have now \(f(q) = q\lg q - (\omega/2 + 1)q + \omega
\geqslant 0 \Leftrightarrow (2-q)\omega \geqslant 2q(1-\lg q)\). Since
\(q \leqslant 2\), this is equivalent to \(\omega \geqslant 2q(\lg q -
1)/(q-2)\). Let \(g(q) := q(\lg q - 1)/(q-2)\). Then \(g'(q) = (-(2\ln
2)\lg q + q + 2\ln 2 - 2)/(q-2)^2/\ln 2\). The denominator is
positive, so let us focus on determining the sign of the numerator by
setting \(h(q) := -(2\ln 2)\lg q + q + 2\ln 2 - 2\), yielding the
derivative \(h'(q) = 1 - 2/q\), and \(q \leqslant 2 \Rightarrow h'(q)
\leqslant 0\). Therefore, \(h\)~decreases and, because \(h(1) = 2\ln2
- 1 > 0\) and \(h(2) = 0\), we conclude that \(h(q) \geqslant 0\) and
then \(g'(q) \geqslant 0\), which means that \(g(q)\)~is increasing
and reaches its maximum at the limit \({q=2}\). This limit can be
found by applying \emph{L'H{\^o}pital's Rule}: \(\lim_{q \rightarrow
  2}g(q) = 1/\ln 2\). Therefore, \(2.88 \simeq 2/\ln 2 \geqslant
2q(\lg q - 1)/(q - 2)\), when \(q \in [1,2]\) and the bound is tight
when \(q \rightarrow 2\). If we restrict the range of \(\omega\) so
\(\omega \geqslant 3\) (it is an integer), then we prove \(f(q)
\geqslant 0\), for all \(q \in [1,2]\) and \(\lambda = \psi - \omega/2
- 1\). Consider again \(\Bideal(n) \leqslant \tfrac{1}{2}{n\floor{\lg
    n}} + 3n - 2^{\floor{\lg n}+1} - 1\). This is equivalent to \(2
\cdot \Bideal(n) \leqslant n\floor{\lg n} + 6n - 4 \cdot 2^{\floor{\lg
    n}} - 2\), which fits the conditions of the previous theorem,
where \(\psi = 6\) and \(\omega = 4 \geqslant 3\). Then \(\lambda =
3\) and
\[
\Bideal(n) \leqslant \tfrac{1}{2}{n\lg n} + \tfrac{3}{2}n - 1.
\]
The case~\(\omega = 2\) is problematic and requires a solution because
it occurs in the upper bound of the worst and average delays (it is
the same bound): \(\Wideal(n) \leqslant n\floor{\lg n} + 3n -
2^{\floor{\lg n}+1} - 1\). A simple workaround consists in slightly
weakening the bound so \(\omega \geqslant 3\): \(-2^{\floor{\lg n}+1}
= -3 \cdot 2^{\floor{\lg n}} + 2^{\floor{\lg n}} \leqslant -3 \cdot
2^{\floor{\lg n}} + n\). In other words, whenever~\({\omega = 2}\), we
change~\(\psi\) into~\(\psi+1\) and \(\omega\)~into~\(\omega+1\), then
apply the theorem, so \(\lambda = \psi - 3/2\). Then \(\Wideal(n)
\leqslant n\floor{\lg n} + 4n - 3 \cdot 2^{\floor{\lg n}} - 1\) and
\[
\Wideal(n) \leqslant n\lg n + \tfrac{3}{2}{n} - 1.
\]
From Question~\ref{ex:merge_sort:1}, we know that \(\Wcomp(n) =
\Wideal(n) - n + 1\), therefore \(\Wcomp(n) \leqslant n\lg n +
\tfrac{1}{2}n\). Furthermore, we have \(S(n) \leqslant \Wcomp(n)\),
hence, together with the lower bound from
Question~\ref{ex:merge_sort:3}, we conclude
\[
n\lg n - \frac{n}{\ln 2} + \frac{1}{\ln 2}
\leqslant S(n) \leqslant n\lg n + \frac{1}{2}{n}.
\]

\medskip

\paragraph{Question 5.}

From definition~\eqref{def:nu} \vpageref{def:nu} of the bit sum
\(\nu_{n}\),
\[
\nu_{0} = 0,\quad \nu_{2n} = \nu_{n},\quad \nu_{2n+1} = \nu_{n} + 1,
\]
we derive a straightforward implementation in \Erlang as follows:
\begin{verbatim}
nu(0) -> 0;
nu(N) -> nu(N div 2) + N rem 2.
\end{verbatim}
where \erlcode{N~div~2} realises \(\floor{n/2}\) and \erlcode{N~rem~2}
realises \(n - 2\floor{n/2}\). (By the way, we have \(\comp{nu}{n} =
\comp{exp2}{n}\). See definition \eqref{def:exp2}
\vpageref{def:exp2}.)

\medskip

\paragraph{Question 6.}

First, let us explicitly express the dependence on~\(n\) by defining
\({e_0 := \rho_n}\), where \(\rho_n\)~is \emph{the number of trailing
  zeros} of~\(n\) and \(\rho\)~is called \emph{the ruler
  function}. Next, we define~\(\rho\) recursively as
\begin{align*}
\rho_0 &:= \infty, & \rho_{2n} &:= \rho_{n} + 1,
& \rho_{2n+1} &:= 0.
\intertext{Finally, we recall the recursive definition of~\(\nu\)
  \eqref{def:nu} \vpageref{def:nu}:}
\nu_{0} &:= 0, & \nu_{2n} &:= \nu_{n}, &
\nu_{2n+1} &:= \nu_{n} + 1.
\end{align*}
At first sight, a simple relationship between these two functions is
not obvious. Without loss in generality, we can assume that \(2n+1 :=
(\alpha 01^a)_2\), where \(\alpha\)~is an arbitrary bit string and
\(1^a\)~is a 1-bit string of length~\(a\). Then \(2n+2 =
(\alpha10^a)_2\) and
\[
\nu_{2n+1} = \nu_{\alpha} + a,\quad \rho_{2n+1} = 0,\quad
\nu_{2n+2} = \nu_{\alpha} + 1,\quad \rho_{2n+2} = a.
\]
Now, we can relate \(\rho\)~and~\(\nu\) by means of~\(a\):
\[
\rho_{2n+2} = \nu_{2n+1} - \nu_{\alpha} = \nu_{2n+1} - (\nu_{2n+2} -
1) = 1 + \nu_{2n+1} - \nu_{2n+2}.
\]
We can check now that the same pattern also works for~\(\rho_{2n+1}\)
by simply using the definitions of \(\rho\)~and~\(\nu\): \(\rho_{2n+1}
= 1 + \nu_{2n} - \nu_{2n+1}\). This achieves to establish, for any
integer \({n>0}\), that \(e_0 := \rho_n = 1 + \nu_{n-1} -
\nu_{n}\). Summing on both sides leads to \(\sum_{k=1}^{n}{\rho_k} = n
- \nu_n\).

\medskip

\paragraph{Question 7.}

There are two cases: either \(n+1\) is odd or it is even. If the
former, we need to draw the trees corresponding to those in
\fig~\vref{fig:Bcomp_2p}, to be found in
\fig~\vref{fig:Bcomp_2p_succ}; if the latter, we draw trees from those
in \fig~\vref{fig:Bcomp_2p_1}, as seen in
\fig~\vref{fig:Bcomp_2p_1_succ}.
\begin{figure}[t]
\centering
\subfloat[Nodes with list lengths\label{fig:bcomp_2p_merge_succ}]{%
  \includegraphics[bb=71 611 196 721]{bcomp_2p_merge_succ}
}
\quad
\subfloat[Nodes with best delays\label{fig:bcomp_2p_succ}]{%
  \includegraphics[bb=71 607 208 721]{bcomp_2p_succ}
}
\caption{Best case for \(n+1\) odd (see \fig~\vref{fig:Bcomp_2p})
\label{fig:Bcomp_2p_succ}}
\end{figure}
\begin{figure}[!b]
\centering
\subfloat[Nodes with list lengths\label{fig:bcomp_2p_1_merge_succ}]{%
\includegraphics[bb=71 661 142 721]{bcomp_2p_1_merge_succ}
}
\qquad
\subfloat[Nodes with best delays\label{fig:bcomp_2p_1_succ}]{%
\includegraphics[bb=71 657 146 721]{bcomp_2p_1_succ}
}
\caption{Best case for \(n+1\) even (see \fig~\ref{fig:Bcomp_2p_1} on
  page~\pageref{fig:Bcomp_2p_1})
\label{fig:Bcomp_2p_1_succ}}
\end{figure}
Next, we compare all the best delays involved in the trees in
\fig~\vref{fig:bcomp_2p}, whose sum of the nodes is~\(\Bcomp(n)\), and
\fig~\vref{fig:bcomp_2p_succ}, whose sum of the nodes is
\(\Bcomp(n+1)\), when \(n\) is even. We can clearly see that the
latter tree contains an excess of \(r + 1 + \Bcomp(2^0) = r+1 :=
\nu_n\). This means that we have \(\Bcomp(n+1) = \Bcomp(n) + \nu_n\),
when \(n\)~is even. We need to apply the same scrutiny to the
complementary case, \(n\)~odd, which corresponds to the trees in
\fig~\vref{fig:bcomp_2p_1} and \fig~\vref{fig:bcomp_2p_1_succ}. The
latter contains an excess of~\(\Bcomp(2^q)\) (we do not find this
node\hyp{}delay in the other tree) and a default of
\(\sum_{i=0}^{q-1}{\Bcomp(2^i)} - r + \sum_{j=1}^{q-1}{2^j}\). That
is, when \(n\)~is odd:
\begin{equation*}
\Bcomp(n+1) - \Bcomp(2^q) + \sum_{i=0}^{q-1}{\Bcomp(2^i)}
- r + \sum_{j=1}^{q-1}{2^j} = \Bcomp(n).
\end{equation*}
Because we already found that \(\Bcomp(2^p) = p 2^{p-1}\), this is
equivalent to
\begin{equation*}
\Bcomp(n+1) = \Bcomp(n) + q2^{q-1} - \sum_{i=0}^{q-1}{i2^{i-1}} + r - (2^q - 2).
\end{equation*}
From Question~\ref{ex:merge_sort:2}, we know that
\(\sum_{i=0}^{r}{i2^{i-1}} = (r-1)2^{r} + 1\), so
\begin{align*}
\Bcomp(n+1) &= \Bcomp(n) + q2^{q-1} - (q-2)2^{q-1} - 1 + r - 2^q + 2\\
            &= \Bcomp(n) + r + 1,
\end{align*}
that is, \(\Bcomp(n+1) = \Bcomp(n) + \nu_n\), just as in the case
\(n\)~even. Therefore, the equality is independent of the parity
of~\(n\). By summing both sides from \(0\) to \(n-1\), we obtain the
pretty \emph{exact} formula
\[
\Bcomp(n) = \Bcomp(0) + \sum_{i=0}^{n-1}{\nu_i} = \sum_{i=0}^{n-1}{\nu_i}.
\]
Let us look at \fig~\vref{fig:bit_table}. Note that \(n=0\) is missing
in the table, but it is important to consider it here and also to fill
the blanks in the table with \(0\)s. We can see bit strings whose
pattern is \(0^{2^{i}}1^{2^{i}}\) repeating themselves downward, where
\(i\)~is the exponent of~\(2\) of the binary notation in the rows. Let
us call them \emph{\(01\)\hyp{}strings}. For example, in the third
column, that is, \({i=2}\), we see \(\underline{00001111}00001\dots\)
The bit string from~\(0\) to~\(n-1\) has length~\(n\) and the length
of the \(01\)\hyp{}strings is \(2^{i+1}\), hence the number of
\(01\)\hyp{}strings it contains is \(\floor{n/2^{i+1}}\). For example,
if~\(n=13\), that is, \(n=(1101)_2\), and \({i=1}\), the number of
\(01\)\hyp{}strings is~\(\floor{13/2^2} = 3\). Since half the bits in
a \(01\)\hyp{}string are \(1\)\hyp{}bits, each of these strings
contains \(2^i\)~\(1\)\hyp{}bits, for instance, there are at least \(3
\cdot 2^1 = 6\) such bits between~\(0\) and~\(12\), on the second
column from the right. We wrote ``at least'' because there may be
other \(1\)\hyp{}bits which could not make up half a complete
\(01\)\hyp{}string, as, for instance, in the bit string from~\(0\)
to~\(10\) included, when \(i=1\). This happens when \(n\)~is not a
power of~\(2\). These extra bits can at most make up a
\(1\)\hyp{}string of length~\(2^{i}-1\), which is the greatest
remainder of the Euclidean division by~\(2^{i}\). Let us
define~\(\sigma_{n-1,i}\) as the sum of the bits of the \(i\)th
column, where the column~\(0\) is the rightmost, from the row~\(0\)
to~\(n-1\). Then
\begin{equation*}
\left\lfloor\frac{n}{2^{i+1}}\right\rfloor 2^i 
\leqslant \sigma_{n-1,i} \leqslant
\left\lfloor\frac{n}{2^{i+1}}\right\rfloor 2^i + (2^i - 1).
\end{equation*}
By means of \(x - 1 < \floor{x} \leqslant x\), we deduce the weaker,
but simpler
\begin{equation}
n/2 - 2^{i} < \sigma_{n-1,i} \leqslant n/2 + 2^{i} - 1.
\label{ineq:sigma}
\end{equation}
Remains to sum on all the columns so the leftmost bit of \(n-1\) is
accounted for. Equation~\eqref{eq:e_r} \vpageref{eq:e_r} tells us that
\(n\)~is made of \(\floor{\lg n}+1\) bits, so its leftmost bit is at
position \(\floor{\lg n}\). Accordingly, the leftmost bit of~\(n-1\)
is at position \(\floor{\lg(n-1)}\). This expression can be simplified
by a simple variation of the argument that led to
equation~\eqref{eq:e_r}. Let us assume that \(n\)~is made of
\(m\)~bits, that is, \(n := (b_{m-1}\dots b_{0})_2\).
\begin{gather*}
2^{m-1} \leqslant n \leqslant 2^{m} - 1
\Rightarrow
2^{m-1} < 2^{m-1} + 1 \leqslant n + 1 \leqslant 2^{m}\\
\Rightarrow
m - 1 < \lg(n+1) \leqslant m
\Rightarrow
\ceiling{\lg(n+1)} = m = \floor{\lg n} + 1.
\end{gather*}
Substituting~\(n-1\) for~\(n\) leads to \(\floor{\lg(n-1)} =
\ceiling{\lg n} - 1\). By summing all the sides of inequations
\eqref{ineq:sigma} from~\({i=0}\) to~\(\ceiling{\lg n}-1\), we cover
all the \(1\)\hyp{}bits of all the binary expansions of the integers
in~\([0..n-1]\):
\begin{equation*}
\sum_{i=0}^{\ceiling{\lg n}-1}{\left(\frac{n}{2} - 2^{i}\right)}
< \sum_{i=0}^{\ceiling{\lg n}-1}{\sigma_{n-1,i}} \leqslant
\sum_{i=0}^{\ceiling{\lg n}-1}{\left(\frac{n}{2} + 2^{i} - 1\right)}.
\end{equation*}
The sum in the middle is none other than \(\Bcomp(n) =
\sum_{i=0}^{n-1}{\nu_{i}}\), therefore 
\begin{equation*}
\tfrac{1}{2}{n\ceiling{\lg n}} - 2^{\ceiling{\lg n}} + 1 
< \Bcomp(n) \leqslant
\tfrac{1}{2}{n\ceiling{\lg n}} + 2^{\ceiling{\lg n}} - \ceiling{\lg n}
- 1.
\end{equation*}
Using the inequalities \(x \leqslant \ceiling{x} < x + 1\) leads to
\begin{equation*}
\tfrac{1}{2}{n\lg n} - 2n + 1 
< \Bcomp(n) \leqslant
\tfrac{1}{2}{n\lg n} + \tfrac{5}{2}{n} - \lg n - 1.
\end{equation*}
The same approach, based upon delay trees, leads to
\[
\Wcomp(2p) = \Wcomp(2p-1) + \nu_{2p-1};\quad 
\Wcomp(2p+1) = \Wcomp(2p) + 2^{\rho_{2p}} + \nu_{2p} - 1.
\]
In Question~\ref{ex:merge_sort:2}, we derived equation
\eqref{eq:Wcomp}: \(\Wcomp(2^p) = \Wcomp(2^p-1) + p\), which is
consistent with the recurrent equation for \(\Wcomp(2p)\), but
\(\Wcomp(n+1) = \Wcomp(n) + \nu_n\) does \emph{not} hold for even
values of~\(n\).

\medskip

\paragraph{Question 8.} Skipped.

\medskip

\paragraph{Question 9.}

If~\(n=2^p\), then \(\floor{\lg n} = \ceiling{\lg n} = \lg n\), thus
\(n\floor{\lg n} + 2n - 2^{\floor{\lg n}+1} = n\lg n = n\ceiling{\lg
  n} + n - 2^{\ceiling{\lg n}};\) otherwise, \(\floor{\lg n} + 1 =
\ceiling{\lg n}\), thus \(n\floor{\lg n} + 2n - 2^{\floor{\lg n}+1} =
n\ceiling{\lg n} + n - 2^{\ceiling{\lg n}}\).~\textsc{qed.} This
identity can be used to derive a legible upper bound for
\(\Wideal(n)\)~and~\(\Aideal(n)\):
\begin{align*}
\Wideal(n) &\leqslant n\floor{\lg n} + 3n - 2^{\floor{\lg n}+1} - 1 
= n\ceiling{\lg n} + 2n - 2^{\ceiling{\lg n}} - 1\\
 &\leqslant n\ceiling{\lg n} + n - 1.
\end{align*}

\medskip

\paragraph{Question 10.}

We have \(\ave{sort}{n} \geqslant n\lg n - (\alpha - 1) n + 2\lg n +
9/2\). A computer algebra system leads to \(1 \leqslant n \leqslant 26
\Rightarrow \ave{i2wb}{n} < \ave{sort}{n}\).
