%%-*-latex-*-

\chapter{Delay}

When a function call is computed, several outcomes are possible: the
rewrites may
\begin{itemize}

  \item never end (\emph{loop}),

  \item end with a memory (call stack or heap) overflow,

  \item end with a type error (as \erlcode{1 + []}),

  \item end with a match failure (a body contains a call which is
  matched by no head),

  \item end with a \emph{value} (that is, an expression which is a
    constant, for example, a list or an integer, and therefore cannot
    be further rewritten).

\end{itemize}
In the latter case, it is sometimes feasible to express the number of
rewrites needed to reach the result in terms of the input
size. Accordingly, a loop requires an infinite number of steps and a
value none. Also, a measure of the input must be decided upon in order
to be able to apprehend the number of rewrite steps, which, in turn,
can be interpreted temporally as a \emph{delay} (between the moment
when the computation starts and when it ends successfully with a
value). The choice of a measure of the data depends entirely on the
programmer, but it is generally considered meaningful, for instance,
to take the number of items in a list as its size, and an \Erlang
integer \erlcode{N} is measured by its mathematical correspondent \(n
\in \mathbb{N}\). The length of a list, that is, the number of items
it contains, can be defined in several ways in \Erlang, one being
\begin{verbatim}
len(   []) -> 0;
len([_|L]) -> 1 + len(L).
\end{verbatim}
It does not matter here that this definition is not in tail form
because its purpose is to prove how a peculiar measure on lists can
easily be implemented.

Given an expression \(e\), we denote the number of rewrites to reach
its value, assuming it exists, by \(\delay{\(e\)}\) and the value
itself by \(\val{e}\). The strategy of \Erlang, which consists in
computing the arguments of a call before the call itself and then
selecting clauses for matching according to the order of definition,
implies that if a value exists, then it is unique. Hence, computing a
value~\(\val{v}\) incurs a null delay: \(\delay{\(\val{v}\)} = 0\).
Moreover, the delay of a function call is the sum of the delays of its
arguments plus the delay of rewriting the call with the resulting
values. For example, let \(P\) and \(Q\) be two expressions whose
values are lists. Then the delay of \erlcode{join(\(P\),\(Q\))} is
denoted by \(\delay{join(\(P\),\(Q\))}\) and equals the sum of the
delays of \(P\)~and~\(Q\), plus the delay of
\erlcode{join(\(\val{P}\),\(\val{Q}\))}:
\[
\delay{join(\(P\),\(Q\))} := \delay{\(P\)} +
\delay{\(Q\)} + \delay{join(\(\val{P}\),\(\val{Q}\))}.
\]
We need now to connect these notions to function definitions: the
delay of a call matching a head is~\(1\) plus the delay of the
corresponding body in the clause.

\medskip

\paragraph{Joining two Lists.} 

Let us consider the definition of \erlcode{join/2} which is not in
tail form:

\verbatiminput{join_2.def}

\noindent Applying the previous definition of the delay to each clause
leads straightforwardly to the following \emph{recurrence
  equations}. Roughly speaking, ``recurrence'' is the mathematical
alter ego of ``recursive'' in computer science.
\begin{align*}
\delay{join([],Q)}    &:= 1 + \delay{Q},\\
\delay{join([I|P],Q)} &:= 1 + \delay{[I|join(P,Q)]}.
\end{align*}
The relation \(a := b\) means ``\(a\) equals \(b\) by
definition;'' \(a = b\) means ``\(a\) equals \(b\) by means of some
mathematical deduction'' and \(a \mathrel{\equiv} b\) means ``\Erlang
expressions \(a\)~and~\(b\) are rewritten into the same value,'' more
precisely, their abstract syntax trees are identical. Here, all
\Erlang variables, \erlcode{I}, \erlcode{P}~and~\erlcode{Q}, denote
values, since they are bound to the arguments during the matching of
the call. But what about \(\delay{[I|join(P,Q)]}\)? The \Erlang
operator~(\erlcode{|}) can be considered as a function which is not
defined by any clause, therefore it incurs no delay and only the
delays of its two arguments matter. A more detailed analysis would
take into account the time needed to push an item on top of a list,
but we assume here that
\[
\delay{[I|L]} = \delay{I} + \delay{L}.
\]
Therefore
\[
\delay{[I|join(P,Q)]} = \delay{I} + \delay{join(P,Q)}.
\]
We have \(\delay{I} = 0\) and the equations are equivalent to
\[
\delay{join([],Q)}    = 1,\quad
\delay{join([I|P],Q)} = 1 + \delay{join(P,Q)}.
\]
Note that we must use here~\((=)\) instead of~\((:=)\). We
need now to define the delay in terms of the size of the input, that
is, the arguments. The function call \erlcode{join(\(P\),\(Q\))} has
two lists as arguments. The usual way to measure the size of a list is
to consider the number of elements it contains, that is, its
length. Here, the size of the input is characterised by a
pair~\((n,m)\), where \(n, m \in \mathbb{N}\) are the mathematical
abstractions of the values of \erlcode{len(\(P\))} and
\erlcode{len(\(Q\))}. For practical purposes, we mean to identify
\Erlang integers and mathematical integers. Let us define the delay of
\erlcode{join/2} in terms of the input size as
\[
\comp{join}{n,m} := \delay{join(\(P\),\(Q\))},
\]
where \(n\)~is the length of~\(P\) and \(m\)~is the length of~\(Q\).
Using this definition, the previous equations are equivalent to
\[
\comp{join}{0,q}   = 1,\quad
\comp{join}{p+1,q} = 1 + \comp{join}{p,q}.
\]
where \erlcode{P}~and~\erlcode{Q} are supposed to contain respectively
\(p\)~and~\(q\) items. It is not too late to observe or recall that
\erlcode{Q}~is constant all along the rewrites, which can be seen here
too because all the occurrences of~\(q\) are \(q\)~itself. In other
words: \(q=m\). Thus, for the sake of clarity, let us define
\(\comp{join}{p} := \comp{join}{p,m}\); then we simply have
\[
\comp{join}{0}   = 1,\quad
\comp{join}{p+1} = 1 + \comp{join}{p}.
\]
As we wish to express \(\comp{join}{n}\) in terms of~\(n\) alone, that
is, we want a \emph{closed form}, let us write the first terms and
guess the general form when~\(p+1=n\) in the second equation:
\begin{align*}
\comp{join}{0} &= 1,\\
\comp{join}{1} &= 1 + \comp{join}{0} = 1 + 1 = 2,\\
\comp{join}{2} &= 1 + \comp{join}{1} = 1 + 2 = 3,\\
&\;\;\vdots\\
\comp{join}{n} &= 1 + \comp{join}{n-1} = 1 + n.
\end{align*}
But we must check if our guess is correct by replacing
\(\comp{join}{p} = 1 + p\) back into the original equations, which
must be satisfied:
\[
\comp{join}{0}   = 1 + 0 = 1,\quad
\comp{join}{p+1} = 1 + (p+1) = 1 + \comp{join}{p}.
\]
And they are. So we can trust that \(\comp{join}{n} = n + 1\).

We would remiss if we do not mention that, whilst the previous
reckoning is perfectly valid, there exists a direct approach that does
not suppose to guess the result and later check it. The technique
needed here, called \emph{telescoping}, consists in forming the
differences between two successive terms in the series
\((\comp{join}{p})_{p \in \mathbb{N}}\) and then summing these
differences. The second equation implies the following ones:
\begin{align}
\comp{join}{p+1} - \comp{join}{p} &= 1, & \text{where} \;\, p
\geqslant 0,\notag\\
\sum_{p=0}^{n-1}{\left(\comp{join}{p+1} - \comp{join}{p}\right)}
  &= \sum_{p=0}^{n-1}{1}, & \text{where} \;\, n > 0,\notag\\
\comp{join}{n} - \comp{join}{0} &= n,\notag\\
\comp{join}{n} &= n + 1.\label{delay:join}
\end{align}
The clue that the reasoning has been tidied up is that it contains no
more ellipses \((\ldots)\), English or Latin locutions like ``and so
on'' and ``etc.'' Furthermore, it is important to have a clear
understanding of the behaviour of the delay when the input becomes
significantly large with res<pect to the chosen measure. This insight
is provided by considering an \emph{equivalent function} when the
variable takes arbitrarily large values:
\[
\comp{join}{n} \mathrel{\sim} n, \,\; \text{as} \;\, n
\rightarrow \infty.
\]
By definition, two functions \(f(n)\)~and~\(g(n)\) are
\emph{asymptotically equivalent} if \(\lim_{n \rightarrow
  \infty}{f(n)/g(n)} = 1\). Our assumption all along has been that the
number of steps necessary to fully rewrite a function call, which we
call delay and other authors often call \emph{time complexity} or
\emph{cost}, is proportional to the real, wall\hyp{}clock, delay. This
means, for example, that the time required to join two lists is almost
proportional to the length of the first list and the more the list
grows, the more accurate the proportionality becomes. In particular,
doubling the size of the first list will approximately double the time
to get the result, whereas doubling the length of the second list will
have no impact at all. Despite being quite reasonable, this
proportionality assumption is slightly weak because many untold
details have been abstracted away so as to get a simple introduction
to functional programming. A much stronger case is found in comparing
two functions, especially if they compute the same results on the same
inputs (this is another sort of equivalence). Indeed, both functions
then assume the same abstract execution model (think of the
implementation of some virtual machine, for example) and no direct
reference to the physical world is needed for validating the
comparison: knowing which function is faster \emph{in the same
  environment} implies that this function is also faster with respect
to the wall\hyp{}clock. This kind of question can sometimes be
answered very reliably and the asymptotic expressions, that is,
involving the equivalence sign~\((\sim)\), are especially useful
because real life data is usually big, assuming that the measure
chosen is pertinent. If small inputs are likely, it can be wise to
have a version of the function dedicated to be efficient on such data
and another version specialised for large cases.

Is it mandatory to go through all these mathematics to reach the
result? Most of the time, such thorough investigation is not necessary
and sometimes not even tractable. We could have think as follows. Let

\verbatiminput{join_2.def}

\noindent In the calls \erlcode{join(\(P\),\(Q\))}, we know that the
size of~\(Q\) does not matter. By focusing on the recursive call in
the definition, we see that the first parameter changes from
\erlcode{[I|P]} (in the head) to~\erlcode{P}. Thus, the sensible
measure on lists is the length because it decreases by one at each
recursive call. Let us suppose that \erlcode{P}~is bound to an
argument which contains \(p\)~items. Let \(\comp{join}{p}\) be the
number of steps to rewrite \erlcode{join(\(P\),\(Q\))} to a
value. Then the second clause implies that \(\comp{join}{p+1} = 1 +
\comp{join}{p}\) and the first one \(\comp{join}{0} = 1\). The
number~\(1\) in each case comes from the fact that one clause is used
(an arrow ``\erlcode{->}''). Then the recurrence relations are solved
and the result is \(\comp{join}{n} = n + 1\).

Let us consider again the definition of \erlcode{join\_tf/2}
\vpageref{code:join_tf}: \begin{alltt}
join_tf(P,Q)        \(\smashedrightarrow{\alpha}\) join(P,Q,[]).
join(   [],Q,   []) \(\smashedrightarrow{\beta}\) Q;
join(   [],Q,[I|A]) \(\smashedrightarrow{\gamma}\) join([],[I|Q],A);
join([I|P],Q,    A) \(\smashedrightarrow{\delta}\) join(P,Q,[I|A]).
\end{alltt}
 It was named
\erlcode{join\_tf} because it is a version of \erlcode{join} which is
in \textbf{t}ail \textbf{f}orm. Clause \clause{\gamma} pushes
item~\erlcode{I} on top of~\erlcode{Q} but the actual contents
of~\erlcode{Q} is never used. So let us find \(\comp{join\_tf}{n}\),
that is, the delay of \erlcode{join\_tf(\(P\),\(Q\))}, where \(n\) is
the length of list~\(P\). Instead of deducing a set of recurrence
equations and solve them, let us endeavour with a direct approach
consisting in counting, for each clause, how many times it is used:
\begin{itemize}

  \item clause~\clause{\alpha} is used once;

  \item clause~\clause{\delta} is used \(n\)~times (the first argument
    is reversed on the accumulator~\erlcode{A}),

  \item clause~\clause{\gamma} is used \(n\)~times (the accumulator is
    empty at the beginning and the first list was reversed on it, so
    it contains \(n\)~items when clause~\clause{\eta} is used),

  \item clause~\clause{\beta} is used once.

\end{itemize}
Note that the order of the clauses in the previous list follows the
order of computation, for example, clause~\clause{\beta} is always
used last. This allows us to express compactly this execution trace as
the composition \(\alpha\delta^n\gamma^n\beta\), where order
matters. The total number of steps is
\(\abs{\alpha\delta^n\gamma^n\beta} = \abs{\alpha} + n \cdot
\abs{\delta} + n \cdot \abs{\gamma} + \abs{\beta} = 2n + 2 =
\comp{join\_tf}{n}\). We can now compare \(\comp{join}{n}\) and
\(\comp{join\_tf}{n}\):
\[
\comp{join\_tf}{n} = 2 \cdot \comp{join}{n}.
\]
In other words, the version in tail form is twice as slow as the
other, \emph{even if we do not know the wall\hyp{}clock delay of
  none.}

\medskip

\paragraph{List Reversal.}

Let us reconsider now the following definition of \erlcode{srev/1},
and investigate its delay:

\verbatiminput{srev.def}

\noindent First, let us go for a mathematical approach. For any
expression~\(L\) whose value is a list, the delay
\(\delay{srev(\(L\))}\) is, by definition, the sum of the costs of
rewriting~\(L\) and the cost of rewriting the call with the
corresponding value~\(\val{L}\) as an argument:
\[
\delay{srev(\(L\))} := \delay{\(L\)} +
\delay{srev(\(\val{L}\))}.
\]
Considering each clause separately, we can therefore write the
equations defining the delays:
\begin{align*}
\delay{srev([])} &:= 1,\\
\delay{srev([H|T])} 
  &:= 1 + \delay{join(srev(T),[H])}\\
  &= 1 + (\delay{srev(T)} +
\delay{join(\(\val{\erlcode{srev(T)}}\),[H])}).
\end{align*}
Let us define the delay of \erlcode{srev/1} in terms of the input size
as
\[
\comp{srev}{n} := \delay{srev(\(L\))},
\]
where \(n\)~is the length of \(L\). Using this definition and assuming
that \erlcode{T}~is bound to a list containing \(p\)~items, then the
length of \erlcode{srev(T)} is also~\(p\) and the previous equations,
for~\(p~\geqslant~0\), are equivalent to
\begin{align*}
\comp{srev}{0}   &= 1,\\
\comp{srev}{p+1}
  &= 1 + \comp{srev}{p} + \comp{join}{p}
   = 1 + \comp{srev}{p} + (p+1)
   = 2 + p + \comp{srev}{p}.
\end{align*}
We would rather like to express \(\comp{srev}{p}\) in terms of~\(p\)
only---in other words, we want to solve the equations. The technique
consists in making up all the differences of two successive terms in
the series and then summing up these differences, so that terms cancel
pairwise:
\begin{align*}
\comp{srev}{p+1} - \comp{srev}{p} &= 2 + p, & \text{where}\;\, p > 0,\\
   \sum_{p=0}^{n-1}{\left(\comp{srev}{p+1} - \comp{srev}{p}\right)}
&= \sum_{p=0}^{n-1}{(2 + p)}, & \text{where} \;\, n > 0.
\end{align*}
Let us change variable \(p\) into \(p-2\) in the
  right\hyp{}hand side:
\[
   \comp{srev}{n} - \comp{srev}{0} = \sum_{p=2}^{n+1}{p}
\;\Leftrightarrow\;
   \comp{srev}{n} - 1 = \sum_{p=2}^{n+1}{p}
\;\Leftrightarrow\;
   \comp{srev}{n} = \sum_{p=1}^{n+1}{p}.
\]
Let us double each side of the equality:
\[
  2 \cdot \comp{srev}{n}
= 2 \cdot \sum_{p=1}^{n+1}{p} = \sum_{p=1}^{n+1}{p} + \sum_{p=1}^{n+1}{p}.
\]
Let us change the variable~\(p\) into \(n-p+2\) in the second sum:
\begin{align}
   2 \cdot \comp{srev}{n}
&= \sum_{p=1}^{n+1}{p} + \sum_{p=1}^{n+1}{(n-p+2)}
 = \sum_{p=1}^{n+1}{(p+(n-p+2))}\notag\\
&= \sum_{p=1}^{n+1}{(n+2)} = (n+1)(n+2).\notag\\
   \comp{srev}{n}
&= (n+1)(n+2)/2,\,\; \text{where} \;\, n > 0.\notag
\intertext{Since that replacing \(n\) by \(0\) in the latter formula
  gives}
\comp{srev}{0} &= (0+1)(0+2)/2 = 1,\notag
\intertext{we can therefore gather the two cases into one as}
\comp{srev}{n} &= (n+1)(n+2)/2,\,\; \text{where} \;\, n \geqslant 0.
\label{delay:srev}
\end{align}
As an example, the number of rewrites for the call
\erlcode{srev([3,2,1])} \vpageref{rev_321bis} was found to be~\(10\)
and our formula indeed reckons this number: since
\erlcode{len([3,2,1])}~\(\mathrel{\equiv}\)~\erlcode{3}, we need
\(\comp{srev}{3} = (3+1)(3+2)/2 = 4 \cdot 5/2 = 10\).

Would a lightweight approach have been possible? Yes. What we need to
do is to use ellipses when we do not want to detail too much, but we
must make sure that we do not get sloppy either. Let us resume from
the recurrence equations and change \(p\)~into~\(p-1\):
\begin{equation}
\comp{srev}{0} = 1;\quad
\comp{srev}{p} = 1 + p + \comp{srev}{p-1},\,\; \text{where}\;\, p >
0,\label{eq:srev_p}
\end{equation}
and simply write a few terms of the series until \(p=n\):
\begin{align*}
\comp{srev}{0} &= 1,\\
\comp{srev}{1} &= 1 + 1 + \comp{srev}{0},\\
\comp{srev}{2} &= 1 + 2 + \comp{srev}{1},\\
&\;\;\vdots\\
\comp{srev}{n} &= 1 + n + \comp{srev}{n-1}, & \text{where}\;\, n >
0.
\end{align*}
By looking at the left\hyp{}hand sides we find the terms
\(\comp{srev}{0}\), \(\comp{srev}{1}\), \dots, \(\comp{srev}{n}\),
whereas, in the right\hyp{}hand sides, we can see \(\comp{srev}{0}\),
\(\comp{srev}{1}\), \dots, \(\comp{srev}{n-1}\), \emph{which are all
  present in the left\hyp{}hand side.} This gives us the idea to sum
all the equations into one whose left\hyp{}hand side will be
\(\comp{srev}{0} + \comp{srev}{1} + \dots + \comp{srev}{n}\) and the
right\hyp{}hand side \(\comp{srev}{0} + \comp{srev}{1} + \dots +
\comp{srev}{n-1}\) can be subtracted on both sides. That is, the
technique is based on
\[
\comp{srev}{0} + \comp{srev}{1} + \dots + \comp{srev}{n} = x +
\comp{srev}{0} + \comp{srev}{1} + \dots + \comp{srev}{n-1}
\]
being simply equivalent to \(\comp{srev}{n} = x\). We need to
explicit~\(x\) now. It is made of two parts, which can be seen as two
columns in the list of equations: the first is \(1 + 1 + \dots + 1\)
(\(n+1\)~times, since it starts at \(\comp{srev}{0}\) and ends at
\(\comp{srev}{n}\)) and the second is~\(1 + 2 + \dots + n\). The
former is simply the value~\(n+1\), while the latter should be
familiar by now. If not, here is another way to calculate it. Let
\(S_n = 1 + 2 + \dots + n\). By writing the sum from right to left,
this is still \(S_n = n + (n-1) + \dots + 1\). By adding side by side
these two equalities yield \(S_n + S_n = (1 + n) + (2 + (n-1)) + \dots
+ (n + 1) = (1 + n) \cdot n\), so \(S_n = n(n+1)/2\). Finally, we can
gather all these findings into the expected result
\[
\comp{srev}{n} = (n+1) + n(n+1)/2 = (n+1)(n+2)/2, \,\;
\text{where}\;\, n > 0.
\]
This formula can be extended to~\(n=0\) since it implies
\(\comp{srev}{0} = (0+1)(0+2)/2 = 1\), as expected. We further have
\[
\comp{srev}{n} = \frac{1}{2}{n^2} + \frac{3}{2}{n} + 1
\;\Rightarrow\;
\frac{2}{n^2}{\comp{srev}{n}} = 1 + \frac{3}{n} + \frac{2}{n^2}
\rightarrow 1,\,\; \text{as} \,\; n \rightarrow \infty.
\]
By definition of~\((\sim)\), this implies
\[
\comp{srev}{n} \mathrel{\sim} \frac{1}{2}{n^2}, \,\; \text{as} \;\,
n \rightarrow \infty.
\]
When a function~\(f\) satisfies \(f(n) \mathrel{\sim} a n^2\) for some
constant~\(a\) and with~\(n \rightarrow \infty\), it is said to be
\emph{asymptotically quadratic}. In our example at hand, multiplying
by~\(10\) the size of a large input list will multiply at least
by~\(50\) the computation time.

Let us reconsider now the definition of \erlcode{rev/1}
\vpageref{code:rev_join}: \begin{alltt}
rev(L)            \(\smashedrightarrow{\alpha}\) rev_join(L,[]).
rev_join(   [],Q) \(\smashedrightarrow{\beta}\) Q;
rev_join([I|P],Q) \(\smashedrightarrow{\gamma}\) rev_join(P,[I|Q]).
\end{alltt}
 and assess the number
of steps to compute \erlcode{rev(\(L\))}, with \(L\)~being an
expression whose value is a list containing \(n\) items. Considering
each clause separately, we can then write the equations defining the
delays
\begin{align*}
\delay{rev(L)}             &\eqn{\alpha} 1 + \delay{rev\_join(L,[])},\\
\delay{rev\_join([],Q)}    &\eqn{\beta} 1,\\
\delay{rev\_join([I|P],Q)} &\eqn{\gamma} 1 + \delay{rev\_join(P,[I|Q])}.
\end{align*}
Let us now define the delay in terms of the size of the input, that
is, the argument. The function call~\erlcode{rev(\(L\))} has one list
as argument. The usual way to measure the size of a list is to
consider the number of elements it contains, that is, its
length. Here, the size of the input~\(L\) is characterised by an
integer~\(n\). The same measure fits \erlcode{rev\_join/2}. Let us
define the delay of \erlcode{rev/1} and \erlcode{rev\_join/2} in terms
of the input size:
\[
\comp{rev}{n} := \delay{rev(\(L\))};\quad
\comp{rev\_join}{n,m} := \delay{rev\_join(\(P\),\(Q\))},
\]
where \(n\)~is the length of~\(P\) and \(m\)~is the length
of~\(Q\). Using these definitions and assuming that
\erlcode{T}~contains \(p\)~items and \erlcode{A}~contains \(q\)~items,
then the previous equations are equivalent to the following system:
\begin{equation}
\comp{rev}{n} \eqn{\alpha} 1 + \comp{rev\_join}{n,0};\quad
\comp{rev\_join}{0,q} \eqn{\beta} 1,\quad
\comp{rev\_join}{p+1,q} \eqn{\gamma} 1 + \comp{rev\_join}{p,q+1}.
\label{eq:rev_join_p_q}
\end{equation}
We would rather like to express~\(\comp{rev}{p}\) (respectively
\(\comp{rev\_join}{p,q}\)), in terms of~\(p\) only (respectively
\(p\)~and~\(q\)). The solution consists in using the telescoping
technique we have seen before, that is, in making the differences of
two successive terms in the series \((\comp{rev\_join}{p,q})_{p \in
  \mathbb{N}}\) and summing them:
\begin{align*}
  && \boxed{\comp{rev\_join}{p,q}} - \comp{rev\_join}{p-1,q+1} &= 1\\
+ && \comp{rev\_join}{p-1,q+1} - \comp{rev\_join}{p-2,q+2} &= 1\\
+ &&&\;\;\vdots\\
+ &&\comp{rev\_join}{1,q+(p-1)} - \boxed{\comp{rev\_join}{0,q+p}} &= 1
\intertext{\rule{230pt}{0.4pt}}
\Rightarrow 
  &&\boxed{\comp{rev\_join}{p,q}} - \boxed{\comp{rev\_join}{0,q+p}}
  &= \underbrace{1 + 1 + \dots + 1}_{p \; \text{times}} = p\\
\Leftrightarrow &&\comp{rev\_join}{p,q} - 1 &= p\\
\Leftrightarrow &&\comp{rev\_join}{p,q} &= p + 1,\,\; \text{where}
\;\, p > 0.
\intertext{Since that replacing~\(p\) by~\(0\) in the last formula
  leads to the expected}
&&\comp{rev\_join}{0,q} &= 0 + 1 = 1,
\end{align*}
we can gather all the cases into just one formula:
\[
\comp{rev\_join}{n,m} = n + 1,\,\; \text{where} \;\, n, m
\geqslant 0.
\]
Clearly, \(m\)~is useless because it is absent from the
right\hyp{}hand side. Then
\begin{align}
  \comp{rev\_join}{n} &= n + 1 \mathrel{\sim} n;
& \comp{rev}{n} &= n + 2 \mathrel{\sim} n,\,\; \text{as} \;\, n
\rightarrow \infty.\label{delay:rev_join}
\end{align}
When a function~\(f\) satisfies \(f(n) \mathrel{\sim} an\) for some
constant~\(a\) and with~\(n \rightarrow \infty\), it is said to be
\emph{asymptotically linear}. In other words, the delay will be
proportional to the size of the input, for large inputs; that is a
much better than a quadratic relationship.

Actually, a little bit more linguistic argument would have lead us
quickly to the result, by counting the number of times each clause is
used when rewriting a non\hyp{}empty list of \(n\)~items:
\begin{itemize}

  \item clause~\(\alpha\) is used once;

  \item clause~\(\gamma\) is used once for each item in the input,
    that is, \(n\)~times;

  \item clause~\(\beta\) is used once,

\end{itemize}
that is, the execution trace is \(\alpha\gamma^n\beta\) and the total
number of rewrites is indeed \(\abs{\alpha\gamma^n\beta} =
\abs{\alpha} + n \cdot \abs{\gamma} + \abs{\beta} = n+2\).

We can now compare \erlcode{srev/1} and \erlcode{rev/1} in terms of
delays:
\[
\comp{srev}{n} = \frac{(n+1)(n+2)}{2} = (n+1)\frac{\comp{rev}{n}}{2} =
\frac{n+1}{2} \cdot \comp{rev}{n}.
\]
It is clear how much more efficient is~\erlcode{rev/1} with respect
to \erlcode{srev/1} and this comparison does not rely on any
assumption about the speed of the \Erlang run\hyp{}time, for example,
the virtual machine, with respect to the wall\hyp{}clock time. Again,
let us hammer down that the efficiency of \erlcode{rev/1} is
\emph{not} due to its definition being in tail form, but because the
algorithm it implements is more efficient. That is why we did not
rename it~\erlcode{rev\_tf/1} because we only do this to signal that a
definition is the tail form version of another and, this is not the
case of \erlcode{rev/1} in regard to \erlcode{srev/1} (whose name is
now revealed as standing for \emph{\textbf{s}low \textbf{rev}ersal}).

\medskip

\paragraph{Counting the Pushes}
\label{counting_the_pushes}

We said above that we do not count the delay incurred by pushing items
on top of a list, that is, we suppose
\[
\delay{[I|L]} = \delay{I} + \delay{L},
\]
because the construction of the value~\erlcode{[I|L]} is not the
result of a user\hyp{}defined function. We cannot compare or mix
delays of predefined and user\hyp{}defined functions because their
implementation most probably rely on different execution
mechanisms. But it is possible to make a complementary analysis of
some user\hyp{}defined function in order to count the number of items
that are pushed on a list, for instance. Usually, this is done using
the same approach we used for counting the number of rewrites to reach
the results. For example,
\begin{alltt}
srev(   [])   -> [];\hfill% \emph{No push}
srev([I|L])   -> join(srev(L),[I]).\hfill% \emph{At least one}
join(   [],Q) -> Q;\hfill% \emph{No push}
join([I|P],Q) -> [I|join(P,Q)].\hfill% \emph{At least one}
\end{alltt}
Assuming that \erlcode{P}~contains \(p\)~items, we deduce the
recurrence equations
\begin{align*}
\push{srev}{0}   &= 0, & \push{join}{0}   &= 0,\\
\push{srev}{p+1} &= 1 + \push{srev}{p} + \push{join}{p},
&\push{join}{p+1} &= 1 + \push{join}{p},
\intertext{where \(\push{srev}{p}\) and \(\push{join}{p}\) are the
  number of pushes needed to compute, respectively,
  \erlcode{srev(\(L\))} and \erlcode{join(\(P\),\(Q\))}, with \(L\),
  \(P\)~and~\(Q\) denoting lists and \(L\)~and~\(P\) contain
  \(p\)~items each. The technique seen before lead to the solutions,
  for all~\(n \geqslant 0\),}
\push{srev}{n} &= n(n+1)/2, & \push{join}{n} &= n.
\end{align*}
It is meaningless to consider delays of the form \(\comp{srev}{n} +
\push{srev}{n}\) since there is no reason a priori that the time for
processing one rewrite is the same as the time to process one
push. Only the pair \((\comp{srev}{n},\push{srev}{n})\) is meaningful
and could be used to compare two functions.

\medskip

\paragraph{Exercises.}
\label{ex:delay}

\noindent [Answers \vpageref{ans:delay}.]
\begin{enumerate}

  \item Find the delay for \erlcode{rev\_ter/1} defined as
\begin{verbatim}
rev_ter(   []) -> [];
rev_ter([I|L]) -> join_tf(rev_ter(L),[I]).

join_tf(P,Q)        -> join(P,Q,[]).
join(   [],Q,   []) -> Q;
join(   [],Q,[I|A]) -> join([],[I|Q],A);
join([I|P],Q,    A) -> join(P,Q,[I|A]).
\end{verbatim}

  \item Assess the maximum memory usage, including the call stack, of
    the previous functions. Justify your answer as precisely as you
    can.

  \item Which of the two following equivalent definitions is the most
    efficient in terms of delay?
\begin{verbatim}
join_3a(P,Q,R) -> join(P,join(Q,R)).
join_3b(P,Q,R) -> join(join(P,Q),R).
\end{verbatim}

\end{enumerate}
