%%-*-latex-*-

\chapter{Aliasing and Tail\hyp{}Call Optimisation}

We arbitrarily defined, \vpageref{def:call_stack_heap}, two distinct
memory zones: the call stack and the heap. We would like now to
provide a rationale for this scheme. Perhaps the best way is to forget
now about the call stack and assume that everything, program and data,
is stored in the heap, which is under the exclusive supervision of the
garbage collector. The garbage collector can be thought of as an
invisible manager which has full access to the contents of the memory
at any moment and whose task consists in finding the nodes in the
abstract syntax trees which are forever useless at some point during
run\hyp{}time. It consequently gets rid of them, so that subsequent
nodes can find enough room in which to be stored. The remainder of
this chapter will hopefully demonstrate that the concept of control
stack, as a complement to the heap, arises naturally enough from a
refined computational model solely based on the heap and that its
purpose is a more efficient memory management.

We showed \vpageref{counting_the_pushes} how the technique used to
find or estimate the delay of a call can also be applied to determine
the number of pushes performed during the evaluation. This number
gives, in general, an overestimation of how much memory needs to be
allocated for the computation. Up to now, we assumed that the result
of a rewrite step constitutes all the data in memory and, somehow, the
function call is wholly replaced by the body of the first clause that
matches it. In terms of abstract syntax trees, the tree on the
left\hyp{}hand side of the arrow is wholly replaced by the one on the
right\hyp{}hand side. For instance, let us recall the following
definition
\begin{alltt}
sum(  [N]) \(\smashedrightarrow{\alpha}\) N;
sum([N|L]) \(\smashedrightarrow{\beta}\) N + sum(L).
\end{alltt}
and consider again the example in \fig~\vref{fig:sum_1234_bis}.
\begin{figure}[t]
\centering
\includegraphics[bb=71 605 374 721]{sum_1234}
\caption{Computation of \erlcode{sum([1,2,3,4])}\label{fig:sum_1234_bis}}
\end{figure}
Nothing is said about how nodes are created or erased. For instance,
on the first rewrite, one node~(\erlcode{|}) has disappeared while a
node~(\erlcode{+}) has been created and connected to a node
\erlcode{sum}. Of the latter, we don't know if it is the same as the
one before the rewrite or whether a copy has been made and
reconnected---because the argument is different. All these details are
considered to be low\hyp{}level, that is, they depend heavily on
implementation issues, as how the run\hyp{}time environment operates,
and can be ignored if one only wants to understand the evaluation
process and assess its delay. Nevertheless, when the concern about
memory usage arises, for example, node creation and deletion, maximum
amount of memory used etc., a more detailed view of the rewriting
process is needed. This finer view relies on the abstract syntax
trees, as they are a canonical representation of both data and
programs. For example, the definition of \erlcode{sum/1} is best
interpreted in terms of rewrites on the abstract syntax trees in
\fig~\vref{fig:sum_ast}.
\begin{figure}[b]
\centering
\includegraphics[bb=71 666 248 721]{sum_ast}
\caption{Definition of \erlcode{sum/1} with abstract syntax trees
\label{fig:sum_ast}}
\end{figure}
It seems that the node~(\erlcode{|}) in the head of
clause~\clause{\beta} can be freed, that is, reclaimed by the garbage
collector, but the argument of \erlcode{sum/1} may be in fact shared
in the body where the function call is located by means of a variable
with multiple occurrences. For example, let us imagine something like
the following definition:
\begin{alltt}
f(L) -> g(L,sum(L)).\hfill% \emph{Variable} L \emph{is repeated.}
\end{alltt}
If we assume that the value of the parameter~\erlcode{L} is not
duplicated but, instead, that the two arguments~\erlcode{L} in the
body refer to the same node in memory, then the root of the abstract
syntax tree of~\erlcode{L} cannot be deleted just after the rewrite
step~\clause{\beta} because it is still necessary to compute the first
argument of the call to~\erlcode{g/2} later. In general, it is best to
assume that all values may be \emph{shared} and, if not, the garbage
collector will find out and delete the useless nodes. Consequently,
when applying a rewrite rule, let us assume that the abstract syntax
tree of the left\hyp{}hand side remains in memory together with the
right\hyp{}hand side \emph{and} that identical subtrees are shared by
both, that is, are not duplicated. The garbage collector will detect
and take care of useless nodes at unpredictable moments.

In other words, the rewrite rules do not explicitly delete any node
and identical subtrees are shared. In order to visualise this data
sharing, we need a kind of abstract syntax tree where identical
subtrees occur only once, which supposes that some nodes can have two
or more incoming edges. Of course, this cannot be a proper tree since,
by definition, all nodes except the root must have exactly one
parent. The adequate data structure which extends the abstract syntax
tree with sharing is the finite \emph{Directed Acyclic Graph}
(DAG). Actually, a DAG is a collection of trees, called a
\emph{forest}, potentially sharing some of their subtrees. Therefore,
such a graph has not, in general, a single root.
\begin{figure}[b]
\centering
\includegraphics[bb=71 666 237 721]{sum_dag}
\caption{Definition of \erlcode{sum/1} with maximum sharing
\label{fig:sum_dag}}
\end{figure}
In \fig~\vref{fig:sum_dag}, we have the same definition
of \erlcode{sum/1} as in \fig~\vref{fig:sum_ast} but with two DAGs,
one for each clause, allowing maximum node sharing. Sharing is, as in
\fig~\vref{fig:fib5_sharing} implemented by arrowed edges. Notice how
this lower\hyp{}level feature does not imply any explicit deletion of
nodes or edges---which, at this level, can be thought of as
references, or indirections, to nodes. Let us make use of this
implementation of the definition of \erlcode{sum/1} on the evaluation
of the call \erlcode{sum([3,7,5])}, whose first stages are shown in
\fig~\vref{fig:sum_375_push}, where the full state of the memory is
given as snapshots between dashed arrows and the last created node
\erlcode{sum} is framed.
\begin{figure}[t]
\centering
\includegraphics[bb=71 625 365 721]{sum_375_push}
\caption{Computation of \erlcode{sum([3,7,5])} with a DAG (phase 1/2)
\label{fig:sum_375_push}}
\end{figure}
The arrows are dashed as to distinguish them from the ones in the
definition, since clauses apply only to parts of the memory, in
general, and we wanted to display all the data after each
rewrite. Notice that all the nodes must be considered useful by the
garbage collector. This property is entailed if the \emph{roots} of
all the abstract syntax trees are deemed useful.

Naturally, this gives rise to the question on how the value of the
original function call is finally computed. Examining
\fig~\vref{fig:sum_375_push} and comparing memory snapshots from left
to right, we realise that roots~(\erlcode{+}) have been accumulating
at the right of the original call, until a reference to a value has
been reached---here, the integer~\erlcode{5}. This process is
analogous to pushing items on a list, although a list containing not
only values, but also expressions as \erlcode{7 + sum([5])}. This
invites us to perform the inverse operation, that is, popping items
from the list in question, in order to finish the calculation. More
precisely, we want to compute \emph{values} from the trees composing
the DAG from right to left, the roots of which are considered an item
in a special list, until the tree of the original call is reached and
associated with its own value---which is, by definition, the final
result. A value can either be an \emph{immediate} value like integers,
atoms and empty lists, or a \emph{constructed} value like
non\hyp{}empty lists and tuples. We may also deal with
\emph{references} to values, which are graphically represented by
edges; for instance, the rightmost tree in the DAG is a reference to
the immediate value~\erlcode{5}. \emph{When the rightmost tree is an
  immediate value or a reference to a value, the second phase of the
  computation (leftward) can take place.} In the following, for the
sake of conciseness, we shall write ``value'' when a ``reference to a
value'' is also acceptable.

While our presently refined computational model forbids erasing nodes,
because this is the exclusive task of the garbage collector, it does
allow for the edge ending in the latest rewritten call to be
overwritten by its value. As explained before, these calls have been
successively framed in \fig~\vref{fig:sum_375_push}. The process
therefore consists here in replacing the edge to the previous node
\erlcode{sum} with an edge to the current value. Then the patched tree
is evaluated, perhaps leading to more trees to be pushed, and the
process resumes.

This algorithm is shown at play in \fig~\vref{fig:sum_375_pop},
\begin{figure}[t]
\centering
\includegraphics[bb=71 626 376 721]{sum_375_pop}
\caption{Computation of \erlcode{sum([3,7,5])} (phase 2/2,
  heap\hyp{}based)
\label{fig:sum_375_pop}}
\end{figure}
which is to be read from right to left. The rightmost memory state is
the result of the prior ``pushing phase,'' on the last state of
\fig~\ref{fig:sum_375_push}. Note how all the nodes~\erlcode{sum}
and~(\erlcode{+}) become useless, step by step. For illustration
purposes, we made all the nodes~(\erlcode{+}) disappear as soon as
possible and three nodes~\erlcode{sum} are reclaimed by the garbage
collector, including the original one, that is, the leftmost. The
leftmost dashed arrow has the superscript~\(2\) because it combines
two steps (\(3+12 \rightarrow 15\) and discarding the original
node~\erlcode{sum}) at once, for the sake of room. A node is useful if
it can be reached from one of the roots of the DAG. Keep in mind that
the argument of the original call, that is, \erlcode{[3,7,5]}, may or
may not be freed by the garbage collector, depending on it being
shared or not (from outside the figure, that is, by some control
context). The intermediary node containing the value~\erlcode{12} has
been freed as well along the way. As far as the programmer is
concerned, the moments when nodes are collected are arbitrary, as the
garbage collector is exclusively in charge of this task, but we wanted
to bring to the fore that garbage collection is interleaved with the
computation or, if adhering to a multiprocessing view, we would say
that collection and computation run in parallel, sharing the same
memory space but without interferences from the programmer's point of
view---only nodes that are forever unreachable from a point onwards
during the execution are swept away.

The previous example is actually worth another, closer look. Indeed,
we can predict exactly when the nodes~\erlcode{sum} can be reclaimed:
after each step backwards (from right to left in
\fig~\vref{fig:sum_375_pop}), the rightmost node~\erlcode{sum} becomes
useless. Same for the intermediary value~\erlcode{12}: it becomes
unreachable from the roots of the DAG as soon is has been used to
compute~\erlcode{15}. The same observation can be made about the
nodes~(\erlcode{+}). All these facts mean that, in our example, we do
not need to rely on the garbage collector to identify these nodes as
useless: let us really implement a list of expressions, \emph{separate
  from the heap}, instead of solely relying on an analogy of push and
pop and storing everything in the heap. This list is called the
\emph{control stack}. \emph{For implementation reasons, the control
  stack never contains constructed data but references to constructed
  data.} As a consequence, a garbage collector is still needed for
checking the constructed data---the heap and the control stack are
complementary and make up the whole memory.

Consider how the computation in \fig~\vref{fig:sum_375_pop} can be
improved with automatic deallocation of nodes based on a
stack\hyp{}based policy (``Last In, First Out'') in
\fig~\vref{fig:sum_375_pop_bis}. Remember that the value
\erlcode{[3,7,5]} is stored in the heap, not in the call stack, and
that it may be shared. Also, due to space limitations on the page, the
last step is actually twofold, as it was in
\fig~\vref{fig:sum_375_pop}.
\begin{figure}[t]
\centering
\includegraphics[bb=71 626 369 721]{sum_375_pop_bis}
\caption{Computation of \erlcode{sum([3,7,5])} (phase 2/2, stack and heap)
\label{fig:sum_375_pop_bis}}
\end{figure}
We can seize the growth of the call stack in
\fig~\vref{fig:sum_375_stack_push}.
\begin{figure}[b]
\centering
\includegraphics[bb=71 673 377 721]{sum_375_stack_push}
\caption{Call stack while computing \erlcode{sum([3,7,5])} (phase 1/2)
\label{fig:sum_375_stack_push}}
\end{figure}
The poppings, from right to left, are presented in
\fig~\vref{fig:sum_375_stack_pop}. The corresponding algorithm
consists in the following steps. Let us first assume, as a result of
the previous steps, that the call stack is not empty and that the top
item is a non\hyp{}contructed value or a reference to a value,
although we shall refer to them both as values.
\begin{enumerate}

  \item While the call stack contains at least two objects, pop out
    the value, but without losing it, so another tree becomes the top;
    \begin{enumerate}
      
    \item if the root of the top tree is a node~\erlcode{sum}, then
      pop it and push instead the value;
      
    \item else, the node~\erlcode{sum} in the tree has an incoming
      edge:
      \begin{enumerate}
        
      \item change its destination so it reaches the value and discard
        the node~\erlcode{sum};
        
      \item evaluate the patched top tree and resume the loop.
        
      \end{enumerate}
      
    \end{enumerate}
    
  \item The only item remaining in the call stack is the result.
    
\end{enumerate}
Actually, we allowed for atoms and integers to be stored in the call
stack, so we could replace any tree which consists solely of a
reference to such kind of value in the heap by a copy of the value.
We can see in \fig~\vref{fig:sum_375_stack_push} that the call stack
grows at every step until a value is reached.
\begin{figure}[b]
\centering
\includegraphics[bb=71 682 373 721]{sum_375_stack_pop}
\caption{Call stack while computing \erlcode{sum([3,7,5])} (phase 2/2)
\label{fig:sum_375_stack_pop}}
\end{figure}

\begin{figure}
\centering
\includegraphics[bb=71 580 377 721]{sum_tf_375_0}
\includegraphics[bb=71 580 377 721]{sum_tf_375_1}
\includegraphics[bb=71 580 238 721]{sum_tf_375_2}
\includegraphics[bb=71 625 249 721]{sum_tf_375_3}
\caption{Computation of \erlcode{sum\_tf([3,7,5])},
no tail\hyp{}call optimisation\label{fig:sum_tf_375}}
\end{figure}

\begin{figure}
\centering
\includegraphics[bb=71 580 340 721]{sum_tf_375_tco_0}
\includegraphics[bb=71 580 362 721]{sum_tf_375_tco_1}
\includegraphics[bb=71 580 228 721]{sum_tf_375_tco_2}
\includegraphics[bb=71 625 230 721]{sum_tf_375_tco_3}
\caption{Computation of \erlcode{sum\_tf([3,7,5])}, 
tail\hyp{}call optimisation\label{fig:sum_tf_375_tco}}
\end{figure}
Let us investigate what happens when using an equivalent definition in
tail form. We found \vpageref{code:sum_tf}:
\begin{alltt}
sum_tf([N|L])   \(\smashedrightarrow{\alpha}\) sum_tf(L,N).
sum_tf(   [],A) \(\smashedrightarrow{\beta}\) A;
sum_tf([N|L],A) \(\smashedrightarrow{\gamma}\) sum_tf(L,A+N).
\end{alltt}
For the sake of clarity in \fig~\vref{fig:sum_tf_375}, where we can
follow the computation of \erlcode{sum\_tf([3,7,5])}, let us swap the
arguments of \erlcode{sum\_tf/2}, that is, we shall use the following
equivalent definition instead:
\begin{alltt}
sum_tf([N|L])   \(\smashedrightarrow{\alpha}\) sum_tf(N,L).
sum_tf(A,   []) \(\smashedrightarrow{\beta}\) A;
sum_tf(A,[N|L]) \(\smashedrightarrow{\gamma}\) sum_tf(A+N,L).
\end{alltt}
\Fig~\vref{fig:sum_tf_375} only shows the first phase, which consists
in pushing in the call stack the tree newly produced by a clause and
sharing the subtrees denoted by variables occurring both in the head
and the body. The second phase consists in popping the accumulated
roots in order to resume suspended control contexts and, in the end,
only the final result remains in the control stack. In the case
of \erlcode{sum\_tf/1}, we notice that we already found the result
after the first phase:~\erlcode{15}. Therefore, in this case, the
second phase does not contribute to build the value, which raises the
question: why keep the previous trees in the first place? Indeed, they
are useless and a common optimisation, named \emph{tail\hyp{}call
  optimisation} and implemented by compilers of functional languages,
consists in popping the previous tree (matched by the head of the
clause) and pushing the new one (created by the body of the
clause). This way the control stack contains only one item at all
times and, since we are using a stack, the garbage is collected
immediately without the need for an external garbage collector, as in
the heap. This optimisation is shown in \fig~\vref{fig:sum_tf_375_tco}
and should be contrasted with the series in
\fig~\vref{fig:sum_tf_375}. \emph{Tail\hyp{}call optimisation can be
  applied to all functions in tail form.}

Perhaps the previous discussions gave the impression that sharing is
always maximum, that is, given a clause, the compiler will determine
the largest abstract syntax subtrees common to the head and the body
and create a directed acyclic graph (DAG) to implement this data
sharing. That is not the case. Compilers, for example, the \Erlang
compilers, only rely on multiple occurrences of variables to define
sharing, so it may be not maximum. Take for instance the function
\erlcode{compress/1} such that the call \erlcode{compress(\(L\))}
results in a list identical to~\(L\), except that successively
repeated items are reduced to one occurrence, for example,
\erlcode{compress([4,a,b,b,b,a,a])} is rewritten into
\erlcode{[4,a,b,a]}. One possible definition is the following:
\begin{alltt}
compress(      []) \(\smashedrightarrow{\alpha}\) [];
compress([I,I|L]]) \(\smashedrightarrow{\beta}\) compress([I|L]);
compress(   [I|L]) \(\smashedrightarrow{\gamma}\) [I|compress(L)].
\end{alltt}
In clause~\clause{\beta}, we can see that~\erlcode{[I|L]} both occurs
in the pattern and the body because \erlcode{[I,I|L]}~is the same
as~\erlcode{[I|[I|L]]}. In \fig~\vref{fig:compress_dag1} is shown the
abstract syntax view of clause~\clause{\beta}.
\begin{figure}
\centering
\includegraphics[bb=71 627 191 721]{compress_dag1}
\caption{Directed acyclic graph (DAG) of clause \clause{\beta} of
  \erlcode{compress/1}\label{fig:compress_dag1}}
\end{figure}
Notice that the two occurrences of~\erlcode{I} in the left\hyp{}hand
side of the clause are not shared because we can only be certain of
sharing when it is obtained by a known clause and the origin of the
two~\erlcode{I}s is unknown. We can also see that no
node~(\erlcode{|}) is shared, despite two lists sharing the same
head~\erlcode{I} and same tail~\erlcode{L}. This shows that the
sharing is not maximal and relies on the programmer. In order to
increase the level of sharing, we need some special \Erlang construct
to express ``This part of the pattern is given a name which can be
used in the body.'' The notation we need is the operator~(\erlcode{=})
with a variable on its left\hyp{}hand side and a pattern on its
right\hyp{}hand side. Consider the changes in
bold:\label{code:compress}
\begin{alltt}
compress(         []) -> [];
compress([I|\textbf{L=}[I|_]]) -> compress(L);\hfill% L \emph{is an alias}
compress(      [I|L]) -> [I|compress(L)].
\end{alltt}
The variable \erlcode{L} is now an \emph{alias} for the pattern
\erlcode{[I|\_]}, so what the compiler understands now is displayed in
\fig~\vref{fig:compress_dag2}.
\begin{figure}
\centering
\includegraphics[bb=71 647 191 721]{compress_dag2}
\caption{DAG of clause \clause{\beta} of
  \erlcode{compress/1} with maximum sharing\label{fig:compress_dag2}}
\end{figure}
Note how the alias~\erlcode{L} does not occur in the figure, as it was
used with the sole intent to have the edge representing the argument
of \erlcode{compress/1} connected to the second node~(\erlcode{|}),
thus specifying that the entire sub\hyp{}list is shared and this
without changing the meaning of the head of \erlcode{compress/1}, that
is, it matches the same input as before.

There is another definition, found \vpageref{code:sflat_par_alpha},
which would benefit from using an alias: \begin{alltt}
sflat(       []) \(\smashedrightarrow{\alpha}\) [];
sflat(   [[]|L]) \(\smashedrightarrow{\beta}\) sflat(L);
sflat([[I|M]|L]) \(\smashedrightarrow{\gamma}\) join(sflat([I|M]),sflat(L));
sflat(    [I|L]) \(\smashedrightarrow{\delta}\) [I|sflat(L)].
\end{alltt}

In clause~\clause{\gamma}, the list~\erlcode{[I|M]} is repeated in the
head and the body, so the node~(\erlcode{|}) in it is not shared, thus
incurring one useless node creation every time this clause is
applied. Let us change it to
\begin{alltt}
sflat(       [])   \(\smashedrightarrow{\alpha}\) [];
sflat(   [[]|L])   \(\smashedrightarrow{\beta}\) sflat(L);
sflat([\textbf{P=}[_|_]|L]) \(\smashedrightarrow{\gamma}\) join(sflat(\textbf{P}),sflat(L));\hfill% \emph{Alias} P
sflat(      [I|L]) \(\smashedrightarrow{\delta}\) [I|sflat(L)].
\end{alltt}
Now, the number of created nodes~(\erlcode{|}) is the number of
non\hyp{}list items in the input, or, in terms of the output, it is
\erlcode{len(sflat(\(L\)))}, where \(L\)~is the input list to be
flattened.

It is interesting to revisit previous definitions with a focus on
memory use instead on the delay. Let us consider again the following
definition of \erlcode{join/2}:
\begin{alltt}
join(   [],Q) \(\smashedrightarrow{\alpha}\) Q;
join([I|P],Q) \(\smashedrightarrow{\beta}\) [I|join(P,Q)].
\end{alltt}
In \fig~\vref{fig:join_dag}, we can see the same definition making use
of two DAGs in order to explicitly display node sharing.
\begin{figure}[t]
\centering
\includegraphics[bb=71 653 292 721]{join_dag}
\caption{Definition of \erlcode{join/2} with maximum
sharing\label{fig:join_dag}}
\end{figure}
Let us detail the computation of \erlcode{join([a,b],[c,d])} by
evincing the node sharing and the call stack management. The first
phase, consisting in pushing the newly created trees in the control
stack is shown in \fig~\vref{fig:join_abcd_push}.
\begin{figure}[h]
\centering
\includegraphics[bb=71 612 295 735]{join_abcd_push}
\caption{Computation of \erlcode{join([a,b],[c,d])} (end of phase 1/2)
\label{fig:join_abcd_push}}
\end{figure}

Note that, for the sake of room, we didn't use dashed arrows as
before, but solid ones which are exactly the arrows used in the
definition, thus the figure as a whole represents the last state of
the control stack, its top being the rightmost value or reference to a
value and its bottom being the leftmost call (the call originally
evaluated). The second phase of the computation of
\erlcode{join([a,b],[c,d])} is shown in \fig~\ref{fig:join_abcd_pop}
\vpageref{fig:join_abcd_pop}. It consists in replacing from right to
left the reference to the previous call by the current (reference to
a) value, until the initial call itself is removed and only the final
result remains. Notice how the second argument, \erlcode{[c,d]}, is
actually shared with the output \erlcode{[a,b,c,d]}.
\begin{figure}[!h]
\centering
\includegraphics[bb=71 600 267 716]{join_abcd_pop_0}
\includegraphics[bb=71 600 245 716]{join_abcd_pop_1}
\includegraphics[bb=71 610 245 713]{join_abcd_pop_2}
\caption{Computation of \erlcode{join([a,b],[c,d])} (phase 2/2)
\label{fig:join_abcd_pop}}
\end{figure}

Let us go back to the definition of \erlcode{rev/1}, as given
\vpageref{code:rev_join}: \verbatiminput{rev.def} The last clause
allocates a new node to hold~\erlcode{I}. In the end, whilst the list
items are shared, the structure itself is not, that is, new
nodes~(\erlcode{|}) are built. If the list contains \(n\)~items, the
delay is~\(n+2\) and the number of extra nodes is~\(n\). There is
perhaps nothing surprising since these two lists are actually
different, but if we reverse the reversed list, we end up with a list
equal to the original \emph{with a distinct, non\hyp{}overlapping
  structure}. Therefore, reversing a list twice incurs a penalty both
in terms of memory and time, although the input and the output are
logically indistinguishable. This unfortunate situation may seem
infrequent, but reversing twice \emph{part of a list} has been done
quite often up to this point. Think of a list accumulator in a
definition in tail form.

Let us revisit~\erlcode{rm\_fst/2}, as defined
\vpageref{code:rm_fst_alpha}:
\verbatiminput{rm_fst.def} This definition allows the output list to
share the sub\hyp{}list starting after the first occurrence of the
item in the input list. This list is denoted by the
variable~\erlcode{L} in the second clause. As a consequence, if the
item occurs at position~\(k\), the head of the list being located at
position~\(0\), this function creates \(k\)~extra nodes. Its worst
case for the delay is the same as for memory consumption, that is, it
happens when the item is absent from the list. Then the result is a
list equal to the input, sharing its items with the input but not the
structure, that is, the nodes (\erlcode{|}).

What about \erlcode{rm\_fst\_tf/2}, as found
\vpageref{code:rm_fst_tf}? We have
\verbatiminput{rm_fst_tf.def} The accumulator~\erlcode{A} stores in
reverse order the first items of the input which are not equal to the
item that is searched. This list is finally reversed on top of the
remaining part of the input, in order to build the result, so the
situation is the same as with~\erlcode{rm\_fst/2}, except that
\(k\)~more nodes have been created (two reversals are composed). This
is too much if the item is missing in the list, because in this case
the output is exactly the same as the input. What we want is to return
the original list if the sequential search fails, instead of reversing
the accumulator in the second clause. Hence, we need to add a
parameter which is this original list and use it in case of failure:
\begin{alltt}
rm_fst_tf(I,L)      -> rm_fst(I,L,[],\textbf{L}).
rm_fst(_,   [],_,\textbf{P}) -> \textbf{P};\hfill% \emph{Here}
rm_fst(I,[I|L],A,\textbf{_}) -> rev_join(A,L);
rm_fst(I,[J|L],A,\textbf{P}) -> rm_fst(I,L,[J|A],\textbf{P}).
\end{alltt}
Incidentally, this change for the sake of memory improves also the
delay in the case of a failure, since there is no more a reversal of a
list whose size~\(n\) is those of the initial list, thus saving
\(n+2\) steps. Furthermore, this modification cannot be performed on
\erlcode{rm\_fst/2}, \vpageref{code:rm_fst}:
\verbatiminput{rm_fst.def} If we try
\begin{alltt}
rm_fst(P)         -> rm_fst(P,P).
rm_fst(_,   [],P) -> P;\hfill% Error
rm_fst(I,[I|L],_) -> L;
rm_fst(I,[J|L],P) -> [J|rm_fst(I,L,P)].
\end{alltt}
we make a mistake because of the control context of the last clause
(so the result is, wrongly, the original list \emph{doubled}). Having
a definition in tail form is a prerequisite for this kind of
optimisation.

Let us also consider the original definition of \erlcode{rm\_lst/2},
as given \vpageref{rm_lst}:
\begin{verbatim}
rm_lst(I,L) -> rev(rm_fst(I,rev(L))).
\end{verbatim}
At this point, we know that \erlcode{rev(\(L\))}~always allocates
\(n\)~nodes if list~\(L\) contains \(n\)~items, and that
\erlcode{rm\_fst(\(I\),rev(\(L\)))} does as well in the worst case,
which happens when \(I\)~is missing in~\(L\). Under this assumption,
\erlcode{rev(rm\_fst(\(I\),rev(\(L\))))} also allocates \(n\)~nodes,
because \erlcode{rm\_fst(\(I\),rev(\(L\)))} holds \(n\)~items.
Therefore, \erlcode{rm\_lst(\(I\),\(L\))} creates \(3n\)~nodes in the
worst case. It is worth noticing that the worst case for the delay
coincides with the worst case for the memory consumption. The input
and the output do not share any node~(\erlcode{|}), because of the
reversals, only the items.

What about \erlcode{rm\_lst\_tf(\(I\),\(L\))}, as found
\vpageref{code:rm_lst_tf}? We have \begin{alltt}
rm_lst_tf(I,L)      \(\xrightarrow{\alpha}\) rev_join(L,[],I).
rev_join(   [],Q,I) \(\xrightarrow{\beta}\) rm_fst(I,Q,[]);
rev_join([J|P],Q,I) \(\xrightarrow{\gamma}\) rev_join(P,[J|Q],I).
rm_fst(_,   [],A)   \(\xrightarrow{\delta}\) A;
rm_fst(I,[I|Q],A)   \(\xrightarrow{\epsilon}\) rev_join(Q,A);
rm_fst(I,[J|Q],A)   \(\xrightarrow{\zeta}\) rm_fst(I,Q,[J|A]).
\end{alltt}

Clause~\clause{\alpha} does not allocate in the
heap. Clause~\clause{\gamma} creates one node~(\erlcode{|}), that is,
\(n\)~nodes if the input list contains
\(n\)~items. Clause~\clause{\beta} performs no allocation. We found
that the worst case for the delay occurs when the item~\erlcode{I} is
present \emph{anywhere} in the input list. How does that translates to
pushes?  Remarkably, one rewrite by means of clause~\clause{\zeta}
results in exactly one node allocation. Same for the recursive clause
of \erlcode{rev\_join/2} (see definition above). Therefore, the worst
case in terms of delay is also the worst case in terms of memory
allocation. Let us assume arbitrarily, but without loss of generality,
that the item is the head of the input list. Then
clause~\clause{\zeta} is unused and \erlcode{rev\_join/2} performs
\(n\)~node creations in the body of clause \clause{\epsilon}. So the
total number of node allocations is~\(2n\) in the worst case, of which
\(n\)~hold the result and \(n\)~are temporary nodes which can be
reclaimed by the garbage collector.

Consider another example. We want to make a list containing the head
of a given list repeated as many times as there are items in the given
list. For instance, \erlcode{[b,3,1]}~results
in~\erlcode{[b,b,b]}. There are at least two ways to implement this
effect, with the same
% The delay is 1+((n-1)+1)=n+1.
\begin{verbatim}
rep_fst1(   []) -> [];
rep_fst1([I|L]) -> [I|repeat(I,L)].
repeat(_,   []) -> [];
repeat(I,[_|L]) -> [I|repeat(I,L)].
\end{verbatim}
and
\begin{verbatim}
rep_fst2(     []) -> [];
rep_fst2(    [I]) -> [I];
rep_fst2([I,_|L]) -> [I|rep_fst2([I|L])].
\end{verbatim}
Both functions have delay~\(n+1\). The difference lies in the memory
used: the last clause of \erlcode{rep\_fst2/1} performs two pushes
instead of one for the last clause of~\erlcode{repeat/2}. As a
consequence, \erlcode{rep\_fst2/1} allocates \(2(n-1)+1=2n-1\) nodes
if~\(n \neq 0\), of which~\(n-1\) are temporary. By contrast,
\erlcode{rep\_fst1/1} only allocates \(n\)~nodes.

\medskip

\paragraph{Exercise.}
\label{ex:aliasing_and_tail-call_optimisation}

\noindent [See answer
  page~\pageref{ans:aliasing_and_tail-call_optimisation}.]

Consider now the definition of \erlcode{srev/1}:
\begin{verbatim}
srev(   []) -> [];
srev([I|L]) -> join(srev(L),[I]).
\end{verbatim}
Determine what part of the input is shared with the output. Is there a
best and worst case? If so, are they different from the delay cases?

