%%-*-latex-*-

\chapter{Persistence and Backtracking}
\label{chap:persistence_and_backtracking}

A distinctive feature of purely functional data structures is that
they are \emph{persistent}. Persistence is the property of values to
remain constant over time. It is a consequence of \Erlang functions
always creating a new version of a data structure to be updated,
instead of modifying it in place. Moreover, as we learnt previously,
only the nodes which actually differ between the input and the output
are actually created, whereas the others are shared by means of
aliases or variables occurring both in the pattern and the body of a
clause. Sharing is sound because it relies on persistence: since the
data cannot be modified in place from different access points in the
program (the roots of the directed acyclic graph), there is no way to
tell apart a perfect copy from a reference to the original---in a
chronological sense. In this chapter, we show how persistence allows
us to undo operations, that is, \emph{to backtrack}, quite
straightforwardly and how sharing makes it affordable in terms of
memory consumption. As a simple case study, let us consider a list
whose different versions we want to record, as items are pushed in and
popped out, starting from the empty list.

\medskip

\paragraph{Version\hyp{}based persistence.}

A simple idea to implement data structures enabling backtracking
consists in keeping all successive states; then, backtracking means
accessing a previous version. A list can be used to keep such a
record, called \emph{history}. Since, from an algorithmic standpoint,
a list is a stack, the most recent version is the top and is thus
accessed in constant time. The older a version is, the longer it takes
to access it, but always in proportion to the time elapsed
in\hyp{}between. Furthermore, two successive versions share as much
structure as the clauses that generated the new one specify. As a
consequence, it is affordable in terms of memory to thread the
history, instead of only passing around the latest version
alone. Consider the following definitions of \erlcode{push/2} and
\erlcode{pop/1}. The call \erlcode{push(\(I\),\(H\))} evaluates in a
history which is made of history~\(H\) on top of which the last
version, extended with~\(I\), has been pushed. The
call~\erlcode{pop(\(H\))} is rewritten in a pair whose first component
is the item~\(I\) on the top of the last version in~\(H\) and second
component is~\(H\) on top of which the last version without~\(I\) is
pushed. In discussing recursive definitions, we may also refer to the
\emph{current version}, meaning the version corresponding to the input
of the clause at hand. Let us assume in the following examples that
the history always contains an empty list at the beginning, so it is
never empty. We write
\begin{alltt}
push(I,H=[L|_])  \(\smashedrightarrow{\alpha}\) [[I|L]|H].
pop(H=[[I|L]|_]) \(\smashedrightarrow{\beta}\) \{I,[L|H]\}.
\end{alltt}
Note the crucial use of \emph{aliases} (\erlcode{H}) to guarantee that
the history before the change is not duplicated by the function
calls. Imagine now the following series of operations on a list
originally empty: push~\erlcode{a}, push~\erlcode{b},
push~\erlcode{c}, pop, push~\erlcode{d}. Let us ignore the popped item
by means of a function projecting the second component of a pair:
\begin{alltt}
snd({_,Y}) -> Y.
\end{alltt}
Then the whole history of the operations can be obtained by the
following composition:
\begin{center}
\texttt{push(d,snd(pop(push(c,push(b,push(a,[])))))).}
\end{center}
This expression evaluates, as expected, into
\begin{center}
\texttt{[[d,b,a],[b,a],[c,b,a],[b,a],[a],[]].}
\end{center}
In order to understand how the data and the structure is shared among
versions in this example, let us reconsider the clauses above under
the aspect of the Directed Acyclic Graphs (DAG) built upon the
abstract syntax trees in \fig~\vref{fig:memo_push_pop}.
\begin{figure}[t]
\centering
\includegraphics[bb=71 622 297 720]{memo_push_pop}
\caption{Definition of \erlcode{push/2} and \erlcode{pop/1} with DAGs
\label{fig:memo_push_pop}}
\end{figure}
The graph of the result is shown in
\fig~\vref{fig:history}. The~\(k\)th previous version in history~\(H\)
is \erlcode{ver(\(k\),\(H\))}:
\begin{verbatim}
ver(K,H) when K >= 0 -> ver__(K,H).

ver__(0,[V|_]) -> V;
ver__(K,[_|H]) -> ver__(K-1,H).
\end{verbatim}
\begin{figure}[t]
\centering
\includegraphics[bb=71 574 226 721]{history}
\caption{DAG of \erlcode{[[d,b,a],[b,a],[c,b,a],[b,a],[a],[]]}\label{fig:history}}
\end{figure}
It seems interesting to remark that the technique presented allows the
newest version in the history to be modified, but not the older ones:
just pop the head of the history and push some other version:
\begin{alltt}
change_last(Update,[Last|History]) -> [Update|History].
\end{alltt}
When a data structure allows every version in its history to be
modified, it is said \emph{fully persistent}; if only the newest
version is updatable, it is said \emph{partially persistent}. 

\medskip

\paragraph{Update\hyp{}based persistence.}
\label{update_based_persistence}

In order to achieve full persistence, we could keep on record the list
of changes to the data structure, instead of the successive states, as
we just did. In our continued example, we have two kinds of updates:
\erlcode{\{push,\(I\)\}}, where~\(I\)~is some item to push, and
\erlcode{pop}. Now history is a list of such updates. But not all
series of \erlcode{push} and \erlcode{pop} are valid, a trivial
example being~\erlcode{[pop]}, which does not allow a version to be
extracted. Another is \erlcode{[pop,pop,\{push,a\}]}. How can we
caracterise the valid histories? It is useful to consider a graphical
representation of \erlcode{push} and \erlcode{pop}, as proposed in
\fig~\vref{fig:push_pop}.
\begin{figure}
\centering
\subfloat[\erlcode{\{push,a\}}]{%
  \includegraphics[bb=71 662 130 721,scale=0.8]{push}
}
\qquad
\subfloat[\erlcode{pop}]{%
  \includegraphics[bb=71 662 130 721,scale=0.8]{pop}
}
\caption{Graphical representations of updates\label{fig:push_pop}}
\end{figure}
A history is, graphically, a broken line composed with these two kinds
of segments, oriented from right to left in order to follow the
\Erlang syntax for lists. Consider for instance the representation of
the history
\erlcode{[pop,pop,\{push,d\},\{push,c\},pop,\{push,b\},\{push,a\}]} in
\fig~\vref{fig:dyck_meander_reversed}.
\begin{figure}[!b]
\centering
\includegraphics[bb=71 577 298 721]{dyck_meander_reversed}
\caption{\erlcode{[pop,pop,\{push,d\},\{push,c\},pop,\{push,b\},\{push,a\}]}
\label{fig:dyck_meander_reversed}}
\end{figure}
Horizontally are the updates and vertically the lengths of the
versions. The line starts at the origin of the axes because the
version at that point is supposed to be an empty list (the caller may
change this assumption at her own expenses). It ends at the present
moment, distinguished by a dot, where the latest version has
length~\(1\). \emph{A valid history is a line which never crosses the
  absissa axis.} The definition of~\erlcode{push/2} is
straightforward:
\begin{alltt}
push(I,H) -> [\{push,I\}|H].
\end{alltt}
The design of \erlcode{pop/1} requires more care as
\begin{alltt}
pop(H) -> [pop|H].\hfill% \emph{Incomplete}
\end{alltt}
does not return the item supposed to be popped and also may create an
impossible configuration of the data structure, as~\erlcode{pop([])}
is accepted instead of being rejected. A better try is
\begin{alltt}
pop(H=[\{push,I\}|_]) -> \{I,[pop|H]\}.\hfill% \emph{Still incomplete}
\end{alltt}
but this definition is incomplete, as it doesn't handle valid
histories like
\begin{alltt}
pop([pop,\{push,b\},\{push,a\}]) \(\nrightarrow\).
\end{alltt}
The missing case is thus when the last update is a~\erlcode{pop}. It
is not necessary to extract the newest version in order to take its
topmost item: it is sufficient to deduce this item from the
updates. If it exists, we know that we can record a~\erlcode{pop}
update because the resulting version is constructible. Graphically, it
means that the historical line reached a point of ordinate~\(1\) at
least, so a~\erlcode{pop} will not cross the absissa axis. The task of
finding this item is delegated to the function~\erlcode{top/1}, hence
\begin{alltt}
pop(H) -> \{top(H),[pop|H]\}.\hfill% \emph{Complete}
\end{alltt}
Note that the alternative \erlcode{[\{pop\}|H]} would waste memory for
one node. The algorithm implemented by~\erlcode{top/1} is best
understood on the graphical representation of updates. Consider again
our running example in \fig~\vref{fig:dyck_meander_reversed}. The last
update was a~\erlcode{pop}. What is the topmost item of the latest
version? By following the historical line backwards, that is, from
left to right, we see that the penultimate update was also
a~\erlcode{pop}, so we cannot conclude yet. The ante\-penultimate
update is a~\erlcode{push}, but its associated item,~\erlcode{d},
cannot be the item we are looking for because it is popped by the
following update (which is the penultimate). Following back in time we
find another \erlcode{push}, but its item,~\erlcode{c}, is popped by
the last update. The previous update is a~\erlcode{pop}, so we are
back to square one. Following backwards is a~\erlcode{push} but its
item,~\erlcode{b}, is popped by the next update. Finally, the previous
update is a~\erlcode{push} and its item,~\erlcode{a}, is the one we
want, because it is the first we found which is not popped by further
updates. Graphically, this update is easily found by drawing a
horizontal line from the present version to the right. There are
several intersections with the historical line. The one we are
interested in is the first end of a~\erlcode{push} update. If such a
point does not exist, we reach the origin. It means that the last
version (the leftmost point of the line) is an empty list and,
consequently, the~\erlcode{pop} update must fail, as
expected. Consider in \fig~\vref{fig:topmost} the geometrical
solution.
\begin{figure}[!t]
\centering
\includegraphics[bb=71 577 298 721]{topmost}
\caption{Finding the topmost item~\erlcode{a} of the last version
\label{fig:topmost}}
\end{figure}
This results in the following code:
\begin{alltt}
top(H)              -> top(0,H).
top(0,[\{push,I\}|\_]) -> I;
top(K,[\{push,\_\}|H]) -> top(K-1,H);
top(K,     [pop|H]) -> top(K+1,H).
\end{alltt}
The counter~\erlcode{K} is used to keep track of the number
of \erlcode{pop} updates (or, equivalently, the height of the old
versions with respect to the last one), so, when it
reaches~\erlcode{0}, we check whether the current update is
a~\erlcode{push} (we know that we are on the same horizontal line as
the last version). If so, we are done; otherwise, we resume the
search. Encountering the origin leads to a match failure. Now, the
call
\begin{center}
\texttt{push(d,snd(pop(push(c,push(b,push(a,[]))))))}
\end{center}
evaluates, as expected, into
\begin{center}
\texttt{[\{push,d\},pop,\{push,c\},\{push,b\},\{push,a\}].}
\end{center}
Let us pop twice more:
\begin{alltt}
pop(snd(pop(push(d,snd(pop(push(c,push(b,push(a,[])))))))))
\end{alltt}
and the evaluation yields the expected pair
\begin{center}
\texttt{\{b,[pop,pop,\{push,d\},pop,\{push,c\},\{push,b\},\{push,a\}]\}.}
\end{center}
Let \erlcode{ver(\(k\),\(H\))} be the~\(k\)th version from the last
moment in history~\(H\), counting the last version as~\(0\). When a
past version is needed, that is, we walk back through history until
the time requested (searching phase),
\begin{alltt}
% Searching
ver(0,    H) -> last(H);
ver(K,[_|H]) -> ver(K-1,H).
\end{alltt}
and we rebuild the version from that point (building phase). How do we
define~\erlcode{last/1}, which builds the last version? There are
several ways. Perhaps the first idea is to recursively construct the
penultimate version and if the last update was a push, then perform a
push on it, otherwise, its top item is popped:\label{code:ver}
\begin{alltt}
% Building
last(          []) -> [];
last([\{push,I\}|H]) -> [I|last(H)];
last(     [pop|H]) -> tail(last(H)).
tail([_|L])       -> L.
\end{alltt}
Note how \erlcode{tail/1}~is used to discard the popped item on our
way, but this creates an asymmetry in the delay between the processing
of a push update and a pop (one more rewrite). Let us determine the
delays of \erlcode{ver/2} and \erlcode{last/1}. First, let us note
\(\comp{ver}{k,n}\) and \(\comp{last}{p}\) the respective delays of
\erlcode{ver(\(k\),\(H_1\))} and \erlcode{last(\(H_2\))}, where
histories \(H_1\)~and~\(H_2\) have respective lengths
\(n\)~and~\(p\). The \Erlang definition of \erlcode{ver/2} directly
leads to \(\comp{ver}{k,n} = (k + 1) + \comp{last}{n-k}\). The delay
of \erlcode{last/1} depends on the nature of the updates: if
a~\erlcode{push}, the second clause adds a delay of~\(1\), otherwise
the third clause adds a delay of~\(2\), due to the additional delay of
calling~\erlcode{tail/1}. Formally:
\[
\comp{last}{0} = 1,\qquad
\comp{last}{p+1} = \begin{cases}
                     1 + \comp{last}{p} & \text{if \erlcode{push}},\\
                     2 + \comp{last}{p} & \text{if \erlcode{pop}}.
                   \end{cases}
\]
This asymmetry can be remedied by a special construct of \Erlang,
commonly found in other programming languages: a~\emph{case}, also
known as a~\emph{switch} in \Java~and~\C:
\begin{alltt}
last(          []) -> [];
last([\{push,I\}|H]) -> [I|last(H)];
last(     [pop|H]) -> \textbf{case} last(H) \textbf{of} [_|V] -> V \textbf{end}.
\end{alltt}
The meaning is ``Compute the value of \erlcode{last(H)}, which must be
a non\hyp{}empty list. Let us name~\erlcode{V} its tail and
\erlcode{V}~is the value of the body.'' The arrow in the case
construct is not compiled as a function call and thus its delay
is~\(0\). When exactly one case of the value being tested
(here~\erlcode{last(H)}) is expected, like here, \Erlang permits a
shortcut notation:
\begin{alltt}
last(          []) -> [];
last([\{push,I\}|H]) -> [I|last(H)];
last(     [pop|H]) -> \textbf{[_|V] = last(H), V}.
\end{alltt}
Now it is very easy to derive the delay of \erlcode{last(H)}:
\(\comp{last}{p} = p + 1\). Hence \(\comp{ver}{k,n} = (k+1) +
\comp{last}{n-k} = k+1+(n-k)+1 = n + 2\). Interestingly, the delay
does not depend on what version is computed. What about memory usage?
As usual, since we do not know anything about the strategy of the
garbage collector, we cannot deduce the exact amount of memory
necessary to compute the result, but we can count the number of pushes
needed, which allows us to compare different versions of the same
function. (Incidentally, this number can be considered a contribution
to the overall delay, as creating a node takes some time, therefore it
complements our delay analysis as it only accounts for the number of
function calls.) The only clause of \erlcode{last/1} performing a push
is the second one and it applies when a~\erlcode{push} update is
found. Therefore, the number of push nodes created is the number
of \erlcode{push} updates in the history. This is a waste in certain
cases, for example, when the version built is empty (consider the
history \erlcode{[pop,\{push,6\}]}). The optimal situation would be to
allocate only as much as the computed version actually contains. This
brings us back to the function~\erlcode{top/1}, which computes the top
of the last version. What we need is to call it recursively on the
remainder of the history, until the empty list is reached. Here is the
modification:
\begin{alltt}
ver(0,    H) -> last(0,H);
ver(K,[_|H]) -> ver(K-1,H).

last(0,[\{push,I\}|\textbf{H}]) -> \textbf{[I|last(0,H)]};\hfill% \emph{Keep going back}
last(M,[\{push,_\}|H]) -> last(M-1,H);
last(M,     [pop|H]) -> last(M+1,H);
\textbf{last(_,          []) -> []}.\hfill% \emph{Origin of history}
\end{alltt}
We still have \(\comp{last}{p} = p + 1\), but the number of push nodes
created is now the length of the last version
\erlcode{ver(\(0\),\(H\))}. This is an example of
\emph{output\hyp{}dependent} measure. There is still room for some
improvement in the special case when the historical line meets the
absissa axis---in other words, when a past version is the empty
list. There is no need to visit the updates \emph{before}
a~\erlcode{pop} resulting into an empty version, for instance, in
\fig~\vref{fig:return_to_zero}, it is useless to go past
\erlcode{\{push,c\}} to find the last version to be~\erlcode{[c]}.
\begin{figure}[!b]
\centering
\includegraphics[bb=69 606 298 721]{return_to_zero}
\caption{Last version \erlcode{[c]} found in four steps
\label{fig:return_to_zero}}
\end{figure}
In order to detect whether the historical line meets the absissa axis,
let us keep, jointly with the history of updates, the length of the
last version. First, let us modify \erlcode{push/2} and
\erlcode{pop/1} accordingly:
\begin{verbatim}
push(I,{N,H}) -> {N+1,[{push,I}|H]}.
pop({N,H}) -> {top(H),{N-1,[pop|H]}}.
\end{verbatim}
We have to rewrite~\erlcode{ver/2} to keep track of the length of the
last version and, while we are at it, we can check that \erlcode{K}~is
not negative:
\begin{verbatim}
ver(K,NH) when K >= 0     -> ver__(K,NH).
ver__(0,              NH) -> last(0,NH);
ver__(K,{N,     [pop|H]}) -> ver__(K-1,{N+1,H});
ver__(K,{N,[{push,_}|H]}) -> ver__(K-1,{N-1,H}).
\end{verbatim}
We can reduce the memory footprint by separating the length of the
current version~\erlcode{N} from the current history~\erlcode{H}, so
no pair nodes~(\erlcode{\{\textvisiblespace,\textvisiblespace\}}) are
allocated:
\begin{verbatim}
ver(K,{N,H}) when K >= 0 -> ver__(K,N,H).
ver__(0,N,           H)  -> last(0,N,H);
ver__(K,N,     [pop|H])  -> ver__(K-1,N+1,H);
ver__(K,N,[{push,_}|H])  -> ver__(K-1,N-1,H).
\end{verbatim}
Let us change the definition of \erlcode{last/2} so it processes the
length of the current version:
\begin{alltt}
\textbf{last(_,0,           _) -> [];}\hfill% \emph{Absissa \(0\)}
last(0,N,[\{push,I\}|H]) -> [I|last(0,N-1,H)];
last(M,N,[\{push,_\}|H]) -> last(M-1,N-1,H);
last(M,N,     [pop|H]) -> last(M+1,N+1,H).
\end{alltt}
Notice how the last clause of the previous version of \erlcode{last/2}
\begin{alltt}
last(_,          []) -> [].\hfill% \emph{Useless now}
\end{alltt}
became redundant with the new first clause and therefore removed. The
delay now presents a worst and best cases. The worst case is when the
bottom item of the last version is the first item pushed in the
history, so \erlcode{last/2}~has to recur until the origin of
time. This incurs the same delay as previously: \(\worst{last}{p} = p
+ 1\). The best case happens when the last version is empty. In this
case \(\best{last}{p} = 1\), and this is an occurrence of the kind of
improvement we sought.

\medskip

\paragraph{Changing the past.}
\label{changing_the_past}

The \emph{update\hyp{}based approach} is fully persistent, that is, it
allows us to modify the past as follows: traverse history until the
required moment, pop the update at that point, push another one and
simply put back the previously traversed updates, which must have been
kept in some accumulator. It is of utmost importance that, by changing
the past, we do not create a history with a non\hyp{}constructible
version in it, that is, we must check that the historical line does
not cross the absissa axis after the modification. If the change
consists in replacing a~\erlcode{pop} by a~\erlcode{push}, there is no
need to worry, as this will raise by~\(2\) the end point of the
line. It is the converse change that requires special attention, as
this will lower by~\(2\) the end point. The~\(\pm 2\) offset comes
from the vertical difference between the ending points of
a~\erlcode{push} and a~\erlcode{pop} update of same origin, as can be
easily figured out by looking back at \fig~\vref{fig:push_pop}. As a
consequence, in \fig~\vref{fig:dyck_meander_reversed}, the last
version has length~\(1\), which implies that it is impossible to
replace a~\erlcode{push} by a~\erlcode{pop}, anywhere in the past.
Let us consider the history in \fig~\vref{fig:short_dyck}
\begin{figure}
\centering
\includegraphics[bb=69 578 270 721]{short_dyck}
\caption{\erlcode{[pop,\{push,d\},\{push,c\},pop,\{push,b\},\{push,a\}]}
\label{fig:short_dyck}}
\end{figure}
Let us note \erlcode{change(\(k\),\(U\),\{\(n\),\(H\)\})}, where
\(k\)~is the index of the update we want to change, indexing the last
one at~\(0\); \(U\)~is the new update we want to set; \(n\)~is the
length of the last version of history~\(H\). The call
\begin{alltt}
  change(3,\{push,e\},
         \{2,[pop,\{push,d\},\{push,c\},pop,\{push,b\},\{push,a\}]\})
\end{alltt}
is rewritten into (changes in bold)
\begin{center}
\texttt{\{\textbf{4},[pop,\{push,d\},\{push,c\},\textbf{\{push,e\}},\{push,b\},\{push,a\}]\})}
\end{center}
\begin{figure}%[H]
\centering
\includegraphics[bb=69 521 270 721]{pop_to_push}
\caption{Changing \erlcode{pop} (\erlcode{b}) into \erlcode{\{push,e\}}
\label{fig:pop_to_push}}
\end{figure}
This call succeeds because, as we can see graphically in
\fig~\vref{fig:pop_to_push}, the new historical line does not cross
the absissa axis. We can see in \fig~\vref{fig:push_to_pop}
\begin{figure}[!b]
\centering
\includegraphics[bb=69 577 270 721]{push_to_pop}
\caption{Changing \erlcode{\{push,c\}} into \erlcode{pop}
  (\erlcode{a}).
\label{fig:push_to_pop}}
\end{figure}
the result of the call
\begin{alltt}
 change(2,pop,
        \{2,[pop,\{push,d\},\{push,c\},pop,\{push,b\},\{push,a\}]\}).
\end{alltt}
It should be clear now that
\begin{alltt}
change(4,pop,
      \{2,[pop,\{push,d\},\{push,c\},pop,\{push,b\},\{push,a\}]\})\(\nrightarrow\),
change(5,pop,
      \{2,[pop,\{push,d\},\{push,c\},pop,\{push,b\},\{push,a\}]\})\(\nrightarrow\)
\end{alltt}
because the line would cross the absissa axis. All these examples help
in guessing the characteristic property for a replacement to be valid:
\begin{itemize}

  \item the replacements of a~\erlcode{pop} by a~\erlcode{push},
    a~\erlcode{pop} by a~\erlcode{pop}, a~\erlcode{push} by
    a~\erlcode{push} are always valid;

  \item the replacement of a~\erlcode{push} by a~\erlcode{pop} at
    update~\(k>0\) is valid if and only if the historical line between
    updates \(0\)~and~\(k-1\) remains above or reaches without
    crossing the horizontal line of ordinate~\(2\).

\end{itemize}
We can divide the algorithm in two phases, as we did with
\erlcode{ver/2} and \erlcode{ver\_\_/3}, on the one hand, and
\erlcode{last/3}, on the other hand. First, the update to be replaced
must be found, but, the difference with \erlcode{ver/2} (or, more
accurately, \erlcode{ver\_\_/3}), is that we may need to know if the
historical line before reaching the update lies above the horizontal
line of ordinate~\(2\). This is easy to check if we maintain across
recursive calls the lowest ordinate reached by the line. The second
phase performs the change of update and checks if the resulting
history is valid. Let us implement the first phase. First, we check
that the index of the update is not negative; the length of the last
version is separated from the history, in order to save some memory
space, and the lowest ordinate is this length, which we pass as an
additional argument to another function,~\erlcode{chg/5}:
\begin{verbatim}
change(K,U,{N,H}) when K >= 0 -> chg(K,U,N,H,N).
\end{verbatim}
Function~\erlcode{chg/5} traverses~\erlcode{H} while
decrementing~\erlcode{K}, until the latter reaches~\erlcode{0}. At the
same time, the length of the current version is computed (third
argument) and compared to the previous lowest ordinate (the fifth
argument), which is updated according to the outcome. When the update
to be changed is found, \erlcode{K}~is~\erlcode{0}.
\begin{alltt}
chg(0,U,N,           H,M)          -> \fbcode{repl(U,H,M)};
chg(K,U,N,     [pop|H],M)          -> chg(K-1,U,N+1,H,M);
chg(K,U,N,[\{push,\_\}|H],M) when M<N -> chg(K-1,U,N-1,H,M);
chg(K,U,N,[\{push,\_\}|H],\_)          -> chg(K-1,U,N-1,H,N-1).
\end{alltt}
The problem is that we forget the history down to the update we are
looking for. There are two methods to record it: either we use an
accumulator and stick with a definition in tail\hyp{}form, or we put
back a visited update after each return of recursive call. The latter
is faster, as there is no need to reverse the accumulator when we are
done; the former allows us to share the history up to the update, at
the cost of an extra parameter being the original history. Let us opt
for limiting the number of arguments:
\begin{alltt}
chg(0,U,\_,             H,M)            -> \textbf{repl(U,H,M)};
chg(K,U,N,       [pop|H],M)            ->
             \textbf{\{N1,H1\} =} chg(K-1,U,N+1,H,  M)\textbf{, \{N1,[pop|H1]\}};
chg(K,U,N,[\textbf{P=}\{push,\_\}|H],M) when M < N -> 
             \textbf{\{N1,H1\} =} chg(K-1,U,N-1,H,  M)\textbf{, \{N1,  [P|H1]\}};
chg(K,U,N,[\textbf{P=}\{push,\_\}|H],\_)            ->
             \textbf{\{N1,H1\} =} chg(K-1,U,N-1,H,N-1)\textbf{, \{N1,  [P|H1]\}}.
\end{alltt}
Notice how the length~\erlcode{N1} of the changed history is
invariant, because we can simply make it out once the update to change
is found:
\begin{itemize}

  \item replacing a~\erlcode{pop} by a~\erlcode{pop} or
    a~\erlcode{push} by a~\erlcode{push} leaves the original length
    invariant;

  \item replacing a~\erlcode{pop} by a~\erlcode{push} increases the
    original length by~\(2\);

  \item replacing a~\erlcode{push} by a~\erlcode{pop}, assuming this
    is valid, decreases the original length by~\(2\).

\end{itemize}
This task is up to the new function~\erlcode{repl/3} (``replace''),
which implements the second phase (the replacement itself). The idea
is that it returns a pair made of the differential in
length~\erlcode{D} and the new history~\erlcode{H1}, which implies
that we need to change~\erlcode{change/3}:
\begin{alltt}
change(K,U,\{N,H\}) when K >= 0 ->
\hfill\textbf{\{D,H1\} =} chg(K,U,N,H,N)\textbf{, \{N+D,H1\}}.

repl(       pop,   H=[pop|\_],\_)            -> \{ 0,      H\};
repl(P=\{push,\_\},[\{push,\_\}|H],\_)            -> \{ 0,  [P|H]\};
repl(P=\{push,\_\},     [pop|H],\_)            -> \{ 2,  [P|H]\};
repl(       pop,[\{push,\_\}|H],M) when M > 1 -> \{-2,[pop|H]\}.
\end{alltt}

\medskip

\paragraph{Improvement.}

Just like \vpageref{tuples_vs_list} we saved memory by nesting tuples
instead of putting tuples in a list when transforming systematically
definitions into tail form, we can here save memory by implementing a
history by nested updates. Instead of having
\begin{center}
\texttt{[pop,pop,\{push,d\},pop,\{push,c\},\{push,b\},\{push,a\}]}
\end{center}
we shall prefer
\begin{center}
\texttt{\{pop,\{pop,\{push,d,\{pop,\{push,c,\{push,b,\{push,a,\{\}\}\}\}\}\}\}\}}
\end{center}
This alternative encoding leads to save a node and an edge for each
\erlcode{push}~update. We wrote \verbatiminput{pers.erl} Now we can
proceed to improve the definitions as intended:
\verbatiminput{pers_opt.erl} Notice how this change of data structure
has an impact on the maintenance of \erlcode{last/3} because we cannot
anymore match any update with an underscore, thus the new version of
this function is longer.

\medskip

\paragraph{Persistent associative arrays.}

\hspace*{-5pt} Let us explore the possibilities offered by the
persistence of purely functional data structures by implementing a
\emph{persistent associative array}. A traditional array is a
contiguous chunk of memory which can be accessed randomly by means of
integer indexes ranging over an interval. An array has its length
fixed at its creation and contains as many values as indexes are
allowed. Associative arrays are commonly found in scripting languages,
where they generalise arrays by accepting indexes to be any kind of
value, not just integers. As a consequence, the notion of interval of
indexes is dropped. All these arrays have in common the fact that they
are updated destructively, that is, the value associated to some
index, once updated, is lost. Arrays of both kinds are a staple
feature of \emph{imperative programming}. To obtain a persistent
associative array, our first attempt will make use of the
update\hyp{}based persistence we studied above. Let us suppose that we
only have two updates on arrays:
\begin{itemize}

  \item \erlcode{set}~records an assignment of a given value at a
    given index, for instance, \erlcode{\{set,d,5\}} specifies that
    the value at index~\erlcode{d} is~\erlcode{5} and any previous
    value at that index is now part of an old version;

  \item \erlcode{unset}~records the cancellation of a previous
    \erlcode{set}~update at a given index, for instance,
    \erlcode{\{unset,d\}} means to unset any previous
    \erlcode{set}~update at index~\erlcode{d}. If the index was not in
    use or was already unset, \erlcode{unset}~has not effect,
    otherwise the previous value at that index becomes the latest,
    that is, it becomes part of the current version (if constructed).

\end{itemize}
A history is made of a list of such updates, starting with the empty
list, which is interpreted as an empty array. Notice that, in contrast
with the histories we dealt with previously, \emph{we record here the
  \erlcode{unset} update itself}. Here, it is not allowed to change
the past directly: the only way to cancel a previous
\erlcode{set}~update is to add an \erlcode{unset} update to the
history---history grows monotonically. Of course, sometimes a version
of the array needs to be built, so a list of pairs index\hyp{}value of
the latest assigned indexes will be constructed from the history. This
kind of list is called an \emph{association list} and the indexes are
then often called \emph{keys} in this context. The reason why we
started with the denomination ``associative array'' is because we
wanted to provide the same functionalities as an array, but using a
list. Hence, the array can be considered as the specification here,
whilst the list is the actual implementation. Note that, since in
general keys are not required to be totally ordered, several
up\hyp{}to\hyp{}date lists may correspond to one array. Let us
consider two basic functions on these persistent associative arrays.
\begin{itemize}

  \item The function~\erlcode{get/2} accesses a value by means of its
    given key, for instance, \erlcode{get(d,\(H\))} evaluates to the
    value associated with the key~\erlcode{d} in the history~\(H\). If
    the key was not last assigned by~\erlcode{set}, the
    atom~\erlcode{none} results. Notice that the key in this example
    is an atom, but any kind of value is permited.

  \item The function~\erlcode{ver/2} is such that
    \erlcode{ver(\(n\),\(H\))} builds the version \(n\)~steps before
    the present moment in history~\(H\), the current version
    corresponding to \(0\)~steps.

\end{itemize}
Let us start with the former. We need to make five cases: one for the
empty history and two for each update, depending on whether the sought
key is present in the last update or not:
\begin{alltt}
get(_,           []) \(\smashedrightarrow{\alpha}\) absent;
get(X,[\{set,X,Y\}|H]) \(\smashedrightarrow{\beta}\) \fbcode{skip(X,H)};\hfill% \emph{Index found}
get(X,[\{set,A,B\}|H]) \(\smashedrightarrow{\gamma}\) \fbcode{skip(X,H)};
get(X,[\{unset,X\}|H]) \(\smashedrightarrow{\delta}\) \fbcode{skip(X,H)};\hfill% \emph{Index found}
get(X,[\{unset,A\}|H]) \(\smashedrightarrow{\epsilon}\) \fbcode{skip(X,H)}.
\end{alltt}
Clause~\clause{\beta} corresponds to the case when we found the value
associated with the key~\erlcode{X}: it is~\erlcode{Y}:
\begin{alltt}
get(X,[\{set,X,Y\}|_]) \(\smashedrightarrow{\beta}\) Y;
\end{alltt}
Clause~\clause{\gamma} matches when we find an assignement about a
different key, so we have to keep looking in~\erlcode{H}:
\begin{alltt}
get(X,[\{set,_,_\}|H]) \(\smashedrightarrow{\gamma}\) get(X,H);
\end{alltt}
Clause~\clause{\epsilon} finds a different key being unset, so this is
not our concern and we also should pass our way:
\begin{alltt}
get(X,[\{unset,_\}|H]) \(\smashedrightarrow{\epsilon}\) get(X,H).
\end{alltt}
The difficult one is clause~\clause{\delta} because we found the key
we are seeking to be unset. This means that looking forward in the
history~\erlcode{H}, that is, the past, \emph{the next assignment to
  the key~\erlcode{X} must be ignored}, and the search should
resume. This sounds very much like a variant of \erlcode{get/2}, we
could call~\erlcode{skip/2}. Here are the final definitions:
\begin{alltt}
get(_,           []) -> absent;
get(X,[\{set,X,Y\}|_]) -> Y;
get(X,[\{set,_,_\}|H]) -> get(X,H);
get(X,[\{unset,X\}|H]) -> skip(X,H);\hfill% \emph{Skip the next} set...
get(X,[\{unset,_\}|H]) -> get(X,H).

skip(_,           []) -> absent;
skip(X,[\{set,X,_\}|H]) -> get(X,H);
skip(X,[\{set,_,_\}|H]) -> skip(X,H);\hfill% \emph{...done.}
skip(X,[\{unset,_\}|H]) -> skip(X,H).
\end{alltt}
The usual improvement in terms of memory usage consists in getting rid
of the list:
\begin{verbatim}
get(_,         {}) -> absent;
get(X,{set,X,Y,_}) -> Y;
get(X,{set,_,_,H}) -> get(X,H);
get(X,{unset,X,H}) -> skip(X,H);
get(X,{unset,_,H}) -> get(X,H).

skip(_,         {}) -> absent;
skip(X,{set,X,_,H}) -> get(X,H);
skip(X,{set,_,_,H}) -> skip(X,H);
skip(X,{unset,_,H}) -> skip(X,H).
\end{verbatim}

\medskip

\paragraph{Exercises.}
\label{ex:persistence_and_backtracking}

\noindent [See answers page~\pageref{ans:persistence_and_backtracking}.]
\begin{enumerate}

  \item Consider the following variant of \erlcode{ver/2}:
\begin{verbatim}
ver(0,H) -> last([],rev(H)).

last(    V,          []) -> V;
last(    V,[{push,I}|R]) -> last([I|V],R);
last([_|V],     [pop|R]) -> last(V,R).

rev(L)            -> rev_join(L,[]).
rev_join(   [],Q) -> Q;
rev_join([I|P],Q) -> rev_join(P,[I|Q]).
\end{verbatim}
    Find the delay and the number of pushes.

  \item The function call~\erlcode{ver(\(n\),\(H\))}
    \vpageref{code:ver} builds the version at \(n\)~steps in the
    history~\(H\) counted from the present version. Define a
    function~\erlcode{mk\_ver/2} such that
    \erlcode{mk\_ver(\(n\),\(H\))}~is the version at position~\(n\)
    \emph{counted from the origin} of the history~\(H\). Find the
    delay.

  \item Write a definition of \erlcode{ver/2} for persistent
    associative arrays and compare it with the one
    \vpageref{code:ver}.

  \item Let us define a variation on the theme of the persistent
    array. We wish to grant the user the possibility to think an array
    to be a contiguous piece of memory whose basic components, called
    \emph{cells}, are accessed through their \emph{index}. Contrary to
    associative arrays, indexes must range over an integer interval,
    from a \emph{lower bound} to an \emph{upper bound}. For example,
    the following array is made of five cells containing \(a\), \(b\),
    \(c\), \(d\)~and~\(e\), which have the respective indexes~\(4\),
    \(5\), \(6\), \(7\)~and~\(8\), the lower bound being~\(4\) and the
    upper bound~\(8\):
    \[
    \begin{array}{|c|c|c|c|c|}
      \multicolumn{1}{c}{4} &
      \multicolumn{1}{c}{5} &
      \multicolumn{1}{c}{6} &
      \multicolumn{1}{c}{7} &
      \multicolumn{1}{c}{8}\\
      \hline
      a & b & c & d & e\\
      \hline
    \end{array}
    \]
    An empty array is implemented by an empty tuple. A non\hyp{}empty
    array is a triple made of
    \begin{enumerate}

      \item a pair of integers representing the index bounds;

      \item a default value for unassigned cells;

      \item a series of updates in the shape of nested tuples.

    \end{enumerate}
    Contrary to persistent associative arrays, there is only one
    update, \erlcode{\{set,\(E\),\(I\),\(A\)\}}, which represents the
    assignment of the value~\(E\) at index~\(I\) in the
    array~\(A\). It is possible to reassign a cell, that is, to
    specify two updates for the same index (this enables the
    interpretation of the data structure as an imperative array) and
    also to undo an update (which supposes the array to be persistent
    as well). The function~\erlcode{init/3} is such that
    \erlcode{init(\(L\),\(U\),\(D\))} creates an array whose indexes
    range from the lower bound~\(L\) to the upper bound \(U \geqslant
    L\) and containing the default value of~\(D\). We simply have
\begin{verbatim}
init(L,U,D) when L =< U -> {{L,U},D,{}}.
\end{verbatim}
    Define the following functions:
    \begin{itemize}

      \item \erlcode{ver/1} is such that \erlcode{ver(\(A\))} is
        rewritten into an association list containing the bindings
        index\hyp{}item lastly assigned in the array \(A\). In
        particular, values of reassigned indexes are ignored. The
        order of the bindings is not significant.

      \item \erlcode{set/3} is such that
        \erlcode{set(\(E\),\(I\),\(A\))} is an array obtained by
        assigning the value of~\(E\) to the index~\(I\) of the
        non\hyp{}empty array~\(A\). If the index is out of bounds,
        including if the array is empty, then the result is the
        atom~\erlcode{out}.

      \item \erlcode{get/2}~is such that \erlcode{get(\(A\),\(I\))} is
        the content of the cell in the array~\(A\) at index~\(I\). If
        the index is out of bound, the atom~\erlcode{out} results.

      \item \erlcode{nth/2} is such that \erlcode{nth(\(A\),\(I\))} is
        the content of the \(I\)th cell in the array~\(A\). If the
        intended index is out of bounds, the atom~\erlcode{out}
        results.

      \item \erlcode{mem/2} is such that \erlcode{mem(\(A\),\(E\))}
        rewrites to the last (temporally) index in the array~\(A\) to
        have been assigned~\(E\). If the sought value has been
        reassigned, it is not part of the last version of the array,
        thus it is not found. If not found, the atom~\erlcode{absent}
        is the final value.

      \item \erlcode{inv/1} is such that
        \erlcode{inv(\(A\))}~evaluates to an array whose cell contents
        are in reverse order with respect to array~\(A\). The history
        of all updates must be preserved, only indexes change.

      \item \erlcode{unset/2} is such that
        \erlcode{unset(\(A\),\(I\))} returns an array identical
        to~\(A\) except that the last assignment to the cell of
        index~\(I\) is undone. If the cell in question was never
        assigned, the result is~\(A\). If the index is out of bounds,
        the atom~\erlcode{out} results. Note that~\erlcode{unset} was
        an update when we presented earlier the persistent associative
        arrays.

    \end{itemize}

\end{enumerate}


