%%-*-latex-*-

\chapter{Queues}

A list can only be augmented with new items on only one of its ends,
called the \emph{top}. A \emph{queue} is like a list where items are
pushed on one end but popped \emph{at the other end}. Adding an item
to a queue is called \emph{enqueuing}, whereas removing one is called
\emph{dequeuing}. The end of the queue used for input is called the
\emph{rear} and the other end is the \emph{front}. Compare the
following figures, where the arrows~\((\rightsquigarrow)\) indicate
the way the items move:
\[
\begin{array}{@{}l@{}r@{\;}cc|c|c|c|c|c|cc@{\;}l@{}}
\cline{4-10}
\textsf{Queue:} & \text{Enqueue (rear)} 
                & \rightsquigarrow & & a & b & c & d & e &
& \rightsquigarrow & \text{Dequeue (front)}\\
\cline{4-10}\\
\cline{4-9}
\textsf{List:} & \text{Push, Pop (top)} 
               & \leftrightsquigarrow & & a & b & c & d & e &&&\\
\cline{4-9}
\end{array}
\]
\noindent Let \erlcode{enqueue/2} be a function such that the call
\erlcode{enqueue(I,Q)} is rewritten into a queue whose last item in
was~\erlcode{I} and remaining queue \erlcode{Q}. Dually, let
\erlcode{dequeue/1} be a function such that the call
\erlcode{dequeue(Q)} is rewritten into the pair \erlcode{\{R,I\}}
where \erlcode{I}~is the first item out of queue~\erlcode{Q} (if there
is none, \erlcode{dequeue(Q)} is undefined) and \erlcode{R}~is the
remaining queue.
\[
\begin{array}{c@{\;\,}c@{\,\;}cc|c|c|c|ccc|c|c|c|c|cc@{}}
\cline{4-8}
\cline{10-15}
\erlcode{enqueue} & a & \text{in} & \phantom{c} & b & c & d &
\phantom{c} 
& \text{results in} & \phantom{c} & a & b & c & d & \phantom{c}\\
\cline{4-8}
\cline{10-15}\\
\cline{4-8}
\cline{10-13}
\erlcode{dequeue} &   & \text{from} & & b & c & d & 
& \text{results in} & & b & c & \multicolumn{1}{c}{} &
\multicolumn{2}{c@{}}{\text{and} \,\; d.}\\
\cline{4-8}
\cline{10-13}
\end{array}
\]

\medskip

\paragraph{One\hyp{}list implementation.}

Let us first try to implement these functions on queues by means of
the usual functions on one list. Let us rename \erlcode{enqueue/2}
into \erlcode{enq1/2} and \erlcode{dequeue/1} into \erlcode{deq1/2} to
remind that they rely on a single list. In this case, \erlcode{enq1/2}
is simply a push operation and \erlcode{deq1(Q)} is a \emph{pair} made
of the last item in the list~\erlcode{Q} and the longest prefix
of~\erlcode{Q} excluding it (that is, all the previous items in the
same order). In other words, \erlcode{deq1/1} evaluates in a pair
whose first component is made of all the original items in the list
except the last one. The second component is the last item in
question. Consider the following:
\begin{align*}
\erlcode{enq1(a,[])}        &\twoheadrightarrow \erlcode{[a]};\\
\erlcode{enq1(a,[b,c])}     &\twoheadrightarrow \erlcode{[a,b,c]};\\
\erlcode{deq1([])}          &\nrightarrow;\\
\erlcode{deq1([a])}         &\twoheadrightarrow \erlcode{\{[],a\}};\\
\erlcode{deq1([a,b,c,d,e])} &\twoheadrightarrow \erlcode{\{[a,b,c,d],e\}}.
\end{align*}
We induce the following \Erlang definitions:
\begin{alltt}
enq1(I,Q)   -> [I|Q].\hfill% \emph{Queue} Q \emph{is a list}
deq1(  [I]) -> \{[],I\};
deq1([I|Q]) -> \{R,J\}=deq1(Q), \{[I|R],J\}.
\end{alltt}
If the queue contains \(n\)~items, then the delay \(\comp{deq1}{n}\)
is defined by the following recurrent equations:
\[
\comp{deq1}{1} = 1;\quad
\comp{deq1}{n} = 1 + \comp{deq1}{n-1}, \,\; \text{with} \,\; n > 0.
\]
This is a recurrence extremely easy to solve: \(\comp{deq1}{n} = n\),
for~\(n \geqslant 0\). So this one\hyp{}list implementation of queues
provides constant time for enqueuing and linear time for dequeuing.

\medskip

\paragraph{Two\hyp{}list implementation.}

Let us now implement a more sophisticated approach with two lists,
instead of one: one for enqueuing, called the \emph{rear list}, and
one for dequeuing, called the \emph{front list}.
\[
\begin{array}{r@{\;}cc|c|c|c|c|c|c|cc@{\;}l}
  \cline{3-6}\cline{8-10}
  \text{Enqueue (rear)} & \rightsquigarrow & & a & b & c & & d & e & &
  \rightsquigarrow & \text{Dequeue (front)}\\
  \cline{3-6}\cline{8-10}
\end{array}
\]
\noindent So enqueuing is pushing on the rear list and dequeuing is
popping on the front list. In the latter case, if the front list is
empty and the rear list is not, we swap the lists and reverse the
(new) front. Pictorially, this is best summed up as
\[
\begin{array}{lcc|c|c|c|c|cl}
  \cline{3-6}\cline{8-8}
  \text{Given} & & & a & b & c & & &
  \text{how to dequeue?}\\
  \cline{3-6}\cline{8-8}\\
  \cline{3-3}\cline{5-8}
  \text{Make} & & & & a & b & c & &
   \text{and dequeue now.}\\
  \cline{3-3}\cline{5-8}
\end{array}
\]
\noindent Let the pair \erlcode{\{In,Out\}} denote the queue where
\erlcode{In}~is the rear list and~\erlcode{Out} the front
list. Enqueuing is as easy as with one list: just push the item
on~\erlcode{In}:
\begin{verbatim}
enq2(E,{In,Out}) -> {[E|In],Out}.
\end{verbatim}
Dequeuing requires some care when the front list is empty. This
suggests an approach like this:
\begin{alltt}
deq2(\{In,     []\}) -> \fbcode{deq2(\{[],rev(In)\})};
deq2(\{In,[E|Out]\}) -> \fbcode{deq2(\{[],rev(In)\})}.
\end{alltt}
In the first clause, we must start over after moving the rear list to
the front and reversing it (or reverse first and swap next, since both
operations commute and are performed in one rewrite):
\begin{alltt}
deq2(\{In,     []\}) -> \textbf{deq2(\{[],rev(In)\})};
deq2(\{In,[E|Out]\}) -> \fbcode{deq2(\{[],rev(In)\})}.
\end{alltt}
\verbatiminput{rev.def}
In the second clause, we found the item expected to be
dequeued:~\erlcode{E}. Therefore we draw
\begin{alltt}
deq2(\{In,     []\}) -> deq2(\{[],rev(In)\});
deq2(\{In,[E|Out]\}) -> \textbf{\{\{In,Out\},E\}}.
\end{alltt}
When testing a definition, it is wise to always envisage the ``empty
case,'' whatever that means for the data structures at hand. In this
particular case, the empty queue is uniquely implemented by the
\Erlang value \erlcode{\{[],[]\}}. If we try to rewrite
\erlcode{deq2(\{[],[]\})}, we realise that the program never
terminates because the first clause always matches the empty
queue. What should have been expected for the empty queue? Simply put,
the function \erlcode{deq2/1} should not be defined on the empty
queue. Thus, let us change the pattern of the first clause in order to
exclude any match for the empty queue:
\begin{alltt}
deq2(\{\textbf{In=[\_|\_]},     []\}) -> deq2(\{[],rev(In)\});
deq2(\{      In,[E|Out]\}) -> \{\{In,Out\},E\}.
\end{alltt}
Let \(\comp{deq2}{p,q}\) be the delay for computing the call
\erlcode{deq2(\{\(P\),\(Q\)\})}, where \(p\)~and~\(q\) are,
respectively, the lengths of the rear list~\(P\) and the front
list~\(Q\). From the definition, two recurrent equations follow:
\begin{align*}
\comp{deq2}{p,0} &= 1 + \comp{rev}{p} + \comp{deq2}{0,p}
                  = p + 3 + \comp{deq2}{0,p},\,\;
                    \text{with} \,\; p > 0;\\
\comp{deq2}{p,q} &= 1 ,\,\; \text{with} \,\; q > 0.
\end{align*}
The best case for dequeuing happens when the front list is not empty,
that is, \({q > 0}\): the delay is then \(\best{deq2}{n} = 1\), where
\({n>0}\) is the total number of items in the queue. The worst case
consists in having the front list empty, that is,~\(q=0\). Then the
delay is \(\worst{deq2}{n} = n + 4\).

\medskip

\paragraph{Comparison.}

We have \(\best{deq2}{n} \leqslant \comp{deq1}{n} < \worst{deq2}{n}\),
therefore it is not possible to conclude for which values of \(n\)
\erlcode{deq1/1} is faster than \erlcode{deq2/1}. A finer analysis is
needed. Because the worst delay of \erlcode{deq/2} depends on the
number of items in the rear list, that is, on the number of previous
enqueuings, it makes sense to study the delay of \emph{a sequence of
  operations} (here, enqueuings and dequeuings) on a two\hyp{}list
queue, \emph{in the worst case}. Up to now, we studied worst case
scenarios of \emph{one} function, taken in isolation, and we just
found that this is sometimes insufficient. In general, it is indeed
possible that the worst case for one specific operation leads to a
good case for other subsequent operations; therefore, \emph{adding the
  worst case delays of all operations taken in isolation may be much
  more than the worst delay of the series considered as a whole.} This
kind of analysis is called \emph{amortised} because it takes into
account the fact that, in a series of operations on a data structure,
some configurations lead to slow operations but the treatment of these
create good configurations for others, so the high delay is amortised
on a long run. Note that amortised analysis is not a statistic or
probabilistic argument, as the randomness of a configuration has
nothing to do with the phenomenon targeted by amortised analysis: we
still consider the worst case, but of a sequence of operations, not a
single one.

To contrast these two analyses, let us study the delay
\(\mathcal{D}_n\) of a series of \(n\)~operations on a queue,
consisting of enqueuings and dequeuings implemented, respectively, by
\erlcode{enq2/2} and \erlcode{deq2/1}. First, let us consider the
worst case of a single operation. We already know that the worst case
is when the operation is a dequeuing and the front list is empty:
\(\worst{deq2}{i} = i + 4\), where \(i\)~is the number of items in the
rear list. This means that
\[
\mathcal{D}_n \leqslant \sum_{i=1}^{n-1}{\worst{deq2}{i}} =
\frac{1}{2}{(n-1)(n+8)} \mathrel{\sim} \frac{1}{2}{n^2}.
\]
There exists a real number~\(\alpha\) and an integer~\(n_0\) such
that, for all \({n > n_0}\), \(\abs{(n-1)(n+8)/2} \leqslant \alpha
n^2\). This is usually written concisely in the notation introduced by
the mathematician Paul Bachmann\label{Bachmann} as \(\mathcal{D}_n \in
\bachmann{n^2}\). That is, the delay of \(n\)~operations in the worst
case is bounded from above by a quadratic polynomial as \(n\)~gets
large. Contrast this notation with Landau's notation in
equation~\vref{eq:Landau}, which is the set of functions
\(\landau{g(n)}\) such that \(f \in \landau{g(n)} \Leftrightarrow
\lim_{n \rightarrow \infty}{f(n)/g(n)} = 0\).

Actually, the worst case above is too pessimistic and cannot even
exist. Indeed, two important constraints were left aside. First, one
cannot dequeue on an empty queue. As a consequence, at any time, the
number of enqueuings since the beginning is always greater or equal
than the number of dequeuings. In particular, the series of operations
must start with at least one enqueuing. Second, when dequeuing on a
queue whose front list is empty, all the items of the rear list are
reversed into the front list, so they cannot be reversed again during
the next dequeuing, whose delay will be~\(1\). Moreover,
\(\comp{enq2}{} \leqslant \comp{deq2}{n}\), so the first consequence
is that the worst case for a series of \(n\)~operations is when the
number of dequeuings is maximum, that is, when it is
\(\lfloor{n/2}\rfloor\). Thus, let us distinguish an even number of
operations and an odd number. If we denote \(e\)~and~\(d\) the number
of enqueuings and, respectively, the number of dequeuings, we have \(n
= e + d\) and the two requisites for a worst case become \(e=d\) or
\(e=d+1\).

\medskip

\paragraph{Dyck path (\(\boldsymbol{e=d}\)).}

This refinement still does not take into account all the information
we have, for instance, we do not use the fact that \emph{at any time}
the past number of enqueuings is greater or equal than the past number
of dequeuings (we only assumed this to hold after
\(n\)~operations). This constraint can be visualised by representing
an enqueuing by an opening parenthesis and a dequeuing by a closing
parenthesis, so, for example, a valid series of operations when
\(e=d=7\) could be \erlcode{((()()(()))())}. All the prefixes of this
string of parentheses are
\begin{center}
\tt
(\quad
((\quad
(((\quad
((()\quad
((()(\quad
((()()\quad
((()()(\quad
((()()((\quad
((()()(()\\
((()()(())\quad
((()()(()))\quad
((()()(()))(\quad
((()()(()))()\quad
((()()(()))())
\end{center}
We can check that for each of them the number of closing parentheses
never exceeds the number of opening parentheses. Another way to
visualise this example is graphically, using a similar convention to
the one found in section \ref{chap:persistence_and_backtracking}
\vpageref{chap:persistence_and_backtracking}. See 
\fig~\vref{fig:enq_deq}.
\begin{figure}
\centering
\subfloat[Enqueue\label{fig:enqueue}]{%
  \includegraphics[scale=0.75]{enqueue}
}
\qquad
\subfloat[Dequeue\label{fig:dequeue}]{%
  \includegraphics[scale=0.75]{dequeue}
}
\caption{Graphical representations of operations on queues\label{fig:enq_deq}}
\end{figure}
Then, \erlcode{((()()(()))())} can be represented in
\fig~\vref{fig:dyck_path} as a \emph{Dyck path}, named in the honour
of the logician Walther (von)~Dyck.
\begin{figure}[t]
\centering
\includegraphics[bb=75 572 505 725,scale=0.71]{dyck_path}
\caption{Dyck path modelling queue operations (delay \(30\))
\label{fig:dyck_path}}
\end{figure}
For a broken line to qualify as a Dyck path of \emph{length}~\(n\), it
has to start at the origin~\((0,0)\) and end at coordinates~\((n,0)\),
that is, on the abscissa axis. In terms of a \emph{Dyck language}, an
enqueuing is called a \emph{rise} (see \fig~\vref{fig:enqueue}) and a
dequeuing is called a \emph{fall}. A rise followed by a fall is called
a \emph{peak}. The ordinate of a point on a path is called its
\emph{height}. (The continued analogy is that of a mountain range.)
For instance, in \fig~\vref{fig:dyck_path}, there are four peaks. The
number near each rise or fall is the delay incurred by the
corresponding operation on the queue, that is, either an enqueuing or
a dequeuing. The height is interpreted as the number of items in the
queue at a given time. Time is the meaning of the abscissas.

When the number of enqueuings equals the number of dequeuings, that
is, \({e=d}\), the graphical representation of the operations is a
Dyck path of length \(n=2e=2d\). In order to deduce the total delay in
this case, we must find a decomposition of the path that is
workable. By decomposition we either mean to identify patterns whose
delays are easy to compute and which make up any Dyck path, or to
associate any path to another path whose delay is the same but easy to
find. Actually, we are going to do both. \Fig~\vref{fig:dyck_eq1}
\begin{figure}[b]
\centering
\includegraphics[bb=75 572 505 725,scale=0.71]{dyck_eq1}
\caption{Dyck path equivalent to \fig~\vref{fig:dyck_path}
\label{fig:dyck_eq1}}
\end{figure}
shows how the previous path can be mapped to an equivalent path only
made of a series of isosceles triangles whose bases belong to the
abscissa axis. Let us call them \emph{mountains} and their series a
\emph{range}. The algorithm for the mapping is simple: after the first
fall, if we are back to the abscissa axis, we have a mountain and we
start again with the rest of the path. Otherwise, the next operation
is a rise and we exchange it with the first fall after it. This brings
us down by~\(1\) and we can repeat the procedure until the bottom line
is reached. We call this process \emph{rescheduling} because it
amounts, in operational terms, to reordering subsequences of
operations a posteriori. For instance, see in
\fig~\vref{fig:rescheduling} the rescheduling of
\fig~\vref{fig:dyck_path}.
\begin{figure}
\centering
\subfloat[Initial\label{fig:initial}]{%
  \includegraphics[bb=56 590 218 724,scale=0.75]{mountain0}
}
\qquad
\subfloat[Swapping \(4\nearrow 5\) and \(5\searrow 6\)]{%
  \includegraphics[bb=56 590 218 724,scale=0.75]{mountain1}
}
\qquad
\subfloat[Swapping \(5\nearrow 6\) and \(8 \searrow
  9\)\label{fig:valley}]{%
  \includegraphics[bb=56 590 218 725,scale=0.75]{mountain4}
}
\qquad
\subfloat[Last one]{%
  \includegraphics[bb=18 646 180 782,scale=0.75]{mountain5}
}
\caption{Rescheduling of \fig~\vref{fig:dyck_path}\label{fig:rescheduling}}
\end{figure}
Rescheduling is not an \emph{injection}, that is, two different Dyck
paths can be rescheduled to the same path. What makes
\fig~\vref{fig:valley}, called a \emph{low valley}, equivalent to
\fig~\vref{fig:initial} is that the delay is invariant because all
operations have individual delays of~\(1\). This is always the case
because, on the one hand, enqueuings always have delay~\(1\) and, on
the other hand, the dequeuings involved in a rescheduling have
delay~\(1\) because they found the front list non\hyp{}empty after a
peak. We proved that all series of queue operations corresponding to a
Dyck path are equivalent to a range, whose associated queue operations
have the same delay as the original one. Therefore, the worst case can
be found on ranges alone and it happens that their delays are easy to
compute. Let us note \(e_1, e_2, \dots, e_k\) the series of rises; for
example, in \fig~\vref{fig:dyck_eq1}, we have \({e_1=3}\), \({e_2=3}\)
and \({e_3=1}\), meaning: ``Three rises from the bottom line, then
again and finally one rise.'' Of course, \(e=e_1 + e_2 + \dots +
e_k\). Then the fall making up the \(i\)th~peak incurs the delay
\(e_i+4\) due to the front list being empty because we started the
rise from the abscissa axis. The next \(e_i-1\) falls have all delay
\(1\), because the front list is not empty. For the \(i\)th~mountain,
the delay is thus \(e_i+(e_i+4)+(e_i-1) = 3e_i+3\). In total, the
delay \(\comp{}{e,k}\) is thus
\[
\comp{}{e,k} = \sum_{i=1}^{k}{(3e_i+3)} = 3(e+k).
\]
We can check the example in \fig~\vref{fig:dyck_eq1}, where \({e=7}\)
and \({k=3}\) and find \(\comp{}{e,k}=3\cdot(7+3)=30\), which is
correct. The worst case is obtained by maximising \(\comp{}{e,k}\) for
a given~\(e\), because~\(k\), not being a parameter of the problem, is
free to vary. We have
\[
\max_{1 \leqslant k \leqslant e}{\comp{}{e,k}}
  = \comp{}{e,e}
  = 3(e+e) = 3n = \worst{}{e,e},\,\; \text{with} \,\; n=2e,
\]
where \(\worst{}{e,e}\) is the worst delay when there are \(e\)
enqueuings and \({d=e}\) dequeuings. In other words, the worst case
when \(e=d=7\) is the saw\hyp{}toothed Dyck path shown in
\fig~\vref{fig:worst_dyck}.
\begin{figure}[t]
\centering
\includegraphics[bb=86 657 503 724,scale=0.7]{worst_dyck}
\caption{Worst case when \(e=d=7\) (delay \(42\))
\label{fig:worst_dyck}}
\end{figure}
Importantly, there are no other Dyck paths whose rescheduling lead to
this worst case and the reason is that the reverse transformation from
ranges to general Dyck paths works on dequeuings of delay~\(1\) and
the solution we found is the only one with no dequeuing equal
to~\(1\).

\medskip

\paragraph{Dyck meander (\(\boldsymbol{e=d+1}\)).}

The other possibility for a worst case is that \(e=d+1\) and the
graphical representation is then a \emph{Dyck meander} whose extremity
ends at ordinate \(e-d=1\). An example is given in
\fig~\vref{fig:dyck_meander1}, where the last operation is a
dequeuing.
\begin{figure}[t]
\centering
\includegraphics[bb=76 572 475 725,scale=0.76]{dyck_meander1}
\caption{Dyck meander modelling queue operations (total delay
  \(25\))\label{fig:dyck_meander1}}
\end{figure}
The dotted line delineates the result of applying the rescheduling we
used on Dyck paths. Here, the last operation becomes an
enqueuing. Another possibility is shown in
\fig~\vref{fig:dyck_meander2},
\begin{figure}[b]
\centering
\includegraphics[bb=76 572 475 725,scale=0.76]{dyck_meander2}
\caption{Dyck meander modelling queue operations (total delay
  \(29\))\label{fig:dyck_meander2}}
\end{figure}
where the last operation is left unchanged. The difference between the
two examples lies in the fact that the original last dequeuing has, in
the former case, a delay of~\(1\) (thus is changed) and, in the latter
case, a delay greater than~\(1\) (thus is invariant). The range
resulting from the rescheduling leads to peaks whose falls have a
delay greater than~\(1\). The third kind of Dyck meander to consider
is one ending with an enqueuing, but because this enqueuing must start
from the abscissa axis, this is the same situation as the result of
rescheduling a meander ending with a dequeuing with delay~\(1\) (see
dotted line in \fig~\ref{fig:dyck_meander1} again). Therefore, we are
left to compare the results of rescheduling meanders ending with a
dequeuing, that is, we have two cases: either a range of \({n-1}\)
operations (\({e-1}\) enqueuings and \(d=e-1\) dequeuings) followed by
an enqueuing or a range of \({n-3}\) operations (\({e-2}\) enqueuings
and \(d-1=e-2\) dequeuings) followed by two rises and one fall
(totalling a delay of~\(8\)). In the former case, the worst case of
the range is a saw\hyp{}toothed Dyck path of delay
\(\comp{}{e-1,e-1}=3((e-1)\cdot 2)=3(n-1)\), because
\(n=e+d=2e-1\). The worst delay of the whole meander is then simply
\(3(n-1)+1 = 3n-2\) and there are \(k=e-1\) peaks. In the latter case,
the worst case of the range is a saw\hyp{}toothed Dyck path of delay
\(\comp{}{e-2,e-2}=3(n-3)\), so the worst delay of the whole meander
is \(3(n-3)+8=3n-1\). This delay is slightly worse than the previous
one, so we can conclude
\[
\worst{}{e,e-1} = 3n-1,\,\; \text{with} \,\; n = 2e-1.
\]

\smallskip

\paragraph{Conclusion.}

As a conclusion, the delay \(\mathcal{D}_n\) of a series of
\(n\)~queue operations, that is, enqueuings and dequeuings starting on
an empty queue, is bounded as follows:
\[
 n \leqslant \mathcal{D}_n \leqslant 3n.
\]
The lower bound corresponds to the case where all the operations are
enqueuings, the upper bound occurs when there are \(e\)~enqueuings and
\(e\)~dequeuings, with \({n=2e}\), in a saw\hyp{}toothed
pattern. Because these bounds are tight, that is, they correspond to
actual configurations of the input, it would be less precise to state
that \(\mathcal{D}_n \in \bachmann{n}\), where the multiplicative
constant is hidden and unknown. We proved that, despite some
dequeuings having a delay linear in the size of the queue in the worst
case, \emph{a series of enqueuings and dequeuings has a linear delay
  in terms of the number of operations in the worst case,} because
individual worst cases lead to favourable dequeuings later. If we
average the delay over the number of operations, we obtain the
\emph{amortised delay} of one operation, \(\mathcal{D}_n/n\), which
lies between \(1\)~and~\(3\).

\medskip

\paragraph{Exercises.}
\label{ex:queues}

\noindent [See answers page~\pageref{ans:queues}.]
\begin{enumerate}

  \item Write a definition for a function \erlcode{head/1} such that
    the call \erlcode{head(Q)}, where \erlcode{Q}~is a queue
    implemented with two lists, is rewritten into the first item to be
    dequeued. For example,
    \[
     \erlcode{head(\{[4,3],[1,2]\})} \twoheadrightarrow \erlcode{1}.
    \]
    Make sure that the delay of \erlcode{head/1} is
    constant. (\emph{Hint.} Revise \erlcode{enq2/2} and
    \erlcode{deq2/1} so that they enforce the invariant property that
    the front list is empty only if the rear list is empty. In other
    words, suppose that this is true when the heads match and make
    sure it is still true after the body is computed.)

  \item A \emph{double\hyp{}ended queue} is a queue where enqueuing
    and dequeuing are allowed on both ends. Use two lists for the
    implementation and write enqueuing and dequeuing on both ends
    efficiently. (\emph{Hint.} Extend the invariant of the previous
    exercise so that none of the two lists is empty if the
    double\hyp{}ended queue contains at least two items and when a
    list becomes empty, split the other list in half and reverse one
    of the halves.)

\end{enumerate}

