%%-*-latex-*-

\chapter{Merge Sort}

\paragraph{Merging.}

Let us recall one\hyp{}way insertion sort, as seen
\vpageref{code:insert}:
\begin{verbatim}
insert(I,[J|S]) when I > J -> [J|insert(I,S)];
insert(I,    S)            -> [I|S].
isort(   [])               -> [];
isort([I|L])               -> insert(I,isort(L)).
\end{verbatim}
The purpose of \erlcode{insert/2} is to insert one item into a list of
items ordered nondecreasingly. Sorting by insertion consists in
inserting items to be sorted one by one into an already sorted list. A
generalisation is concerned with inserting a series of items
\emph{already sorted} in nondecreasing order. This basic operation is
called \emph{merging} and it is more efficient than inserting items
one at a time, because we don't need to move backwards to
insert. Consider the following example, where two increasingly sorted
lists, \erlcode{[22,47,50]} and \erlcode{[16,36,59,70]} are merged
step by step into one increasingly sorted list which appears at the
right. The numbers being compared are the heads of the lists and they
are set in a bold typeface:
\begin{equation*}
\left.
\begin{aligned}
&\erlcode{[\textbf{22},47,50]}\\
&\erlcode{[\textbf{16},36,59,70]}
\end{aligned}
\right\}
\;\erlcode{[]}
\quad \text{then} \quad
\left.
\begin{aligned}
&\erlcode{[\textbf{22},47,50]}\\
&\erlcode{[\textbf{36},59,70]}
\end{aligned}
\right\}
\;\erlcode{[16]}
\end{equation*}
\begin{equation*}
\left.
\begin{aligned}
&\erlcode{[\textbf{47},50]}\\
&\erlcode{[\textbf{36},59,70]}
\end{aligned}
\right\}
\;\erlcode{[16,22]}
\quad \text{then} \quad
\left.
\begin{aligned}
&\erlcode{[\textbf{47},50]}\\
&\erlcode{[\textbf{59},70]}
\end{aligned}
\right\}
\;\erlcode{[16,22,36]}
\end{equation*}
\begin{equation*}
\left.
\begin{aligned}
&\erlcode{[\textbf{50}]}\\
&\erlcode{[\textbf{59},70]}
\end{aligned}
\right\}
\;\erlcode{[16,22,36,47]}
\quad \text{then} \quad
\left.
\begin{aligned}
&\erlcode{[]}\\
&\erlcode{[59,70]}
\end{aligned}
\right\}
\;\erlcode{[16,22,36,47,50]}.
\end{equation*}
When a list is empty, we can append it to the current result and
obtain \erlcode{[16,22,36,47,50,59,70]}. Otherwise, we compare the
heads of both lists, select the smallest as being the head of the
result from now on and recur to the same process without that
number. Here is the \Erlang code to implement this algorithm:

\label{code:merge}
\begin{alltt}
merge(   [],      Q)            \(\smashedrightarrow{\alpha}\) Q;
merge(    P,     [])            \(\smashedrightarrow{\beta}\) P;
merge([I|P],Q=[J|_]) when I < J \(\smashedrightarrow{\gamma}\) [I|merge(P,Q)];
merge(    P,  [J|Q])            \(\smashedrightarrow{\delta}\) [J|merge(P,Q)].
\end{alltt}


\noindent Notice that we do not actually need to join lists, as the
control context of the recursive calls suffice for obtaining the
desired order. The difference with \erlcode{insert/2} above is that we
don't stop after the first item is inserted, but keep inserting
forward until one list or the other is empty, because the remaining
list is already sorted. Let us illustrate this method on a short
example:
\begin{alltt}
\erlcode{merge([3,4,7],[1,2,5,6])} \(\smashedrightarrow{\delta}\) \erlcode{[1|merge([3,4,7],[2,5,6])]}
                         \(\smashedrightarrow{\delta}\) \erlcode{[1,2|merge([3,4,7],[5,6])]}
                         \(\smashedrightarrow{\gamma}\) \erlcode{[1,2,3|merge([4,7],[5,6])]}
                         \(\smashedrightarrow{\gamma}\) \erlcode{[1,2,3,4|merge([7],[5,6])]}
                         \(\smashedrightarrow{\delta}\) \erlcode{[1,2,3,4,5|merge([7],[6])]}
                         \(\smashedrightarrow{\delta}\) \erlcode{[1,2,3,4,5,6|merge([7],[])]}
                         \(\smashedrightarrow{\beta}\) \erlcode{[1,2,3,4,5,6,7]}\textrm{.}
\end{alltt}
As can be surmised now, merging can efficiently be used as a basic
operation for sorting, instead of insertion. Before we proceed to see
how, note that we could have defined \erlcode{merge/2} as follows:
\begin{verbatim}
merge(    P,     [])            -> P;
merge([I|P],Q=[J|_]) when I < J -> [I|merge(P,Q)];
merge(    P,  [J|Q])            -> [J|merge(P,Q)].
\end{verbatim}
This is the previous definition without its first clause
\clause{\alpha}. This shorter version is slower when the first list is
shorter than the other: the second list will have all its items
examined, although it is unnecessary since they are the only remaining
ones. It is faster to stop as soon as the first list becomes empty. We
recall that, \vpageref{code:join}, we defined \erlcode{join/2} as
\begin{verbatim}
join(   [],Q) -> Q;
join([E|P],Q) -> [E|join(P,Q)].
\end{verbatim}
instead of
\begin{alltt}
\textbf{join(    P,[]) -> P;}
join(   [], Q) -> Q;
join([E|P], Q) -> [E|join(P,Q)].
\end{alltt}
even if the later is faster when the second list is empty from the
beginning. The reason was that this simplification, despite slowing
down the program in this special case, made the expression of the
delay dependent only upon the length of the first list. This avoided
making two cases and allowed us to simply define \(\comp{join}{n}\)
instead of \(\comp{join}{n,m}\). The difference with \erlcode{merge/2}
is that \erlcode{join/2} is not symmetric in general, that is,
\erlcode{join(\(P\),\(Q\))} \(\not\equiv\)
\erlcode{join(\(Q\),\(P\))}, when both \(P\) and \(Q\) are not empty,
whilst \erlcode{merge(\(P\),\(Q\))} \(\equiv\)
\erlcode{merge(\(Q\),\(P\))} always holds. The symmetry of
\erlcode{merge/2} means that we should not arbitrarily distinguish one
list with respect to the other, so the delay of
\erlcode{merge(\(P\),\(Q\))} is unconditionally the same as the delay
of \erlcode{merge(\(Q\),\(P\))}, that is, \(\comp{merge}{m,n} =
\comp{merge}{n,m}\). This results in easier delay calculations and
faster computations.

\begin{figure}[H]
\centering
\includegraphics[bb=75 702 289 723]{merged}
\caption{Two ordered lists merged into one\label{fig:merged}}
\end{figure}

%\medskip

\paragraph{Best delay.}

Let \(\best{merge}{m,n}\) be the delay in the best case of
\erlcode{merge(\(P\),\(Q\))}, where \(P\) and \(Q\) contain
respectively \(m\) and \(n\) items. The goal here is to minimise the
use of clauses \clause{\gamma} and \clause{\delta}, that is, the
number of comparisons. Graphically, we represent an item from \(P\) as
a \emph{white node} \((\circ)\) and an item from \(Q\) as a
\emph{black node} \((\bullet)\). Nodes of these kinds are printed in
an horizontal line, the leftmost node being the smallest. Comparisons
are always performed between black and white nodes and are represented
as \emph{edges} in \fig~\vref{fig:merged}. An incoming arrow means
that the node is smaller than the other end of the edge, so all edges
point leftward and the number of comparisons is the number of nodes
with an incoming edge. This abstract representation suggests that the
more items from one list we have at the end of the result, the less
comparisons we needed for merging (there are two consecutive white
nodes at the right without any edges). Indeed, \emph{the minimum delay
  is achieved when the shortest list comes first in the result, that is,
  it is a \emph{prefix.}} See \fig~\vref{fig:best_merge}
\begin{figure}[!b]
\centering
\includegraphics[bb=74 698 290 722]{best_merge}
\caption{Best\hyp{}case merging: \(\min\{m,n\}\)
  comparisons\label{fig:best_merge}}
\end{figure}
for the best case with \(m=9\) and \(n=4\), which are the same lengths
as in \fig~\vref{fig:merged}.  If both lists have the same length, any
of them can be considered the shortest and the previous statement
applies as well.\label{merge:best_case}
\begin{equation}
\best{merge}{m,n} = 1 + \min\{m,n\}.\label{eq:best_merge}
\end{equation}
We can check that \(\best{merge}{m,n} = \best{merge}{n,m}\) and
\(\best{merge}{0,n} = \best{merge}{m,0} = 1\), as
expected.

\medskip

\paragraph{Worst delay.}

Let \(\worst{merge}{m,n}\) be the delay of
\erlcode{merge(\(P\),\(Q\))} in the worst case, where \(P\) and \(Q\)
contain respectively \(m\) and \(n\) items. Clearly, this delay is
one, by clause \clause{\alpha} or \clause{\beta}, plus the number of
times clauses \clause{\gamma} and \clause{\delta} are used, which is
the number of comparisons needed. For the purpose of illustration, let
us consider an example where \(m=9\) and \(n=4\). A possible merging
of \(P\) and \(Q\) is shown in \fig~\vref{fig:merged}. We can see that
we can increase the number of comparisons with respect to \(m+n\) by
removing those nodes from \(Q\) which find their place at the end of
the result \emph{without being compared}. This is shown in
\fig~\vref{fig:worst_merged2}.
\begin{figure}[t]
\centering
\includegraphics[bb=74 702 255 723]{worst_merged2}
\caption{A worst\hyp{}case merging in \(m+n-1\)
  comparisons\label{fig:worst_merged2}}
\end{figure}
After this shortening, we have \(m+n=11\), that is, down by \(2\), and
the number of comparisons is unchanged (\(m+n-1=10\)). This is the
maximum number because it corresponds to a situation where all the
nodes, except the rightmost, are the destination of an edge. (The last
node cannot be pointed at because edges must go from right to left.)
Actually, it is possible to interchange the last two nodes as well,
because this removes one out\hyp{}going edge from the white node but
adds one to the black, so the number of comparisons is still
\(m+n-1\). This can be visualised in \fig~\vref{fig:worst_merged1}.
\emph{The worst case of merging \(P\) and \(Q\) is when the last item
  of one list becomes the last item of the result and the last item of
  the other list becomes the penultimate item of the result.} The
total number of comparisons is then \(m+n-1\), where \(m\) and \(n\)
are, respectively, the number of items in \(P\) and \(Q\). When one
list is empty, the delay is \(1\). Therefore, when \(m > 0\) and \(n >
0\),
\begin{figure}[b]
\centering
\includegraphics[bb=74 700 255 725]{worst_merged1}
\caption{The two rightmost nodes in \fig~\vref{fig:worst_merged2} have
  been swapped\label{fig:worst_merged1}}
\end{figure}
\begin{equation}
\worst{merge}{0,n} = \worst{merge}{m,0} = 1;\quad
\worst{merge}{m,n} = 1 + (m + n - 1) = m + n.\label{eq:worst_merge}
\end{equation}
It is important to realise that the above reckoning is unchanged if we
swap \(P\) and \(Q\), thus \(m\) and \(n\); formally, we expect
\(\worst{merge}{m,n} = \worst{merge}{n,m}\).

\medskip

\paragraph{Average delay.}

One question remains to be answered: Is the average delay closer to
the lower or the upper bound of the delay? Let us consider a smaller
example in \fig~\vref{fig:average_merged}, 
\begin{figure}
\centering
\includegraphics[bb=67 649 350 723]{average_merged}
\caption{All possible merges with \(m=3\) (\(\circ\)) and \(n=2\)
  (\(\bullet\))
\label{fig:average_merged}}
\end{figure}
with \(m=3\) white nodes and \(n=2\) black nodes which are interleaved
in all possible manners.  Note how the figure was structured. The
first column lists the configurations where the rightmost black node
is the last of the result. The second column lists the cases where the
rightmost black node is the penultimate node of the result. The third
column is divided in two groups itself, the first of which lists the
cases where the rightmost black node is the antepenultimate. The total
number of comparisons is \(35\) and the number of configurations is
\(10\), thus the average number of comparisons is \(35/10 =
7/2\).\label{seven_two} Let us devise a method to find this ratio for
any \(m\) and \(n\). First, the number of configurations: how many
ways are there to combine \(m\) white nodes and \(n\) black nodes?
This is the same as asking how many ways there are to paint in black
\(n\) nodes picked amongst \(m+n\) white nodes. More abstractly, this
is equivalent to wonder how many ways there are to choose \(n\)
objects amongst \(m+n\). This number is usually a \emph{combination}
and noted \(\binom{m+n}{n}\). For example, let us consider the set
\(\{a,b,c,d,e\}\) and the combinations of \(3\) objects taken from it
are
\begin{gather*}
\{a,b,c\},\{a,b,d\},\{a,b,e\},\{a,c,d\},\{a,c,e\},\{a,d,e\},\\
\{b,c,d\},\{b,c,e\},\{b,d,e\},\\
\{c,d,e\}.
\end{gather*}
This enumeration establishes that \(\binom{5}{3} = 10\). (Notice that
we use mathematical sets, therefore the order of the elements or their
repetition are not meaningful.) It is not difficult to count the
combinations if we recall how we counted the permutations,
\vpageref{permutations}. Let us determine \(\binom{r}{k}\). We can
pick the first object amongst \(r\), the second amongst \(r-1\)
etc. until we pick the \(r\)th object amongst \(r-k+1\), so there are
\(r(r-1)\dots(r-k+1)\) choices. But these arrangements contain
duplicates, for example, we may form \(\{a,b,c\}\) and \(\{b,a,c\}\),
which are to be considered identical combinations. Therefore, we must
divide the number we just obtained by the number of redundant
arrangements, which is the number of permutations of \(k\) objects,
\(k!\). In the end:
\[
\binom{r}{k} := \frac{r(r-1)\ldots(r-k+1)}{k!} =
\frac{r!}{k!(r-k)!}.
\]
We can check now that in \fig~\vref{fig:average_merged}, we must have
\(10\) cases: \(\binom{5}{2} = 5!/(2!3!) = 10\). The symmetry of the
problem means that merging a list of \(m\) items with a list of \(n\)
leads to exactly the same results as merging a list of \(n\) items
with a list of \(m\) items.
\[
\binom{m+n}{n} = \binom{m+n}{m}.
\]
This can be easily proved by means of the definition:
\[
\binom{m+n}{n} := \frac{(m+n)!}{n!(m+n-n)!} =
\frac{(m+n)!}{m!n!} := \binom{m+n}{m},\quad
\text{\textsc{qed.}}
\]
Going forth, let us determine the total number \(\mathcal{L}(m,n)\) of
comparisons needed to merge \(m\) and \(n\) items in all possible
manners. This is the number of times clauses \clause{\gamma} and
\clause{\delta} are used, so we have to count the nodes which are the
destination of an edge in our graph representation. In
\emph{enumerative combinatorics}, that is, the art of counting
configurations, it is often worthy to consider the complementary
set. In this case, it may be easier to count the nodes without
incoming edges, that is, the number of times clauses \clause{\alpha} and
\clause{\beta} are used. These nodes are circled, so they are easily
distinguished, in \fig~\vref{fig:average_merge_circled}.
\begin{figure}[b]
\centering
\includegraphics[bb=68 650 354 723]{average_merge_circled}
\caption{All merges with nodes handled by clauses \clause{\alpha} and
  \clause{\beta} circled\label{fig:average_merge_circled}}
\end{figure}
We can check for each merge that their number is always \(m+n\) minus
the number of nodes with incoming edges. This is easy to justify
because there are \(m+n\) nodes and each is either processed by
clauses \clause{\alpha}\hyp{}\clause{\beta} or else clauses
\clause{\gamma}\hyp{}\clause{\delta}. Summing over all the
\(\binom{m+n}{n}\) merges, we draw that the total number of circled
nodes is
\[
(m + n) \binom{m+n}{n} - \mathcal{L}(m,n).
\]
It is simple to characterise these circled nodes: they make up the
longest rightmost contiguous series of nodes of the same colour. Since
there are only two colours, the problem of determining the total
number \(W(m,n)\) of white circled nodes is symmetric to the
determination of the total number \(B(m,n)\) of black circled nodes:
\(W(m,n) = B(n,m)\). Bear in mind that, in our running example, \(m\)
is the number of white nodes and \(n\) the number of black nodes. We
expect then
\[
(m + n) \binom{m+n}{n} - \mathcal{L}(m,n) = W(m,n) + B(m,n),
\]
hence,\label{average_delay_merge} dividing by \(\binom{m+n}{n}\) and
reordering the terms yields
\[
\biggl[\mathcal{L}(m,n)\biggr]\left/\binom{m+n}{n}\right.
= m + n - \biggl[W(m,n) + B(m,n)\biggr]\left/\binom{m+n}{n}\right..
\]
The left\hyp{}hand side is the average number of comparisons we are
looking for in the end. We can decompose \(W(m,n)\) by counting the
circled white nodes vertically, which is another common technique used
in combinatorics. In \fig~\vref{fig:average_merge_circled}, \(W(3,2)\)
is the sum of the numbers of merges with
\begin{itemize}

  \item one ending circled white node (\(6\)),

  \item two ending circled white nodes (\(3\)),

  \item three ending circled white nodes (\(1\)).
\end{itemize}
Indeed, the examination of the two last columns in
\fig~\vref{fig:average_merge_circled} brings \(W(3,2) = 6 + 3 + 1 =
10\) and the first column \(B(3,2) = 4 + 1 = 5\). How can we
generalise to arbitrary \(m\) and \(n\)? The number of merges with one
ending circled white node is the answer to the question: ``How many
ways are there to combine \(n\) black nodes with \(m-1\) white
nodes?'' We know, by definition of combinations, this to be
\(\binom{n+(m-1)}{n}\). Similarly, the number of merges with two
ending circled white nodes is the answer to the question: ``How many
ways are there to combine \(n\) black nodes with \(m-2\) white nodes?''
This is \(\binom{n+(m-2)}{n}\). So on until
\(\binom{n+(m-m)}{n}\):
\begin{align}
W(m,n) %&= \sum_{i=1}^{m}{\binom{n+m-i}{n}}\\
  &= \binom{n+m-1}{n} + \binom{n+m-2}{n} + \dots + \binom{n+0}{n}\notag\\
  &= \binom{n+0}{n} + \dots + \binom{n+m-2}{n} + \binom{n+m-1}{n}\notag,\\
W(m,n)
  &= \sum_{j=0}^{m-1}{\binom{n+j}{n}}.\label{eq:W}
\end{align}
This sum can actually be simplified, more precisely, it has a closed
form, but in order to understand its underpinnings, we need firstly to
develop our intuition about combinations. By computing combinations
\(\binom{r}{k}\) for small values of \(r\) and \(k\) using the
definition, we can fill a table traditionally known as \emph{Pascal's
  triangle} and displayed in \fig~\vref{fig:pascal_triangle}.
\begin{figure}
\centering
\includegraphics{pascal}
\caption{The corner of Pascal's Triangle (in bold)\label{fig:pascal_triangle}}
\end{figure}
Note how we set the convention \(\binom{r}{k} = 0\) if \(k >
r\). Pascal's Triangle contains many interesting properties relative
to the sum of some of its values. For instance, if we choose a number
in the triangle and look at the one on its right, then the one below
the latter is their sum. For the sake of illustration, let us extract
from \fig~\vref{fig:pascal_triangle} the lines \(r=7\) and \(r=9\):
\begin{center}
\begin{tabular}{||c|c||rrrrrrrrrr||}
\cline{5-6}\cline{8-9}
      & 7 & \textbf{1} & \textbf{7} & \multicolumn{1}{|c}{\textbf{21}} & \multicolumn{1}{c|}{\textbf{35}} &  \textbf{35} &  \multicolumn{1}{|c}{\textbf{21}} & \multicolumn{1}{c|}{\textbf{7}} &  \textbf{1} & 0 & 0\\
\cline{5-5}\cline{8-8}
      & 8 & \textbf{1} & \textbf{8} & \textbf{28} & \multicolumn{1}{|c|}{\textbf{56}} &  \textbf{70} &  \textbf{56} & \multicolumn{1}{|c|}{\textbf{28}} &  \textbf{8} & \textbf{1} & 0\\
\cline{6-6}\cline{9-9}
\end{tabular}
\end{center}
We surrounded two examples of the additive property of combinations we
discussed: \(21 + 35 = 56\) and \(21 + 7 = 28\). We would then bet
that
\[
\binom{r-1}{k-1} + \binom{r-1}{k} = \binom{r}{k}.
\]
This is actually not difficult to prove if we go back to the
definition:
\begin{align*}
\binom{r}{k} &:= \frac{r!}{k!(r-k)!}
              = \frac{r}{k} \cdot \frac{(r-1)!}{(k-1)!((r-1)-(k-1))!}
              = \frac{r}{k} \binom{r-1}{k-1}.\\
\binom{r}{k} &:= \frac{r!}{k!(r-k)!}
              = \frac{r}{r-k} \cdot \frac{(r-1)!}{k!((r - 1) - k)!}
              = \frac{r}{r-k} \binom{r-1}{k}.
\end{align*}
The first equality is valid if \(k > 0\) and the second if \(r \neq
k\). We can now replace \(\binom{r-1}{k-1}\) and \(\binom{r-1}{k}\) in
terms of \(\binom{r}{k}\) in the sum
\[
\binom{r-1}{k-1} + \binom{r-1}{k} = \frac{k}{r}\binom{r}{k}
+ \frac{r-k}{r}\binom{r}{k} = \binom{r}{k},\quad \text{\textsc{qed.}}
\]
The sum is valid if \(r > 0\). There is direct proof of the formula,
which does not involve computations but pure logic. Let us suppose
that we \emph{already} have all the subsets of \(k\) items chosen
among \(r\). By definition, there are \(\binom{r}{k}\) of them. We
choose to distinguish an arbitrary item amongst \(r\) and we want to
split the collection in two subsets: on one side, all the combinations
containing this particular item, on the other side, all the
combinations without it. The former subset has cardinal
\(\binom{r-1}{k-1}\) because its combinations are built from the fixed
item and further completed by choosing \(k-1\) remaining items amongst
\(r-1\). The latter subset is made of \(\binom{r-1}{k}\) combinations
which are made from \(r-1\) items, of which \(k\) have to be selected
because we ignore the distinguished item. This yields the same
additive formula. Now, let us return to our pending sum
\[
\sum_{j=0}^{m-1}{\binom{n+j}{n}}.
\]
\parpic[l][bl]{\includegraphics[bb=75 659 104 721]{triangle1}}
\noindent In terms of navigation across Pascal's Triangle, we
understand that this sum operates on numbers in the same column. More
precisely, it starts from the diagonal with the number
\(\binom{n}{n}\) and goes down until a total of \(m\) numbers have
been added. So let us choose a simple example and fetch two adjacent
columns were the sum is small. On the left is an excerpt for \(n=4\)
(the left column is the fifth in Pascal's Triangle) and \(m=4\)
(height of the left column). Interestingly, the sum of the numbers in
bold in the left column, that is, the sum under study, equals the number
in bold at the bottom of the second column: \(1 + 5 + 15 + 35 =
56\). By checking other columns, we may feel justified to think that
this is a general pattern. Before attempting a general proof, let us
see how it may work on our particular example. Let us start from the
bottom of the second column, that is, \(56\), and use the addition
formula in reverse, that is, express \(56\) as the sum of the numbers
in the row above it: \(56 = 35 + 21\).

\parpic(30pt,32pt)[lb][bl]{\includegraphics[bb=75 695 104 721]{triangle2}}
\noindent Graphically, this sum is seen as the triangle at the left,
extracted from the previous one. The number \(35\) has just appeared
and we would like to keep it because it is part of the equality to
prove. So let us apply the addition formula again to \(21\) and draw
\(21 = 15 + 6\). Let us keep \(15\) and resume the same procedure on
\(6\) so \(6 = 5 + 1\). Finally, \(1 = 1 + 0\). We just checked \( 56
= 35 + (15 + (5 + (1 + 0))), \) which is exactly what we
wanted. Because we want the number corresponding to \(35\) in our
example to be \(\binom{n+m-1}{n}\), we have the derivation
\begin{align*}
\binom{n+m}{n+1}
  &= \binom{n+m-1}{n} + \binom{n+m-1}{n+1}\\
  &= \binom{n+m-1}{n} + \left[\binom{n+m-2}{n} +
     \binom{n+m-2}{n+1}\right]\\
  &= \binom{n+m-1}{n} + \binom{n+m-2}{n} + \dots +
     \left[\binom{n}{n} + \binom{n}{n+1}\right],\\
\binom{n+m}{n+1}
  &= \sum_{j=0}^{m-1}{\binom{n+j}{n}} = W(m,n),\,\; \text{by eq.
     \eqref{eq:W} on page \pageref{eq:W}}.
\end{align*}
By symmetry, \(B(m,n) = W(n,m)\), therefore \(B(m,n) =
\binom{m+n}{m+1}\). We can check these formulas for \(m=3\) and
\(n=2\): \(W(3,2) = \binom{5}{3} = 5!/(3!2!) = 10\) and \(B(3,2) =
\binom{5}{4} = 5!/(4!1!) = 5\), both being correct. Let
\(\Aideal^{\gamma\delta}(m,n)\) be the average number of
comparisons to merge \(m\) and \(n\) numbers, which is the same as the
number of times clauses \clause{\gamma} or \clause{\delta} are used.
We found \vpageref{average_delay_merge} that
\begin{align*}
\Aideal^{\gamma\delta}(m,n)
  &= m + n - \biggl[W(m,n) + B(m,n)\biggr]\left/\binom{m+n}{n}\right..
\intertext{We can now replace \(W(m,n)\) and \(B(m,n)\) and obtain a
  closed form:}
\Aideal^{\gamma\delta}(m,n)
  &= m + n - \left[\binom{m+n}{n+1} + \binom{m+n}{m+1}\right]\left/\binom{m+n}{n}\right.\\
  &= m + n - \frac{n!m!}{(n+1)!(m-1)!} - \frac{n!m!}{(m+1)!(n-1)!},\\
\Aideal^{\gamma\delta}(m,n)
  &= m + n - \frac{m}{n+1} - \frac{n}{m+1} = \frac{mn}{m+1} +
\frac{mn}{n+1}.
\end{align*}
Let us check this pretty formula with \(\Aideal^{\gamma\delta}(3,2) =
6/4 + 6/3 = 7/2\), which is the value we computed
\vpageref{seven_two}. Let us recall the program as found
\vpageref{code:merge}:

\begin{alltt}
merge(   [],      Q)            \(\smashedrightarrow{\alpha}\) Q;
merge(    P,     [])            \(\smashedrightarrow{\beta}\) P;
merge([I|P],Q=[J|_]) when I < J \(\smashedrightarrow{\gamma}\) [I|merge(P,Q)];
merge(    P,  [J|Q])            \(\smashedrightarrow{\delta}\) [J|merge(P,Q)].
\end{alltt}


\noindent Because each call to \erlcode{merge/2} ends either with a
rewrite by clause \clause{\alpha} or \clause{\beta}, the relationship
between the average delay of \erlcode{merge/2}, noted
\(\ave{merge}{m,m}\), and \(\Aideal^{\gamma\delta}(m,n)\) is simply
\begin{equation}
\ave{merge}{m,n} = 1 + \Aideal^{\gamma\delta}(m,n)
                 = \frac{mn}{m+1} + \frac{mn}{n+1} + 1.
\label{eq:ave_merge}
\end{equation}
First, let us check that \(\ave{merge}{m,n} = \ave{merge}{n,m}\),
which was expected because every symmetric problem has a symmetric
solution. Next, we can now verify what happens when \(m=0\) or
\(n=0\), as we expect the number of comparisons to be zero and we
indeed deduce \(\ave{merge}{0,n} = \ave{merge}{m,0} = 1\). This proves
that the average delay in this case coincides with the minimum delay
and, consequently, the lower bound in \(\best{merge}{m,n} \leqslant
\ave{merge}{m,n} \leqslant \worst{merge}{m,n}\). In fact, if \(m > 0\)
and \(n \geqslant 0\) or \(n > 0\) and \(m \geqslant 0\), we have
\[
1 + \min\{m,n\} \leqslant \ave{merge}{m,n} \leqslant m + n,
\]
and the bounds are actually tight. Another case is worth pondering:
\(m=1\) or \(n=1\). Indeed, we may realise then that merging a
singleton list with another list leads to the same result as inserting
the single item into the other list. Let us recall the
\erlcode{insert/2} function used in insertion sort,
\vpageref{code:insert}:
\begin{alltt}
insert(I,[J|S]) when I > J \(\smashedrightarrow{\beta}\) [J|insert(I,S)];
insert(I,    S)            \(\smashedrightarrow{\gamma}\) [I|S].
\end{alltt}
We then have, for all items \(I\) and all lists \(L\), the equivalence
\[
\erlcode{insert(\(I\),\(L\))} \mathrel{\equiv}
\erlcode{merge([\(I\)],\(L\))}.
\]
Therefore, \emph{insertion is a special case of merging.}
Nevertheless, the average delays are not exactly the same:
\[
\ave{insert}{n} = \frac{1}{2}{n} + 1,\quad
\ave{merge}{1,n} = \frac{1}{2}{n} + 2 - \frac{1}{n+1},
\]
as we found \vpageref{ave_insert}. Therefore, \erlcode{merge/2} is
slightly slower in average than \erlcode{insert/2} in this special
case:
\[
\ave{merge}{1,n} - \ave{insert}{n} = 1 - \frac{1}{n+1} < 1.
\]
Nonetheless, asymptotically, this difference is insignificant:
\[
\ave{merge}{1,n} \mathrel{\sim} \ave{insert}{n},
\,\; \text{as} \;\; n \rightarrow \infty.
\]
Also, it may be interesting to see what happens when \(m=n\), that is,
when the two lists to be merged have the same length:
\begin{equation}
\ave{merge}{n,n}
  = \frac{2n^2}{n+1} + 1
  \mathrel{\sim} 2n,\,\; \text{as} \,\; n \rightarrow \infty.
\label{eq:ave_merge_n_n}
\end{equation}
In other words, the average delay of merging two lists of identical
length is asymptotically the total number of items being merged.

Merging can be used to sort a list of items if this list can be
previously split evenly in two sub\hyp{}lists which are in turn
sorted, taking into account that a singleton list is always sorted by
definition. This recursive approach to sorting is said
\emph{top\hyp{}down} because of the way it is usually depicted, the
original list being written at the top of the page, the singletons
being located at the bottom. Another view first maps the list to be
sorted into a list of singletons, directly at the bottom. As we
noticed before, these can be readily merged pairwise and their results
too etc. until one sorted list remains at the top. Here, the sorted
list is at the top, not the input list. This approach is said
\emph{bottom\hyp{}up} because, after the singletons have been
constructed, the merging takes place directly. In the top\hyp{}down
way, these singletons are reached after a succession of
splittings. Anyway, the same detail proves thorny: how to manage the
cases when, in the top\hyp{}down fashion, a list to be evenly divided
contains an odd number of items or how, in the bottom\hyp{}up way, an
odd number of lists can be merged pairwise.

\medskip

\paragraph{Bottom\hyp{}up merge sort of \(\boldsymbol{2^p}\) items.} 

Let us ignore these problems for now and explore a small example with
a number of items which is a power of \(2\), like
\erlcode{[7,3,5,1,6,8,4,2]}. Sorting by merging, be it top\hyp{}down
or bottom\hyp{}up, results in the same kind of graphical
representation: a \emph{complete binary tree}. We introduced trees
\vpageref{def:tree} and a binary tree is a tree whose nodes have
either no subtree or exactly two, as shown in
\fig~\vref{fig:bottom_up1}. This structure mirrors faithfully the
strategy: if reading top\hyp{}down, we divide a problem (to sort a
list) into two subproblems (to sort two smaller lists); if reading
bottom\hyp{}up, we reduce two solutions (two small sorted lists) into
one solution (a bigger sorted list). \Fig~\vref{fig:bottom_up1}
illustrates the bottom\hyp{}up approach of starting off from
singletons and merging them and the resulting lists in turn etc. until
the root of the tree is an ordered list.
\begin{figure}
\centering
\includegraphics[bb=73 637 252 721]{bottom_up1}
\caption{Sorting bottom\hyp{}up \erlcode{[7,3,5,1,6,8,4,2]} by merging
\label{fig:bottom_up1}}
\end{figure}
Perhaps the first question arising is how to obtain all these
singletons at the leaves of the tree. This is actually quite easy if
we already rely on a map, as seen \vpageref{maps}:
\begin{verbatim}
map(_,   []) -> [];
map(F,[I|L]) -> [F(I)|map(F,L)].
\end{verbatim}
Then, the function \erlcode{solo/1} reduces simply to one call:
\begin{verbatim}
solo(L) -> map(fun(I) -> [I] end,L).
\end{verbatim}
Otherwise, we opt for the equivalent handmade
\begin{verbatim}
solo(   []) -> [];
solo([I|L]) -> [[I]|solo(L)].
\end{verbatim}
The delay of \(\erlcode{solo(\(L\))}\) when \(L\) contains \(n\) items
is \(\comp{solo}{n} = n + 1\) with the latter definition. We can now
refocus on \fig~\vref{fig:bottom_up1} and try to quantify some of its
aspects. For instance, we may wonder about the \emph{height} of the
tree or the total number of merges. We assumed that the we sorted
\(2^p\) numbers, with \(p > 0\). In our example \(p=3\). Since at each
step upward in the tree, all the nodes at one level are merged
pairwise, there are half the number of parents. There are \(2^p\)
leaves, which have \(2^{p-1}\) parents, which have \(2^{p-2}\) parents
etc. until \(2^0 = 1\), which is the root. This observation allows us
to answer our two questions at once. First, it means that there are
\(p+1\) levels in the tree, including the root. Second, it implies
that there are \(p\) merge phases making, bottom\hyp{}up, \(2^{p-1}\),
\(2^{p-2}\), \ldots, \(2^0\) merges, amounting to \(2^{p-1} + 2^{p-2}
+ \dots + 2^0\) merges in total. This sum has a simple closed
form. Let us consider the general case \(Q_x(p) := x^{p-1} +
x^{p-2} + \dots + x^0\). Then \(xQ_x(p) - Q_x(p) = x^p - x^0\), that
is, when \(x \neq 1\),
\begin{equation}
\sum_{k=0}^{p-1}{x^k} = \frac{x^p - 1}{x-1}.\label{eq:sum_power}
\end{equation}
In particular, \(x=2\) let us know that there are \(2^p-1\) merges in
total. If, as it is custom, we express the size of the input as
\(n=2^p\), we prefer to express these previous expressions in terms of
\(n\) instead of \(p\). This is easy, since we then have \(p = \lg
n\), where \(\lg x\) is the \emph{binary logarithm} of \(x\), which
can be defined as the inverse function of \(x \mapsto 2^x\), for \(x >
0\). Now we can conclude that the height of the tree is \(1 + \lg n\)
and the number of merges is \(n-1\). We may also want to quantify the
amount of memory needed to sort by merging \(n=2^p\)~keys. First, the
number of stacks created is the number of nodes of the merge tree:
\(2^p + 2^{p-1} + \ldots + 2^0 = 2^{p+1}-1 = 2n - 1\). Because the
keys and the empty stack are shared with the input, we should count
the references to them. Also, we have to account for the pushes,
namely the nodes~(\texttt{|}): there is one for each reference to a
key. This leads us to determine the sum of the lengths of all the
created stacks: \((p+1)2^p = n\lg n + n\). We must add \(2n - 1\) for
all the references to~\texttt{[]}, so the total number of references
created is \(n\lg n + 3n - 1\) and the number of nodes~(\texttt{|}) is
\(n\lg n + n\). Finally, we might compute the total delay to sort
\(2^p\)~items because we know that, bottom\hyp{}up, the length of the
lists doubles and we already studied the delay of merging, in
particular when the two lists have the same length,
\(\comp{merge}{n,n}\). Consequently, all we need is to resume our
calculation of the number of merges whilst making sure to ponder each
stage with the corresponding delay of merging. In order to focus on
the concepts at stake, we shall leave out the delay \(\comp{solo}{2^p}
= 2^p + 1\) for now, which can be thought of as preprocessing time. So
let us note \(\Dideal(n)\) the delay to sort \(n\) items without
taking into account \erlcode{solo/1}. We have, for \(p > 0\),
\begin{align}
\Dideal(2^p)
  &= 2^{p-1} \cdot \comp{merge}{1,1} + 2^{p-2} \cdot
     \comp{merge}{2,2} + \dots + 2^0 \cdot
     \comp{merge}{2^{p-1},2^{p-1}}\notag\\
  &= \sum_{k=0}^{p-1}{\left(2^{p-1-k} \cdot \comp{merge}{2^k,2^k}\right)}
   = 2^{p-1}
   \sum_{k=0}^{p-1}{\frac{1}{2^k}\comp{merge}{2^k,2^k}}.
\label{eq:delay_power_2}
\end{align}
This sum spurs us to investigate further the best and worst cases of
\erlcode{merge/2} when its arguments have the same length. The result
we deduced \vpageref{merge:best_case} applies here as well, so
\[
\best{merge}{n,n} = 1 + \min\{n,n\} = n + 1.
\]
This most favourable configuration arises when all the numbers in the
shortest list are smaller than the smallest number of the longest. In
the present case, both lists have the same length, so it happens
whenever all the numbers in one list are smaller than the smallest
number in the other. For example, if both lists are singletons, we
find ourselves in the best case and the delay is \(2\). This gives us
already \(\comp{merge}{1,1} = \best{merge}{1,1} = 2\). One possible
configuration is when the list is already sorted, as shown in
\fig~\vref{fig:bottom_up_best1}.
\begin{figure}[t]
\centering
\includegraphics[bb=73 637 252 721]{bottom_up_best1}
\caption{A best case for bottom\hyp{}up merge sort
\label{fig:bottom_up_best1}}
\end{figure}
If the list is sorted in decreasing order, it is also a best case, as
can be verified in \fig~\vref{fig:bottom_up_best2}.
\begin{figure}[b]
\centering
\includegraphics[bb=73 637 252 721]{bottom_up_best2}
\caption{Another best case for bottom\hyp{}up merge sort
\label{fig:bottom_up_best2}}
\end{figure}
Other variations are possible: in \fig~\vref{fig:bottom_up_best1},
start from the root and exchange any pair of subtrees which have the
same parent and this is still a best case. Importantly, \emph{a best
  case has the property that it is made of subtrees which are
  themselves instances of a best case}. This allows us to construct
all the best cases. (How many are there when \(n\) numbers are given?)
Let us note \(\Bideal(n)\) the best delay for sorting \(n\) items
without taking into account the preprocessing by \erlcode{solo/1}.  We
can write the delay of bottom\hyp{}up merge sort in the best case as
follows:
\[
\Bideal(2^p)
  = 2^{p-1}\!\sum_{k=0}^{p-1}{\frac{1}{2^k}\best{merge}{2^k,2^k}}
   = 2^{p-1}\!\sum_{k=0}^{p-1}{\frac{1}{2^k}(2^k+1)}\\
  = 2^{p-1} \!\left(p +\!\!\sum_{k=0}^{p-1}{\frac{1}{2^k}}\right)\!.
\]
We can reuse equation \eqref{eq:sum_power} \vpageref{eq:sum_power}
with \(x=1/2\), which implies
\begin{equation}
\sum_{k=0}^{p-1}{\frac{1}{2^k}} = 2 - \frac{1}{2^{p-1}}.
\label{eq:sum0.5k}
\end{equation}
Let us finish now:
\begin{align}
\Bideal(2^p)
  &= 2^{p-1} \left(p +  2 - \frac{1}{2^{p-1}}\right)
   = p2^{p-1} + 2^p - 1;\label{eq:best_power}\\
\Bideal(n)
  &= \frac{1}{2}n\lg n + n - 1,\,\; \text{with} \,\; n=2^p.\notag
\end{align}
How does \(n \lg n\) compares to the more familiar \(n^2\)? Let us
prove that, for all real \(x > 0\), we have \(x < 2^x\). Let us define
a function \(\xi(x) = 2^x - x\), for \(x \geqslant 0\). Its derivative
is \(\xi'(x) = 2^x - 1\). Because \(x \mapsto 2^x\) is a strictly
increasing function, \(x > 0 \Rightarrow 2^x > 2^0 = 1\). Thus,
\(\xi'(x) > 0\). Since \(\xi(0) = 0\), all this implies \(\xi(x) > 0\)
for \(x > 0\), hence the expected result. Because the logarithm is the
inverse of the exponentiation, which is an increasing function, the
logarithm is an increasing function. Therefore we can apply it on both
sides of an inequality, provided they are positive: \(0 < x < 2^x\)
implies \(\lg x < x\). Furthermore, \(n \lg n < n^2\) for all \(n>0\).

Let us delve now into the worst case of sorting bottom\hyp{}up by
merging when \(n=2^p\), for \(p>0\). Equation \eqref{eq:worst_merge}
\vpageref{eq:worst_merge} was \(\worst{merge}{m,n} = m + n\), thus
\(\worst{merge}{n,n} = 2n\). The corresponding configuration supposes
that the last number of one of the two lists being merged is the last
item of the result and that the last number of the other list finds
its place as the penultimate in the result. Obviously, this happens
when both lists are singleton, that is, \(n=1\), that is, for the leaves
of any tree capturing the essence of bottom\hyp{}up sorting by
merging: \(\comp{merge}{1,1} = \worst{merge}{1,1} = 2\). It is easy to
derive a worst case from the one presented in
\fig~\vref{fig:bottom_up1}: just exchange the leaves \erlcode{[4]} and
\erlcode{[6]}. The result is shown in
\fig~\vref{fig:bottom_up_worst1}.
\begin{figure}
\centering
\includegraphics[bb=73 637 252 721]{bottom_up_worst1}
\caption{A worst case for bottom\hyp{}up merge sort
\label{fig:bottom_up_worst1}}
\end{figure}
Just as with the best case, constructing a global worst case is
achieved by constructing worst cases for all the subtrees. Let us note
\(\Wideal(n)\) the worst delay for sorting \(n\) items without
counting in the delay due to \erlcode{solo/1}.
\begin{align}
\Wideal(2^p)
  &= 2^{p-1}\!\sum_{k=0}^{p-1}{\frac{1}{2^k}\worst{merge}{2^k,2^k}}
  = 2^{p-1}\!\sum_{k=0}^{p-1}{\frac{1}{2^k}(2^k+2^k)}
  = p2^p.\label{eq:worst_power}\\
\Wideal(n)
  &= n\lg n,\;\, \text{with} \,\; n = 2^p.\notag
\end{align}
In sum, we have found the following bounds when \(n=2^p\):
\begin{equation}
\Bideal(n) = \tfrac{1}{2}{n \lg n} + n - 1
\leqslant \Dideal(n) \leqslant 
n \lg n = \Wideal(n).\label{ineq:D}
\end{equation}
Notice how the lower bound is only a bit more than half the upper
bound. Contrast this with insertion sort, where the lower bound was
linear in the input size, whilst the upper bound was quadratic. Before
jumping to conclusions, bear in mind that we have analysed only a
special case of merge sort, when \(n\) is a power of \(2\). We shall
deal with the general case latter. But, before doing so, let us make
full profit of our analysis of the delay of merging by drawing the
average delay of bottom\hyp{}up merge sort when \(n=2^p\). Equation
\eqref{eq:ave_merge_n_n} \vpageref{eq:ave_merge_n_n} was:
\[
\ave{merge}{n,n} = \frac{2n^2}{n+1} + 1 = 2n - 1 + \frac{2}{n+1}.
\]
Let us note \(\Aideal(n)\) the average delay to sort \(n\) items
without adding up the delay of the calls to \erlcode{solo/1}. The
average delay, being a delay, obeys the same recurrent equations as
the general delay, so it is made of all the average delays of
processing the sub\-structures:
  %% &= 2^{p-1}\left(2p + \sum_{k=0}^{p-1}{\frac{1}{2^k}}
  %%    - 2\sum_{k=0}^{p-1}\frac{1}{2^k+1}\right)\notag\\
  %% &= 2^{p-1}\left(2p + \left(2 - \frac{1}{2^{p-1}}\right)
  %%    - 2\sum_{k=0}^{p-1}\frac{1}{2^k+1}\right),\notag\\
  %% = 2^{p-1}
  %%   \sum_{k=0}^{p-1}{\frac{1}{2^k}\left(2\cdot2^k - 1
  %%   + \frac{2}{2^k+1}\right)}\notag\\
\begin{align}
\Aideal(2^p)
  &= 2^{p-1}\sum_{k=0}^{p-1}{\frac{1}{2^k}\ave{merge}{2^k,2^k}}
  = 2^{p-1}\left(2p - \sum_{k=0}^{p-1}{\frac{1}{2^k}}
     + 2\sum_{k=0}^{p-1}{\frac{1}{2^k(2^k+1)}}\right)\notag\\
  &= 2^{p-1}\left(2p - \sum_{k=0}^{p-1}{\frac{1}{2^k}}
     + 2\sum_{k=0}^{p-1}\left(\frac{1}{2^k}
     - \frac{1}{2^k+1}\right)\right),\notag\\
\Aideal(2^p)
  &= p2^p + 2^p - 1 - 2^p \sum_{k=0}^{p-1}\frac{1}{2^k+1},\,\;
\text{with} \,\; p > 0.
\label{eq:ave_power}
\end{align}
There is no closed form known for the remaining sum but, for \(k
\geqslant 0\), we can work out rough bounds for it and thus for
\(\Aideal(2^p)\):
\begin{gather*}
2^{k+1} = 2^k + 2^k \geqslant 2^k + 1 > 2^k
\Rightarrow
\frac{1}{2}\sum_{k=0}^{p-1}{\frac{1}{2^{k}}}
\leqslant \sum_{k=0}^{p-1}{\frac{1}{2^k+1}} <
\sum_{k=0}^{p-1}{\frac{1}{2^k}}\\
1 - \frac{1}{2^{p}}
\leqslant \sum_{k=0}^{p-1}{\frac{1}{2^k+1}} <
2 - \frac{1}{2^{p-1}},\,\; \text{by eq.}~\eqref{eq:sum0.5k}.
\end{gather*}
Because \(1/(2^k + 1) > 0\), the partial sums
\(\sum_{k=0}^{p-1}\frac{1}{2^k+1}\) are positive and increasing in
function of \(p\). They are also converging (absolutely) because the
limit of the upper bound we found is finite: \(\lim_{p \rightarrow
  \infty}{(2 - 1/2^{p-1})} = 2\). The limit of the lower bound being
\(1\), it proves that \(\sum_{k \geqslant 0}\frac{1}{2^k+1}\) is a
number between \(1\) and \(2\) and Peter Borwein proved that it is
actually an irrational number. Let us call it \(\alpha \simeq
1.2644997803484442092\). Then, we can discard the upper bound \(2 -
1/2^{p-1}\) in favour of \(\alpha\):
\begin{gather*}
1 - \frac{1}{2^p} \leqslant \sum_{k=0}^{p-1}{\frac{1}{2^k+1}}
< \alpha
\Rightarrow
-\alpha2^p < -2^p\sum_{k=0}^{p-1}{\frac{1}{2^k+1}} \leqslant
-2^p\left(1 - \frac{1}{2^p}\right).
\end{gather*}
%(p2^p + 2^p - 1) - \alpha 2^p &< \Aideal(2^p) \leqslant (p2^p + 2^p - 1) - 2^p + 1\notag\\
\begin{equation}
p2^p - (\alpha - 1)2^p - 1 < \Aideal(2^p) \leqslant
p2^p.\label{ineq:A_power}
\end{equation}
Finally, setting \(n=2^p\), we conclude
\begin{equation}
n \lg n - (\alpha - 1) n - 1 < \Aideal(n) \leqslant n \lg n.
\label{ineq:A}
\end{equation}
This means that the average delay cannot be as small as the best
delay, whilst it can be as large as the worst delay. In fact,
inequalities \eqref{ineq:A} allow us at last to derive the asymptotic
approximation of \(\Aideal(n)\):
\[
\Aideal(n) \mathrel{\sim} n \lg n \mathrel{\sim}
\Wideal(n),\;\, \text{as} \,\; n=2^p \,\; \text{and} \,\; p
\rightarrow \infty.
\]
Therefore, \emph{the average delay is asymptotically equivalent to the
  worst case delay when sorting \(2^p\) items.} It can be proved that
the worst delay of \erlcode{merge/2} is the worst delay of any
\emph{optimal} merge algorithm when both lists have the same length
(the maximum delay is minimum in this case) and all merges are done on
lists of equal lengths when \(n=2^p\). Therefore, the worst delay
\(\Wideal(n)\) is minimum. After the general case of merge sort has
been presented, we will also show that the worst delay of merge sort
is asymptotically as good as the worst delay of any optimal sorting
algorithm. Another interesting fact we can already underline is that
the worst case of merge sort is not sensitive to a possible lack of
randomness of the input data. For instance, we saw that the worst
delay of insertion sort happens when the data is already sorted. We
also mentioned before that insertion was a special case of merging,
which may lead to a quadratic delay. Therefore, it is crucial for
merge sort not to degenerate into insertion sort, by making sure that
the maximum number of merges are performed on lists of almost the same
length. See the problem in \fig~\vref{fig:wrong_mergings}, where the
delay is quadratic and, thus, is not an instance of merge sort. This
observation will be our guideline in the design of the general
bottom\hyp{}up merge sort.
\begin{figure}[t]
\centering
\includegraphics[bb=73 612 170 721]{wrong_mergings}
\caption{Mergings equivalent to the worst case of insertion sort
\label{fig:wrong_mergings}}
\end{figure}

\medskip

\paragraph{General bottom\hyp{}up merge sort.}

In general, any number can be uniquely expressed as the sum of
distinct powers of \(2\), for example, \(7=2^2+2^1+2^0\), and this is the
foundation of the \emph{binary number system}. Let us assume
\(n=\sum_{i=0}^{m-1}{b_i2^i}\), where the \(b_i \in \{0,1\}\) are
called \emph{bits}. Then \(n = 2(b_{m-1}2^{m-2} + \dots + b_1) +
b_0\), which means that \(b_0\) is the remainder of the Euclidean
division of \(n\) by \(2\), traditionally noted \(b_0 = n \mod
2\). This uniquely defines \(b_0\) because a number is either odd or
even. The same division process is repeated on the quotient
\(\floor{n/2}\), establishing the remaining bits. In binary, \(7\) is
expressed as the series of bits \((111)_2\).

How can we manage the general case of merge sort, that is, when the
number of items is not a power of \(2\)?  \Fig~\vref{fig:bottom_up_7}
\begin{figure}[b]
\centering
\includegraphics[bb=73 637 228 721]{bottom_up_7}
\caption{Sorting seven numbers bottom\hyp{}up by merging
\label{fig:bottom_up_7}}
\end{figure}
shows an example where seven numbers are sorted. Notice how the
rightmost leaf, that is, \erlcode{[4]}, is connected by an egde without
an arrow to a node above containing \erlcode{[4]} as well because it
is not merged. The list \erlcode{[4]} is only merged at the next level
with \erlcode{[2,6]}, a list twice its length. The imbalance in length
is propagated upward, as \erlcode{[1,3,5,7]} is merged with
\erlcode{[2,4,6]}. Let us rely on the special case \(n=2^p\) to find
an upper bound on \(\Bideal(n)\) for arbitrary \(n\), tight enough to
yield an asymptotic equivalence with the help of the lower bound.

The general situation can be guessed by abstracting away unnecessary
details from the example of \fig~\vref{fig:bottom_up_7}. We can see
that there are three special
\begin{figure}[t]
\centering
\includegraphics[bb=71 633 149 721]{msort_abstract}
\caption{Abstraction of \fig~\vref{fig:bottom_up_7}
\label{fig:msort_abstract}}
\end{figure}
sub\-cases: the subtree rooted at \erlcode{[1,3,5,7]}, figured by a
triangle and containing \(2^2\) numbers at its leaves; the subtree
rooted at \erlcode{[2,6]}, also depicted by a triangle and containing
\(2^1\) leaves, and the subtree whose root is \erlcode{[4]},
containing \(2^0\) leaf. See \fig~\vref{fig:msort_abstract}, where
only edges corresponding to merges, that is, ending with an arrow, are
represented. In all generality, let \(n=2^{e_r} + \dots + 2^{e_1} +
2^{e_0} > 0\), with \(e_r > \dots > e_1 > e_0 \geqslant 0\) and \(r
\geqslant 0\). The natural numbers \(e_i\)~are the positions of the
\(1\)\hyp{}bits in the binary representation of \(n\). In particular,
this means that \(i \leqslant e_i\) and \(r+1\) is the number of
\(1\)\hyp{}bits or, equivalently, the sum of the bits of \(n\). Also,
\(2^{e_r}\) being the largest power in the decomposition of \(n\), it
is identical to the largest power in the binary expansion of \(n\), so
\(e_r+1\) is equal to the number of bits of \(n\). Let \(m\) be this
number. Then the smallest integer expressible with \(m\) bits is \((10
\ldots 0)_2 = 2^{m-1} = 2^{e_r}\). Dually, the greatest integer is
\((1 \ldots 1)_2 = 2^{m-1}+\dots+2^1+2^0 = 2^{m}-1\). For \(n > 0\),
\begin{equation}
2^{m-1} \leqslant n \leqslant 2^m -  1 
< 2^m \Rightarrow m - 1 \leqslant \lg n
< m \Rightarrow e_r = \floor{\lg n}.\label{eq:e_r}
\end{equation}
Now let us consider in \fig~\vref{fig:msort_gen}
\begin{figure}[!b]
\centering
\includegraphics[bb=71 631 202 721]{msort_gen}
\caption{Sorting by merging \(n=2^{e_r}+ \dots + 2^{e_0}\) items
\label{fig:msort_gen}}
\end{figure}
the tree of all the merges when we abstract away the details. The
triangles are trees made of \emph{balanced merges}, that is, merges
performed on lists of the same length, for which we already found the
delays. The nodes from the root \(2^{e_r}+\dots+2^{e_0}\) down to
\(2^{e_1}+2^{e_0}\) are the lengths of the results of \emph{unbalanced
  merges}. The total delay is the sum of the balanced and unbalanced
merges:
\begin{equation}
\Dideal(n)
  = \sum_{i=0}^{r}{\Dideal(2^{e_i})}
    +
    \sum_{i=1}^{r}{\comp{merge}{2^{e_i},2^{e_{i-1}}+\dots+2^{e_0}}}.
\label{eq:msort_gen}
\end{equation}
If \(r=0\), we found again the special case \(n=2^{e_0}\), as the
second sum is zero. Let us bet that the general case is asymptotically
the same as the special case \(n=2^p\) and strive for upper and lower
bounds on the best, worst and average delays which converge to the
same functions. This hope should not been considered as obvious, but
really optimistic. For instance, even though \(2^{\floor{\lg n}} =
n\), when \(n\)~is a power of \(2\), it is \emph{false} to state that
\(2^{\floor{\lg n}} \mathrel{\sim} n\), as \(n \rightarrow \infty\).
Consider the function \(x(p)=2^p-1\) ranging over the positive
integers. First, let us notice that, for all \(p>0\),
\begin{equation*}
2^{p-1} \leqslant 2^p - 1 < 2^p \Rightarrow p-1 \leqslant \lg(2^p-1) <
p \Rightarrow \floor{\lg(2^p-1)} = p-1.
\end{equation*}
Therefore, \(2^{\floor{\lg(x(p))}} = 2^{p-1} = (x(p)+1)/2
\mathrel{\sim} x(p)/2 \mathrel{\nsim} x(p)\), which proves that
\(2^{\floor{\lg(n)}} \nsim n\) when \(n=2^p-1 \rightarrow \infty\).

\medskip

\paragraph{Best delay.} 

The best case of \erlcode{merge/2} is given by equation
\eqref{eq:best_merge} \vpageref{eq:best_merge}, that is,
\(\best{merge}{m,n} = 1 + \min\{m,n\}\), so
\begin{equation*}
\Bideal(n)
  =   \sum_{i=0}^{r}{\Bideal(2^{e_i})}
    + \sum_{i=1}^{r}{\left(1+\min\{2^{e_i},2^{e_{i-1}} + \dots +
                                 2^{e_0}\}\right)}.
\end{equation*}
Let us commence by noting that the following inequality holds:
\begin{equation*}
\sum_{j=0}^{i}{2^{e_j}}
\leqslant
\sum_{j=0}^{e_i}{2^j} = 2 \cdot 2^{e_i} - 1.
\end{equation*}
This is the same as saying that a given binary number is always lower
or equal than the number with the same number of bits all set to
\(1\), for example, \((10110111)_2 \leqslant (11111111)_2\). By definition of
\(e_i\): \(e_{i-1} + 1 \leqslant e_i\), hence
\begin{equation}
\sum_{j=0}^{i-1}{2^{e_j}} \leqslant 2^{e_{i-1}+1} - 1
                       \leqslant 2^{e_i} - 1 < 2^{e_i},
\label{ineq:up_bin}
\end{equation}
therefore, \(\min\{2^{e_i},2^{e_{i-1}} + \dots + 2^{e_0}\} =
2^{e_{i-1}} + \dots + 2^{e_0}\). We can reformulate the original
definition of the best delay now as
\begin{equation}
\Bideal(n)
   =   \sum_{i=0}^{r}{\Bideal(2^{e_i})}
     + r + \sum_{i=1}^{r}\sum_{j=0}^{i-1}{2^{e_j}}.
\label{eq:Bideal}
\end{equation}
Equation \eqref{eq:best_power} \vpageref{eq:best_power} states
\(\Bideal(2^p) = p2^{p-1} + 2^p - 1\), hence
\begin{gather}
\sum_{i=0}^{r}{\Bideal(2^{e_i})} =
\frac{1}{2}\sum_{i=0}^{r}{e_i2^{e_i}} + n - (r + 1),\notag\\
\Bideal(n) =
   \frac{1}{2}\sum_{i=0}^{r}{e_i2^{e_i}} + n - 1
 + \sum_{i=1}^{r}\sum_{j=0}^{i-1}{2^{e_j}}.\label{eq:B_n_tmp}
\end{gather}
By reusing inequations \eqref{ineq:up_bin}, let us work out a small
upper bound:
\begin{align}
\sum_{i=1}^{r}\sum_{j=0}^{i-1}{2^{e_j}}
&\leqslant \sum_{i=1}^{r}\left(2^{e_{i-1}+1}-1\right)
= 2 \sum_{i=0}^{r-1}{2^{e_i}} - r\notag\\
&= 2 (n - 2^{e_r}) - r
= 2n - 2^{\floor{\lg n} + 1} - \nu_{n} + 1.
\label{ineq:double_upper}
\end{align}
The function \(\nu_{n} := r+1\) is the number of \(1\)\hyp{}bits in
the binary expansion of \(n\), or, equivalently, the sum of its
bits. This function of \(n\) is called \emph{population count},
\emph{sideways sum}, \emph{bit sum} or \emph{Hamming weight}. It can
be defined recursively as follows:
\begin{equation}
\nu_{0} := 0,\quad \nu_{2n} := \nu_{n},\quad
\nu_{2n+1} := \nu_{n} + 1.\label{def:nu}
\end{equation}
We have the following intuitive tight bounds for any \(n>0\):
\[
1 \leqslant \nu_{n} \leqslant \floor{\lg n} + 1,
\]
because deduction \eqref{eq:e_r} \vpageref{eq:e_r} establishes that
\(\floor{\lg n} + 1\) is the number of bits of \(n\). Notice that
\(\nu\) is a deceptively simple function: first, it is periodic
because \(\nu_{2^p} = 1\) and, second, \(\nu_{2^p-1}=p\). We can check
that \(n=2^{e_0}\) leads to both sides to equal \(0\), as
expected. The remaining sum \(\sum_{i=0}^{r}{e_i2^{e_i}}\) can be
simply upper\hyp{}bounded as follows:
\begin{equation}
\sum_{i=0}^{r}{e_i2^{e_i}}
 \leqslant \sum_{i=0}^{r}{e_r2^{e_i}} = e_r\sum_{i=0}^{r}{2^{e_i}}
= n \floor{\lg n}.\label{ineq:upper_n_lg_n}
\end{equation}
This bound is tight only if \(n=2^{e_0}\). We can conclude now with
the following bound, tight when \(n\) is a power of \(2\):
\begin{equation*}
\Bideal(n) 
  \leqslant \tfrac{1}{2}n\floor{\lg n} + 3n
             - 2^{\floor{\lg n} + 1} - \nu_{n}
  \leqslant \tfrac{1}{2}n \floor{\lg n} + 3n - 2^{\floor{\lg n} + 1} - 1.
\end{equation*}
In order to ease finding the limit of the bound when \(n\) gets large,
we need to get rid of the integer parts and the bit sum. We have
\begin{equation}
x - 1 <\! \floor{x} \leqslant x \Rightarrow 2^x < 2^{\floor{x}+1}
\!\leqslant 2^{x+1} \Rightarrow n + 1 \leqslant 2^{\floor{\lg n}+1}
\!\leqslant 2n.
\label{ineq:upper_minus_2}
\end{equation}
The lower bound is tight only when \(n=2^p-1\) and the upper bound
when \(n=2^p\). In other words, the expression \(2^{\floor{\lg n}+1}\)
increases the most between consecutive integers of shape \(2^p-1\) and
\(2^p\). Now we can weaken the upper bound of \(\Bideal(n)\), so it
becomes simpler:
\begin{equation}
\Bideal(n) \leqslant \tfrac{1}{2}{n \lg n} + 3n - (n+1) - 1
                   = \tfrac{1}{2}{n \lg n} + 2n - 2.
\label{ineq:B_upper_simpl}
\end{equation}
The bound is tight if and only if \(n=1\) because it assumed that
\(n\) is both of the form \(2^p\) and \(2^p-1\): this is worse than it
could be but we are not looking for the smallest upper bound here,
only one which helps us to derive the asymptotic behaviour. Let us
endeavour now for a lower bound whose leading term is, hopefully,
asymptotically equivalent to \(\frac{1}{2}n \lg n\). Let us resume
from equation \eqref{eq:B_n_tmp} \vpageref{eq:B_n_tmp} and look for a
lower bound of the first summation \(\sum_{i=0}^{r}{e_i2^{e_i}}\). Up
to this point, we have expressed an arbitrary number \(n>0\) in two
ways: first, as a sum of distinct powers of \(2\) and we wrote \(n
:= \sum_{i=0}^{r}{2^{e_i}}\), where \(e_r > \ldots > e_1 > e_0
\geqslant 0\) and \(r \geqslant 0\) (this point of view focuses on the
\(1\)\hyp{}bits of the binary representation of \(n\)); second, as a
binary string \(n := \sum_{i=0}^{m-1}{b_i2^i} = (b_{m-1}\ldots
b_0)_2\), where \(b_i \in \{0,1\}\) and \(b_{m-1} = 1\). The sequence
\((e_i)_i\) is an increasing subsequence of the positive integers, so
\[
\sum_{i=0}^{r}{e_i2^{e_i}} = \sum_{i=0}^{m-1}{ib_i2^i}.
\]
It is easy to derive a formula for \(b_i\). Let \(n=(b_{m-1} \dots b_1
b_0)_2\). We have 
%= \frac{1}{2^{i+1}}\sum_{k=0}^{i}{b_k2^k} + \sum_{k=i+1}^{m-1}{b_k2^{k-i-1}}\\
\begin{equation*}
\frac{n}{2^{i+1}}
= \frac{1}{2^{i+1}}\sum_{k=0}^{m-1}{b_k2^{k}}
= \frac{1}{2^{i+1}}\sum_{k=0}^{i}{b_k2^k} + (b_{m-1}\dots b_{i+1})_2.
\end{equation*}
We want to prove that \(\floor{n/2^{i+1}} = (b_{m-1}\dots
b_{i+1})_2\), that is,
\[
\frac{1}{2^{i+1}}\sum_{k=0}^{i}{b_k2^k} < 1
\;\Leftrightarrow\;
\sum_{k=0}^{i}{b_k2^k} < 2^{i+1}
\;\Leftarrow\;
\sum_{k=0}^{i}{2^k} = 2^{i+1} - 1 < 2^{i+1},
\]
which obviously holds. So \(2\floor{n/2^{i+1}} = (b_{m-1}\dots
b_{i+1}0)_2\). Similarly, \(\floor{n/2^i} = (b_{m-1}\dots
b_{i+1}b_i)_2\), so, in general, we have
\[
b_i = \left\lfloor\frac{n}{2^i}\right\rfloor
      - 2\left\lfloor\frac{n}{2^{i+1}}\right\rfloor.
\]
We can now use this latter equation to expand the former:
% &= \sum_{i=0}^{m-1}{i\left\lfloor\frac{n}{2^i}\right\rfloor2^i}
%   - \sum_{i=0}^{m-1}{i\left\lfloor\frac{n}{2^{i+1}}\right\rfloor2^{i+1}}\\
 %% &= \sum_{i=0}^{m-1}{i\left\lfloor\frac{n}{2^i}\right\rfloor2^i}
 %%   - \sum_{i=1}^{m}{i\left\lfloor\frac{n}{2^{i}}\right\rfloor2^{i}}
 %%   + \sum_{i=1}^{m}{\left\lfloor\frac{n}{2^{i}}\right\rfloor2^{i}}\\
\begin{align*}
\sum_{i=0}^{r}{e_i2^{e_i}}
 &= \sum_{i=0}^{m-1}{i\left(\left\lfloor\frac{n}{2^i}\right\rfloor - 2
                   \left\lfloor\frac{n}{2^{i+1}}\right\rfloor\right)2^i}\\
 &= \sum_{i=0}^{m-1}{i\left\lfloor\frac{n}{2^i}\right\rfloor2^i}
   - \sum_{i=1}^{m}{(i-1)\left\lfloor\frac{n}{2^{i}}\right\rfloor2^{i}}\\
 &= -m\left\lfloor\frac{n}{2^m}\right\rfloor 2^m
   + \sum_{i=1}^{m}{\left\lfloor\frac{n}{2^{i}}\right\rfloor2^{i}}
  = \sum_{i=1}^{m-1}{\left\lfloor\frac{n}{2^{i}}\right\rfloor2^{i},}
\end{align*}
because \(\floor{n/2^m} = 0\). Moreover, \(\floor{x} > x - 1\) implies
\(\floor{n/2^i} 2^i > n - 2^i\), so we can proceed as follows:
\begin{align}
\sum_{i=0}^{r}{e_i2^{e_i}}
&= \sum_{i=1}^{m-1}{\left\lfloor\frac{n}{2^{i}}\right\rfloor2^{i}}
 \geqslant \sum_{i=1}^{m-1}{(n-2^i)}
 = n(m-1) - \sum_{i=0}^{m-1}{2^i} + 2^0\notag\\
&= n(m-1) - (2^m - 1) + 1
 = n\floor{\lg n} - 2^{\floor{\lg n} + 1} + 2,
\label{ineq:sum_ei_2ei_up}
\end{align}
by equation \eqref{eq:e_r} \vpageref{eq:e_r}: \(m-1=e_r=\floor{\lg
  n}\). Notice that the bound is tight if and only if \(n=1\) (then
\(m=1\) and the two sums are null). This allows us to deduce, from
equation~\eqref{eq:B_n_tmp} \vpageref{eq:B_n_tmp}:
\begin{equation*}
\Bideal(n) \geqslant 
(\tfrac{1}{2}n\floor{\lg n} - 2^{\floor{\lg n}} + 1)
+ n - 1 + \sum_{i=1}^{r}\sum_{j=0}^{i-1}{2^{e_j}}.
\end{equation*}
We can easily obtain a lower bound of the remaining double summation
by remarking that \(e_j \geqslant j\), therefore
\begin{align}
\sum_{i=1}^{r}\sum_{j=0}^{i-1}{2^{e_j}}
&=   \sum_{j=0}^{r-1}{2^{e_j}}
   + \sum_{i=1}^{r-1}\sum_{j=0}^{i-1}{2^{e_j}}
 = (n - 2^{e_r}) + \sum_{i=1}^{r-1}\sum_{j=0}^{i-1}{2^{e_j}}\notag\\
&\geqslant n - 2^{\floor{\lg n}} + \sum_{i=1}^{r-1}\sum_{j=0}^{i-1}{2^j}
= n - 2^{\floor{\lg n}} + \sum_{i=1}^{r-1}{(2^i-1)}\notag\\
%&= n - 2^{\floor{\lg n}} + ((2^r-1)-2^0) - (r-1)\notag\\
&= n - 2^{\floor{\lg n}} + 2^{\nu_n-1} - \nu_n.
\label{ineq:double_sum}
\end{align}
because \(e_r=\floor{\lg n}\) and \(\nu_{n} := r + 1\). The
bound is tight when \(r=0\), that is, \(n=2^{e_0}\). Let us resume:
\begin{align*}
\Bideal(n) 
&\geqslant (\tfrac{1}{2}n\floor{\lg n} - 2^{\floor{\lg n}} +
  1) + n - 1 + (n - 2^{\floor{\lg n}} + 2^{\nu_n-1} - \nu_n)\\
&= \tfrac{1}{2}n\floor{\lg n} + 2n - 2^{\floor{\lg n}+1} + 2^{\nu_n-1} - \nu_n.
\end{align*}
Let \(f(x) := 2^{x-1} - x\) a function defined on real numbers
\(x\). The derivative is \(f'(x) = 2^{x-1}\ln 2 - 1\), which is zero
for \(x = -\lg(\ln 2) \simeq 1.53\) and positive afterwards. We also
easily check that \(f(1) = f(2) = 0, f(3) = 1\) and \(f(4) = 4\), so
we are now certain that \(2^{\nu_n-1} - \nu_n \geqslant 0\) for any
integer \(n>0\), as shown in \fig~\vref{fig:exponential4}.
\begin{figure}
\centering
\includegraphics{exponential4}
\caption{\(f(x) := 2^{x-1} - x\)\label{fig:exponential4}}
\end{figure}
\begin{align}
\Bideal(n)
&\geqslant \tfrac{1}{2}n\floor{\lg n} + 2n - 2^{\floor{\lg n}+1}.
\label{ineq:B_lower}
\intertext{We can weaken the bound with the
inequalities \eqref{ineq:upper_minus_2} \vpageref{ineq:upper_minus_2}:}
\Bideal(n)
&> \tfrac{1}{2}n(\lg n - 1) + 2n - 2n
= \tfrac{1}{2}{n\lg n} - \tfrac{1}{2}{n}.\notag
\end{align}
Together with inequation \eqref{ineq:B_upper_simpl}
\vpageref{ineq:B_upper_simpl}, we finally have proved
\begin{align*}
\tfrac{1}{2}{n\lg n} - \tfrac{1}{2}{n}
&< \Bideal(n) \leqslant
\tfrac{1}{2}{n \lg n} + 2n - 2.
\end{align*}
Now we can draw the asymptotic approximation
\[
\Bideal(n) \mathrel{\sim} \tfrac{1}{2}n\lg n,\;\, \text{as} \,\; n
\rightarrow \infty.
\]
As a reminder, the case \(n=2^{e_0}\) in equation
\eqref{eq:best_power} \vpageref{eq:best_power} was
\[
\Bideal(n) = \tfrac{1}{2}{n\lg n} + n - 1.
\]

\smallskip

\paragraph{Worst delay.}

Let us turn our attention now to the worst delay. We resume the
mathematical definition at equation \eqref{eq:msort_gen}
\vpageref{eq:msort_gen}:
\begin{equation*}
\Dideal(n)
  = \sum_{i=0}^{r}{\Dideal(2^{e_i})}
    +
    \sum_{i=1}^{r}{\comp{merge}{2^{e_i},2^{e_{i-1}}+\dots+2^{e_0}}}.
\end{equation*}
The worst delay of \erlcode{merge/2} is given by equation
\eqref{eq:worst_merge} \vpageref{eq:worst_merge}, that is,
\(\worst{merge}{m,n} = m + n\), with \(m > 0\) and \(n > 0\),
therefore, using equation \eqref{eq:worst_power} on page
\pageref{eq:worst_power}, that is, \(\Wideal(2^p) = p2^p\), we draw
\begin{equation}
\Wideal(n)
  = \sum_{i=0}^{r}{\Wideal(2^{e_i})}
     + \sum_{i=1}^{r}\sum_{j=0}^{i}{2^{e_j}}
  = \sum_{i=0}^{r}{e_i2^{e_i}} + \sum_{i=1}^{r}\sum_{j=0}^{i}{2^{e_j}}.
\label{eq:Wideal}
\end{equation}
If \(r=0\), this equation degenerates into the correct \(\Wideal(n) =
\Wideal(2^{e_0})\) because the double sum cancels out completely. Let
us focus on it:
\begin{align}
  \sum_{i=1}^{r}\sum_{j=0}^{i}{2^{e_j}}
&= \sum_{i=1}^{r}\biggl(\!2^{e_i} + \sum_{j=0}^{i-1}{2^{e_j}}\!\biggr)
 = (n - 2^{e_0}) + \sum_{i=1}^{r}\sum_{j=0}^{i-1}{2^{e_j}}\notag\\
&\geqslant n - 2^{e_0} + (n - 2^{\floor{\lg n}}
          + 2^{\nu_{n}-1} - \nu_{n})\notag\\
&\geqslant 2n - 2^{\floor{\lg n}+1} + 2^{\nu_{n}-1} - \nu_{n}.
\label{ineq:lower_r_i_tmp}
\end{align}
We made use of inequality \eqref{ineq:double_sum}
\vpageref{ineq:double_sum}. We also have \(e_0 \leqslant \floor{\lg
  n}\). The bound is tight only when \(n=2^{e_0}\). We can replace
this inequality back into the definition of \(\Wideal(n)\):
\begin{align*}
\Wideal(n) &\geqslant \sum_{i=0}^{r}{e_i2^{e_i}}
 + 2n - 2^{\floor{\lg n}+1} + 2^{\nu_{n}-1} - \nu_{n}.
\intertext{We already bounded the remaining summation in inequation
\eqref{ineq:sum_ei_2ei_up} \vpageref{ineq:sum_ei_2ei_up}, so we
deduce}
\Wideal(n) &\geqslant n\floor{\lg n} + 2n - 2^{\floor{\lg n}+2}
   + 2^{\nu_{n}-1} - \nu_{n} + 2.
\end{align*}
The bound is tight if \(n=1\). Let us weaken it as we did before in
order to obtain a more legible bound. We already know that
\(2^{\nu_n-1} - \nu_n \geqslant 0\) for any integer \(n>0\), as shown
in \fig~\vref{fig:exponential4}, so
\begin{equation}
\Wideal(n) \geqslant n\floor{\lg n} + 2n - 2^{\floor{\lg n}+2} + 2.
\label{ineq:W_lower}
\end{equation}
We can weaken it further in the usual manner:
\begin{equation}
\Wideal(n) > n(\lg n - 1) + 2n - 2(2n) + 2 = n\lg n - 3n + 2.
\label{ineq:W_lower_simpl}
\end{equation}
Let us turn now to finding a small upper bound for \(\Wideal(n)\).  We
already know how to bound \(\sum_{i=1}^{r}\sum_{j=0}^{i-1}{2^{e_j}}\)
by means of inequation \eqref{ineq:double_upper}
on page~\pageref{ineq:double_upper}, hence
\begin{equation*}
\sum_{i=1}^{r}\sum_{j=0}^{i}{2^{e_j}}
\leqslant (n - 2^{e_0})
           + (2n - 2^{\floor{\lg n} + 1} - \nu_{n} + 1)
\leqslant 3n - 2^{\floor{\lg n}+1} - \nu_{n}.
\end{equation*}
The remaining sum was bounded in inequation \eqref{ineq:upper_n_lg_n}
on page~\pageref{ineq:upper_n_lg_n}, that is,
\(\sum_{i=0}^{r}{e_i2^{e_i}} \leqslant n \floor{\lg n}\), so we can
draw an upper bound for \(\Wideal(n)\):
\begin{equation*}
\Wideal(n) \leqslant 
n\floor{\lg n} + 3n - 2^{\floor{\lg n}+1} - \nu_{n}.
\end{equation*}
The bound is tight only when \(n=1\). As we did previously, we trade
some weakeness for some legibility:
\begin{equation}
\Wideal(n) \leqslant n\floor{\lg n} + 3n - 2^{\floor{\lg n}+1} - 1.
\label{ineq:W_upper}
\end{equation}
Inequations~\eqref{ineq:upper_minus_2} \vpageref{ineq:upper_minus_2}
lead to \(-(n+1) \geqslant -2^{\floor{\lg n}+1}\), so
\begin{equation}
\Wideal(n) \leqslant n \lg n + 3n - (n+1) - 1
= n \lg n + 2n - 2.
\label{ineq:W_upper_simpl}
\end{equation}
Together with inequation \eqref{ineq:W_lower_simpl}, we thus proved
\begin{equation*}
n\lg n - 3n + 2 < \Wideal(n) \leqslant n \lg n + 2n - 2,
\end{equation*}
which imply the same asymptotic behaviour in the special case
\(n=2^{e_0}\):
\[
\Wideal(n) \mathrel{\sim} n\lg n,\;\, \text{as} \,\; n \rightarrow
\infty.
\]
As a reminder, the worst delay when \(n=2^{e_0}\) in
\eqref{eq:worst_power} \vpageref{eq:worst_power} is
\[
\Wideal(n) = n\lg n.
\]

\medskip

\paragraph{Average delay.}

The remaining general case to investigate is the average delay
\(\Aideal(n)\). We can restart with equation \eqref{eq:msort_gen} on
page \pageref{eq:msort_gen}
\begin{equation*}
\Aideal(n)
  = \sum_{i=0}^{r}{\Aideal(2^{e_i})}
    +
    \sum_{i=1}^{r}{\ave{merge}{2^{e_i},2^{e_{i-1}}+\dots+2^{e_0}}}.
\end{equation*}
The average delay of \erlcode{merge/2} is given by equation
\eqref{eq:ave_merge} \vpageref{eq:ave_merge}:
\[
\ave{merge}{p,q} = \frac{pq}{p+1} + \frac{pq}{q+1} + 1.
\]
We draw
\begin{equation}
\Aideal(n)
  = \sum_{i=0}^{r}{\Aideal(2^{e_i})}
     + 
     \sum_{i=1}^{r}{\left(\frac{2^{e_i}\sum_{j=0}^{i-1}{2^{e_j}}}{2^{e_i}+1}
     +
     \frac{2^{e_i}\sum_{j=0}^{i-1}{2^{e_j}}}{\sum_{j=0}^{i-1}{2^{e_j}}
       + 1}
     + 1\right)}.\label{eq:Aideal}
\end{equation}
From inequations \eqref{ineq:up_bin} \vpageref{ineq:up_bin},
\(\sum_{j=0}^{i-1}{2^{e_j}} \leqslant 2^{e_{i}} - 1\), we deduce
\begin{equation*}
\frac{1}{\sum_{j=0}^{i-1}{2^{e_j}} + 1} \geqslant \frac{1}{2^{e_{i}}}
\;\Longrightarrow\;
\frac{1}{2^{e_{i}}+1} + \frac{1}{\sum_{j=0}^{i-1}{2^{e_j}} + 1}
\geqslant
\frac{2^{e_i+1}+1}{2^{e_i}(2^{e_i}+1)}.
\end{equation*}
The left\hyp{}hand side is \(\ave{merge}{p,q}/p/q\). Multiplying by
\(pq\):
\begin{gather*}
2^{e_i}\sum_{j=0}^{i-1}2^{e_j}\left(\frac{1}{2^{e_{i}}+1} +
\frac{1}{\sum_{j=0}^{i-1}{2^{e_j}} + 1}\right) 
\geqslant \frac{2^{e_i+1}+1}{2^{e_{i}}+1}\sum_{j=0}^{i-1}{2^{e_j}}\\
= \left(2 - \frac{1}{2^{e_i}+1}\right)\sum_{j=0}^{i-1}{2^{e_j}}
> \left(2 - \frac{1}{2^{e_i}}\right)\sum_{j=0}^{i-1}{2^{e_j}}
= 2\sum_{j=0}^{i-1}{2^{e_j}} - \sum_{j=0}^{i-1}{2^{e_j-e_i}}.
\end{gather*}
In the remaining sums, we have \(j < i\), and we know that the integer
sequence \((e_i)_i\) is strictly increasing from \(e_0 \geqslant 0\),
therefore \(e_j + 1 \leqslant e_i\) and \(2^{e_j-e_i} \leqslant 1/2\).
We draw
\begin{equation*}
2^{e_i}\sum_{j=0}^{i-1}2^{e_j}\bigg(\frac{1}{2^{e_{i}}+1} +
\frac{1}{\sum_{j=0}^{i-1}{2^{e_j}} + 1}\bigg) 
> 2\sum_{j=0}^{i-1}{2^{e_j}} - \frac{i}{2}.
\end{equation*}
Furthermore
\begin{gather}
\sum_{i=1}^{r}{\ave{merge}{2^{e_i},2^{e_{i-1}}+\dots+2^{e_0}}}
\geqslant
\sum_{i=1}^{r}{\bigg(2\sum_{j=0}^{i-1}{2^{e_j}} - \frac{i}{2} + 1\bigg)}\notag\\
= 2\sum_{i=1}^{r}\sum_{j=0}^{i-1}{2^{e_j}} - \frac{1}{4}{r(r+1)} + r
= 2\sum_{i=1}^{r}\sum_{j=0}^{i-1}{2^{e_j}} -\frac{1}{4}{r(r-3)}.\notag
\end{gather}
The bound is tight only when \(r=0\), that is, \(n=2^{e_0}\). We can
use this inequation on equation \eqref{eq:Aideal} \vpageref{eq:Aideal}:
\begin{equation*}
\Aideal(n) \geqslant \sum_{i=0}^{r}{\Aideal(2^{e_i})} +
2\sum_{i=1}^{r}\sum_{j=0}^{i-1}{2^{e_j}}
-\frac{1}{4}{(\nu_{n}-1)(\nu_{n}-4)},
\end{equation*}
by definition of \(\nu\). We already know how to bound the double sum
in the right\hyp{}hand side by mean of inequation
\eqref{ineq:double_sum} \vpageref{ineq:double_sum}, therefore
\begin{align*}
\Aideal(n)
&\geqslant \sum_{i=0}^{r}{\Aideal(2^{e_i})}
           + 2(n - 2^{\floor{\lg n}} + 2^{\nu_{n}-1} - \nu_{n})
           - \frac{1}{4}{(\nu_{n}-1)(\nu_{n}-4)}\\
&= \sum_{i=0}^{r}{\Aideal(2^{e_i})}
   + 2n - 2^{\floor{\lg n}+1} + 2^{\nu_{n}}
   - \frac{1}{4}(\nu_{n}^2 + 3\nu_{n} + 4).
\end{align*}
We can now make use of the lower bound of \(\Aideal(2^p)\) in
inequation \eqref{ineq:A_power} \vpageref{ineq:A_power}, where
\(\alpha := \sum_{k \geqslant 0}{1/(2^k+1)} \simeq
1.2644997803484442092\):
\begin{align*}
\Aideal(n)
&\geqslant \sum_{i=0}^{r}{(e_i2^{e_i}\!
  -\!(\alpha\!-\!1)2^{e_i}\!-\!1)}
   + 2n - 2^{\floor{\lg n}+1}\! + 2^{\nu_n}\!
   - \frac{\nu_{n}^2+3\nu_{n}+4}{4}\\
%% &= \sum_{i=0}^{r}{e_i2^{e_i}} - (\alpha-1)n 
%%    - \nu_{n} + 2n - 2^{\floor{\lg n}+1}\!
%%    + 2^{\nu_n}\! - \frac{\nu_{n}^2+3\nu_{n}+4}{4}\\
&= \sum_{i=0}^{r}{e_i2^{e_i}} + (3-\alpha)n - 2^{\floor{\lg n}+1}
   + 2^{\nu_n} - \frac{1}{4}{(\nu_{n}^{2} + 7\nu_{n} + 4)}.
\end{align*}
Note that \(3 - \alpha > 0\). We already found a lower bound for the
remaining sum, in inequation \eqref{ineq:sum_ei_2ei_up}
\vpageref{ineq:sum_ei_2ei_up}, therefore
\begin{equation*}
\Aideal(n) \geqslant n \floor{\lg n} + (3 - \alpha)n
- 2^{\floor{\lg n}+2} + 2^{\nu_{n}}
- (\nu_{n}^2 + 7\nu_{n} - 4)/4.
\end{equation*}
Let \(h(x) := 2^x - (x^2 + 7x - 4)/4\) be a function over the
positive real numbers. It has no positive roots and its derivative is
\(h'(x) = 2^x\ln 2 -(2x+7)/4\), whose unique positive root is
approximately \(1.98399\). We have \(h(1) = 1, h(2) = 1/2\), and
\(h(3) = 3/2\). See \fig~\vref{fig:exponential3},
\begin{figure}
\centering
\includegraphics{exponential3}
\caption{\(h(x) := 2^x - (x^2 + 7x - 4)/4\)
\label{fig:exponential3}}
\end{figure}
where we should mind that the minimum is not exactly \(h(2)\). We
deduce that, for any \(n>0\), we have \(2^{\nu_{n}} - (\nu_{n}^2 +
7\nu_{n} - 4)/4 \geqslant 1/2\). We can weaken the lower bound
accordingly:
\begin{equation}
\Aideal(n) \geqslant n \floor{\lg n} + (3 - \alpha)n
             - 2^{\floor{\lg n}+2} + 1/2.
\label{ineq:A_lower}
\end{equation}
We can further weaken it to get a rough, but more legible, bound as
\begin{align*}
\Aideal(n) &> n (\lg n - 1) + (3 - \alpha)n - 4n + 1/2
            = n\lg n - (\alpha + 2)n + 1/2.
\end{align*}
Let us strive now to find a small upper bound of \(\Aideal(n)\) from
equation \eqref{eq:Aideal} \vpageref{eq:Aideal}:
\begin{align*}
\Aideal(n)
  &= \sum_{i=0}^{r}{\Aideal(2^{e_i})}
     + 
     \sum_{i=1}^{r}{\frac{2^{e_i}}{2^{e_i}+1}\sum_{j=0}^{i-1}{2^{e_j}}}
     +
     \sum_{i=1}^{r}{\frac{2^{e_i}\sum_{j=0}^{i-1}{2^{e_j}}}{\sum_{j=0}^{i-1}{2^{e_j}
       + 1}}}
     + r\\
  &\leqslant \sum_{i=0}^{r}{\Aideal(2^{e_i})}
     + 
     \sum_{i=1}^{r}{\sum_{j=0}^{i-1}{2^{e_j}}}
     +
     \sum_{i=1}^{r}{2^{e_i}}
     + r\\
  &= \sum_{i=0}^{r}{\Aideal(2^{e_i})}
     + 
     \sum_{i=1}^{r}{\sum_{j=0}^{i-1}{2^{e_j}}}
     + (n - 2^{e_0}) + (\nu_{n} - 1).     
\intertext{We bounded the double sum in inequation
  \eqref{ineq:double_upper} \vpageref{ineq:double_upper}, thus}
\Aideal(n)
 &\leqslant \sum_{i=0}^{r}{\Aideal(2^{e_i})}
    + (2n - 2^{\floor{\lg n}+1} - \nu_{n} + 1)
+ n + \nu_{n} - 2^{e_0} - 1\\
 &\leqslant \sum_{i=0}^{r}{\Aideal(2^{e_i})} + 3n - 2^{\floor{\lg n}+1}
    - 1.
\end{align*}
The bound is tight only if \(n=1\). Using inequation
\eqref{ineq:A_power} \vpageref{ineq:A_power}, that is, \(\Aideal(2^p)
\leqslant p2^p\), and inequation \eqref{ineq:upper_n_lg_n}
\vpageref{ineq:upper_n_lg_n}, that is, \(\sum_{i=0}^{r}{e_i2^{e_i}}
\leqslant n\floor{\lg n}\), yields
\begin{equation}
\Aideal(n) \leqslant n\floor{\lg n} + 3n - 2^{\floor{\lg n}+1} - 1.
\label{ineq:A_upper}
\end{equation}
Let us eliminate the integer parts and bit sums, so the asymptotic
analysis becomes easier. Inequations \eqref{ineq:upper_minus_2}
\vpageref{ineq:upper_minus_2} yield
\[
\Aideal(n) \leqslant n\lg n + 3n - (n + 1) - 1 = n\lg n + 2n - 2.
\]
In summary, we found the following bounds
\begin{equation*}
n \lg n - (\alpha+2)n + 1/2 < \Aideal(n) \leqslant n\lg n + 2n - 2,
\end{equation*}
which are tight enough to allow us to derive the asymptotic
equivalences
\[
\Aideal(n) \mathrel{\sim} n \lg n \mathrel{\sim}
\Wideal(n),\;\; \text{as} \,\; n \rightarrow \infty.
\]
Again, we find that this is the same as the special case
\(n=2^{e_0}\).

\medskip

\paragraph{The program, finally.}

It may come as a surprise that we have not written the program, except
for the functions \erlcode{solo/1} and \erlcode{merge/2}, but we have
deduced its delay nevertheless. In fact, the delays we found are not
the delays of the program yet to be written, but the delays of an
ideal version without \erlcode{solo/1} and without dealing with all
the empty lists and singletons ending many recursive calls: only
\erlcode{merge/2} was exactly defined and accounted for. Here are the
definitions again:
\begin{verbatim}
merge(   [],      Q)            -> Q;
merge(    P,     [])            -> P;
merge([I|P],Q=[J|_]) when I < J -> [I|merge(P,Q)];
merge(    P,  [J|Q])            -> [J|merge(P,Q)].

solo(   []) -> [];
solo([I|L]) -> [[I]|solo(L)].
\end{verbatim}
The extra delay incurred by \erlcode{solo/1} is known or obvious:
\(\comp{solo}{n} = n + 1\). We need now to complete the program and
sum the extra delays each new function entails. Let us call
\erlcode{level/1} the function which computes the level in the tree
just above the current one, which is the argument.
\begin{verbatim}
level([P,Q|S]) -> [merge(P,Q)|level(S)];
level(      L) -> L.
\end{verbatim}
Note how we do not need to write a specific clause handling the case
when the list is empty: the singleton list and the empty list are
treated likewise, which is really neat. In our previous detailed
analysis, we only counted the delays of the calls to
\erlcode{merge/2}, therefore, we have now to find how many times in
total \erlcode{level/1} is called to sort \(n\) items. The definition
shows that, to one call \erlcode{level([P,Q|S])} corresponds one call
\erlcode{merge(P,Q)}, so, if a level contains \(k\) lists, then the
calls total \(\floor{k/2}+1\). First, let us assume that we sort
\(n=2^p\) items. Previous study revealed that there are \(2^{p-1}\)
merges on the leaves, then \(2^{p-2}\) on the level above etc. until
\(2^0\). The extra delay due to \erlcode{level/1} is thus
\(\sum_{k=0}^{p-1}{2^k} + p\), the additional \(p\) is the number of
empty lists terminating each level, which trigger the second clause of
\erlcode{level/1}. Hence, in total, the extra delay is \(2^p + p -
1\).  Then, the function \erlcode{level/1} must be composed with
itself until one list remains in the level, corresponding to the
root. It follows:
\begin{verbatim}
all([P]) -> P;
all(  L) -> all(level(L)).
\end{verbatim}
Notice that, just as with the definition of \erlcode{level/1}, the
order of the clauses of \erlcode{all/1} is crucial. The extra delay
incurred is the number of levels in the merge tree. Since we assume
that the number of items is \(2^p\), we already know that there are
\(p+1\) levels. Finally, the merge sort function, \erlcode{sort/1},
starts by creating the singletons and then merging all of them and the
resulting lists as well, etc. until one list remains.
\begin{verbatim}
sort([]) -> [];
sort( L) -> all(solo(L)).
\end{verbatim}
Here, the extra delay due to \erlcode{sort/1} is simply \(1\). In sum,
the total extra delay with respect to the ideal version of merge sort
which only took into account the delay of \erlcode{merge/2} is
\[
(2^p + 1) + (2^p + p - 1) + (p+1) + 1 = 2^{p+1} + 2p + 2 
= 2n + 2\lg n + 2.
\]
Let us recall the delays we found earlier in the case \(n=2^p\):
\begin{gather*}
\Bideal(n) = \tfrac{1}{2}{n\lg n} + n - 1;\;
\Wideal(n) = n\lg n;\;\\
n\lg n - (\alpha - 1)n - 1 < \Aideal(n) \leqslant n\lg n.
\end{gather*}
Let us add them the extra delay we just deduced:
\begin{gather*}
\best{sort}{n}  = \Bideal(n) + (2n + 2\lg n + 2)
                = \tfrac{1}{2}{n\lg n} + 3n + 2\lg n + 1;\\
\worst{sort}{n} = \Wideal(n) + (2n + 2\lg n + 2)
                = n\lg n + 2n + 2\lg n + 2;\\
n\lg n + (3-\alpha)n + 2\lg n + 1
< \ave{sort}{n} \leqslant n\lg n + 2n + 2\lg n + 2.
\end{gather*}
It is important to check these involved calculations against some
experiments. Let us thread an integer counter into all function calls
and run a few tests. We do not need to transform the previous program
into tail form. Consider the following method, consisting in handling
each inner call separately and projecting their result in order to
separate the value and the counter, then reconstructing the next call
or the final value.  \verbatiminput{ms_c.erl} \Erlang permits
simplifying the syntax of the \erlcode{case} constructs with only one
clause: instead of
\begin{center}
\erlcode{case \emph{Exp}\(_1\) of \emph{Pattern} ->
  \emph{Exp}\(_2\) end}
\end{center}
we can equivalently write
\begin{center}
\erlcode{\emph{Pattern} = \emph{Exp}\(_1\), \emph{Exp}\(_2\)}.
\end{center}
Now we can check the best case of \fig~\vref{fig:bottom_up_best1}:
\begin{center}
\erlcode{ms\_c:sort([1,2,3,4,5,6,7,8])} \(\twoheadrightarrow\) \erlcode{\{[1,2,3,4,5,6,7,8],43\}},
\end{center}
which means \erlcode{ms:sort([1,2,3,4,5,6,7,8])}
\(\smashedrightarrow{43}\) \erlcode{[1,2,3,4,5,6,7,8]}, where
\erlcode{ms} is the original module without the counter, so this is in
accordance with the prediction \(\best{sort}{8} = 8 \cdot 3/2 + 3
\cdot 8 + 2 \cdot 3 + 1 = 43\). Similarly, the worst case in
\fig~\vref{fig:bottom_up_worst1} leads to the test
\begin{center}
\erlcode{ms\_c:sort([7,3,5,1,4,8,6,2])} \(\twoheadrightarrow\) \erlcode{\{[1,2,3,4,5,6,7,8],48\}},
\end{center}
which means \erlcode{ms:sort([7,3,5,1,4,8,6,2])}
\(\smashedrightarrow{48}\) \erlcode{[1,2,3,4,5,6,7,8]}, in accordance
with the instance \(\worst{sort}{8} = 8 \cdot 3 + 2 \cdot 8 + 2 \cdot
3 + 2 = 48\). We could further use the program we designed in chapter
\vpageref{permutations} to generate all permutations of
\erlcode{[1,2,3,4,5,6,7,8]}, feed them to merge sort and take the
arithmetic mean of the delays. We wrote \verbatiminput{perm.erl} Now
we can write a module \erlcode{ms\_ave} which exports a function
\erlcode{means/1} such that \erlcode{means(\(n\))} rewrites in the
list of the mean values of \erlcode{ms:sort/1} from \(n\) down to
\(1\).  \verbatiminput{ms_ave.erl} Module \erlcode{ms\_ave} depends
upon independent modules \erlcode{ms\_c} and \erlcode{perm}, so these
must be compiled first. Notive how we interleaved the computations of
the sum and the length in one function, \erlcode{ms\_ave:sum\_len/1},
performing only one traversal of the list of permutations and how,
being in tail form, it uses a small, constant amount of the control
stack. A possible session under the \Erlang shell:
\begin{alltt}
Erlang (BEAM) emulator version 5.6.3 [source] [smp:2]
 [async-threads:0] [kernel-poll:false]

Eshell V5.6.3  (abort with ^G)
1> c(ms_c).
\{ok,ms_c\}
2> c(perm).
\{ok,perm\}
3> c(ms_ave).
\{ok,ms_ave\}
4> ms_ave:means(10).
[62.84444444444444,56.62222222222222,46.733333333333334,
 40.733333333333334,34.93333333333333,29.466666666666665,
 21.666666666666668,16.666666666666668,10.0,4.0]
5> \(\talloblong\)
\end{alltt}
Since we expect the resulting numbers to be rational, with the help of
some \emph{computer algebra system}, we can find the exact expression
of these numbers in \fig~\vref{fig:mean_table}.
\begin{figure}
\centering
\includegraphics{mean_table}
\caption{Values of \erlcode{mean(\(n\))} for
         \(1 \leqslant n \leqslant 10\)
         \label{fig:mean_table}}
\end{figure}
\Erlang has the peculiarity that integer arithmetic is implicitly
performed with arbitrary precision, so we can compute integers as big
as the computer memory can hold. Usually, programming languages
require a dedicated library to do this and the integers are mapped by
the run\hyp{}time system to the hardware integers, whose
representation may lead to underflows and overflows. In this respect,
\Erlang is quite a high\hyp{}level language. We have to compare now
\erlcode{mean(\(n\))} with \(\ave{sort}{n}\). We can compute exactly
the latter with help of equation \eqref{eq:ave_power}
\vpageref{eq:ave_power}:
\[
\Aideal(2^p)
  = p2^p + 2^p - 1 - 2^p \sum_{k=0}^{p-1}\frac{1}{2^k+1}.
\]
Therefore,
\[
\ave{sort}{2^p}
  = \Aideal(2^p) + (2^{p+1}+2p+2)
  = 2^p\left(p + 3 - \sum_{k=0}^{p-1}\frac{1}{2^k+1}\right) + 2p + 1.
\]
This yields the comfort to know that
\begin{align*}
\ave{sort}{2} &= 2\left(1 + 3 - \frac{1}{1+1}\right) + 2 \cdot 1 + 1 =
10 = \erlcode{mean(\(2\))};\\
\ave{sort}{4} &= 4\left(2 + 3 - \frac{1}{2} - \frac{1}{2+1}\right) +
2 \cdot 2 + 1 = \frac{65}{3} = \erlcode{mean(\(4\))};\\
\ave{sort}{8} &= 8\left(3 + 3 - \frac{1}{2} - \frac{1}{3} -
\frac{1}{4+1}\right) + 2 \cdot 3 + 1 = \frac{701}{15} =
\erlcode{mean(\(8\))}.
\end{align*}
Now, what if \(n\) is not a power of \(2\)? Let us assume that \(n
\neq 2^p\) is expressed as a binary number with \(m\) bits as
\((b_{m-1}b_{m-2}\dots{b_1}{b_0})_2\), that is, \(n :=
\sum_{i=0}^{m-1}{b_i2^i}\). We have shown by equation \eqref{eq:e_r}
\vpageref{eq:e_r} that \(m=\floor{\lg n} + 1\). Let \(L_n\) be the
contribution of the arrows of function \erlcode{level/1}. As we
pointed out earlier, if level \(k\) contains \(I(k)\) items, there are
\(\floor{I(k)/2}+1\) calls to \erlcode{level/1}, so remains to
determine \(I(k)\). A look back at the example in
\fig~\vref{fig:bottom_up_7}, where \(n=7\), helps us to guess the
answer:
\[
I(k+1) = \ceiling{I(k)/2}\,\; \text{and} \,\; I(0) = n,
\]
meaning that \(k=0\) represents the level of the leaves. This
recurrent definition is equivalent to the closed form \(I(k) =
\ceiling{n/2^k}\) because \(\ceiling{\ceiling{x}/q} = \ceiling{x/q}\).
The proof is as follows. The equality is equivalent to the conjunction
of the two complementary inequalities \(\ceiling{\ceiling{x}/q}
\geqslant \ceiling{x/q}\) and \(\ceiling{\ceiling{x}/q} \leqslant
\ceiling{x/q}\). The former is straightforward because it is a
consequence of \(\ceiling{y} \geqslant y\), for any real number \(y\):
\(\ceiling{x} \geqslant x \Rightarrow \ceiling{x}/q \geqslant x/q
\Rightarrow \ceiling{\ceiling{x}/q} \geqslant \ceiling{x/q}\). Next,
because both sides of the inequality are integers,
\(\ceiling{\ceiling{x}/q} \leqslant \ceiling{x/q}\) is equivalent to
state that \(p \leqslant \ceiling{\ceiling{x}/q} \Rightarrow p
\leqslant \ceiling{x/q}\), for any integer \(p\). An obvious lemma is
that if \(i\) is an integer and \(y\) a real number, \(i \leqslant
\ceiling{y} \Leftrightarrow i \leqslant y\), so the original
inequality is transitively equivalent to \(p \leqslant \ceiling{x}/q
\Rightarrow p \leqslant x/q\), for any integer \(p\), which is
trivially equivalent to \(pq \leqslant \ceiling{x} \Rightarrow pq
\leqslant x\). This implication is true from the same lemma above,
achieving the proof. In the end, \(L_n =
\sum_{k=0}^{?}\left(\floor{\ceiling{n/2^k}/2} + 1\right)\). What is
the greatest value of \(k\)? This is where the hypothesis \(n \neq
2^p\) comes into play. If we compare \fig~\ref{fig:bottom_up_7} on
page~\pageref{fig:bottom_up_7} with \fig~\vref{fig:bottom_up1}, where
\(n=2^3\), we find that \(k=m\) for the former and \(k=m-1\) for the
latter. Therefore, if \(n \neq 2^p\),
\[
L_n = \sum_{k=0}^{m}
\left\lfloor\frac{1}{2}\left\lceil\frac{n}{2^k}\right\rceil\right\rfloor
+ (m + 1)
= \sum_{k=0}^{m}
\left\lfloor\frac{1}{2}\left\lceil\frac{n}{2^k}\right\rceil\right\rfloor
+ \floor{\lg n} + 2.
\]
For all integers \(q\), \(q = \floor{q/2} + \ceiling{q/2}\) holds,
hence \(q=\ceiling{n/2^k}\) yields
\begin{align*}
L_n &= \sum_{k=0}^{m}\left(\left\lceil\frac{n}{2^k}\right\rceil -
\left\lceil\frac{1}{2}\left\lceil\frac{n}{2^k}\right\rceil\right\rceil\right)
+ \floor{\lg n} + 2\\ 
&= \sum_{k=0}^{m}{\!\left\lceil\frac{n}{2^k}\right\rceil} \! - \!
\sum_{k=0}^{m}{\!\left\lceil\frac{n}{2^{k+1}}\right\rceil} + \floor{\lg n} + 2
 = \left\lceil\frac{n}{2^0}\right\rceil \! - \!
 \left\lceil\frac{n}{2^{m+1}}\right\rceil + \floor{\lg n} + 2\\
&= n + \floor{\lg n} + 2.
\end{align*}
As far as \erlcode{all/1} is concerned, we determined previously that
it contributes as much as the number of levels in the merge tree,
which is here \(m+1=\floor{\lg n} + 2\). We have the extra \(1\)
due to \erlcode{sort/1} and the \(n+1\) from \erlcode{solo/1}. In sum,
the additional delay to reach the actual implementation delay is
\[
(n + \floor{\lg n} + 2) + (\floor{\lg n} + 2) + 1 + (n+1)
= 2n + 2\floor{\lg n} + 6.
\]
Let us remind of the intervals for the delays of the ideal version of
merge sort and add this extra time to get the actual intervals:
\begin{align}%%BOUNDS
\best{sort}{n} &\geqslant n\floor{\lg n}/2 + 4n - 2^{\floor{\lg n}+1} 
    + 2\floor{\lg n} + 5,\notag\\
\best{sort}{n} &\leqslant n\floor{\lg n}/2 + 5n
    - 2^{\floor{\lg n}+1} + 2\floor{\lg n} + 5;\notag\\
\worst{sort}{n} &\geqslant n\floor{\lg n} + 4n - 2^{\floor{\lg n}+2}
   + 2\floor{\lg n} + 8,\notag\\
\worst{sort}{n} &\leqslant n\floor{\lg n} + 5n
                 - 2^{\floor{\lg n}+1} + 2\floor{\lg n} + 5;\notag\\
\ave{sort}{n} &\geqslant n \floor{\lg n} + (5 - \alpha)n
- 2^{\floor{\lg n}+2} + 2\floor{\lg n} + 13/2,\notag\\
\ave{sort}{n} &\leqslant n\floor{\lg n} + 5n - 2^{\floor{\lg n}+1}
               + 2\floor{\lg n} + 5.
\label{ineq:Asort_upper}
\end{align}
The derivations ensure that these delays are valid for any integer
\(n>0\). It is complicated to caracterise the best and worst cases but
we can put to the test the average delay because we already have some
exact values from instrumenting the actual code by adding a counter
and a permutation generator. Let us use \Erlang to compute the bounds:
\verbatiminput{mean.erl} Function \erlcode{trunc/1} is
predefined. Module \erlcode{math} is also predefined and
\erlcode{math:log/1} implements the Napierian logarithm. The
approximate value of \(\alpha\) can be compute in \Erlang, but we
preferred to use a computer algebra system. Note how we efficiently
computed the binary exponentiation \(2^n\) by means of the recurrent
equations
\begin{equation}
2^0 = 1,\quad 2^{2m} = (2^m)^2,\quad 2^{2m+1} =  2(2^m)^2.
\label{def:exp2}
\end{equation}
The delay \(\comp{exp2}{n}\) thus satisfies the equations
\[
\comp{exp2}{0} = 1;\quad \comp{exp2}{n} = 1 +
\comp{exp2}{\floor{n/2}},\;\, \text{if} \;\, n > 0.
\]
Therefore, if \(n>0\), it is \(1\) plus the number of bits of \(n\),
that is, \(\comp{exp2}{n} = \floor{\lg n} + 2\), else \(\comp{exp2}{0} =
1\). We can now complete the table of the bounds of \(\ave{sort}{n}\)
and check them against the experimental results \erlcode{mean(\(n\))}
in \fig~\vref{fig:mean_ms}.
\begin{figure}
\centering
\includegraphics[bb=71 662 341 721]{mean_ms}
\caption{Checking average delay of merge sort when \(n \neq
  2^p\)\label{fig:mean_ms}}
\end{figure}
\medskip

\paragraph{Stability.}

This implementation of merge sort is unstable because
\erlcode{merge/2} does not preserves the relative order of equal
numbers. In order to obtain stability, we should write instead
\begin{alltt}
merge(   [],      Q)            -> Q;
merge(    P,     [])            -> P;
merge(P=[I|_],[J|Q]) when \textbf{I > J} -> [J|merge(P,Q)];\hfill% \emph{Here}
merge(  [I|P],    Q)            -> [I|merge(P,Q)].
\end{alltt}
This works because the first list we pass to \erlcode{merge/2} always
contains items which were located \emph{before} the items of the
second list.

\medskip

\paragraph{Improvement.}

It is easy to improve our implementation of merge sort by avoiding the
construction of the initial list of singletons, that is, the leaves of
the merge tree, and instead build directly the lists with two items,
\emph{without using \erlcode{merge/2}}. Let us name the function in
charge of this \erlcode{duo/1}. The program is as follows:
\verbatiminput{ms_opt.erl} This simple modification saves
\(\comp{solo}{n} = n+1\) function calls and the first call to
\erlcode{all/1} and \erlcode{level/1}, that is, \(1\) for the call to
\erlcode{all/1}, plus \(\floor{n/2}\) calls to \erlcode{merge/2} on
pairs of singletons, plus \(1\) for the last singleton or the empty
list in \erlcode{level/1}, that is, \(\floor{n/2}\comp{merge}{1,1} + 2
= 2 \floor{n/2} + 2\). Instead, we have the delay of \erlcode{duo/2},
that is, an additional delay of \(\floor{n/2} + 1\). The total delay
is hence decreased by
\[
(n + 1 + 2\floor{n/2} + 2) - (\floor{n/2} + 1) = n + \floor{n/2} + 2.
\]
By modifying module \erlcode{ms\_opt} so that all functions thread a
counter and by testing some cases we already know, we can reinforce
our conviction that our deduction is correct.

\medskip

\paragraph{Higher\hyp{}order merge sort.}

Just as we made insertion sort able to sort any kind of data as long
as it is totally ordered by a supplied function, we can make merge
sort by threading the comparison: \verbatiminput{ms_ho.erl} Now the
following calls become possible:
\begin{center}
\erlcode{ms\_ho:sort(fun(I,J)->I>J end,[4,3,1,0,2])}
\(\twoheadrightarrow\) \erlcode{[0,1,2,3,4]};\\
\erlcode{ms\_ho:sort(fun(I,J)->I<J end,[4,3,1,0,2])}
\(\twoheadrightarrow\) \erlcode{[4,3,2,1,0]}.
\end{center}
Stability is useless if the only type of data to be sorted is integer,
so let us take an association list whose keys are integers but data is
not:
\begin{alltt}
ms_ho:sort(fun(\{I,_\},\{J,_\}) -> I > J end,
           [\{6,six\},\{1,one\},\{6,seis\},\{2,two\}]).
\end{alltt}
The result is \erlcode{[\{1,one\},\{2,two\},\{6,six\},\{6,seis\}]}
because the relative order of \erlcode{\{6,six\}} and
\erlcode{\{6,seis\}} is preserved.

\medskip

\paragraph{Comparison with insertion sort.}

We introduce Landau's notation, which we define as
\begin{equation}
f(n) \in \landau{g(n)} \Leftrightarrow \lim_{n \rightarrow
  \infty}{f(n)/g(n)} = 0.\label{eq:Landau}
\end{equation}
Equivalence~\eqref{ineq:ln} \vpageref{ineq:ln} imply \(\lim_{n
  \rightarrow \infty}(\lg n)/n = 0\), which we can equivalently
express with Landau's notation as \(\lg n \in \landau{n}\). Thus,
\(n\lg n \in \landau{n^2}\). Moreover, a consequence of inequations
\eqref{ineq:i2wb} \vpageref{ineq:i2wb} is that the average delay
\(\ave{i2wb}{n}\) of the two\hyp{}way insertion sort is
lower\hyp{}bounded as follows:
\begin{equation*}
\tfrac{1}{8}(n^2 - 13n + \ln n - 10 + \ln 2) < \ave{i2wb}{n}.
\end{equation*}
In this chapter, we established inequation \eqref{ineq:Asort_upper}
\vpageref{ineq:Asort_upper} about the upper bound of merge sort, which
we can weaken a little bit to make it more wieldy:
\begin{align*}
\ave{sort}{n} 
&\leqslant n\floor{\lg n} + 5n - 2^{\floor{\lg n}+1}
+ 2\floor{\lg n} + 5\\
&\leqslant n\lg n + 4n + 2\lg n + 4.
\end{align*}
Therefore, we expect the existence of \(n_0 > 0\), such that, for all
\(n \geqslant n_0\),
\[
\ave{sort}{n} < \ave{i2wb}{n}.
\]
The problem is to determine \(n_0\), that is, we want to solve the
inequation
\[
n\lg n + 4n + 2\lg n + 4
< \tfrac{1}{8}(n^2 - 13n + \ln n - 10 + \ln 2).
\]
This is where a computer algebra system comes handy to deduce that it
is equivalent to \(n \geqslant 100\), that is, \(n_0 = 100\). Since we
worked on bounds of the delays, this result does not mean that the
crossing of the averages happens at \(n=100\), only that we are
certain that it happened whenever \(n \geqslant 100\). In order to get
a lower bound, let us consider the dual situation: we want the upper
bound of \(\ave{i2wb}{n}\) to be lower than the lower bound of
\(\ave{sort}{n}\). To make the point clearer, we still consider the
fastest version of insertion sort and the slowest version of merge
sort we found. Another implication of \eqref{ineq:i2wb}
\vpageref{ineq:i2wb} is
\begin{equation*}
\ave{i2wb}{n} < \tfrac{1}{8}(n^2 - 13n + \ln n - 10 + \ln 2 + 7).
\end{equation*}
Furthermore, we found \vpageref{ineq:Asort_upper} a lower bound we can
now weaken as usual:
\begin{align*}
\ave{sort}{n} 
&> n\floor{\lg n} + (5-\alpha)n - 2^{\floor{\lg n}+2}
+ 2\floor{\lg n} + 13/2\\
&> n\lg n - \alpha n + 2\lg n + 5/2.
\end{align*}
With a computer algebra system, we obtain
\[
1 \leqslant n \leqslant 50 \Rightarrow \ave{i2wb}{n} < \ave{sort}{n}.
\]
In other words, insertion sort is certainly faster in average than
merge sort for less than \(50\)~items and it is certainly slower in
average for more than \(100\)~items. We do not have enough precision
to decide in\hyp{}between. Anyhow, this shows that, when comparing the
asymptotic delays of two sorting algorithms, we should not conclude of
the superiority of one over the other for small inputs. Also, this
result entices us to drop the function \erlcode{solo/2} (even
\erlcode{duo/2}) in favour of a function which constructs chunks of
\(50\)~items from the original list, then sort them using balanced
two\hyp{}way insertions and, finally, if there are more than
\(50\)~items, start merging these sorted lists. In reference to the
example in \fig~\vref{fig:bottom_up_7}, this improvement amounts to
not constructing the first \(50\)~levels in the tree, starting from
the leaves, and instead use two\hyp{}way insertion sort to directly
build the \(50\)th level.

\medskip

\paragraph{Minimum\hyp{}comparison sorting.}

In several instances, we claimed that a good sorting algorithm has a
worst delay asymptotically equivalent to \(n\lg n\), without any
justification. In order to substantiate this statement, we need to
investigate \emph{minimum\hyp{}comparison sorting}. First, it is
important to realise that there are sorting algorithms which do not
rely on comparisons, although they are not suited for being
implemented in purely functional languages. Second, for the sake of
simpli\-city, we shall restrict the present theoretical discussion to
\emph{distinct items}. We model the outcome of sorting \(n\) distinct
objects \(a_1\), \(a_2\), \ldots, \(a_n\), as a permutation of
\((a_1,\dots,a_n)\). There are \(n!\) possible sorts, as we found out
in chapter \vref{permutations}. For instance, let us represent in the
tree of \fig~\vref{fig:comparison_tree} all the comparisons for
sorting three numbers.
\begin{figure}[t]
\centering
\includegraphics[bb=72 626 335 721]{comparison_tree}
\caption{A comparison tree for sorting three items
\label{fig:comparison_tree}}
\end{figure}
The \emph{external nodes} are all the permutations of
\((a_1,a_2,a_3)\). The \emph{internal nodes} are comparisons between
two items, say, in general, \(a_i\) is compared to \(a_j\) and noted
\(a_i\)?\(a_j\). If \(a_i<a_j\), then this property holds everywhere
in the left subtree, otherwise \(a_i>a_j\) holds in the right
subtree. Note that this tree is one possible amongst many: it
corresponds to an algorithm which starts by comparing \(a_1\) and
\(a_2\) and there are, of course, many other strategies. It does not
perform redundant comparisons, though. \Fig~\vref{fig:redudant_comp}
shows an excerpt of a comparison tree with such a useless comparison.
\begin{figure}[b]
\centering
\includegraphics[bb=72 638 238 721]{redundant_comp}
\caption{Comparison of \(a_1\) to \(a_3\) is useless
\label{fig:redudant_comp}}
\end{figure}
The special external node \(\bot\) corresponds to no permutation
because the comparison \(a_1<a_3\) cannot fail as it is implied by
transitivity of the previous comparisons on the path from the
root. \emph{A comparison tree for \(n\) items without redundancy has
  \(n!\) external nodes.} Because we are investigating
minimum\hyp{}comparison sorting, we shall consider henceforth
comparison trees with \(n!\) external nodes. Furthermore, amongst
them, we want to determine the trees such that the maximum number of
comparisons is minimum.

The maximality constraint means that we must consider the longest
paths from the root to a leaf (a leaf is an internal node whose two
children are external nodes, see definitions \vpageref{def:tree})
because the number of comparison nodes along those paths is an upper
bound for sorting all the permutations. For example, in
\fig~\vref{fig:comparison_tree}, there are four paths satisfying this
maximality constraint and they contain \(3\) comparisons each, for example,
\((a_1<a_2)\rightarrow(a_2<a_3)\rightarrow(a_1<a_3)\). Let us note
\(S(n)\) the number of nodes on a maximum path, which is usually
called the \emph{height} of the tree, for example, \(S(3)=3\) in
\fig~\vref{fig:comparison_tree}. The minimality constraint in the
problem statement above then means that \emph{we want a lower bound on
  the height of a comparison tree with \(n!\) external nodes.}

A \emph{complete binary tree} is a binary tree whose internal nodes
have children which are either two internal nodes or two external
nodes. If such a tree has height \(h\), then it has \(2^h\) external
nodes. For instance, \fig~\vref{fig:complete_tree} shows the case
where the height \(h\) is 3 and there are indeed \(2^h=8\) external
nodes, figured as \(\Box\).
\begin{figure}[b]
\centering
\includegraphics[bb=71 635 294 721]{complete_tree}
\caption{Complete binary trees of height \(h\) have \(2^h\) external
  nodes.\label{fig:complete_tree}}
\end{figure}
Since we know that the minimum\hyp{}comparison trees have \(n!\)
external nodes and height \(S(n)\), they must contain less external
nodes than a complete binary tree of identical height, that is, the
following holds:
\begin{equation*}
n! \leqslant 2^{S(n)}.
\end{equation*}
This inequality implies the following (perhaps best read leftward):
\begin{equation*}
\sum_{k=1}^{n}{(\floor{\lg k} + 1)} - n
= \sum_{k=1}^{n}{\floor{\lg k}}
\leqslant \sum_{k=1}^{n}{\lg k}
= \lg n! \leqslant \lg\left(2^{S(n)}\right) = S(n).
\end{equation*}
The reason for manifesting the term \(\floor{\lg k} + 1\) is that its
value is the number of bits of \(k\), so the remaining sum is the
total number of bits of the integers from~\(1\) to~\(n\). The table in
\fig~\vref{fig:bit_table} 
\begin{figure}[t]
\centering
\includegraphics[bb=71 463 146 721]{bit_table}
\caption{Binary numbers from \(1\) to \(n\)\label{fig:bit_table}}
\end{figure}
shows the enumeration of the first integers in binary. The greatest
power of \(2\) smaller than \(n\) is \(2^{\floor{\lg n}}\) because it
is the number \((10\dots0)_2\) with the same number of bits as \(n\)
(so it appears in the same section of the table as \(n\)). The trick
consists in counting the bits in columns, from top to bottom, and
leftward. In the rightmost column, we find \(n\) bits. In the second
column, from the right, we find \(n-2^1+1\) bits. The third from the
right contains \(n-2^2+1\) bits etc. until the leftmost column
containing \(n-2^{\floor{\lg n}}+1\) bits. The total number of bits in
the table is
\begin{align*}
\sum_{k=1}^{n}{\!(\floor{\lg k}+1)}
  &= \sum_{k=0}^{\floor{\lg n}}{\!(n-2^k+1)}
   = (n+ 1)(\floor{\lg n} + 1) - 2^{\floor{\lg n}+1} + 1\\
  &\geqslant (n\!+\!1)(\floor{\lg n}\! +\! 1) - 2n + 1
  = n\floor{\lg n} - n + \!\floor{\lg n} + 3.
\end{align*}
We can finally weaken a little bit the lower bound of \(S(n)\) as
follows:
\[
n\lg n - 3n + \lg n + 2 < n\floor{\lg n} - 2n + \floor{\lg n} + 3
                        \leqslant S(n),
\]
\noindent 
because \(x - 1 < \floor{x}\). This proves that any optimal sorting
algorithm, solely based on comparisons, requires at least
\(n\floor{\lg n} - 2n\) comparisons to sort any list of \(n\)
items. Since we found in inequation~\eqref{ineq:W_upper_simpl} on
page~\pageref{ineq:W_upper_simpl} that the worst delay of merge sort,
\(\Wideal(n)\), satisfies
\[
\Wideal(n) \leqslant n\lg n + 2n - 2
\]
and, since \(S(n) \leqslant \Wideal(n)\), we now have converging
bounds for the so\hyp{}called \emph{minimax} delay:
\[
n\lg n - 3n + \lg n + 2 < S(n) \leqslant n\lg n + 2n - 2
\]
and thus \(S(n) \mathrel{\sim} n \lg n \mathrel{\sim} \Wideal(n)
\mathrel{\sim} \Aideal(n)\). This means that merge sort is, in the
worst case and asymptotically, as good as any optimal sorting
algorithm. However, this does not mean that it cannot be beaten by
another algorithm for small values of~\(n\) or in average.

\medskip

\paragraph{Exercises.}
\label{ex:merge_sort}

\noindent [See answers page~\pageref{ans:merge_sort}.]
\begin{enumerate}

  \item \label{ex:merge_sort:1} A theoretical way to compare sorting
    algorithms based on comparisons is to compare only the number of
    comparisons they perform. Determine \(\Bcomp(n)\), \(\Wcomp(n)\)
    and \(\Acomp(n)\), which are, respectively, the best, worst and
    average number of comparisons in the ideal merge sort
    algorithm. Deduce a better upper bound for \(S(n)\).

  \item \label{ex:merge_sort:2} Find better approximations of
    \(\Bideal(n)\) and \(\Wideal(n)\) when \(n=2^p-1\). Compare
    \(\Bcomp(2^p-1)\) with \(\Bcomp(2^p)\) and \(\Wcomp(2^p-1)\) with
    \(\Wcomp(2^p)\).

  \item \label{ex:merge_sort:3} When looking for a lower bound of
    \(S(n)\), we used the inequality \(\sum_{k=1}^{n}{\floor{\lg k}} <
    \sum_{k=1}^{n}{\lg k}\), which is quite rough. If we allow
    ourselves to work temporarily on real numbers, we can harvest the
    fruits of real analysis as follows. Let two integers \(a \leqslant
    b\) and \(f:[a,b] \rightarrow \mathbb{R}\) a real\hyp{}valued,
    increasing and Riemann\hyp{}integrable function. Then
    \[
      \sum_{k=a}^{b-1}f(k) \leqslant \int_{a}^{b}{\!\!f(x)} \,dx
                         \leqslant \sum_{k=a+1}^{b}{\!\!\!f(k)}.
    \]
    Assume this theorem and derive a better lower bound for \(S(n)\).

  \item \label{ex:merge_sort:4} The derivation of simple bounds is not
    as accurate as we would like because we made over\hyp{}pessimistic
    assumptions. For instance, the inequalities
    \[
    \Wideal(n) \geqslant n\floor{\lg n} + 2n - 2^{\floor{\lg n}+2} + 2
                       > n(\lg n - 1) + 2n - 4n + 2
    \]
    rely on \(\floor{\lg n} > \lg n - 1\) and \(n \geqslant
    2^{\floor{\lg n}}\). The latter is tight if \(n\) is a power of
    \(2\). But if it is so, we could have used an equality in stead of
    the former inequality: \(\floor{\lg n} = \lg n\). Prove the finer
    \[
    \begin{array}{r@{\,}c@{\,}c@{\,}c@{\,}l}
    \frac{1}{2}{n\lg n} & \leqslant & \Bideal(n)
       & \leqslant & \tfrac{1}{2}{n \lg n} + \tfrac{3}{2}{n} - 1,\\
    n\lg n - 2n + 2 & \leqslant & \Wideal(n)
                    & \leqslant & n\lg n + \tfrac{3}{2}{n} - 1,\\[1pt]
    n\lg n - (\alpha + 1)n + \tfrac{1}{2} & \leqslant & \Aideal(n)
                               & \leqslant & n\lg n + \tfrac{3}{2}n - 1.
    \end{array}
    \]
    Using Question~\ref{ex:merge_sort:1}, find a smaller upper bound
    for \(S(n)\).

    \noindent \emph{Hint.} In general, we want a lower bound such that
    \(n\lg n + \lambda n \leqslant n\floor{\lg n} + \psi n - \omega
    2^{\floor{\lg n}}\), with \(\psi > 0\), \(\omega > 0\) and
    \(\lambda\) is maximum. In the example above, \(\psi = 2\) and
    \(\omega = 4\) (the constant term can be cancelled on both sides)
    and \(\lambda = -2\). The dual case for upper bounds is obtained
    by reversing the comparison and minimising \(\lambda\). Let us
    allow \(n\) to range over the real numbers and represent it in the
    mantissa\hyp{}exponent notation in base \(2\), that is, we set
    \(n=q2^p\), where \(p \in \mathbb{N}, q \in \mathbb{R}\) and \(1
    \leqslant q < 2\). Therefore \(\lg n = p + \lg q\), with \(0
    \leqslant \lg q < 1\), and thus \(\floor{\lg n} = p\). Use this
    representation in the inequality, study the resulting
    real\hyp{}valued function of \(q\) and maximise \(\lambda\) on the
    real interval \([1,2]\), depending on conditions upon \(\psi\) and
    \(\omega\).

  \item \label{ex:merge_sort:5} Implement in two \Erlang clauses the
    function \(\nu\) (bit sum).

  \item \label{ex:merge_sort:6} Let \(\rho_n\) the largest power of
    \(2\) dividing \(n\). In other words, \(\rho_n = 2^{e_0}\) if we
    let \(n=2^{e_r} + \dots + 2^{e_1} + 2^{e_0} > 0\), with \(e_r >
    \dots > e_1 > e_0 \geqslant 0\) and \(r \geqslant 0\). This
    function is called the \emph{ruler function} and can also be
    understood as computing the number of trailing zeros in the binary
    representation of \(n\). Define recursively \(\rho_n\) and prove
    \[
    \rho_n = 1 + \nu_{n-1} - \nu_{n},
    \]
    where \(\nu_n\) is the sum of the bits of \(n\). Finally conclude that
    \[
      \sum_{k=1}^{n}{\rho_k} = n - \nu_n \sim n,\; \text{as} \; n
      \rightarrow \infty.
    \]
    %% Prove independently that
    %% \[
    %% \sum_{k>0}{\left\lfloor\frac{n}{2^k}\right\rfloor} = n - \nu_n.
    %% \]

  \item \label{ex:merge_sort:7} Let \(\Bcomp(n)\) be the number of
    comparisons in the best case of the ideal merge sort. Prove what
    Question~\ref{ex:merge_sort:2} suggested:
    \[
    \Bcomp(n+1) = \Bcomp(n) + \nu_{n}.
    \]
    Examine vertically \fig~\vref{fig:bit_table} and deduce simple
    bounds on \(\sum_{k=0}^{n-1}{\nu_k}\). (Note that the row \(k=0\)
    is missing in the table but it matters even though \(\nu_0 = 0\).)
    Deduce bounds on \(\Bcomp(n)\) and compare them with the ones in
    Question~\ref{ex:merge_sort:1} and \ref{ex:merge_sort:4}.

    \noindent \emph{Hint.} Let us consider \(n\) expressed as a sum of
    powers of \(2\): \(n := \sum_{i=0}^{r}{2^{e_{i}}}\). When
    \(n\) is even, \(e_0>0\) and all the merge trees in their best
    cases are isomorphic to the tree displayed in
    \fig~\vref{fig:bcomp_2p_merge}, 
\begin{figure}[t]
\centering
\subfloat[Nodes with list lengths\label{fig:bcomp_2p_merge}]{%
  \includegraphics[bb=71 640 165 721]{bcomp_2p_merge}
}
\quad
\subfloat[Nodes with best delays\label{fig:bcomp_2p}]{%
  \includegraphics[bb=71 632 175 721]{bcomp_2p}
}
\caption{Best delay mergings for \(n\) even\label{fig:Bcomp_2p}}
\end{figure}
whose nodes contain the lengths of the lists in the corresponding
merge tree. They are also isomorphic to the tree in
\fig~\vref{fig:bcomp_2p}, whose nodes contain the best delays of the
corresponding merges in the merge tree. The total best delay
\(\Bcomp(n)\) is the sum of the delays of all the nodes. The
complementary situation, \(n\) odd, means that \(n\) can be written in
binary as \(\sigma 01^q\), where \(\sigma\) is some bit sequence and
\(q\) some positive integer. In terms of the decomposition into powers
of \(2\), this is equivalent to say that there exists a positive \(q\)
such that \(e_q>q\) and, for all \(i < q\), \(e_i=i\). This leads to
the trees in \fig~\ref{fig:Bcomp_2p_1} on
page~\pageref{fig:Bcomp_2p_1}. Make the trees for \(n+1\) from the
cases ``\(n\) even'' and ``\(n\) odd.''  Conclude that \(\Bcomp(n+1) =
\Bcomp(n) + \nu_{n}\), whatever the parity of \(n\) is. Does the same
recurrent equation hold for \(\Wcomp\), as
Question~\ref{ex:merge_sort:2} suggested?
\begin{figure}
\centering
\subfloat[Nodes with list lengths\label{fig:bcomp_2p_1_merge}]{%
\includegraphics[bb=71 610 193 721]{bcomp_2p_1_merge}
}
\quad
\subfloat[Nodes with best delays\label{fig:bcomp_2p_1}]{%
\includegraphics[bb=71 607 198 721]{bcomp_2p_1}
}
\caption{Best delay mergings for \(n\) odd\label{fig:Bcomp_2p_1}}
\end{figure}

  \item \label{ex:merge_sort:8} We mentionned that the beauty of our
    implementation of merge sort is the fact that no provision is made
    for handling the unbalanced merges: the merge tree is built
    bottom\hyp{}up in a uniform manner. \Fig~\vref{fig:msort_gen}
    shows the sequence in which the unbalanced merges are
    performed. Abstractly, we could order them as we wish, so the
    question naturally arises: does this scheme lead to the fastest
    algorithm in the worst or average case? \Fig~\ref{fig:msort_gen}
    can be considered as an extreme case, because the unbalanced
    merges are computed from right to left. Another extreme case is
    the exact reverse order: from left to right, as shown in
    \fig~\vref{fig:msort_var}. Would you expect the worst delay to be
    worse than merging leftward? Same question about the best delay.
  \item \label{ex:merge_sort:9} Prove \(n\floor{\lg n} + 2n -
    2^{\floor{\lg n}+1} = n\ceiling{\lg n} + n - 2^{\ceiling{\lg
        n}}\).

  \item \label{ex:merge_sort:10} Use theorem of
    Question~\ref{ex:merge_sort:4} to derive a better lower bound for
    \(\ave{sort}{n}\) and then \(1 \leqslant n \leqslant 26
    \Rightarrow \ave{i2wb}{n} < \ave{sort}{n}\).

\end{enumerate}
\begin{figure}[!b]
\centering
\includegraphics[bb=71 640 164 721]{msort_var}
\caption{Unbalanced merges from left to right\label{fig:msort_var}}
\end{figure}

